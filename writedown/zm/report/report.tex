\documentclass[a4paper,reprint,unsortedaddress,onecolumn]{revtex4-1}
% \documentclass[reprint,unsortedadress,oneclumn]{revtex4-1}

\usepackage{amsmath,amssymb,amsfonts,latexsym}
\usepackage[dvips]{graphicx}
\usepackage{color}
\usepackage{indentfirst} 

\newcommand{\eps}{\varepsilon}
\newcommand{\recheck}[1]{{\color{red} #1}}
\newcommand{\redc}[1]{{\color{red} #1}}            
\newcommand{\bluec}[1]{{\color{blue} #1}}            
\newcommand{\vect}[1]{\textbf{\textit{#1}}}


\begin{document}

\title{A Critical Appraisal of Zero-Multiple Method in Simulation Biomolecular Systems}
\author{Han Wang}
\email{han.wang@fu-berlin.de}
\affiliation{Institute for Mathematics, Freie Universit\"at Berlin, Germany}
\author{Ikuo Fukuda}
\email{ifukuda@protein.osaka-u.ac.jp}
\affiliation{Institute for Protein Research, Osaka University, Japan}

\begin{abstract}
\end{abstract}

\maketitle


\begin{table}
  \centering
  \caption{A list of systems tested in this work. We report, from left to right, the number of water
    molecules in the system (TIP3P model), box size, length of equilibriation, length of productive trajecty and the frequency
    that the configurations are recorded or the phsical properties are calculted.}
  \begin{tabular*}{0.9\textwidth}{@{\extracolsep{\fill}}c rrrrr}\hline\hline
    System      & \# of Water Mols. & Box size [$\textrm{nm}^3$]  & Equilibriation [ps] & Traj. Length [ps] & Freq. Record [ps]  \\\hline
    S1          & 4,500          & $5.15^3$              & 3,000        & 17,000        & 5.0 \\
    S1'         &   4,500        & $5.15^3$              & --           &  6,000        & -- \\
    L0          &   13,824       & $7.49^3$              & 200          & 1,800         & 0.1 \\
    L1          &   13,824       & $7.49^3$              & 3,000        & 17,000        & 10.0 \\
    L1'         &   13,824       & $7.49^3$              & --           & 10,000        & -- \\
    \hline\hline
  \end{tabular*}
  \label{tab:tmp1}
\end{table}

The systems used for MD simulations are listed in
Table.\ref{tab:tmp1}.  The initial configurations are prepared by in
the following steps: (1) A NPT simulation of system L0 with SPME~\cite{darden1993pme, essmann1995spm}
electrostatic. The equilibrium density of the system is calculated
from this simulation. (2) Initial configurations that has 4,500 and
13,824 molecules are generated at the calculated equilibrium
density. (3) NVT simulations with SPME electrostatic for system S1'
and L1' are performed. (4) The output configuration of S1' is used as
initial configuration for system S1, and the output configuration of
L1' is used as initial configuration for system L0 and L1.  The SPME
parameters for all simulations reported by this work is: real space
cut-off 1.90~nm. The cut-off is smoothed from 1.85 to 1.90~nm by the
``\texttt{switch}'' method provide by Gromacs
4.6~\cite{hess2008gromacs, pronk2013gromacs}. The reciprocal space
grid spacing is 0.06~nm. The B-Spline interpolation order is 6. The
splitting parameter is optimized by the error estimate for the SPME method~\cite{wang2010optimizing}.

\subsubsection{Pressure}
The results are plotted in Fig.~\ref{fig:tmp0}.
\begin{figure}
  \centering
  \includegraphics[]{pressure-methods.eps}
  \caption{The pressure convergence with respect to the cut-off radius.
    For ZM method, the splitting parameter $\alpha = 0.00\,\textrm{nm}^{-1}$.
  }
  \label{fig:tmp0}
\end{figure}


\subsubsection{The dielectric constant}

The dielectric constant:
\begin{align}
  \eps = 1 + \frac{1}{3L^3 k_BT} ( \langle \vert \vect M\vert^2\rangle - \vert\langle \vect M\rangle\vert^2 )
\end{align}
where
\begin{align}
  \vect M = \sum_i\boldsymbol\mu_i = \sum_{\alpha\in i} q_\alpha\vect r_\alpha
\end{align}
The results are listed in Tab.~\ref{tab:tmp2}

\begin{table}
  \centering
  \caption{A list of the dielectric constant calculated for different systems by different methods.
    The paremeters are provided. The parenthesises in the last column show the statistical uncertanty
    of the last two digits upto the confidence level of 95~\%.}
  \begin{tabular*}{0.5\textwidth}{@{\extracolsep{\fill}}cccc rr}\hline\hline
    System & Method      &       $\alpha$ [$\textrm{nm}^{-1}$] & $r_c$ [nm] &    $l$     &       $\eps$ \\\hline
    S1  &       ZM          &       0.00    &       1.2     &       1       &       99.8 (3.8)\\ 
    S1  &       ZM          &       0.00    &       1.2     &       2       &       95.2 (3.6)\\ 
    S1  &       ZM          &       0.00    &       1.2     &       3       &       93.9 (3.7)\\ 
    S1  &       ZM          &       0.00    &       1.5     &       3       &       98.7 (3.8)\\ 
    S1  &       ZM          &       0.00    &       1.8     &       3       &      100.7 (3.6)\\ 
    S1  &       ZM          &       0.00    &       1.2     &       4       &       94.7 (3.6)\\
    S1   & SPME          & 2.09  & 1.9   &       --      &       98.8 (4.0) \\
    S1  & RF             & --  & 1.2 & -- & 60.6   (1.7) \\
    L1  &       ZM          &       0.00    &       1.2     &       1       &       97.1 (3.2)\\ 
    L1  &       ZM          &       0.00    &       1.2     &       2       &       95.6 (3.5)\\ 
    L1  &       ZM          &       0.00    &       1.2     &       3       &       95.8 (4.6)\\ 
    L1  &       ZM          &       0.00    &       1.5     &       3       &       98.0 (3.3)\\ 
    L1  &       ZM          &       0.00    &       1.8     &       3       &       98.9 (4.2)\\ 
    L1  &       ZM          &       0.00    &       2.1     &       3       &       98.8 (3.9)\\ 
    L1  &       ZM          &       0.00    &       1.2     &       4       &       97.8 (3.9)\\
   L1   & SPME          & 2.09  & 1.9   &       --      &       98.7 (3.5) \\
    L1  & RF             & --  & 1.2 & -- & 60.8   (2.0) \\
    \hline\hline
  \end{tabular*}
  \label{tab:tmp2}
\end{table}

\subsubsection{Kirkwood G-factor}

The Kirkwood G-factor is defined by~\cite{vanderSpoel2006origin}
\begin{align}
  G_k(r) =
  \Big\langle
  \frac 1N
  \sum_{i=1}^N \sum_{j, r_{ij} < r}
  \frac {\boldsymbol\mu_i \cdot \boldsymbol\mu_j}{\vert \boldsymbol\mu_i\vert \cdot \vert\boldsymbol\mu_j\vert}
  \Big\rangle,
\end{align}
where $r_{ij}$ denotes the oxygen-oxygen distance between two water
molecules.  It is well know that the G-factor is system size
dependent~\cite{vanderSpoel2006origin}. Here we show the G-factor for
both a smaller system S1 (Fig.~\ref{fig:tmp1}) and a larger system L1
(Fig.~\ref{fig:tmp2}).  In the figures, the statistical uncertanty of
the SPME method is denoted by the red error bar, while thoes of the
other methods are essentially the same, so they are not presented for
clarity.  In both the system, the RF method is qualitatively
wrong. The results of the ZM method are consistent with SPME
method. The lower order produces better consistency.  An artificial
oscillation around the cut-off radius presents for $l=1$, but it is
not obvious for $l\geq 2$.  For high order ZM (here $l=3$ is investigated),
using larger cut-off radius can substantially improve the accuracy.

\begin{figure}
  \centering
  \includegraphics[]{fig-gkr-small.eps}
  \includegraphics[]{fig-gkr-small-conv.eps}
  \caption{The Kirkwood G-factor calculated for system S1.
    % The result of different methods are compared in the left plot.
    For all ZM methods the splitting parameter $\alpha = 0.00\,\textrm{nm}^{-1}$.
    In the left plot the cut-off radius for all orders is set to $1.2$~nm, while
    in the right plot, only the order $l=3$ is considered and different cut-off radii are compared.
    The RF method uses a cut-off radius of 1.2~nm, and dielectric constant of 80.
    The statistical uncertainty of the SPME method is presented at 95~\% confidence level with the red bar.
  }
  \label{fig:tmp1}
\end{figure}

\begin{figure}
  \centering
  \includegraphics[]{fig-gkr.eps}
  \includegraphics[]{fig-gkr-conv.eps}
  \caption{The Kirkwood G-factor calculated for system L1.
    % The result of different methods are compared in the left plot.
    For all ZM methods the splitting parameter $\alpha = 0.00\,\textrm{nm}^{-1}$.
    In the left plot the cut-off radius for all orders is set to $1.2$~nm, while
    in the right plot, only the order $l=3$ is considered and different cut-off radii are compared.
    The RF method uses a cut-off radius of 1.2~nm, and dielectric constant of 80.
    The statistical uncertainty of the SPME method is presented at 95~\% confidence level with the red bar.
  }
  \label{fig:tmp2}
\end{figure}





\newpage
\bibliography{ref}{}
\bibliographystyle{unsrt}

\end{document}
