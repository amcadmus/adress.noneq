\documentclass[aip,jcp,a4paper,reprint,unsortedaddress,onecolumn,fleqn]{revtex4-1}
% \documentclass[reprint,unsortedadress,oneclumn]{revtex4-1}

\usepackage{amsmath,amssymb,amsfonts,latexsym}
\usepackage[dvips]{graphicx}
\usepackage{color}
\usepackage{indentfirst} 

\newcommand{\eps}{\varepsilon}
\newcommand{\recheck}[1]{{\color{red} #1}}
\newcommand{\redc}[1]{{\color{red} #1}}            
\newcommand{\bluec}[1]{{\color{blue} #1}}            
\newcommand{\vect}[1]{\textbf{\textit{#1}}}
\newcommand{\exc}{\textrm{ex}}
\newcommand{\systemsa}{S_0}
\newcommand{\systemsb}{S_1}
\newcommand{\systemsbp}{S'_1}
\newcommand{\systemma}{M_0}
\newcommand{\systemmb}{M_1}
\newcommand{\systemmbp}{M'_1}
\newcommand{\systemla}{L_0}
\newcommand{\systemlbp}{L'_1}
\newcommand{\systemlb}{L_1}


\begin{document}

\title{A Critical Appraisal of Zero-Multiple Method in Simulation Biomolecular Systems}
\author{Han Wang}
\email{han.wang@fu-berlin.de}
\affiliation{Institute for Mathematics, Freie Universit\"at Berlin, Germany}
\author{Ikuo Fukuda}
\email{ifukuda@protein.osaka-u.ac.jp}
\affiliation{Institute for Protein Research, Osaka University, Japan}

\begin{abstract}
\end{abstract}

\maketitle

\section{Simulation protocols}

\begin{table}
  \centering
  \caption{A list of systems tested in this work. We report, from left to right, the number of water
    molecules in the system (TIP3P model), box size, length of equilibriation, length of productive trajectory, the thermostat and the coupling time scale of the thermostat.
    }
  \begin{tabular*}{0.9\textwidth}{@{\extracolsep{\fill}}c rrrrrr}\hline\hline
    System      & $N$ & $V$ [$\textrm{nm}^3$]  & $T_{eq}$ [ps] & Traj. Length [ps] & Thermostat & $\tau$ [ps]  \\\hline
    $\systemsb$          & 2,500          & $4.24^3$              & 3,000        & 27,000        & Nose-Hoover & 1.0 \\
    $\systemsbp$         &   2,500        & $4.24^3$              & --           &  6,000        & Nose-Hoover & 1.0\\
    $\systemmb$          & 4,500          & $5.15^3$              & 3,000        & 17,000        & Langevin & 0.1\\
    $\systemmbp$         &   4,500        & $5.15^3$              & --           &  6,000        & Langevin & 0.1\\
    $\systemla$          &   13,824       & $7.49^3$              & 200          & 1,800         & Langevin & 0.1\\
    $\systemlb$          &   13,824       & $7.49^3$              & 3,000        & 17,000        & Langevin & 0.1\\
    $\systemlbp$         &   13,824       & $7.49^3$              & --           & 10,000        & Langevin & 0.1\\
    \hline\hline
  \end{tabular*}
  \label{tab:tmp1}
\end{table}

The systems used for MD simulations are listed in
Table.\ref{tab:tmp1}.  The initial configurations are prepared by in
the following steps: (1) A NPT simulation of system $\systemla$ with SPME~\cite{darden1993pme, essmann1995spm}
electrostatic. The equilibrium density of the system is calculated
from this simulation. (2) Initial configurations that has 4,500 and
13,824 molecules are generated at the calculated equilibrium
density. (3) NVT simulations with SPME electrostatic for system $\systemmbp$
and $\systemlbp$ are performed. (4) The output configuration of $\systemmbp$ is used as
initial configuration for system $\systemmb$, and the output configuration of
$\systemlbp$ is used as initial configuration for system $\systemla$ and $\systemlb$.  The SPME
parameters for all simulations reported by this work is: real space
cut-off 1.90~nm. The cut-off is smoothed from 1.85 to 1.90~nm by the
``\texttt{switch}'' method provide by Gromacs
4.6~\cite{hess2008gromacs, pronk2013gromacs}. The reciprocal space
grid spacing is 0.06~nm. The B-Spline interpolation order is 6. The
splitting parameter is optimized by the error estimate for the SPME method~\cite{wang2010optimizing}.

We add a new system $\systemsb$, and use the Nose-Hoover
thermostat. The reason is that for the fluctuation properties (in this
work heat capacity, dielectric constant, diffusion constant and
viscosity are fluctuation properties), simply increase the size of the
system does not improve the accuracy of the
simulation~\cite{ferrenberg1991statistical}. The only way to improve
the accuracy is to run longer simulation to do more
time-average. Therefore, the optimal way of measuring these properties
is to use smaller system (as far as the finite size effect is not
important) and run as long simulation as possible, as the equilibrium
state is reached. The reason of using Nose-Hoover thermostat is that
it perturbs the dynamics of the system only slightly, while the
Langevin thermostat is likely to substantially change the dynamics
when the coupling is very strong.

\subsection{Static properties}

\subsubsection{Pressure}
The results are plotted in Fig.~\ref{fig:tmp0}. All pressure calculations are done with system $\systemla$.
\begin{figure}
  \centering
  \includegraphics[]{pressure-methods.eps}
  \caption{The pressure convergence with respect to the cut-off radius.
    For ZM method, the splitting parameter $\alpha = 0.00\,\textrm{nm}^{-1}$.
  }
  \label{fig:tmp0}
\end{figure}

\subsubsection{Chemical potential}

The chemical potential of the NVT system is defined by
\begin{align}
  \mu = \frac{\partial A}{\partial N}\Big\vert_{V,T}
\end{align}
In practice the excess chemical potential $\mu^\exc$, which is the
chemical potential abstracted by the kinetic contribution, is of
special interest. We calculate the excess chemical potential by the
thermodynamic integration (TI). In this approach, the interaction
between an inserted molecule and the rest of the system is controlled
by a coupling parameter $\lambda$. $\lambda$ goes from 0 to 1, where 0
indicates no coupling while 1 indicates full coupling. Therefore the
Hamiltonian of the system is function of the coupling parameter. The
excess chemical is calculated by:
\begin{align}
  \mu^\exc = \int_0^1 \Big\langle \frac{\partial \mathcal H(\lambda)}{\partial \lambda} \Big\rangle_\lambda d\lambda
\end{align}
We use a two step coupling approach: firstly couple the van der Waals
interaction (stage vdw), and then couple the electrostatic interaction
(stage ele).  During stage vdw, 21 $\lambda$ values are equally
distributed in $[0,1]$.  During stage ele, 6 $\lambda$ values are
equally distributed in $[0,0.05]$, while the other 20 $\lambda$ values
are equally distributed in $[0.05, 1]$. Therefore, in total 47
simulations are performed for all $\lambda$ values. The simulation was
in system $\systemla$.

\begin{table}
  \centering
  \caption{The excess chemical potential $\mu^\exc$, constant volume molar heat capacity $C_{v,m}$, dielectric constant $\eps$, diffusion constant $D$ and viscosity $\eta$
    calculated by different methods.    
    The parenthesises in the last column show the statistical uncertainty
    at the confidence level of 95~\%.
    The bold numbers indicating that its deviation from the SPME result is larger than the statistical uncertainty.
  }
  \begin{tabular*}{0.9\textwidth}{@{\extracolsep{\fill}}cccc cccc}\hline\hline
    Method      &   $r_c$ [nm] &    $l$     &  $\mu^\exc$ [kJ/mol] &$C_{v,m}$ [J/(mol K)] & $\eps$ (Sys.~$\systemsb$) & $D$ [$10^{-9}\textrm{m}^2/\textrm{s}$] &  $\eta$ [$10^{-3}\textrm{Pa}\cdot\textrm{s}$] \\\hline
    SPME        &         1.9     &       --      &       $-26.08$ (0.24) & 72.21 (0.97)  & 97.6 (3.4)  & 5.86 (0.07)   & 0.315 (0.007)\\
    RF          &         1.2     & --            &       $-26.03$ (0.32) & 72.28 (0.81)  & \textbf{59.4} (1.3)  & \textbf{6.27} (0.19)   & \textbf{0.449} (0.020)\\
    ZM          &         1.2     &       1       &       $-26.22$ (0.30) & 71.75 (0.89)  & 95.5 (2.4)  & \textbf{5.53} (0.11)   & \textbf{0.346} (0.006)\\ 
    ZM          &         1.2     &       2       &       $-26.24$ (0.24) & 71.53 (0.88)  & 95.4 (2.5)  & 5.82 (0.27)   & 0.318 (0.012)\\ 
    ZM          &         1.2     &       3       &       $-{26.18}$ (0.20) & 71.58 (0.88)  & 96.0 (2.7)  & 5.79 (0.09)   & 0.321 (0.008)\\ 
    ZM          &         1.2     &       4       &       $-{26.01}$ (0.26) & 71.07 (0.84)  & 99.7 (2.2)  & 5.89 (0.09)   & 0.318 (0.013)\\
    \hline\hline
  \end{tabular*}
  \label{tab:tmp3}
\end{table}


\subsubsection{Constant volume molar heat capacity}
The constant volume molar heat capacity is defined by:
\begin{align}
  C_{v,m} = \frac{1}{k_BT N} \langle (\mathcal H - \langle\mathcal H\rangle)^2 \rangle,
\end{align}
where $\mathcal H$ is the Hamiltonian of the system. Since constant
volume heat capacity is a extensive thermodynamics quantity, it is
normalized by the number molecules in the system.
It is calculated in system $\systemsb$.

\subsubsection{The dielectric constant}

The dielectric constant:
\begin{align}
  \eps = 1 + \frac{1}{3L^3 k_BT} ( \langle \vert \vect M\vert^2\rangle - \vert\langle \vect M\rangle\vert^2 )
\end{align}
where
\begin{align}
  \vect M = \sum_i\boldsymbol\mu_i = \sum_{\alpha\in i} q_\alpha\vect r_\alpha
\end{align}
The results are listed in Tab.~\ref{tab:tmp2}. The dielectric constant calculated in system $\systemsb$ is reported in Table.~\ref{tab:tmp3}.

\begin{table}
  \centering
  \caption{A list of the dielectric constant calculated for different systems by different methods.
    The parameters are provided. The parenthesises in the last column show the statistical uncertainty
    of the last two digits up to the confidence level of 95~\%.}
  \begin{tabular*}{0.5\textwidth}{@{\extracolsep{\fill}}cccc rr}\hline\hline
    System & Method      &       $\alpha$ [$\textrm{nm}^{-1}$] & $r_c$ [nm] &    $l$     &       $\eps$ \\\hline
    $\systemmb$  &       ZM          &       0.00    &       1.2     &       1       &       99.8 (3.8)\\ 
    $\systemmb$  &       ZM          &       0.00    &       1.2     &       2       &       95.2 (3.6)\\ 
    $\systemmb$  &       ZM          &       0.00    &       1.2     &       3       &       93.9 (3.7)\\ 
    $\systemmb$  &       ZM          &       0.00    &       1.5     &       3       &       98.7 (3.8)\\ 
    $\systemmb$  &       ZM          &       0.00    &       1.8     &       3       &      100.7 (3.6)\\ 
    $\systemmb$  &       ZM          &       0.00    &       1.2     &       4       &       94.7 (3.6)\\
    $\systemmb$   & SPME          & 2.09  & 1.9   &       --      &       98.8 (4.0) \\
    $\systemmb$  & RF             & --  & 1.2 & -- & 60.6   (1.7) \\
    $\systemlb$  &       ZM          &       0.00    &       1.2     &       1       &       97.1 (3.2)\\ 
    $\systemlb$  &       ZM          &       0.00    &       1.2     &       2       &       95.6 (3.5)\\ 
    $\systemlb$  &       ZM          &       0.00    &       1.2     &       3       &       95.8 (4.6)\\ 
    $\systemlb$  &       ZM          &       0.00    &       1.5     &       3       &       98.0 (3.3)\\ 
    $\systemlb$  &       ZM          &       0.00    &       1.8     &       3       &       98.9 (4.2)\\ 
    $\systemlb$  &       ZM          &       0.00    &       2.1     &       3       &       98.8 (3.9)\\ 
    $\systemlb$  &       ZM          &       0.00    &       1.2     &       4       &       97.8 (3.9)\\
   $\systemlb$   & SPME          & 2.09  & 1.9   &       --      &       98.7 (3.5) \\
    $\systemlb$  & RF             & --  & 1.2 & -- & 60.8   (2.0) \\
    \hline\hline
  \end{tabular*}
  \label{tab:tmp2}
\end{table}



\subsubsection{Radial distribution function}
The results are plotted in Fig.~\ref{fig:rdf}. As the order of ZM method goes higher, the precision of RDF improves. In addition, the precision of RF method is between $l=2$ and $l=3$.
\begin{figure}
  \centering
  \includegraphics[width=0.4\textwidth]{fig/result.nvt/fig-rdf.eps}  
  \caption{The RDF of different method. The insert is a zoom-in of range 0.9 -- 1.5~nm. In the insert, the position of the cut-off (1.2~nm) is indicated by a vertical black line.}
  \label{fig:rdf}
\end{figure}


\subsubsection{Kirkwood G-factor}

The Kirkwood G-factor is defined by~\cite{vanderSpoel2006origin}
\begin{align}
  G_k(r) =
  \Big\langle
  \frac 1N
  \sum_{i=1}^N \sum_{j, r_{ij} < r}
  \frac {\boldsymbol\mu_i \cdot \boldsymbol\mu_j}{\vert \boldsymbol\mu_i\vert \cdot \vert\boldsymbol\mu_j\vert}
  \Big\rangle,
\end{align}
where $r_{ij}$ denotes the oxygen-oxygen distance between two water
molecules.  It is well know that the G-factor is system size
dependent~\cite{vanderSpoel2006origin}. Here we show the G-factor for
both a smaller system $\systemmb$ (Fig.~\ref{fig:tmp1}) and a larger system $\systemlb$
(Fig.~\ref{fig:tmp2}).  In the figures, the statistical uncertainty of
the SPME method is denoted by the red error bar, while those of the
other methods are essentially the same, so they are not presented for
clarity.  In both the system, the RF method is qualitatively
wrong. The results of the ZM method are consistent with SPME
method. The lower order produces better consistency.  An artificial
oscillation around the cut-off radius presents for $l=1$, but it is
not obvious for $l\geq 2$.  For high order ZM (here $l=3$ is investigated),
using larger cut-off radius can substantially improve the accuracy.

\begin{figure}
  \centering
  \includegraphics[]{fig-gkr-small.eps}
  \includegraphics[]{fig-gkr-small-conv.eps}
  \includegraphics[]{fig/result.nvt.small/fig-eps-t-3000.eps}
  \includegraphics[]{fig/result.nvt.small/fig-eps-t-zm3-3000.eps}
  \includegraphics[]{fig/result.nvt.small/fig-eps-t-10000.eps}
  \includegraphics[]{fig/result.nvt.small/fig-eps-t-zm3-10000.eps}
  \caption{The Kirkwood G-factor calculated for system $\systemmb$.
    % The result of different methods are compared in the left plot.
    For all ZM methods the splitting parameter $\alpha = 0.00\,\textrm{nm}^{-1}$.
    In the left plot the cut-off radius for all orders is set to $1.2$~nm, while
    in the right plot, only the order $l=3$ is considered and different cut-off radii are compared.
    The RF method uses a cut-off radius of 1.2~nm, and dielectric constant of 80.
    The statistical uncertainty of the SPME method is presented at 95~\% confidence level with the red bar.
  }
  \label{fig:tmp1}
\end{figure}

\begin{figure}
  \centering
  \includegraphics[]{fig-gkr.eps}
  \includegraphics[]{fig-gkr-conv.eps}
  \includegraphics[]{fig/result.nvt/fig-eps-t-3000.eps}
  \includegraphics[]{fig/result.nvt/fig-eps-t-zm3-3000.eps}
  \caption{The Kirkwood G-factor calculated for system $\systemlb$.
    % The result of different methods are compared in the left plot.
    For all ZM methods the splitting parameter $\alpha = 0.00\,\textrm{nm}^{-1}$.
    In the left plot the cut-off radius for all orders is set to $1.2$~nm, while
    in the right plot, only the order $l=3$ is considered and different cut-off radii are compared.
    The RF method uses a cut-off radius of 1.2~nm, and dielectric constant of 80.
    The statistical uncertainty of the SPME method is presented at 95~\% confidence level with the red bar.
  }
  \label{fig:tmp2}
\end{figure}

\begin{figure}
  \centering
  \includegraphics[]{fig-gkr-10000.eps}
  \includegraphics[]{fig-gkr-conv-10000.eps}
  \includegraphics[]{fig/result.nvt/fig-eps-t-10000.eps}
  \includegraphics[]{fig/result.nvt/fig-eps-t-zm3-10000.eps}
  \caption{The Kirkwood G-factor calculated for system $\systemlb$. Equilibriation time is 10~ns, and the production simulation time is 10~ns.
    % The result of different methods are compared in the left plot.
    For all ZM methods the splitting parameter $\alpha = 0.00\,\textrm{nm}^{-1}$.
    In the left plot the cut-off radius for all orders is set to $1.2$~nm, while
    in the right plot, only the order $l=3$ is considered and different cut-off radii are compared.
    The RF method uses a cut-off radius of 1.2~nm, and dielectric constant of 80.
    The statistical uncertainty of the SPME method is presented at 95~\% confidence level with the red bar.
  }
  \label{fig:tmp3}
\end{figure}



\subsection{Dynamical properties}

\subsubsection{Diffusion constant}
The diffusion constant is calculated from the Einstein relation:
\begin{align}
  D = \lim_{t\rightarrow \infty}\frac {1}{6t} \langle \vert \vect r_i(t) - \vect r_i(0)\vert^2\rangle.
\end{align}
In practice, the mean-square-displacement $\langle\vert \vect r_i(t) - \vect
r_i(0)\vert^2\rangle$ is calculated, then the value is linearly fitted. The
resulting slop is used to calculate the diffusion constant. The results are listed in Table~\ref{tab:tmp3}.
The ZM method $l\geq 2$ is consistent with the SPME
result, while the RF method is off. The system for simulation is $\systemsb$.

\subsubsection{Viscosity}
The viscosity is calculated from the Green-Kubo relation:
\begin{align}
  \eta = \frac{V}{k_BT}\int_0^\infty\langle P_{\alpha\beta}(0) P_{\alpha\beta}(t)\rangle\,dt, \quad \alpha,\beta \in \{x, y, z\}
\end{align}
where $\alpha$ and $\beta$ denote the directions, and
$P_{\alpha\beta}$ denotes the off-diagonal components of the pressure
tensor. Since we used the isotropic system setting, then it is obvious
that $P_{xy}$, $P_{yz}$ and $P_{xz}$ are equivalent. Moreover, it has
been pointed out that in addition $(P_{xx} - P_{yy})/2$ and $(P_{xx} -
P_{yy})/2$ are two independent components that are equivalent to the
first three~\cite{alfe1998first}. Therefore, the viscosity is
calculated from the auto-correlation functions of five independent
components, and the statistical error is estimated from the standard
deviation of the viscosities calculated from the five components.

We investigate the convergence of the integral of the auto-correlation function w.r.t.~time:
\begin{align}
  I_\eta(T) = \frac{V}{k_BT}\int_0^T\langle P_{\alpha\beta}(0) P_{\alpha\beta}(t)\rangle\,dt,
\end{align}
and plot the function $I_\eta(T) $ for all method in
Fig.~\ref{fig:tmp4}.  It is clear the integral for the RF method
converges at 6~ps, while the ZM method converges only in 2~ps, which
is consistent with the SPME result.  The value of ZM $l=1$ method is
different from the rest ZM method an SPME result.  The value
$I_\eta(10\,\textrm{ps})$ is given in Table~\ref{tab:tmp3}.  The
system for simulation is $\systemsb$.

\begin{figure}
  \centering
  \includegraphics[]{fig/result.tiny/fig-vis.eps}
  \caption{The convergence of the integrated auto-correlation function $I_\eta(T) $. The error bars indicating 95\% confidence level are plotted with the SPME method.}
  \label{fig:tmp4}
\end{figure}



\newpage
\bibliography{ref}{}
\bibliographystyle{unsrt}

\end{document}
