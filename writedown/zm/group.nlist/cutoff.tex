\documentclass[aps,pre,preprint]{revtex4-1}

\usepackage[fleqn]{amsmath}
\usepackage{amssymb}
\usepackage[dvips]{graphicx}
\usepackage{color}
\usepackage{tabularx}
\usepackage{algorithm}
\usepackage{algorithmic}

\newcommand{\gromacs}[0]{\texttt{Gromacs}}
\newcommand{\rlist}[0]{\texttt{rlist}}
\newcommand{\rcoulomb}[0]{\texttt{rcoulomb}}

\begin{document}

\begin{figure}
  \centering
  \includegraphics[width=0.6\textwidth]{fig/cutoff.eps}
  \caption{The schematic plot of gromacs cut-off method. The two molecules
  in the plot are considered to be neighbors.}
  \label{fig:tmp1}
\end{figure}

For simplicity, in this instruction we only consider a system of water
molecules. In \gromacs, there are two steps of finding a pair of interacting atoms:
\begin{enumerate}
\item Building a neighbor list
\item Finding a pair of interacting atoms out of the neighbor list.
\end{enumerate}
The neighbor list may not be updated every timestep of MD
simulation. When a neighbor list is built, it may be used for several
following time-steps, so the computational cost of building neighbor
list can be saved.  \gromacs{} builds the neighbor list by using the
``group'' method.  The range of building
neighbor list is called \rlist{}, see the Fig.~\ref{fig:tmp1}.
The ``group'' method means that all pair of water
molecules, whoes center-of-geometry (COG) distance is shorter than
\rlist, will be recoreded by the neighbor list.
% In fact, all pair of
% atoms on the pair of molecules are built into the neighbor list.

Whether the a pair of molecules in the neighbor list interacts
depends on the cut-off of the interaction. We only consider the
Coulomb interaction, the cut-off of which is denoted by \rcoulomb, see
Fig~\ref{fig:tmp1}. \gromacs{} actually allows $\rcoulomb = \rlist$,
then the molecules are interacting in a ``group'' fashion:
when the neighbor list is built, all pairs of molecules in the neighbor list
interact as long as the neighbor list is considered valid.
However, when \rcoulomb{} is shorter than \rlist{}, then the interaction
fashion is different.
At this point I was confused: I thought that the interaction between two
molecules in the neighbor list that is built in the ``group'' fashion is also
cut-offed in a ``group'' fashion, however, in fact, the interaction is 
cut-offed in an atomistic fashion. 

I confirm the atomistic cut-off fashion by doing simulation in the
following two cases in Tab.~\ref{tab:tmp1} and~\ref{tab:tmp2}. In both cases, $\rcoulomb
= 1.2$~nm and $\rlist=1.5$~nm. The system contains only 2 water molecules.
The van der Waals interaction is switched off.
In the 1st case, the COG distance between the molecules is shorter than
the \rcoulomb, while it is langer than \rcoulomb{} in the 2nd case.
It is obvious that whether two atoms interact depends only on the atomistic
distance, and is independent with the COG distance. This confirms the
atomistic cut-off fashion.
% \begin{enumerate}
% \item $\rcoulomb = 1.2$~nm, $rlist=1.5$~nm. 
% \item $\rcoulomb = 1.2$~nm, $rlist=1.5$~nm.
% \end{enumerate}
\begin{table}
  \centering
  \caption{The 1st testing cases of the cut-off scheme. First two columns: the atom pairs recoreded in the neighbor list.
    They are from two neighboring molecules.
    COG is the center of geometry. H1 means the first hydrogen and the H2 means the second hydrogen.
    If the two atoms are interacting, we write ``y'' in the fourth column, otherwise, we write ``n''.
  }
  \label{tab:tmp1}
  \begin{tabular*}{0.8\textwidth}{@{\extracolsep{\fill}}cccc}\hline\hline
    1st atom  & 2nd atom & Distance [nm] & Interacting \\\hline
    COG & COG &  1.1413 & --\\
    O	&O	&1.0940	&y \\
    O	&H1	&1.1890	&y \\
    O	&H2	&1.0740	&y \\
    H1	&H1 	&1.2840	&n \\
    H1	&H2 	&1.1687	&y \\
    H2	&H2	&1.0460	&y \\
    \hline\hline
  \end{tabular*}
\end{table}

\begin{table}
  \centering
  \caption{The 2nd testing cases of the cut-off scheme. }
  \label{tab:tmp2}
  \begin{tabular*}{0.8\textwidth}{@{\extracolsep{\fill}}cccc}\hline\hline
    1st atom  & 2nd atom & Distance [nm] & Interacting \\\hline
    COG & COG &  1.2413 & --\\
    O	&O	&1.1940	&y \\
    O	&H1	&1.2890	&n \\
    O	&H2	&1.1740	&y \\
    H1	&H1 	&1.3840	&n \\
    H1	&H2 	&1.2687	&n \\
    H2	&H2	&1.1460	&y \\
    \hline\hline
  \end{tabular*}
\end{table}

If we assume \rlist{} is sufficiently larger than \rcoulomb{}, then
the ``group'' neighbor searching method can mimic the atomistic
neighbor searching method, in total we use the \gromacs{}
implementation of the ZM method is in an atomistic cut-off way.
Here ``sufficiently'' means that
\begin{align}
  d_{COG} + r_{COG,H} < \frac12 (\rlist - \rcoulomb),
\end{align}
where $d_{COG}$ is the COG displacement with respect to the COG position
when the neighbor list was built, and $r_{COG,H}$ is the distance between
the COG and the hydrogen atom (notice that  $r_{COG,H} >  r_{COG,O}$).
For TIP3P, $r_{COG,H}\approx 0.078$~nm, then
in the case of $\rlist - \rcoulomb = 0.3$~nm, we have
\begin{align}\label{eqn:tmp2}
  d_{COG} < 0.072\ \textrm{nm}.
\end{align}
In the numerical tests, the maximum velocity found in our simulation
of TIP3P water at 300~K was bounded by 8~nm/ps. The time step was
0.001~ps, and the neighbor list was updated every 5 time-steps, so
$d_{COG}$ is estimated to be bounded by $0.001 \times 5 \times 8 = 0.04$~nm,
which satisfies Eq.~\eqref{eqn:tmp2}.

A looser condition can be considered. Since only 0.16\% of the molecules
has velocity larger than 5~nm/ps at a certain time frame, the velocity can be
roughly bounded by  5~nm/ps.
If the neighbor list is updated every 10 time-steps, 
$d_{COG}$ is estimated to be roughly bounded $0.001 \times 10 \times 5 = 0.05$~nm.
We can set  $\rlist - \rcoulomb = 0.26$~nm, then we need $d_{COG} < 0.052\ \textrm{nm}$, which is satisfied.

Since all of our simulations had $\rlist - \rcoulomb = 0.3$~nm, the ZM
interaction was indeed treated in an atomistic way.



% when the \rlist{} is sufficiently larger than \rcoulomb{}
% (the region between them is called the buffer region), the atoms are
% interacting in an atomistic fashion.


% however, when we use $\rcoulomb < \rlist$,
% the region between \rcoulomb and \rlist is called the buffer region.
% If the size of the buffer region is large enough, then





\end{document}
