\documentclass[aip,jcp,a4paper,reprint,unsortedaddress,onecolumn,fleqn]{revtex4}
%%%%%%%%%%%%%%%%%%%%%%%%%%%%%%%%%%%%%%%%%%%%%%%%%%%%%%%%%%%%%%%%%%%%%%%%%%%%%%%%%%%%%%%%%%%%%%%%%%%%%%%%%%%%%%%%%%%%%%%%%%%%%%%%%%%%%%%%%%%%%%%%%%%%%%%%%%%%%%%%%%%%%%%%%%%%%%%%%%%%%%%%%%%%%%%%%%%%%%%%%%%%%%%%%%%%%%%%%%%%%%%%%%%%%%%%%%%%%%%%%%%%%%%%%%%%
\usepackage{amsmath,amssymb,amsfonts,latexsym}
\usepackage[dvips]{graphicx}
\usepackage{color}
\usepackage{indentfirst}
\usepackage{rotating,booktabs}

\setcounter{MaxMatrixCols}{10}
%TCIDATA{OutputFilter=LATEX.DLL}
%TCIDATA{Version=5.50.0.2952}
%TCIDATA{Codepage=932}
%TCIDATA{<META NAME="SaveForMode" CONTENT="1">}
%TCIDATA{BibliographyScheme=BibTeX}
%TCIDATA{LastRevised=Friday, November 07, 2014 19:42:26}
%TCIDATA{<META NAME="GraphicsSave" CONTENT="32">}
%TCIDATA{Language=American English}

\newcommand{\eps}{\varepsilon}
\newcommand{\recheck}[1]{{\color{red} #1}}
\newcommand{\redc}[1]{{\color{red} #1}}
\newcommand{\bluec}[1]{{\color{blue} #1}}
\newcommand{\vect}[1]{\textbf{\textit{#1}}}
\newcommand{\exc}{\textrm{ex}}
\newcommand{\systemsa}{S_0}
\newcommand{\systemsb}{S_1}
\newcommand{\systemsbp}{S'_1}
\newcommand{\systemma}{M_0}
\newcommand{\systemmb}{M_1}
\newcommand{\systemmbp}{M'_1}
\newcommand{\systemla}{L_0}
\newcommand{\systemlbp}{L'_1}
\newcommand{\systemlb}{L_1}
% \input{tcilatex}
\begin{document}

\title{\textbf{Supplementary Materials}\\
  A Critical Appraisal of the Zero-Multiple Method: Structural,
Thermodynamic, Dielectric, and Dynamical Properties of a Water System}
\author{Han Wang}
\email{han.wang@fu-berlin.de}
\affiliation{CAEP Software Center for High Performance Numerical Simulation, Huayuan Road
6, 100088 Beijing, China}
\affiliation{Zuse Institute Berlin (ZIB), Germany}
\author{Ikuo Fukuda}
\email{ifukuda@protein.osaka-u.ac.jp}
\affiliation{Institute for Protein Research, Osaka University, 3-2 Yamadaoka, Suita,
Osaka 565-0871, Japan}
\affiliation{RIKEN (The Institute of Physical and Chemical Research), 2-1 Hirosawa, Wako,
Saitama 351-0198, Japan}
\author{Haruki Nakamura}
\affiliation{Institute for Protein Research, Osaka University, 3-2 Yamadaoka, Suita,
Osaka 565-0871, Japan}

\begin{abstract}
\end{abstract}

\maketitle
\affiliation{CAEP Software Center for High Performance Numerical Simulation, Huayuan Road
6, 100088 Beijing, China}
\affiliation{Institute for Protein Research, Osaka University, 3-2 Yamadaoka, Suita,
Osaka 565-0871, Japan}
\affiliation{RIKEN (The Institute of Physical and Chemical Research), 2-1 Hirosawa, Wako,
Saitama 351-0198, Japan}

% A Critical Appraisal of Zero-Multiple Method in Classical Water Simulation}

\begin{figure}[]
\centering
\includegraphics[width=0.5\textwidth]{fig/nvt.pressure.1/pressure-l2.eps}\\
\includegraphics[width=0.5\textwidth]{fig/nvt.pressure.1/pressure-l3.eps}\\
\includegraphics[width=0.5\textwidth]{fig/nvt.pressure.1/pressure-l4.eps}
\caption{The pressure convergence with respect to the cut-off radius for ZM
methods with splitting parameter from $0.0$ to $2.0\,\text{nm}^{-1}$. From
left to right, the ZM method of order 2 to 4. The black
solid line indicates the magnitude of the correct pressure: 1~Bar. The error
bars present the statistical uncertainty at 95\% confidence level.}
\label{fig:pres-l1}
\end{figure}

\begin{table}
  \centering
  \caption{A list of the dielectric constant calculated for different systems by different methods.
    The parameters are provided. The parenthesises in the last column show the statistical uncertainty
    of the last two digits up to the confidence level of 95~\%.}
  \begin{tabular*}{0.8\textwidth}{@{\extracolsep{\fill}}cccc rr}\hline\hline
    System & Method      &       $\alpha$ [$\textrm{nm}^{-1}$] & $r_c$ [nm] &    $l$     &       $\eps$ \\\hline
    $\systemmb$  &       ZM          &       0.0    &       1.2     &       1       &       99.8 (3.8)\\ 
    $\systemmb$  &       ZM          &       0.0    &       1.2     &       2       &       95.2 (3.6)\\ 
    $\systemmb$  &       ZM          &       0.0    &       1.2     &       3       &       93.9 (3.7)\\ 
    $\systemmb$  &       ZM          &       0.0    &       1.5     &       3       &       98.7 (3.8)\\ 
    $\systemmb$  &       ZM          &       0.0    &       1.8     &       3       &      100.7 (3.6)\\ 
    $\systemmb$  &       ZM          &       0.0    &       1.2     &       4       &       94.7 (3.6)\\
    $\systemmb$   & SPME          & 2.1  & 1.9   &       --      &       98.8 (4.0) \\
    $\systemmb$  & RF             & --  & 1.2 & -- & 60.6   (1.7) \\\hline
    $\systemlb$  &       ZM          &       0.0    &       1.2     &       1       &       97.1 (3.2)\\ 
    $\systemlb$  &       ZM          &       0.0    &       1.2     &       2       &       95.6 (3.5)\\ 
    $\systemlb$  &       ZM          &       0.0    &       1.2     &       3       &       95.8 (4.6)\\ 
    $\systemlb$  &       ZM          &       0.0    &       1.5     &       3       &       98.0 (3.3)\\ 
    $\systemlb$  &       ZM          &       0.0    &       1.8     &       3       &       98.9 (4.2)\\ 
    $\systemlb$  &       ZM          &       0.0    &       2.1     &       3       &       98.8 (3.9)\\ 
    $\systemlb$  &       ZM          &       0.0    &       1.2     &       4       &       97.8 (3.9)\\
   $\systemlb$   & SPME          & 2.1  & 1.9   &       --      &       98.7 (3.5) \\
    $\systemlb$  & RF             & --  & 1.2 & -- & 60.8   (2.0) \\
    \hline\hline
  \end{tabular*}
  \label{tab:tmp2}
\end{table}


\begin{figure}[tbp]
\centering
\includegraphics[]{fig/result.nvt.tiny/fig-gkr.eps}
\caption{
  The Kirkwood G-factor calculated for system $S_{1}$.
  ZM method of orders $l=1,2,3$ and 4 with the cut-off 1.2~nm and
splitting parameter $\protect\alpha =0.0\,\text{nm}^{-1}$ are plotted. The
RF method is shown as an comparison, which uses a cut-off radius of 1.2~nm
and dielectric constant of 80. The statistical uncertainty of the SPME
method is presented at 95~\% confidence level with the red error bars. the
statistical uncertainty of the SPME method is denoted by the red error bars.
The statistical uncertainties of the other methods are essentially of the
same size as the SPME, so they are not presented in the figure for clarity.
}
\label{fig:gkr-conv-alpha}
\end{figure}

\begin{figure}[tbp]
\centering
\includegraphics[]{fig/result.nvt/fig-gkr-l2-damp.eps}\\
\includegraphics[]{fig/result.nvt/fig-gkr-l3-damp.eps}
\caption{  The 
  splitting parameter dependency in calculating the Kirkwood-G factor.
  The orders of ZM method $l=2$ (a)  and $l=3$ (b) are presented.
  In both plots the cut-off radius is 1.2~nm. The
  statistical uncertainty is presented at 95~\% confidence level with the
  error bars. }
\label{fig:gkr-conv-alpha}
\end{figure}


\begin{figure}[tbp]
\centering
\includegraphics[]{fig/result.nvt/fig-rdf-l2-damp.eps}\\
\includegraphics[]{fig/result.nvt/fig-rdf-l3-damp.eps}
\caption{  The 
  splitting parameter dependency in calculating the RDF.
  The orders of ZM method  $l=2$ (b)  and $l=3$ (c) are presented.
  In both plots the cut-off radius is 1.2~nm. }
\label{fig:gkr-conv-alpha}
\end{figure}


\newpage 

\bibliography{ref}{}
\bibliographystyle{unsrt}


\end{document}
