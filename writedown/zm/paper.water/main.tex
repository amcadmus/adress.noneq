\documentclass[aip,jcp,a4paper,preprint,unsortedaddress,onecolumn,fleqn]{revtex4-1}
% \documentclass[reprint,unsortedadress,oneclumn]{revtex4-1}

\usepackage{amsmath,amssymb,amsfonts,latexsym}
\usepackage[dvips]{graphicx}
\usepackage{color}
\usepackage{indentfirst} 
\usepackage{rotating,booktabs}

\newcommand{\eps}{\varepsilon}
\newcommand{\recheck}[1]{{\color{red} #1}}
\newcommand{\redc}[1]{{\color{red} #1}}            
\newcommand{\bluec}[1]{{\color{blue} #1}}            
\newcommand{\vect}[1]{\textbf{\textit{#1}}}
\newcommand{\exc}{\textrm{ex}}
\newcommand{\systemsa}{S_0}
\newcommand{\systemsb}{S_1}
\newcommand{\systemsbp}{S'_1}
\newcommand{\systemma}{M_0}
\newcommand{\systemmb}{M_1}
\newcommand{\systemmbp}{M'_1}
\newcommand{\systemla}{L_0}
\newcommand{\systemlbp}{L'_1}
\newcommand{\systemlb}{L_1}


\begin{document}

\title{A Critical Appraisal of Zero-Multiple Method in Classical Water Simulation}
\author{Han Wang}
\email{han.wang@fu-berlin.de}
\affiliation{CAEP Software Center for High Performance Numerical Simulation, Huayuan Road 6, 100088 Beijing, China}
\author{Ikuo Fukuda}
\email{ifukuda@protein.osaka-u.ac.jp}
\affiliation{Institute for Protein Research, Osaka University, 3-2 Yamadaoka, Suita, Osaka 565-0871, Japan}
\affiliation{RIKEN (The Institute of Physical and Chemical Research), 2-1 Hirosawa, Wako, Saitama 351-0198, Japan}

\begin{abstract}
\end{abstract}

\maketitle

\section{Introduction}

A very short overview of the existing electrostatic interaction methods.
Long-range treatment, Ewald and derived methods like SPME.
Short-range treatment.

The importance of developing a short-range method for the electrostatic interaction.

Zero-Multiple method, an overview, achievements, sucessful applications.

The propuse of this work: Extensive and strict tests investigating the reliablility of Zero-Multiple method in water simulation, which is the most important solvent.

The structure of the paper.

\section{Zero-Multiple Method}

Introduction of the ZM method, formula...

A short mention of the group cut-off v.s.~the atom cut-off method. (will discuss the implmentation in gromacs later)



\section{Benchmark quantities}

In order to investigate the performance of ZM method, we simulation
the water system by using both SPME and ZM method, and calculate a set
of \emph{benchmark quantities}.  This set covers a broad range of
water propertis that are of general interest, and includes the
structure property (radial distribution function), the thermodynamic
properties (pressure, excess chemical potential, constant
volume/pressure heat capacity, isothermal compressibility, thermal
expansion coefficient), dielectric properties (dielectric constant and
Kirkwood-G factor) and dynamical properties (diffusion coefficient and
viscosity).  By comparing ZM to SPME, a good consistency in computing
the benchmark quantities would indicate a good accuracy of the ZM
method.  Moreover, being aware of the weakness of the ZM method also
prevents the application the method in wrong applications.  The ZM
method is also compared to the traditional short-range method: RF.
The definitions for computing these quantities are provided in details
in Appendix.~\ref{appendix:benchmark}.


\section{Simulation protocols}

In this work, we investigate the water system modeled by
TIP3P~\cite{jorgensen1983comparison} point-charge water model.
The size of the systems and corresponding number of molecules used for MD simulations are summarized in Tab.~\ref{tab:tmp1}.
The benchmark quantities were calculated in either  canonical (NVT) or isothermal-isobaric (NPT) ensemble.
In order to equilibriate the temperature,
the systems were coupled to either Nose-Hoover~\cite{nose1984molecular,hoover1985canonical} or Langevin thermostat (Tab.~\ref{tab:tmp1})
at 300~K.
In the NPT simulations, the systems were further coupled
to the Parrinello-Rahman barostat~\cite{parrinello1980crystal,parrinello1981polymorphic} (in Gromacs implementation)
with time-scale $\tau_P = 0.5$~ps to equilibriate the system at 1~Bar.
All systems were integrated with leap-frog scheme at the time-step of 0.001~ps.
For the ZM and SPME methods,
the neighbor lists were updated every 5 time-steps.
The ``\texttt{group}'' cut-off scheme in Gromacs was used. In order to mimic the atomistic
cut-off manner, the radius for neighbor list building was set 0.3~nm
larger than the cut-off radius (the maximum of Coulomb cut-off and the van der Waals cut-off).
To make sure the van der Waals interactions were cut-off in a smooth way,
they are modified in a shell of $0.05$~nm width near the cut-off sphere by the Gromacs ``\texttt{shift}'' method.
The short-range part of SPME method is also smoothed in a shell of $0.05$~nm width  by the Gromacs ``\texttt{pme-switch}'' method.
The reciprocal space
grid spacing is 0.06~nm. The B-Spline interpolation order is 6. The
splitting parameter is optimized by the error estimate for the SPME method~\cite{wang2010optimizing}.
In the simulations of the RF method, the interactoins were cut-off in the group manner,
so the radius for neighbor list building is set identical to the cut-off radius.
No cut-off smoothing was applied in the cases of RF simulation.

\begin{table}
  \centering
  \caption{A list of systems simulated in this work. We report, from left to right, the number of water
    molecules in the system (TIP3P model), equilibrium box size, length of equilibriation, length of productive trajectory, the thermostats and the corresponding coupling time-scales.
  }
  \vskip .5cm
  \begin{tabular*}{0.9\textwidth}{@{\extracolsep{\fill}}c rrrrrr}\hline\hline
    System      & $N$ & $V$ [$\textrm{nm}^3$]  & $T_{eq}$ [ps] & Traj. Length [ps] & Thermostat & $\tau$ [ps]  \\\hline
    $\systemsbp$         &   2,500        & $4.24^3$              & --           &  6,000        & Nose-Hoover & 1.0\\
    $\systemsb$          & 2,500          & $4.24^3$              & 3,000        & 27,000        & Nose-Hoover & 1.0 \\\hline
    $\systemmbp$         &   4,500        & $5.15^3$              & --           &  6,000        & Langevin & 0.1\\
    $\systemmb$          & 4,500          & $5.15^3$              & 3,000        & 17,000        & Langevin & 0.1\\\hline
    $\systemla$          &   13,824       & $7.49^3$              & 200          & 1,800         & Langevin & 0.1\\
    $\systemlbp$         &   13,824       & $7.49^3$              & --           & 10,000        & Langevin & 0.1\\
    $\systemlb$          &   13,824       & $7.49^3$              & 3,000        & 17,000        & Langevin & 0.1\\
    \hline\hline
  \end{tabular*}
  \label{tab:tmp1}
\end{table}

The initial configurations were prepared by in
the following steps:
(1) An NPT simulation of system $\systemla$ with SPME
electrostatic was performed. The equilibrium density (984.1~$\textrm{kg}/\textrm{m}^{-3}$) of the system is calculated
from this simulation.
(2) Initial configurations that contains 2500, 4500 and
13824 molecules were generated at the calculated equilibrium
density.
(3) NVT simulations with SPME electrostatic for system $\systemsbp$, $\systemmbp$
and $\systemlbp$ are performed.
(4) The output configurations of $\systemsbp$ and $\systemmbp$ were used as
initial configurations for system $\systemsb$ and $\systemmb$, respectively.  The output configuration of
$\systemlbp$ was used as initial configurations for system $\systemla$ and $\systemlb$.
% The SPME
% parameters for all simulations reported by this work is: real space
% cut-off 1.90~nm.

% The cut-off is smoothed from 1.85 to 1.90~nm by the
% ``\texttt{switch}'' method provide by Gromacs
% 4.6~\cite{hess2008gromacs, pronk2013gromacs}. 

The smallest system $\systemsb$ was adopted,
because for the quantities that are computed  by estimating fluctuations (in this
work heat capacities, compressibility, thermal expansion coefficient, dielectric constant, diffusion constant and
viscosity are fluctuations. See details in Appendix~\ref{appendix:benchmark}), simply increase the size of the
system does not improve the accuracy of the 
simulation~\cite{milchev1986fluctuations,ferrenberg1991statistical}.
The optimal way of measuring these properties
is to simulate with smaller systems (as far as the finite size effect is negligible) and run longer simulations for longer time-averages.
The reason of using Nose-Hoover thermostat is that
it perturbs the dynamics of the system only slightly, while the
Langevin thermostat is likely to substantially change the dynamics
when the coupling is very strong, therefore, the former was used in computing the dynamical
properties like the diffusion constant and viscosity.
The quantities that
are ensemble averages (in this work, pressure, chemical potential, radial distribution function and Kirkwood-G factor are of this type)
were computed by the largest system $\systemlb$,
in order to achieve higher parallel efficiency.
The system $\sustemmb$ is simulated to investigate the size-dependency of the
Kirkwood-G factor~\cite{vanderSpoel2006origin}.

\section{Results and discussions}

The results are plotted in Fig.~\ref{fig:tmp0}. All pressure calculations are done with system $\systemla$.
\begin{figure}
  \centering
  \includegraphics[]{fig/nvt.pressure.1/pressure-methods.eps}
  \caption{The pressure convergence with respect to the cut-off radius.
    For ZM method, the splitting parameter $\alpha = 0.00\,\textrm{nm}^{-1}$.
  }
  \label{fig:tmp0}
\end{figure}


\begin{sidewaystable}
  \centering
  \caption{The excess chemical potential $\mu^\exc$, constant volume molar heat capacity $C_{v,m}$, dielectric constant $\eps$, diffusion constant $D$, viscosity $\eta$, constant pressure molar heat capacity $C_{v,m}$, isothermal compressibility $\kappa_T$ and
    thermal expansion coefficient $\alpha$
    calculated by different methods.    
    The parenthesises in the last column show the statistical uncertainty
    at the confidence level of 95~\%.
    The bold numbers indicating that its deviation from the SPME result is larger than the statistical uncertainty.
  }
  \bigskip
  \centering\small\setlength\tabcolsep{2pt}
  \begin{tabular*}{0.99\textwidth}{@{\extracolsep{\fill}}cccc cccccccc}\hline\hline
    Method      &   $r_c$ &    $l$ & $\alpha$  & $\mu^\exc$  &$C_{v,m}$ & $C_{p,m}$ &   $\kappa_T$  &$\alpha$ &  $\eps$ & $D$ &  $\eta$  \\
                & [nm] & & [$\textrm{nm}^{-1}$] &   [kJ/mol] &[J/(mol K)] & [J/(mol K)] & [$10^{-10}\textrm{m}^2/\textrm{N}$] &  [$10^{-3}\textrm{K}^{-1}$]&  & [$10^{-9}\textrm{m}^2/\textrm{s}$] &  [$10^{-3}\textrm{Pa}\cdot\textrm{s}$]  \\
                &    &    &    &NVT $\systemla$&NVT $\systemsb$&  NPT $\systemsb$          &  NPT $\systemsb$    &  NPT $\systemsb$    & NVT $\systemsb$& NVT $\systemsb$    & NVT $\systemsb$       \\\hline
    SPME        &1.9 & -- &2.1 & $-26.1$ (0.2) & 72.2 (1.0)  &79.2 (0.9)           & 5.94 (0.05)               &1.03 (0.02)          & 98 (3)          &         5.86  (0.07)&         0.315  (0.007)\\
    RF          &1.2 & -- &--  & $-26.0$ (0.3) & 72.3 (0.8)  &\textbf{82.3} (1.0)  & \textbf{6.51} (0.06)      &\textbf{1.15} (0.02) & \textbf{59} (1) & \textbf{6.27} (0.19)& \textbf{0.449} (0.020)\\\hline
    ZM          &1.2 & 1  &0.0 & $-26.2$ (0.3) & 71.8 (0.9)  &78.9 (1.0)           & 5.87 (0.05)               &1.01 (0.02)          & 96 (2)          & \textbf{5.53} (0.11)& \textbf{0.346} (0.006)\\ 
    ZM          &1.2 & 1  &0.5 & $-26.4$ (0.3) & 72.3 (0.8)  &78.9 (0.9)           & 5.92 (0.05)               &\textbf{1.00} (0.02) & 97 (2)          & \textbf{5.52} (0.25)& \textbf{0.339} (0.008)\\ 
    ZM          &1.2 & 1  &1.0 & $-26.0$ (0.2) & 72.4 (0.8)  &78.4 (0.9)           & 5.95 (0.05)               &1.01 (0.02)          & 96 (2)          &        {5.78} (0.25)& \textbf{0.330} (0.011)\\ 
    ZM          &1.2 & 1  &1.5 & $-26.0$ (0.2) & 71.3 (0.8)  &80.2 (0.9)           & 5.97 (0.05)               &1.04 (0.02)          & 99 (3)          & \textbf{5.71} (0.09)&        {0.312} (0.008)\\ 
    ZM          &1.2 & 1  &2.0 & $-26.1$ (0.3) & 71.3 (0.8)  &79.4 (0.9)           & \textbf{6.05} (0.05)      &1.05 (0.02)          & 96 (3)          &        {5.90} (0.04)&        {0.307} (0.007)\\\hline
    ZM          &1.2 & 2  &0.0 & $-26.2$ (0.2) & 71.5 (0.9)  &79.7 (1.0)           & 5.95 (0.05)               &1.03 (0.02)          & 95 (3)          &         5.82  (0.27)&         0.318  (0.012)\\
    ZM          &1.2 & 3  &0.0 & $-26.2$ (0.2) & 71.6 (0.9)  &79.5 (1.0)           & 5.99 (0.05)               &1.02 (0.02)          & 96 (3)          &         5.79  (0.09)&         0.321  (0.008)\\
    ZM          &1.2 & 4  &0.0 & $-26.0$ (0.3) & 71.1 (0.8)  &79.5 (1.0)           & 5.96 (0.05)               &1.03 (0.02)          &100 (2)          &         5.89  (0.09)&         0.318  (0.013)\\
    \hline\hline
  \end{tabular*}
  \label{tab:tmp3}
\end{sidewaystable}


% \begin{table}
%   \centering
%   \caption{The constant pressure molar heat capacity $C_{v,m}$, isothermal compressibility $\kappa_T$ and
%     thermal expansion coefficient $\alpha$
%     calculated by different methods.    
%     The parenthesises in the last column show the statistical uncertainty
%     at the confidence level of 95~\%.
%     The bold numbers indicating that its deviation from the SPME result is larger than the statistical uncertainty.
%   }
%   \begin{tabular*}{0.9\textwidth}{@{\extracolsep{\fill}}cccc ccc}\hline\hline
%     Method      &   $r_c$ [nm] &    $l$ & $\alpha$ [$\textrm{nm}^{-1}$]     &  $\kappa_T$ [$10^{-10}\textrm{m}^2/\textrm{N}$] & $C_{p,m}$ [J/(mol K)] & $\alpha$ [$10^{-3}\textrm{K}^{-1}$]\\\hline
%     SPME        &  1.9     &       -- &    & 5.941 (0.050)               &79.18 (0.94)           &1.026 (0.016)          \\
%     RF          &  1.2     & --       &    & \textbf{6.508} (0.058)      &\textbf{82.34} (0.95)  &\textbf{1.149} (0.018) \\\hline
%     ZM          &  1.2     &       1  & 0.0& 5.872 (0.050)               &78.91 (0.95)           &1.009 (0.018)          \\
%     ZM          &  1.2     &       1  & 0.5& 5.920 (0.050)               &78.90 (0.92)           &\textbf{1.003} (0.016) \\
%     ZM          &  1.2     &       1  & 1.0& 5.946 (0.050)               &78.37 (0.92)           &1.008 (0.016)          \\
%     ZM          &  1.2     &       1  & 1.5& 5.974 (0.050)               &80.23 (0.86)           &1.040 (0.017)          \\
%     ZM          &  1.2     &       1  & 2.0& \textbf{6.052} (0.051)      &79.41 (0.91)           &1.045 (0.016)          \\\hline
%     ZM          &  1.2     &       2  & 0.0& 5.952 (0.051)               &79.70 (0.96)           &1.033 (0.016)          \\
%     ZM          &  1.2     &       3  & 0.0& 5.991 (0.050)               &79.49 (1.01)           &1.024 (0.019)          \\
%     ZM          &  1.2     &       4  & 0.0& 5.961 (0.054)               &79.46 (0.98)           &1.034 (0.017)          \\
%     \hline\hline
%   \end{tabular*}
%   \label{tab:tmp3a}
% \end{table}


% \begin{table}
%   \centering
%   \caption{The excess chemical potential $\mu^\exc$, constant volume molar heat capacity $C_{v,m}$, dielectric constant $\eps$, diffusion constant $D$ and viscosity $\eta$
%     calculated by different methods.    
%     The parenthesises in the last column show the statistical uncertainty
%     at the confidence level of 95~\%.
%     The bold numbers indicating that its deviation from the SPME result is larger than the statistical uncertainty.
%   }
%   \begin{tabular*}{0.95\textwidth}{@{\extracolsep{\fill}}cccc ccccc}\hline\hline
%     Method      &   $r_c$ &    $l$ & $\alpha$  & $\mu^\exc$  &$C_{v,m}$ &  $\eps$ & $D$ &  $\eta$ \\
%                 &    [nm] &        & [$\textrm{nm}^{-1}$] &   [kJ/mol] &[J/(mol K)] & (Sys.~$\systemsb$) & [$10^{-9}\textrm{m}^2/\textrm{s}$] &  [$10^{-3}\textrm{Pa}\cdot\textrm{s}$] \\\hline
%     % Method      &   $r_c$ [nm] &    $l$      & $\mu^\exc$ [kJ/mol] &$C_{v,m}$ [J/(mol K)] & $\eps$ (Sys.~$\systemsb$) & $D$ [$10^{-9}\textrm{m}^2/\textrm{s}$] &  $\eta$ [$10^{-3}\textrm{Pa}\cdot\textrm{s}$] \\\hline
%     SPME        &1.9 & -- &    & $-26.08$ (0.24) & 72.21 (0.97)  & 97.6 (3.4)  & 5.86 (0.07)   & 0.315 (0.007)\\
%     RF          &1.2 & -- &--  & $-26.03$ (0.32) & 72.28 (0.81)  & \textbf{59.4} (1.3)  & \textbf{6.27} (0.19)   & \textbf{0.449} (0.020)\\\hline
%     ZM          &1.2 & 1  &0.0 & $-26.22$ (0.30) & 71.75 (0.89)  & 95.5 (2.4)  & \textbf{5.53} (0.11)   & \textbf{0.346} (0.006)\\ 
%     ZM          &1.2 & 1  &0.5 & $-26.36$ (0.30) & 72.25 (0.78)  & 97.1 (2.2)  & \textbf{5.52} (0.25)   & \textbf{0.339} (0.008)\\ 
%     ZM          &1.2 & 1  &1.0 & $-26.01$ (0.22) & 72.44 (0.84)  & 96.2 (2.3)  &        {5.78} (0.25)   & \textbf{0.330} (0.011)\\ 
%     ZM          &1.2 & 1  &1.5 & $-26.03$ (0.24) & 71.31 (0.78)  & 99.1 (2.5)  & \textbf{5.71} (0.09)   &        {0.312} (0.008)\\ 
%     ZM          &1.2 & 1  &2.0 & $-26.14$ (0.34) & 71.27 (0.78)  & 95.9 (2.5)  &        {5.90} (0.04)   &        {0.307} (0.007)\\ \hline
%     ZM          &1.2 & 2  &0.0 & $-26.24$ (0.24) & 71.53 (0.88)  & 95.4 (2.5)  & 5.82 (0.27)   & 0.318 (0.012)\\ 
%     ZM          &1.2 & 3  &0.0 & $-{26.18}$ (0.20) & 71.58 (0.88)  & 96.0 (2.7)  & 5.79 (0.09)   & 0.321 (0.008)\\ 
%     ZM          &1.2 & 4  &0.0 & $-{26.01}$ (0.26) & 71.07 (0.84)  & 99.7 (2.2)  & 5.89 (0.09)   & 0.318 (0.013)\\
%     \hline\hline
%   \end{tabular*}
%   \label{tab:tmp3}
% \end{table}
    % SPME        &         1.9     &       --      & 1.67049 (0.00014)     $-26.08$ (0.24) & 72.21 (0.97)  & 97.6 (3.4)  & 5.86 (0.07)   & 0.315 (0.007)\\
    % RF          &         1.2     & --            & 1.66733 (0.00016)      $-26.03$ (0.32) & 72.28 (0.81)  & \textbf{59.4} (1.3)  & \textbf{6.27} (0.19)   & \textbf{0.449} (0.020)\\
    % ZM          &         1.2     &       1       & 1.67052 (0.00022)     $-26.22$ (0.30) & 71.75 (0.89)  & 95.5 (2.4)  & \textbf{5.53} (0.11)   & \textbf{0.346} (0.006)\\ 
    % ZM          &         1.2     &       2       & 1.67029 (0.00013)     $-26.24$ (0.24) & 71.53 (0.88)  & 95.4 (2.5)  & 5.82 (0.27)   & 0.318 (0.012)\\ 
    % ZM          &         1.2     &       3       & 1.67024 (0.00018)     $-{26.18}$ (0.20) & 71.58 (0.88)  & 96.0 (2.7)  & 5.79 (0.09)   & 0.321 (0.008)\\ 
    % ZM          &         1.2     &       4       & 1.67031 (0.00010)     $-{26.01}$ (0.26) & 71.07 (0.84)  & 99.7 (2.2)  & 5.89 (0.09)   & 0.318 (0.013)\\
% \subsubsection{Hydrogen bonding}
% We count 


\begin{table}
  \centering
  \caption{A list of the dielectric constant calculated for different systems by different methods.
    The parameters are provided. The parenthesises in the last column show the statistical uncertainty
    of the last two digits up to the confidence level of 95~\%.}
  \begin{tabular*}{0.5\textwidth}{@{\extracolsep{\fill}}cccc rr}\hline\hline
    System & Method      &       $\alpha$ [$\textrm{nm}^{-1}$] & $r_c$ [nm] &    $l$     &       $\eps$ \\\hline
    $\systemmb$  &       ZM          &       0.00    &       1.2     &       1       &       99.8 (3.8)\\ 
    $\systemmb$  &       ZM          &       0.00    &       1.2     &       2       &       95.2 (3.6)\\ 
    $\systemmb$  &       ZM          &       0.00    &       1.2     &       3       &       93.9 (3.7)\\ 
    $\systemmb$  &       ZM          &       0.00    &       1.5     &       3       &       98.7 (3.8)\\ 
    $\systemmb$  &       ZM          &       0.00    &       1.8     &       3       &      100.7 (3.6)\\ 
    $\systemmb$  &       ZM          &       0.00    &       1.2     &       4       &       94.7 (3.6)\\
    $\systemmb$   & SPME          & 2.09  & 1.9   &       --      &       98.8 (4.0) \\
    $\systemmb$  & RF             & --  & 1.2 & -- & 60.6   (1.7) \\
    $\systemlb$  &       ZM          &       0.00    &       1.2     &       1       &       97.1 (3.2)\\ 
    $\systemlb$  &       ZM          &       0.00    &       1.2     &       2       &       95.6 (3.5)\\ 
    $\systemlb$  &       ZM          &       0.00    &       1.2     &       3       &       95.8 (4.6)\\ 
    $\systemlb$  &       ZM          &       0.00    &       1.5     &       3       &       98.0 (3.3)\\ 
    $\systemlb$  &       ZM          &       0.00    &       1.8     &       3       &       98.9 (4.2)\\ 
    $\systemlb$  &       ZM          &       0.00    &       2.1     &       3       &       98.8 (3.9)\\ 
    $\systemlb$  &       ZM          &       0.00    &       1.2     &       4       &       97.8 (3.9)\\
   $\systemlb$   & SPME          & 2.09  & 1.9   &       --      &       98.7 (3.5) \\
    $\systemlb$  & RF             & --  & 1.2 & -- & 60.8   (2.0) \\
    \hline\hline
  \end{tabular*}
  \label{tab:tmp2}
\end{table}



\begin{figure}
  \centering
  \includegraphics[]{fig/result.nvt/fig-rdf.eps}  
  \caption{The RDF of different method. The insert is a zoom-in of range 0.9 -- 1.5~nm. In the insert, the position of the cut-off (1.2~nm) is indicated by a vertical black line.}
  \label{fig:rdf}
\end{figure}


\begin{figure}
  \centering
  \includegraphics[]{fig/result.nvt.small/fig-gkr-small.eps}
  \includegraphics[]{fig/result.nvt.small/fig-gkr-small-conv.eps}
  \includegraphics[]{fig/result.nvt.small/fig-eps-t-3000.eps}
  \includegraphics[]{fig/result.nvt.small/fig-eps-t-zm3-3000.eps}
  \includegraphics[]{fig/result.nvt.small/fig-eps-t-10000.eps}
  \includegraphics[]{fig/result.nvt.small/fig-eps-t-zm3-10000.eps}
  \caption{The Kirkwood G-factor calculated for system $\systemmb$.
    % The result of different methods are compared in the left plot.
    For all ZM methods the splitting parameter $\alpha = 0.00\,\textrm{nm}^{-1}$.
    In the left plot the cut-off radius for all orders is set to $1.2$~nm, while
    in the right plot, only the order $l=3$ is considered and different cut-off radii are compared.
    The RF method uses a cut-off radius of 1.2~nm, and dielectric constant of 80.
    The statistical uncertainty of the SPME method is presented at 95~\% confidence level with the red bar.
  }
  \label{fig:tmp1}
\end{figure}

\begin{figure}
  \centering
  \includegraphics[]{fig/result.nvt/fig-gkr.eps}
  \includegraphics[]{fig/result.nvt/fig-gkr-conv.eps}
  \includegraphics[]{fig/result.nvt/fig-eps-t-3000.eps}
  \includegraphics[]{fig/result.nvt/fig-eps-t-zm3-3000.eps}
  \caption{The Kirkwood G-factor calculated for system $\systemlb$.
    % The result of different methods are compared in the left plot.
    For all ZM methods the splitting parameter $\alpha = 0.00\,\textrm{nm}^{-1}$.
    In the left plot the cut-off radius for all orders is set to $1.2$~nm, while
    in the right plot, only the order $l=3$ is considered and different cut-off radii are compared.
    The RF method uses a cut-off radius of 1.2~nm, and dielectric constant of 80.
    The statistical uncertainty of the SPME method is presented at 95~\% confidence level with the red bar.
  }
  \label{fig:tmp2}
\end{figure}

\begin{figure}
  \centering
  \includegraphics[]{fig/result.nvt/fig-gkr-10000.eps}
  \includegraphics[]{fig/result.nvt/fig-gkr-conv-10000.eps}
  \includegraphics[]{fig/result.nvt/fig-eps-t-10000.eps}
  \includegraphics[]{fig/result.nvt/fig-eps-t-zm3-10000.eps}
  \caption{The Kirkwood G-factor calculated for system $\systemlb$. Equilibriation time is 10~ns, and the production simulation time is 10~ns.
    % The result of different methods are compared in the left plot.
    For all ZM methods the splitting parameter $\alpha = 0.00\,\textrm{nm}^{-1}$.
    In the left plot the cut-off radius for all orders is set to $1.2$~nm, while
    in the right plot, only the order $l=3$ is considered and different cut-off radii are compared.
    The RF method uses a cut-off radius of 1.2~nm, and dielectric constant of 80.
    The statistical uncertainty of the SPME method is presented at 95~\% confidence level with the red bar.
  }
  \label{fig:tmp3}
\end{figure}






\subsection{Damping effect}

In this section we investigate the damping effect. The pressure is plotted as a function of  the cut-off radius is plotted for different choices of
the splitting parameter $\alpha$ in Fig.~\ref{fig:damp-pres}. The system for the simulation was $L_0$.
\begin{figure}
  \centering
  \includegraphics[width=0.49\textwidth]{fig/nvt.pressure.1/pressure-l1.eps}
  \includegraphics[width=0.49\textwidth]{fig/nvt.pressure.1/pressure-l2.eps}\\
  \includegraphics[width=0.49\textwidth]{fig/nvt.pressure.1/pressure-l3.eps}
  \includegraphics[width=0.49\textwidth]{fig/nvt.pressure.1/pressure-l4.eps}
  \caption{The pressure convergence with respect to the cut-off radius.
    For ZM method, different splitting parameter $\alpha$ (in unit of $\textrm{nm}^{-1}$) are shown for different orders:  $l=1$, 2, 4, and 4.
  }
  \label{fig:damp-pres}
\end{figure}

The convergence of the integrated auto-correlation function $I_\eta(T) $ is shown in Fig.~\ref{fig:damp-vis-l1}.
\begin{figure}
  \centering
  \includegraphics[]{fig/result.tiny/fig-vis-l1-damp.eps}
  \caption{The convergence of the integrated auto-correlation function $I_\eta(T) $. The error bars indicating 95\% confidence level are plotted with the SPME method. The different splitting parameters for ZM $l=1$ method are plotted}
  \label{fig:damp-vis-l1}
\end{figure}


The RDF is plotted for different splitting parameters $\alpha$ for ZM $l=2$ and 3 in Fig.~\ref{fig:damp-rdf-l23}. Higher $\alpha$ reproduces the RDF better.
The system was $L_1$.
\begin{figure}[]
  \centering
  \includegraphics[width=0.49\textwidth]{fig/result.nvt/fig-rdf-l2-damp.eps}
  \includegraphics[width=0.49\textwidth]{fig/result.nvt/fig-rdf-l3-damp.eps}
  \caption{Radial distribution function for different splitting parameters $\alpha$ for ZM $l=2$ (left) and $l=3$ (right).}
  \label{fig:damp-rdf-l23}
\end{figure}

The Kirkwood G-factor is plotted for different splitting parameters $\alpha$ for ZM $l=2$ and 3 in Fig.~\ref{fig:damp-gkr-l23}. 
The system was $L_1$.
\begin{figure}[]
  \centering
  \includegraphics[width=0.49\textwidth]{fig/result.nvt/fig-gkr-l2-damp.eps}
  \includegraphics[width=0.49\textwidth]{fig/result.nvt/fig-gkr-l3-damp.eps}
  \caption{Kirkwood G-factor for different splitting parameters $\alpha$ for ZM $l=2$ (left) and 3 (right).}
  \label{fig:damp-gkr-l23}
\end{figure}


It is well know that the G-factor is system size
dependent~\cite{vanderSpoel2006origin}. Here we show the G-factor for
both a smaller system $\systemmb$ (Fig.~\ref{fig:tmp1}) and a larger system $\systemlb$
(Fig.~\ref{fig:tmp2}).  In the figures, the statistical uncertainty of
the SPME method is denoted by the red error bar, while those of the
other methods are essentially the same, so they are not presented for
clarity.  In both the system, the RF method is qualitatively
wrong. The results of the ZM method are consistent with SPME
method. The lower order produces better consistency.  An artificial
oscillation around the cut-off radius presents for $l=1$, but it is
not obvious for $l\geq 2$.  For high order ZM (here $l=3$ is investigated),
using larger cut-off radius can substantially improve the accuracy.

The
resulting slop is used to calculate the diffusion constant. The results are listed in Table~\ref{tab:tmp3}.
The ZM method $l\geq 2$ is consistent with the SPME
result, while the RF method is off. 

\appendix

\section{The computation of benchmark quantities}
\label{appendix:benchmark}

In this section we proivde the formula used for computing the benchmark quantities. We assume that there are
$N$ molecules in the water system. The position and configuration of each molecule are fully described
by $\{\vect r_{3i},\vect r_{3i+1}, \vect r_{3i+2}\},\, i\in 1,\cdots,N$, and the corresponding velocities
are denoted by $\{\vect v_{3i},\vect v_{3i+1}, \vect v_{3i+2}\},\, i\in 1,\cdots,N$. The mass of each
atom is denoted by $m_i$.

\subsection{Radial distribution function}
The radial distribution function (RDF) $g(r)$ is a scalor function of distance $r$, which indicates the
probability of finding two atoms of distance $r$ apart.  It is a very
impotant equilibrium structure property, by which the X-ray scattering
intensity and a mount of thermodynamic properties can be computed.
In this work, we investigate the center-of-mass RDF in the water
system.

% The results are plotted in Fig.~\ref{fig:rdf}. As the order of ZM method goes higher, the precision of RDF improves. In addition, the precision of RF method is between $l=2$ and $l=3$.

\subsection{Pressure}
The pressure of the system calculated by the virial formulus:
\begin{align}
  P = \frac1{3V}
  \Bigg\langle
  \sum_{i=1}^{3N} \Big( m_i\vect v_i^2 + \vect r_i\cdot \vect F_i \Big)
  \Bigg\rangle,
\end{align}
where $V$ is the volume of the system.

\subsection{Chemical potential}

The chemical potential is defined by
\begin{align}
  \mu = \Big(\frac{\partial A}{\partial N}\Big)_{V,T}
\end{align}
In practice the excess chemical potential $\mu^\exc$, which is the
chemical potential abstracted by the kinetic contribution, is of
special interest. We calculate the excess chemical potential by the
thermodynamic integration (TI). In this approach, the interaction
between an inserted testing molecule and the system is
denotedd by $U_t(\lambda)$, where $\lambda$
is a coupling parameter in range $[0,1]$.
When $\lambda = 1$ the testing molecule is fully coupled to the system
with the TIP3P interaction. When $\lambda=0$, the testing molecule
is decoupled with the system.
Therefore the
Hamiltonian of the system is function of the coupling parameter. The
excess chemical is calculated by the free energy difference between
the fully coupled and decoupled states:
\begin{align}
  \mu^\exc = \int_0^1 \Big\langle \frac{\partial U_t(\lambda)}{\partial \lambda} \Big\rangle_\lambda d\lambda
\end{align}
This integral is usually calculated by numerical integration, which
equally paritions the range of $\lambda$, simulates the system at each
discretized $\lambda$, and computes chemical potential by trapezoidal formulus.
Here for the sake of accuracy, 
we adopt a two step coupling approach: firstly couple the van der Waals
interaction of the testing particle with the rest of the system (stage vdw),
and then couple the electrostatic interaction (stage ele).
During stage vdw, 21 $\lambda$ values are equally
distributed in $[0,1]$.  During stage ele, 6 $\lambda$ values are
equally distributed in $[0,0.05]$, while the other 20 $\lambda$ values
are equally distributed in $(0.05, 1]$. Therefore, in total 47
simulations are performed for all $\lambda$ values.
The free energy differences and the error estimates are calculated
by the Bennet's acceptance ratio method (BAR)~\cite{bennett1976efficient}. 
The simulation was
in system $\systemla$.

\subsection{Constant volume/pressure molar heat capacity}

The constant volume heat capacity is defined by the infiniesimal
increment of the energy due to an infiniesimal increment of the temperature
at constnat volume condition: 
\begin{align}
  C_V = \Big(\frac{\partial E}{\partial T}\Big)_V
\end{align}
Since it is a extensive thermodynamics quantity, we  normalize
it by the number molecules in the system $N$ to obtain the
constnat volume molar heat capacity $C_{V,m} = C_V/N$.
In the MD simlations, 
the constant volume molar heat capacity is calculated by estimating the
fulctuation of the Hamiltonian in an NVT simulation:
\begin{align}
  C_{V,m} = \frac{1}{k_BT^2 N} \langle (\mathcal H - \langle\mathcal H\rangle)^2 \rangle,
\end{align}
where $\mathcal H$ is the 
Hamiltonian of the system. 
% It is calculated in system $\systemsb$.

The constant pressure molar heat capacity is defined in
a similar way, but at constant pressure
condition. 
It is calculated by estimating
the fluctuation of the enthalpy in an NPT simulation~\cite{wang2011existence}:
\begin{align}
  C_{p,m} = \frac{1}{k_BT^2 N} \langle ( H - \langle H\rangle)^2 \rangle,
\end{align}
where $ H$ is the enthalpy of the system,
defined by $H = \mathcal H + P\mathcal V$, and $\mathcal V$ is the instantaneous volume
of the system.
It has been shown that the statistical error in estimating
the fluctuations does not decrease with
respect to the system size~\cite{milchev1986fluctuations,ferrenberg1991statistical}.
Therefore, both constant volume and pressure molar heat capacities
are calculated from the simulation of the smallest system: $\systemsb$.

\subsection{Isothermal heat capacity}
The isothermal heat capacity is defined
by the normalized increment of volume due to an infiniesimal
decrement of pressure under constant temperature condition:
\begin{align}
  \kappa_T = - \frac 1V \Big(\frac{\partial V}{\partial P}\Big)_T
\end{align}
It can be calculated by estimating the fluctuation
of the instantaneous volume in an NPT simulation~\cite{wang2011existence}:
\begin{align}
  \kappa_T = \frac{1}{k_BT} \frac{\langle (\mathcal V - \langle \mathcal V\rangle)^2 \rangle}{\langle \mathcal V\rangle}.
\end{align}
Since it is also a fluctuation, the system $\systemsb$ is adopted for simulation.

\subsection{Thermal expansion coefficient}
The thermal expansion coefficient is defined by
the normalized increment of volume due to an infiniesimal
increment of temperature under the constant pressure condition:
\begin{align}
  \alpha_V = - \frac 1V \Big(\frac{\partial V}{\partial T}\Big)_P
\end{align}
It can be calculated by estimating the cross fluctuation
of the enthalpy and instantaneous volume  in an NPT simulation:
\begin{align}
  \alpha_V = \frac{1}{k_BT^2\langle \mathcal V\rangle} \langle (H - \langle H\rangle)\cdot(\mathcal V - \langle \mathcal V\rangle) \rangle
\end{align}
The system $\systemsb$ adopted for simulation.


\subsection{The dielectric constant}

The dielectric constant can be estimated in a direct manner:
\begin{align}
  \eps = 1 + \frac{1}{3L^3 k_BT} ( \langle \vert \vect M\vert^2\rangle - \vert\langle \vect M\rangle\vert^2 )
\end{align}
where
\begin{align}
  \vect M = \sum_i\boldsymbol\mu_i = \sum_{\alpha\in i} q_\alpha\vect r_\alpha
\end{align}
is the total dipole moment of the system. Since it is a fluctuation, the simulated system is $\systemsb$.
% The results are listed in Tab.~\ref{tab:tmp2}. The dielectric constant calculated in system $\systemsb$ is reported in Table.~\ref{tab:tmp3}.

\subsection{Kirkwood G-factor}

The Kirkwood G-factor is defined by~\cite{vanderSpoel2006origin}
\begin{align}
  G_k(r) =
  \Big\langle
  \frac 1N
  \sum_{i=1}^N \sum_{j, r_{ij} < r}
  \frac {\boldsymbol\mu_i \cdot \boldsymbol\mu_j}{\vert \boldsymbol\mu_i\vert \cdot \vert\boldsymbol\mu_j\vert}
  \Big\rangle,
\end{align}
where $r_{ij}$ denotes the oxygen-oxygen distance between two water
molecules.  

\subsection{Diffusion constant}
The diffusion constant is calculated from the Einstein relation:
\begin{align}
  D = \lim_{t\rightarrow \infty}\frac {1}{6t} \langle \vert \vect r_i(t) - \vect r_i(0)\vert^2\rangle.
\end{align}
In practice, the mean-square-displacement $\langle\vert \vect r_i(t) - \vect
r_i(0)\vert^2\rangle$ is calculated, then the value is linearly fitted. The system for simulation is $\systemsb$.

\subsection{Viscosity}
The viscosity is calculated from the Green-Kubo relation:
\begin{align}
  \eta = \frac{V}{k_BT}\int_0^\infty\langle P_{\alpha\beta}(0) P_{\alpha\beta}(t)\rangle\,dt, \quad \alpha,\beta \in \{x, y, z\}
\end{align}
where $\alpha$ and $\beta$ denote the directions, and
$P_{\alpha\beta}$ denotes the off-diagonal components of the pressure
tensor. Since we used the isotropic system setting, then it is obvious
that $P_{xy}$, $P_{yz}$ and $P_{xz}$ are equivalent. Moreover, it has
been pointed out that in addition $(P_{xx} - P_{yy})/2$ and $(P_{xx} -
P_{yy})/2$ are two independent components that are equivalent to the
first three~\cite{alfe1998first}. Therefore, the viscosity is
calculated from the auto-correlation functions of five independent
components, and the statistical error is estimated from the standard
deviation of the viscosities calculated from the five components.

We investigate the convergence of the integral of the auto-correlation function w.r.t.~time:
\begin{align}
  I_\eta(T) = \frac{V}{k_BT}\int_0^T\langle P_{\alpha\beta}(0) P_{\alpha\beta}(t)\rangle\,dt,
\end{align}
and plot the function $I_\eta(T) $ for all method in
Fig.~\ref{fig:conv-vis}.  It is clear the integral for the RF method
converges at 6~ps, while the ZM method converges only in 2~ps, which
is consistent with the SPME result.  The value of ZM $l=1$ method is
different from the rest ZM method an SPME result.
The values listed  in Table~\ref{tab:tmp3} are computed with $I_\eta(10\,\textrm{ps})$.
The
system for simulation is $\systemsb$.

\begin{figure}
  \centering
  \includegraphics[]{fig/result.tiny/fig-vis.eps}
  \caption{The convergence of the integrated auto-correlation function $I_\eta(T) $. The error bars indicating 95\% confidence level are plotted with the SPME method.}
  \label{fig:conv-vis}
\end{figure}


\newpage
\bibliography{ref}{}
\bibliographystyle{unsrt}

\end{document}
