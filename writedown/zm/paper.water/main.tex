%2multibyte Version: 5.50.0.2952 CodePage: 932
% \documentclass[reprint,unsortedadress,oneclumn]{revtex4-1}
%\documentclass[aip,jcp,a4paper,preprint,unsortedaddress,onecolumn,fleqn]{revtex4-1}
% \input{tcilatex}
% \input{tcilatex}
% \input{tcilatex}


\documentclass[a4paper,preprint,unsortedaddress,onecolumn,fleqn]{revtex4}
%%%%%%%%%%%%%%%%%%%%%%%%%%%%%%%%%%%%%%%%%%%%%%%%%%%%%%%%%%%%%%%%%%%%%%%%%%%%%%%%%%%%%%%%%%%%%%%%%%%%%%%%%%%%%%%%%%%%%%%%%%%%%%%%%%%%%%%%%%%%%%%%%%%%%%%%%%%%%%%%%%%%%%%%%%%%%%%%%%%%%%%%%%%%%%%%%%%%%%%%%%%%%%%%%%%%%%%%%%%%%%%%%%%%%%%%%%%%%%%%%%%%%%%%%%%%
\usepackage{amsmath,amssymb,amsfonts,latexsym}
\usepackage[dvips]{graphicx}
\usepackage{color}
\usepackage{indentfirst}
\usepackage{rotating,booktabs}

\setcounter{MaxMatrixCols}{10}
%TCIDATA{OutputFilter=LATEX.DLL}
%TCIDATA{Version=5.50.0.2952}
%TCIDATA{Codepage=932}
%TCIDATA{<META NAME="SaveForMode" CONTENT="1">}
%TCIDATA{BibliographyScheme=BibTeX}
%TCIDATA{LastRevised=Wednesday, December 10, 2014 19:23:08}
%TCIDATA{<META NAME="GraphicsSave" CONTENT="32">}
%TCIDATA{Language=American English}

\newcommand{\eps}{\varepsilon}
\newcommand{\recheck}[1]{{\color{red} #1}}
\newcommand{\redc}[1]{{\color{red} #1}}
\newcommand{\bluec}[1]{{\color{blue} #1}}
\newcommand{\vect}[1]{\textbf{\textit{#1}}}
\newcommand{\exc}{\textrm{ex}}
\newcommand{\systemsa}{S_0}
\newcommand{\systemsb}{S_1}
\newcommand{\systemsbp}{S'_1}
\newcommand{\systemma}{M_0}
\newcommand{\systemmb}{M_1}
\newcommand{\systemmbp}{M'_1}
\newcommand{\systemla}{L_0}
\newcommand{\systemlbp}{L'_1}
\newcommand{\systemlb}{L_1}
% \input{tcilatex}
\begin{document}

\title{A Critical Appraisal of the Zero-Multiple Method: Structural,
Thermodynamic, Dielectric, and Dynamical Properties of a Water System}
\author{Han Wang}
\email{wang_han@iapcm.ac.cn}
\affiliation{CAEP Software Center for High Performance Numerical Simulation, Huayuan Road
6, 100088 Beijing, China}
\affiliation{Zuse Institute Berlin (ZIB), Germany}
\author{Ikuo Fukuda}
\email{ifukuda@protein.osaka-u.ac.jp}
\affiliation{Institute for Protein Research, Osaka University, 3-2 Yamadaoka, Suita,
Osaka 565-0871, Japan}
\affiliation{RIKEN (The Institute of Physical and Chemical Research), 2-1 Hirosawa, Wako,
Saitama 351-0198, Japan}
\author{Haruki Nakamura}
\affiliation{Institute for Protein Research, Osaka University, 3-2 Yamadaoka, Suita,
Osaka 565-0871, Japan}
\date{\today}

\begin{abstract}
\end{abstract}

\maketitle

% A Critical Appraisal of Zero-Multiple Method in Classical Water Simulation}

\section{Introduction}

A very short overview of the existing electrostatic interaction methods. Two
classes in general: Long-range treatment, e.g.~Ewald and derived methods
like SPME. Short-range treatment, like RF and ZM.

The importance of developing a short-range method for the electrostatic
interaction.

Zero-Multiple method, an overview, achievements, successful applications.

The purpose of this work: Extensive and strict tests investigating the
reliability of Zero-Multiple method in water simulation, which is the most
important solvent.

The structure of the paper.

[{\color{blue} COMMENT: I can contribute this section later}]

\section{Zero-Multipole Summation}

Introduction of the ZMM, formula...[ {\color{blue} COMMENT: I can write here
later}]

% {\color{red} If we want to discuss the issue of cut-off method, a short
% introduction of the group cut-off v.s.~the atom cut-off method. (will
% discuss the implementation in gromacs later) }[ {\color{blue} COMMENT: what
% do you mean? {\color{red} I mean the explanation of group and atomistic
% cut-off method. the traditional gromacs uses group cut-off as default. We
% need instead the atomistic cut-off}}] [ {\color{blue} COMMENT: OK, the
% matter for {traditional gromacs} will be discussed in Sec. IIIB, and the
% relevant matter of ZM will be discussed around $\bigstar ${\ in Sec. IIIB.
% If we feel the resulted descriptions are too detailed, they will go to SI.}] 
% } [{\color{red} please see the red part in Sec. III B. The order of the
% paragraphs are also rearranged.}][ {\color{blue} I added comments.}]

\section{Material and Methods}

\subsection{Benchmark quantities}

In order to investigate the physical-chemical properties obtained by ZMM, we
simulate the water system by using both SPME and ZMM, and evaluate a set of 
\emph{benchmark quantities}. This set covers a broad range of water
properties that are of general interest. It includes the structure property
(radial distribution function), the thermodynamic properties (pressure,
excess chemical potential, constant volume/pressure heat capacity,
isothermal compressibility, thermal expansion coefficient), dielectric
properties (dielectric constant and Kirkwood-G factor) and dynamical
properties (diffusion constant and viscosity). The definitions and computing
methods of these quantities are provided in details in Appendix.~\ref%
{appendix:benchmark}. By comparing ZMM to SPME, a good consistency in
computing the benchmark quantities should indicate a good accuracy of the
ZMM. More importantly, the benchmark tests provide knowledge in the
situations when the ZMM does not work so well, and help in finding the way
of tuning the working parameters to achieve higher accuracy. The ZMM is also
compared to the traditional short-range method: RF. Magnitude
of the deviations between the ZMM and SPME is basically
discussed considering the statistical uncertainty.  \recheck{More details of the statistical
uncertainty in computing a physical quantity is provided in Appendix~\ref{app:error}.}
% ....[COMMENTS: Since the critical issue of this ms is the
% accuracy check, it is helpful (around here) to describe briefly how this
% check is done. This is also helpful for readers to understand the PME\
% pressure deviation "($-0.4\pm 1.5$~Bar){" appearing later. A little bit more
% detailed descriptions may be added in Appendix. Duplication to the caption
% of fig 1 is not the problem.]} }

\subsection{Simulation protocols}

\label{sec:protocols} In this work, we focus on a system of water molecules
modeled by TIP3P~\cite{jorgensen1983comparison} point-charge particles. 
% Other point-charge water models differ in the modeling parameters, and
The accuracy of ZMM in other water models is out of the scope of the current
work. However, since the point-charge water models share similar idea, and
only differ in the model parameters, it is highly probable that ZMM works
well with other water models if it works well with TIP3P.

All simulations were performed by Gromacs~4.6.5~\cite{hess2008gromacs,
pronk2013gromacs} complied by GNU C compiler with single-precision floating
point. The size of the systems and corresponding number of molecules used
for MD simulations are summarized in Tab.~\ref{tab:tmp1}. The benchmark
quantities were calculated in either canonical (NVT) or isothermal-isobaric
(NPT) ensemble, with equilibrium temperature $T=300$~K. In order to simulate
at the desired temperature, the systems $S_{1}$ and $S_{1}^{\prime }$ were
coupled to Nos\'{e}-Hoover~\cite{nose1984molecular,hoover1985canonical}
thermostat with time-scale $\tau _{T}=1.0$~ps, while systems $M_{1}$, $%
M_{1}^{\prime }$, $L_{0}$, $L_{1}$ and $L_{1}^{\prime }$ were coupled to
Langevin thermostat with time-scale $\tau _{T}=0.1$~ps. In the NPT
simulations, the systems were further coupled to the Parrinello-Rahman
barostat~\cite{parrinello1980crystal,parrinello1981polymorphic} (in Gromacs
implementation) with time-scale $\tau _{P}=0.5$~ps to equilibrate the system
at 1~Bar. All systems were integrated with leap-frog scheme at a time-step
of $0.001$~ps.

% [{\color{blue} COMMENT: I am confusing. Is "rcoulomb"="rlist" default in
% gromacs? If so, "Gromacs by default uses the group cut-off scheme" is
% correct. Or, here, you want to say the "neighbor list making" is done by
% group cut-off? (I reread your "cutoff.pdf")} {\color{red} In the main text I
% only mention the \textquotedblleft group cut-off\textquotedblright\ and
% \textquotedblleft atom cut-off\textquotedblright\ and put how to implement
% them in the Appendix. So those who are not familiar with Gromacs would not
% be confused by the Gormacs terminology (e.g. rlist, rcoulomb, ...). see the
% revise below: }][ {\color{blue} I see and also modified. check it.]}

The neighbor lists were updated every 5 time-steps. The Gromacs by default
uses the group cut-off scheme, while the ZMM recommends an atom cut-off
scheme~\cite{fukuda2011molecular,fukuda2013zero}. The difference between
these cut-off schemes is discussed in detail in e.g.~Ref.~\cite%
{hunenberger1998alternative,baumketner2009removing}. We adopted the atom
based cut-off scheme in the simulations using the SPME (short-range part)
and ZMM electrostatic. The implementation details of the atom cut-off scheme
in Gromacs are provided in Appendix.~\ref{app:cut-off}. The ZMM was
implemented in Gromacs with the tabulated electrostatic interaction, and the
interaction tables were generated by a home-made code.

We treated the RF with the group cut-off scheme because the group cut-off
scheme is the default scheme in Gromacs. The experimental dielectric
consistency $\varepsilon _{\text{rf}}=80$ was adopted for the RF. 
% In order to mimic the atom cut-off scheme in
% the Gromacs group cut-off implementation, the radius for neighbor list
% updating was set ${0.3}$~nm larger than the cut-off radius (the maximum of
% Coulomb cut-off and the van der Waals cut-off). 
% The \textquotedblleft 
% \texttt{group}\textquotedblright\ cut-off scheme in Gromacs was used.[ {\color{blue} COMMENT:
% Yes,..., this is not easy to understand how we mimic AC. {\color{red} it
%   should be OK as it is now.}} {\color{blue} Please see COMMENT in SecII.}]

The short-range part of SPME method was cut-offed at $1.9$~nm, and was
smoothed in a shell of $0.05$~nm width by the Gromacs \textquotedblleft 
\texttt{pme-switch}\textquotedblright\ method. The reciprocal space grid
spacing was set to $0.06$~nm, with B-Spline interpolation order of $6$.
These parameters were chosen to be more accurate than the Gromacs default
choices (grid spacing $0.12$~nm and interpolation order $4$). The conducting
boundary condition was used with SPME. The splitting parameter was optimized
by the error estimate, which is implemented in Gromacs by tool
\textquotedblleft \texttt{g\_pme\_error}\textquotedblright ~\cite%
{wang2010optimizing}. The optimized real space and reciprocal space force
computing errors were equally distributed as $1.4\times 10^{-4}$~kJ/(mol
nm). Since the magnitude of the forces are of order $10^{2}$~kJ/(mol nm),
this force computing precision almost reaches the limit of the
single-precision representation of the floating point numbers. 
% [ {\color{blue} COMMENT: Did you use the
% conducting boundary conditions? (maybe it is default in gromacs.) Please
% state the thing.}]
% In the simulations using RF method, the electrostatic interactions were cut-off in the
% group manner (Gromacs default), so the radius for neighbor list updating is set identical to
% the cut-off radius. [{\color{blue} COMMENT: Is "rcoulomb"="rlist" default
%   for RF? If so, please describe it.} \recheck{Describing "rcoulomb"="rlist" by human languange} ]
% \recheck{We used $1.2$~nm for both of them}, and the experimental dielectric constant $%
% \varepsilon _{\text{rf}}=80$.

The van der Waals interaction was always cut-offed at $1.2$~nm, and it was
modified in a shell of $0.05$~nm width near the cut-off sphere by the
Gromacs \textquotedblleft \texttt{shift}\textquotedblright\ method.

\begin{table}[tbp]
\caption{A list of systems simulated in this work. We report, from left to
right, the number of water molecules in the system (TIP3P model),
equilibrium box size, length of equilibriation, length of productive
trajectory. }
\label{tab:tmp1}\centering
\vskip .5cm 
\begin{tabular*}{0.5\textwidth}{@{\extracolsep{\fill}}crrrr}
\hline\hline
System & $N$ & $V$ [$\text{nm}^3$] & $T_{\text{eq}}$ [ps] & $T_{\text{prod}}$
[ps] \\ \hline
$S^{\prime }_1$ & 2,500 & $4.24^3$ & -- & 6,000 \\ 
$S_1$ & 2,500 & $4.24^3$ & 3,000 & 27,000 \\ \hline
$M^{\prime }_1$ & 4,500 & $5.15^3$ & -- & 6,000 \\ 
$M_1$ & 4,500 & $5.15^3$ & 3,000 & 17,000 \\ \hline
$L_0$ & 13,824 & $7.49^3$ & 200 & 1,800 \\ 
$L^{\prime }_1$ & 13,824 & $7.49^3$ & -- & 10,000 \\ 
$L_1$ & 13,824 & $7.49^3$ & 3,000 & 17,000 \\ \hline\hline
\end{tabular*}%
\end{table}

The initial configurations were prepared by in the following steps: (1) An
NPT simulation of system $L_{0}$ with SPME electrostatic was performed. The
equilibrium density (984.1~$\text{kg}/\text{m}^{3}$) of the system is
calculated from this simulation. (2) Initial configurations that contains
2500, 4500 and 13824 molecules were generated at the calculated equilibrium
density. (3) NVT simulations with SPME electrostatic for system $%
S_{1}^{\prime }$, $M_{1}^{\prime }$ and $L_{1}^{\prime }$ are performed. (4)
The final configurations of $S_{1}^{\prime }$ and $M_{1}^{\prime }$ were
used as initial configurations for system $S_{1}$ and $M_{1}$, respectively.
The final configuration of $L_{1}^{\prime }$ was used as initial
configurations for system $L_{0}$ and $L_{1}$. % The SPME
% parameters for all simulations reported by this work is: real space
% cut-off 1.90~nm.
% [ {\color{blue} COMMENT: The following has been moved here, OK? If we need a
%   special care, please modify.}]
Using the density in the NVT simulation with SPME electrostatic, the
pressure ($-0.4\pm 1.5$~Bar) is reproducing the target pressure of 1.0~Bar,
which means the system is in desired thermodynamic state.

% The cut-off is smoothed from 1.85 to 1.90~nm by the
% ``\texttt{switch}'' method provide by Gromacs
% 4.6~\cite{hess2008gromacs, pronk2013gromacs}. 

The smallest system $S_{1}$ was adopted for the quantities that are computed
by estimating fluctuations (in this work heat capacities, compressibility,
thermal expansion coefficient, dielectric constant, diffusion constant and
viscosity are fluctuations. See details in Appendix~\ref{appendix:benchmark}%
), because simply increasing the size of the system does not improve the
accuracy of the simulation~\cite%
{milchev1986fluctuations,ferrenberg1991statistical}. The optimal way of
measuring these properties is to simulate with smaller systems 
% (as far as the finite size effect is negligible)
and run longer simulations for longer time-averages. The reason of using Nos%
\'{e}-Hoover thermostat is that it perturbs the dynamics of the system only
slightly, while the Langevin thermostat is likely to substantially change
the dynamics when the coupling is very strong, therefore, the former was
used in computing the dynamical properties like the diffusion constant and
viscosity. The quantities that are ensemble averages (in this work,
pressure, chemical potential, radial distribution function and Kirkwood-G
factor are of this type) were computed by the largest system $L_{0}$ or $%
L_{1}$, in order to achieve higher parallel efficiency. The system $M_{1}$
is simulated to investigate the size-dependency of the Kirkwood-G factor~%
\cite{vanderSpoel2006origin}.

\section{Results and discussions}

In this section we compare the benchmark quantities of the ZMM to the SPME
method to investigate the accuracy of the former. The comparison to the RF
is also mentioned, to show that the ZMM could be a promising replacement of
the RF.

\subsection{Thermodynamic properties}

\begin{sidewaystable}
  \centering
  \caption{The excess chemical potential $\mu^\exc$, constant volume molar heat capacity $C_{v,m}$, dielectric constant $\eps$, diffusion constant $D$, viscosity $\eta$, constant pressure molar heat capacity $C_{v,m}$, isothermal compressibility $\kappa_T$ and
    thermal expansion coefficient $\alpha_V$
    calculated by different methods.    
    The parentheses in the last column show the statistical uncertainty
    at the confidence level of 95~\%.
    The bold numbers indicating that its deviation from the SPME result is larger than the statistical uncertainty.
  }
  \bigskip
  \centering\small\setlength\tabcolsep{2pt}
  \begin{tabular*}{0.99\textwidth}{@{\extracolsep{\fill}}cccc cccccccc}\hline\hline
    Method      &   $r_c$ &    $l$ & $\alpha$  & $\mu^\exc$  &$C_{v,m}$ & $C_{p,m}$ &   $\kappa_T$  &$\alpha_V$ &  $\eps$ & $D$ &  $\eta$  \\
                & [nm] & & [$\textrm{nm}^{-1}$] &   [kJ/mol] &[J/(mol K)] & [J/(mol K)] & [$10^{-10}\textrm{m}^2/\textrm{N}$] &  [$10^{-3}\textrm{K}^{-1}$]&  & [$10^{-9}\textrm{m}^2/\textrm{s}$] &  [$10^{-3}\textrm{Pa}\cdot\textrm{s}$]  \\
                &    &    &    &NVT $\systemla$&NVT $\systemsb$&  NPT $\systemsb$          &  NPT $\systemsb$    &  NPT $\systemsb$    & NVT $\systemsb$& NVT $\systemsb$    & NVT $\systemsb$       \\\hline
    SPME        &1.9 & -- &2.1 & $-26.1$ (0.2) & 72.2 (1.0)  &79.2 (0.9)           & 5.94 (0.05)               &1.03 (0.02)          & 98 (3)          &         5.86  (0.07)&         0.315  (0.007)\\
    RF          &1.2 & -- &--  & $-26.0$ (0.3) & 72.3 (0.8)  &\textbf{82.3} (1.0)  & \textbf{6.51} (0.06)      &\textbf{1.15} (0.02) & \textbf{59} (1) & \textbf{6.27} (0.19)& \textbf{0.449} (0.020)\\\hline
    ZMM          &1.2 & 1  &0.0 & $-26.2$ (0.3) & 71.8 (0.9)  &78.9 (1.0)           & 5.87 (0.05)               &1.01 (0.02)          & 96 (2)          & \textbf{5.53} (0.11)& \textbf{0.346} (0.006)\\ 
    ZMM          &1.2 & 1  &0.5 & $-26.4$ (0.3) & 72.3 (0.8)  &78.9 (0.9)           & 5.92 (0.05)               &\textbf{1.00} (0.02) & 97 (2)          & \textbf{5.52} (0.25)& \textbf{0.339} (0.008)\\ 
    ZMM          &1.2 & 1  &1.0 & $-26.0$ (0.2) & 72.4 (0.8)  &78.4 (0.9)           & 5.95 (0.05)               &1.01 (0.02)          & 96 (2)          &        {5.78} (0.25)& \textbf{0.330} (0.011)\\ 
    ZMM          &1.2 & 1  &1.5 & $-26.0$ (0.2) & 71.3 (0.8)  &80.2 (0.9)           & 5.97 (0.05)               &1.04 (0.02)          & 99 (3)          & \textbf{5.71} (0.09)&        {0.312} (0.008)\\ 
    ZMM          &1.2 & 1  &2.0 & $-26.1$ (0.3) & 71.3 (0.8)  &79.4 (0.9)           & \textbf{6.05} (0.05)      &1.05 (0.02)          & 96 (3)          &        {5.90} (0.04)&        {0.307} (0.007)\\\hline
    ZMM          &1.2 & 2  &0.0 & $-26.2$ (0.2) & 71.5 (0.9)  &79.7 (1.0)           & 5.95 (0.05)               &1.03 (0.02)          & 95 (3)          &         5.82  (0.27)&         0.318  (0.012)\\
    ZMM          &1.2 & 3  &0.0 & $-26.2$ (0.2) & 71.6 (0.9)  &79.5 (1.0)           & 5.99 (0.05)               &1.02 (0.02)          & 96 (3)          &         5.79  (0.09)&         0.321  (0.008)\\
    ZMM          &1.2 & 4  &0.0 & $-26.0$ (0.3) & 71.1 (0.8)  &79.5 (1.0)           & 5.96 (0.05)               &1.03 (0.02)          &100 (2)          &         5.89  (0.09)&         0.318  (0.013)\\
    \hline\hline
  \end{tabular*}
  \label{tab:thermo}
\end{sidewaystable}

Thermodynamic quantities including, the chemical potential, constant
volume/pressure molar heat capacity, isothermal compressibility and heat
expansion coefficient, are compared in Tab.~\ref{tab:thermo}. The ZMM
results are in good consistency with the SPME results. In almost all cases
the deviations from the SPME results are smaller than the statistical
uncertainty. The RF was not so bad, but the results of the constant pressure
molar heat capacity and the isothermal compressibility show considerable
errors. The dependence of the ZMM results on the order\ $l$\ is little.
Although the splitting parameter, $\alpha $, also does not have significant
influence on almost all these thermodynamic quantities for $l=1$, only the
isothermal compressibility admits a weak dependence. Overall, the ZMM\
agrees with the SPME results without any bias and out-performs the RF.

It should be noticed that in the calculation of the chemical potential, the
energy correction plays an important role. In the case of $l=1$, $\alpha
=0.0\,\text{nm}^{-1}$, it contributes $-0.1$~kJ/mol to the final result. By
increasing $l$ or $\alpha $, this contribution becomes more significant, and
reaches $-0.6$~kJ/mol for $l=4$, $\alpha =0.0\,\text{nm}^{-1}$, and $-1.0$%
~kJ/mol for $l=1$, $\alpha =2.0\,\text{nm}^{-1}$, respectively. [{%
\color{blue}COMMENT: We will clarify what is the "energy correction" after
making SecII, and describe the relation to IPSp after making Appendix B.}]

\begin{figure}[tbp]
\caption{The pressure convergence with respect to the cut-off radius. The
ZMM with order from 1 to 4 are compared with the RF. The splitting parameter
of ZMM is set to be $\protect\alpha =0.00\,\text{nm}^{-1}$. The black solid
line indicates the magnitude of the correct pressure: 1~Bar. The error bars
present the statistical uncertainty at 95\% confidence level. {\color{red}
This uncertainty is twice of the standard deviation of the estimates (which
are averages along MD trajectories) that is calculated by the block average
method~\protect\cite{janke2002statistical}.} }
\label{fig:pres-comp}\centering
\includegraphics[]{fig/nvt.pressure.1/pressure-methods.eps}
\end{figure}

We investigated the dependence on the cut-off radius, $r_{\text{c}}$, on the
pressure, simulated with system $L_{0}$. The pressure calculated by ZMM of
order $l=1,2,3$ and $4$ with $\alpha =0$ \ and that by the RF are plotted in
Fig.~\ref{fig:pres-comp}. All ZMM converges to the target pressure at large
enough cut-off radius. Lower order ZMM presents an smaller error in the
pressure calculation than the higher order ZMM. Note ZMM $l=1$ tends to
under-estimate the pressure, while ZMM $l=2,3$ and 4 tends to over-estimate.
However, the accuracy of ZMM in pressure calculation is much higher than
that of the RF. 
\begin{figure}[tbp]
\centering
\includegraphics[width=0.5\textwidth]{fig/nvt.pressure.1/pressure-l1.eps} %
% \includegraphics[width=0.24\textwidth]{fig/nvt.pressure.1/pressure-l2.eps}%
% \newline
% \includegraphics[width=0.24\textwidth]{fig/nvt.pressure.1/pressure-l3.eps} %
% \includegraphics[width=0.24\textwidth]{fig/nvt.pressure.1/pressure-l4.eps}
\caption{ The pressure convergence with respect to the cut-off radius for
ZMM $l=1$ with splitting parameter from $0.0$ to $2.0\,\text{nm}^{-1}$. The
black solid line indicates the magnitude of the correct pressure: 1~Bar. The
error bars present the statistical uncertainty at 95\% confidence level.}
\label{fig:pres-l1}
\end{figure}
The effect of $\alpha $ in the case of $l=1$ is shown in Fig.~\ref%
{fig:pres-l1}. The result of $\alpha =0.5\,$nm$^{-1}$ is similar to $\alpha
=0$ and shows good convergence. However, a higher $\alpha $ shifts the
pressure to higher side, and in particular $\alpha =2.0\,\text{nm}^{-1}$ is
too large to show no convergence to 1~Bar. This implies that $\alpha \leq 1\,%
\text{nm}^{-1}$ should be used in the simulations where the high accuracy in
the pressure is important. This "anti-damping effect" is similarly seen in
the energy accuracy in the TIP3P model~\cite{fukuda2014zero}. 
% [JCP2014 (J. Chem. Phys. 140, 194307 (2014))].
In SI the results of the higher order are shown, which are similar to that
of $l=1$, but indicate less significant dependency on $\alpha $ for a large $%
l$\ and for a small $r_{\text{c}}$, which is consistent with the property of
the pair potential of the ZMM~\cite{fukuda2014zero}. The convergence is not
sensitive to $\alpha $ up to $1.0\,\text{nm}^{-1}$.

{Larger estimations of the pressure seen in the ZMMs with $l>1$
should come from, roughly speaking, the suppression of the negative
contribution of the dipole-dipole (Keesom) attractive interaction from a
long distance. The suppression is larger, as the values of ${l}$ and $\alpha 
$ increase due to the damping in the ZMM pair function, and larger as $r_{%
\text{c}}$ decreases. Specifically, the ignorance of the Fourier part in the
ZMM~\cite{fukuda2011molecular} accounts for this suppression, since
suppressed function erf$(\alpha r)/r$ tends to the original Coulomb $1/r$ as 
$\alpha $ increases (e.g., erf$(\alpha r)\sim 1$ for $\alpha r\gtrsim 2$).
Thus the ignorance causes positive contribution to the pressure estimation
with increasing $\alpha $, as observed in Fig.~\ref{fig:pres-l1}. This
ignorance is validated for a small} { $\alpha $, which limits
the use of large $\alpha $\ in the ZMM. In addition, the dependence on order 
$l$ should be relevant to the excess energy error~\cite{fukuda2014zero},
which depends on the order. }

{The magnitude and origin of the deviations of the pressure in
the ZMM\ are discussed more quantitatively. First note that even with the
largest} system $L_{0}$ that contains 13824 water molecules, the fluctuation
(measured by standard deviation) of the pressure is roughly 100~Bar {%
(see SI fig.1)}, which means that it is very difficult to
achieve very accurate estimate for this quantity. {A pressure
error of 20~Bar only cause a density deviation, which is calculated by the
NPT simulations of ZMM, of roughly }${0.1}${\% (The density
deviation of ${0.1}$\% corresponds to each cell length deviation of about $%
0.03$\%, which should not be important to many applications). In fact,}
taking different orders ZMM at cut-off 1.2~nm with $\alpha =0$%
 for example, the density deviations {were} $0.051$\%, 
$0.061$\%, $0.097$\% and $0.146$\% for $l=1,2,3$ and 4, respectively.
%For $l\leq 3$ the pressure error is less than 20~Bar, while
%for $l=4$ the pressure error is less than 30~Bar (see Fig.~\ref%
%{fig:pres-comp}).
% [{\color{blue} {COMMENT1:} Can we have a trajectory figure of the pressure?
% It will show that the fluctuation is large and the 10 Bar error for ZMM looks
% small. If there is a suitable one, it may be good to put it in SI.} {%
% \color{red} Could you explain why? The SD should be enough for our purpose.}%
% ] [ {\color{blue} COMMENT2: Can we have some data on the (mass) density?
% Some potential users of ZMM\ may worry about the deviation of the density
% from the reality due to the pressure error. I remember that the density
% error for ZMM was small. It is helpful to show that the deviation of the
% density is sufficiently small. These two comments indicates that one wants
% to feel how good or bad about the 10 Bar error.} {\color{red} Done}][ {%
% \color{blue} Thank you very much for your consideration of the density. {%
% COMMENT1: please see my question in the last page. COMMENT2: The deviation
% of the pressure between the ZMM and PME should be due to this density
% deviation between the ZMM and PME (0.1\%). Namely {0.1\% density deviation
% should cause 10 atm deviation.}\ How about this confirmation? ---The
% pressure by ZMM\ in figs1\&2 was calculated with the NTV MD where the cell
% size was determined by the \textit{PME} NTP. So, if we will do NTV MD with
% the cell size determined by the \textit{ZMM} NTP (which was used to calculate
% the current density), and if we calculate the pressure by ZMM with this new
% NTV, then we should have 1 atm pressure for may cutoff lengths. If this is
% true, then we can show that the 10 atm deviation from the PME is not the
% serious problem but a tiny density deviation issue.--- How about this idea? }%
% }]
{Namely, the density of
the current simulation was fixed by the SPME NPT simulation, and this
density was deviated about $0.1$\% from the equilibrated value of the ZMM.} {%
The pressure deviations of the ZMM, although which may seem
large, originated from this slight density deviations between the SPME and
ZMM and from the sensitivity of the pressure to the density.} 
% \begin{table}
%   \centering
%   \caption{A list of the dielectric constant calculated for different systems by different methods.
%     The parameters are provided. The parenthesises in the last column show the statistical uncertainty
%     of the last two digits up to the confidence level of 95~\%.}
%   \begin{tabular*}{0.5\textwidth}{@{\extracolsep{\fill}}cccc rr}\hline\hline
%     System & Method      &       $\alpha$ [$\textrm{nm}^{-1}$] & $r_c$ [nm] &    $l$     &       $\eps$ \\\hline
%     $\systemmb$  &       ZMM          &       0.00    &       1.2     &       1       &       99.8 (3.8)\\ 
%     $\systemmb$  &       ZMM          &       0.00    &       1.2     &       2       &       95.2 (3.6)\\ 
%     $\systemmb$  &       ZMM          &       0.00    &       1.2     &       3       &       93.9 (3.7)\\ 
%     $\systemmb$  &       ZMM          &       0.00    &       1.5     &       3       &       98.7 (3.8)\\ 
%     $\systemmb$  &       ZMM          &       0.00    &       1.8     &       3       &      100.7 (3.6)\\ 
%     $\systemmb$  &       ZMM          &       0.00    &       1.2     &       4       &       94.7 (3.6)\\
%     $\systemmb$   & SPME          & 2.09  & 1.9   &       --      &       98.8 (4.0) \\
%     $\systemmb$  & RF             & --  & 1.2 & -- & 60.6   (1.7) \\
%     $\systemlb$  &       ZMM          &       0.00    &       1.2     &       1       &       97.1 (3.2)\\ 
%     $\systemlb$  &       ZMM          &       0.00    &       1.2     &       2       &       95.6 (3.5)\\ 
%     $\systemlb$  &       ZMM          &       0.00    &       1.2     &       3       &       95.8 (4.6)\\ 
%     $\systemlb$  &       ZMM          &       0.00    &       1.5     &       3       &       98.0 (3.3)\\ 
%     $\systemlb$  &       ZMM          &       0.00    &       1.8     &       3       &       98.9 (4.2)\\ 
%     $\systemlb$  &       ZMM          &       0.00    &       2.1     &       3       &       98.8 (3.9)\\ 
%     $\systemlb$  &       ZMM          &       0.00    &       1.2     &       4       &       97.8 (3.9)\\
%    $\systemlb$   & SPME          & 2.09  & 1.9   &       --      &       98.7 (3.5) \\
%     $\systemlb$  & RF             & --  & 1.2 & -- & 60.8   (2.0) \\
%     \hline\hline
%   \end{tabular*}
%   \label{tab:tmp2}
% \end{table}

\subsection{Dielectric quantities}

\begin{figure}[tbp]
\centering
\includegraphics[width=0.5\textwidth]{fig/result.nvt.tiny/fig-eps-t-3000.eps}
\caption{The dielectric constant convergence with respect to length of the
trajectories used to compute the value. The SPME, ZMM $l=1$ to 4 and RF are
presented. The cut-off distance of ZMM and RF was 1.2~nm. The working
parameters of SPME are given in Sec.~\protect\ref{sec:protocols} The system
used for simulation is $S_{1}$.}
\label{fig:eps-conv}
\end{figure}

The dielectric constant computed by ZMM at different orders are listed in
Tab.~\ref{tab:thermo}, and are consistent with the SPME computed results. In
contrast, the RF has poor accuracy in the dielectric constant. It is well
known that the dielectric feature shows system-size dependence~\cite%
{vanderSpoel2006origin}, therefore we further investigated this dependence
and also the effect of the cutoff length, as shown in Table~I of SI. ZMM
shows good results without any system-size dependency and indicate $r_{\text{%
c}}\sim 1.2\,$nm\ being suffice for the accuracy. Since a slow convergence
is also known for the dielectric constant, we have monitored the behavior as
shown in Fig.~\ref{fig:eps-conv}, and confirm the sufficient convergence
w.r.t.~the length of the trajectories.

\begin{figure}[tbp]
\centering
\includegraphics[]{fig/result.nvt.small/fig-gkr-small.eps}\newline
\includegraphics[]{fig/result.nvt/fig-gkr.eps}
\caption{The Kirkwood G-factor calculated for system $M_{1}$ (a) and $L_{1} $
(b). ZMM of orders $l=1,2,3$ and 4 with the cut-off 1.2~nm and splitting
parameter $\protect\alpha =0.0\,\text{nm}^{-1}$ are plotted. The RF is shown
as an comparison, which uses a cut-off radius of 1.2~nm and dielectric
constant of 80. The statistical uncertainty of the SPME method is presented
at 95~\% confidence level with the red error bars. the statistical
uncertainty of the SPME method is denoted by the red error bars. The
statistical uncertainties of the other methods are essentially of the same
size as the SPME, so they are not presented in the figure for clarity.}
\label{fig:gkr}
\end{figure}

As a measurement of the dipole correlation as a function of molecular
distance, the Kirkwood-G factor is an important quantity of the water
system. Considering the system size dependence~\cite{vanderSpoel2006origin},
we show the G-factor for system $M_{1}$ in Fig.~\ref{fig:gkr}~(a) and a
larger system $L_{1}$ in Fig.~\ref{fig:gkr} (b). In both the system, the RF
is qualitatively wrong, and the accuracy of the ZMM is much higher. In the
system of smaller size ($M_{1}$), ZMM of all orders are highly consistent
with the SPME result, and $l=2$ gave the best result. In the system of
larger size ($L_{1}$), the accuracies of the ZMM is less for order $l\geq 2$%
, while they may be in still acceptable accuracy. In both systems, ZMM $l=1$
presents an artificial oscillation around the cut-off radius (1.2~nm), while
this oscillation is not obvious for $l\geq 2$. The Kirkwood-G factor
computed by ZMM in system $S_{1}$ is as good as that computed in $M_{1}$,
and is presented in SI.

\begin{figure}[tbp]
\centering
\includegraphics[]{fig/result.nvt/fig-gkr-conv.eps}
\caption{ The cut-off dependency in calculating the Kirkwood-G factor. The
order of ZMM is $l=3$. The splitting parameter is $\protect\alpha =0.0\,%
\text{nm}^{-1}$. The statistical uncertainty is presented at 95~\%
confidence level by the error bars. }
\label{fig:gkr-conv}
\end{figure}

\begin{figure}[tbp]
\centering
\includegraphics[]{fig/result.nvt/fig-gkr-l1-damp.eps}
\caption{ The splitting parameter dependency in calculating the Kirkwood-G
factor at order $l=1$. The cut-off radius is 1.2~nm. The statistical
uncertainty is presented at 95~\% confidence level by the error bars. }
\label{fig:gkr-damp-l1}
\end{figure}

We thus investigated the cut-off dependency in calculating the Kirkwood-G
factor and present it in Fig.~\ref{fig:gkr-conv} for ZMM of order $l=3$.
Without surprising, the Kirkwood-G factor converges to the SPME result when
using increasingly large cut-off radius. In Fig.~\ref{fig:gkr-damp-l1}, it
shows that in the case of $l=1$, the largest splitting parameter ($\alpha
=2.0\ \text{nm}^{-1}$) is worse in calculating the Kirkwood-G factor than
smaller splitting parameters. Splitting parameter dependency was not so
clear for $l=2$ and 3 (shown in SI). Note that in many cases the
discrepancies from the SPME were relatively prominent in the middle range of
the distance and they tend small for larger distance.

\subsection{Dynamical quantities}

The dynamical quantities, diffusion constant and viscosity, are listed in
Tab.~\ref{tab:thermo}. The ZMM of order $l\geq 2$ can almost perfectly
reproduce the SPME dynamical quantities. The ZMM order $l=1$ is of less
accurate than higher orders, but is more accurate than the RF. Using larger
splitting parameters improves the accuracy of the order $l=1$, as understood
as the damping effect.

\begin{figure}[tbp]
\centering
\includegraphics[]{fig/result.nvt.tiny/fig-vis.eps}
\caption{The convergence of the integrated auto-correlation function $I_{%
\protect\eta }(t)$. Different orders of ZMM are compared with the SPME as
well as RF results. The error bars indicating 95\% confidence level are
plotted with the SPME method. }
\label{fig:conv-vis}
\end{figure}

\begin{figure}[tbp]
\centering
\includegraphics[]{fig/result.nvt.tiny/fig-vis-l1-damp.eps}
\caption{The convergence of the integrated auto-correlation function $I_{%
\protect\eta }(t)$. Different splitting parameters for ZMM order $l=1$ are
plotted. The error bars indicating 95\% confidence level are plotted with
the SPME method. }
\label{fig:conv-vis-damp}
\end{figure}

Similar tendency is confirmed in the auto-correlation function $I_{\eta }(T)$
as shown in Fig.~\ref{fig:conv-vis}. It is clear that the integral for the
RF converges at 6~ps, while the ZMM converges only in 2~ps, which is
consistent with the SPME result. The value of ZMM $l=1$ is different from
the rest ZMM and SPME result, but the results was improved by the damping
effect as shown in Fig.~\ref{fig:conv-vis-damp}. Similar to the behaviors
seen in Fig.~\ref{fig:conv-vis}, the convergence is quick for a large $%
\alpha $.

\subsection{Structure property}

\begin{figure}[tbp]
\centering
\includegraphics[]{fig/result.nvt/fig-rdf.eps}
\caption{The comparison of the center-of-mass radial distribution functions
calculated by SPME, ZMM and RF. The insert is a zoom-in of range 0.9 --
1.5~nm. In the inset, the position of the cut-off (1.2~nm) is indicated by a
vertical black line. For ZMM, the splitting parameter $\protect\alpha $ was
set to 0.0~$\text{nm}^{-1}$. }
\label{fig:rdf}
\end{figure}

We plot in Fig.~\ref{fig:rdf} the center-of-mass radial distribution
functions (RDFs) of TIP3P water calculated by the SPME, ZMM ($l=1,2,3$ and $%
4 $) with $\alpha =0$\ and RF. It is observed from the figure that the ZMM
RDFs are consist very well with the SPME RDF, except oscillations around the
cut-off radius. This oscillations are not so large, as the maximum magnitude
of the discrepancies from the SPME is about $1.9\%$ for $l=1$ and ${0.3\%}$
for $l=2$, % {[} {\color{blue}
% "$2\%"$ is just my guess. How about stating here the maximum error for
% individual $l$}]
but should be artificial. However, higher order ZMM present lower amplitude
oscillation, and when $l\geq 3$, the improvement is dramatic and the
oscillation is almost negligible. [ {\color{blue}COMMENT: I thank for your
efforts to reveal that the GC is far better than AC in RDF. Complete pursuit
of RF RDF, including the verification of Prof Nakamura's scenario, should be
our future work. Thus I have omitted the descriptions of the difference
between GC and AC in RDF]} %{\color{red} The RF RDF is more accurate
%than the ZMM ones, this is mainly because of the group cut-off used in
%simulating with RF. When changing the cut-off scheme to atom scheme, the
%accuracy is RF is roughly same as ZMM $l=1$ (see SI). We could not explain
%the reason, but the numerical evidence suggests if only the RDF is of
%interest, a group cut-off scheme or higher order of ZMM is preferable. }

% {\color{red} Shall we comment here on the fact that group cut-off produces
% smoother RDF? Add figures in Supplementary material {\color{blue} Do you
% mean the reason of the goodness in RF?} {\color{red} Yes, shall we discuss
% it? or it is too complicated and not very relevant to the current paper?}[} {%
% \color{blue} It is preferable to write it if the reason of the goodness is
% clear. Anyway, please describe.... This is the only matter ZMM is inferior to
% RF. So, it may be interesting if the reason is clear. } {\color{red} I
% agree, if the reason is not clear for us, we should not touch this topic.}][ 
% {\color{blue} COMMENT: Thank you very much for the O-O RDF. If you have time
% to reconsider this issue, it may be useful to calculate the RDF with H-H
% length (MD trajectory also remains). Deviations should be larger for H-H
% than O-O, but this is my guess: (1) eventually COG-COG is near O-O, sorry I
% did not notice.; (2) RDF deviations appear in 0.6 nm width around the cutoff
% and the H-H length is just the twice of this 0.6 nm). Rather, the true
% reason may be other, e.g., group cuotff MD scheme may intrinsically give the
% accracy at least for RDF. }][{\color{blue} COMMENT: Thank you very much for
% the HH trial. The reason may be other. Since we do not have the reason why
% RF was so good, one choice is the deletion of the RF data in the RDF. How do
% you think about it?}]

\begin{figure}[tbp]
\centering
\includegraphics[]{fig/result.nvt/fig-rdf-l1-damp.eps}
\caption{ The splitting parameter dependency in calculating the
center-of-mass RDF for ZMM $l=1$. The insert is a zoom-in of range 0.9 --
1.5~nm. In the insert, the position of the cut-off (1.2~nm) is indicated by
a vertical black line. }
\label{fig:rdf-damp}
\end{figure}

The oscillations for $l=1$\ can also be substantially improved by using a
larger splitting parameter $\alpha $. Figure~\ref{fig:rdf-damp} presents the
RDF of ZMM $l=1$ with splitting parameter from $0.0$ to $2.0$~$\text{nm}%
^{-1} $. The damping effect, viz., a clear trend of better RDF by using a
larger $\alpha $, is observed. At $\alpha =2.0\ \text{nm}^{-1}$ the RDF is
almost overlapping with the SPME result. This damping effect is clearer for
a smaller order $l$\ (see SI for results of {$l=2$ and $3$}), as expected
from the pair potential behavior~\cite{fukuda2014zero}.

% {\color{red} If we want $l=2,3$, add in supplementary material.{>}
% Please add them. }

% \subsection{Damping effect}

% The convergence of the integrated auto-correlation function $I_\eta(T) $ is shown in Fig.~\ref{fig:damp-vis-l1}.
% \begin{figure}
%   \centering
%   \includegraphics[]{fig/result.nvt.tiny/fig-vis-l1-damp.eps}
%   \caption{The convergence of the integrated auto-correlation function $I_\eta(T) $. The error bars indicating 95\% confidence level are plotted with the SPME method. The different splitting parameters for ZMM $l=1$ method are plotted}
%   \label{fig:damp-vis-l1}
% \end{figure}

% The RDF is plotted for different splitting parameters $\alpha$ for ZMM $l=2$ and 3 in Fig.~\ref{fig:damp-rdf-l23}. Higher $\alpha$ reproduces the RDF better.
% The system was $L_1$.
% \begin{figure}[]
%   \centering
%   \includegraphics[width=0.49\textwidth]{fig/result.nvt/fig-rdf-l2-damp.eps}
%   \includegraphics[width=0.49\textwidth]{fig/result.nvt/fig-rdf-l3-damp.eps}
%   \caption{Radial distribution function for different splitting parameters $\alpha$ for ZMM $l=2$ (left) and $l=3$ (right).}
%   \label{fig:damp-rdf-l23}
% \end{figure}

% The
% resulting slop is used to calculate the diffusion constant. The results are listed in Table~\ref{tab:thermo}.
% The ZMM method $l\geq 2$ is consistent with the SPME
% result, while the RF method is off. 

\section{Summary and conclusion}

In summary, ZMM works well for almost all cases. Damping effect of the
splitting parameter, viz., larger $\alpha $\ gives accurate results, was
seen in several quantities. The weak effect is seen in the isothermal
compressibility, {\color{red} } and dynamical quantities including the
diffusion constant and viscosity, while the strong effect is seen in the
RDF. Anti-damping effect was also observed in the pressure when $\alpha >1$
and Kirkwood-G factor (only $l=1$). A larger order is preferable slightly
for dynamical quantities including diffusion constant and viscosity, and
significantly for the RDF. Whereas a smaller order is preferable for the
pressure, though $l=1,2$\ are similar.

At cut-off radius of 1.2~nm, the recommended choices of parameters for TIP3P
system are thus $\alpha \leq 1$~$\text{nm}^{-1}$ and $l=2$ or $l=3$. Using
these parameters, the ZMM achieves almost perfect accuracy in computing the
excess chemical potential, constant volume/pressure heat capacity,
isothermal compressibility, thermal expansion coefficient, dielectric
constant diffusion constant and viscosity. The order $l=2$ performs better
in reproducing the pressure than $l=3$ (10 v.s.~17~Bar), but worse in
calculating the RDF (0.2\% error v.s.~almost negligible). Both of orders 2
and 3 are not sensitive to the splitting parameter up to 1.0~$\text{nm}^{-1}$%
. Although further increasing the splitting parameter will push the accuracy
of RDF in case $l=2 $ to as high as $l=3$, the pressure accuracy decreases
(anti-damping effect), so that $\alpha >1$\ for $l=2$ or $l=3$ is not
recommended. It should be noticed that the recommended parameters indicate $%
5 $\% error (largest) in calculating the Kirkwood-G factor for a larger
system ($L_{1}$ containing 13,824 molecules). If we seek more accuracy for
the G factor, we should use a larger cut-off radius at the expense of the
computational time. % Further increasing the splitting paraemter
% Optimal parameter value for the TIP3P system is thus $l=2,\alpha \leq 1$,
% where the error about $0.2\%$\  remains in the RDF. An
% alternative choice is $l=3,\alpha =0$, where the accuracy in the pressure is
% not the best but highly accurate RDF can be obtained.

[{\color{blue}---$>$ IPSp; I can contribute later. IPSp should be good for
the current water system since IPSp is developed for "polar" molecule (while
IPSn is not)}]

[ {\color{blue} COMMENT0: Although I know this paper's aim is to judge the
accuracy of the ZMM, I also know many people may also be interested in some
different aspects. So please see the following.}]

[ {\color{blue} COMMENT1: The first one is the accuracy comparison between
the ZMM\ and a "default PME" that uses default protocol and parameters,
e.g., group-based cutoff scheme (OK?) and a short cutoff length(about 1 nm?)
for the real part and a lower order spline and default grid spacing for the
reciprocal part. Many peoples are not the specialist of the electrostatic
calculation, and they just uses the default PME. Thus it would be OK\ for
them if the difference between the ZMM\ and a "default PME" is small. Is it
possible to calculate e.g., pressure or KG factor for {$L_{1}$} (n{ot all
the quantities}), using the {"default PME"?}}]

[ {\color{blue} COMMENT2: The second one is the comparison of the
computational timing between the ZMM\ and a "default PME."\ Many people,
including e.g. reviewers, should be interested in it. How about this
comparison using a short time simulation?\ One expects ZMM\ is faster than
SPME. But is SPME with rc=1.0 nm along with the buffer length 0.2nm (are
they default length?) faster than ZMM\ with rc=1.2 nm along with the buffer
length 0.3nm?? }]

\appendix

\section{The definition of statistical uncertainty of a physical quantity}
\label{app:error}

We denote the physical quantity under study by $\mathcal O$. In an MD simulation,
a series of measurements of $\mathcal O$ is denoted by $\{O_0, O_1, \cdots, O_{M-1}\}$.
Then the physical quantity can be estimated by:
\begin{align}\label{eqn:sum-esti}
  \mathcal O \approx \frac 1M \sum_{i=0}^M O_i
\end{align}
The twice the standard deviation of the right-hand-side of Eq.~\eqref{eqn:sum-esti}
is defined to be the \emph{statistical uncertainty} (at 95\% confidence level) in computing
the  physical quantity $\mathcal O$
\begin{align}
  \mathcal E_{O} = 2\sqrt{\Big\langle
  \Big(
  \frac 1M \sum_{i=0}^M O_i - 
  \Big\langle \frac 1M \sum_{i=0}^M O_i \Big \rangle
  \Big)^2
  \Big \rangle}.
\end{align}
If the measurements are identical and independent random variables,
then the statistical uncertainty is
\begin{align}
  \mathcal E_{O} = \frac{2}{\sqrt M} \sigma({O})
\end{align}
where $\sigma({O})$ is the standard deviation of measurements $O_i$.
If the measurements are correlated, the statistical uncertainty is estimated
by the block average method~\cite{janke2002statistical}.


\section{Implementation of atom cut-off scheme in Gromacs}

\label{app:cut-off}

The Gromacs code 4.6 does not directly support the atomistic cut-off scheme, 
{which is recommended in the ZMM, in }using tabulated
electrostatic interaction. The atom cut-off scheme can be implemented by
using the default \textquotedblleft \texttt{Group}\textquotedblright\
neighbor searching method, {% {%
% How much? (just my interest)}
{by }}setting the radius of neighbor list \textquotedblleft 
\texttt{rlist}\textquotedblright\ 0.3~nm larger than the cut-off radius,
which is the maximum of the electrostatic cut-off \textquotedblleft \texttt{%
rcoulomb}\textquotedblright\ and the van der Waals cut-off \textquotedblleft 
\texttt{rvdw}\textquotedblright . {The size of the thickness
0.3~nm of the buffering region was determined from the molecular size,
average speed of molecules, and the neighbor-searching timing. Due to this buffering region}, the atom cut-off scheme in
Gromacs is in general { twice} expensive { in
computing the electrostatic interaction} as the group cut-off scheme. {%
 It should be pointed out that this implementation is not
optimal, {but an effective realization of atom cut-off scheme in
the environment of the \textquotedblleft \texttt{Group}\textquotedblright\
neighbor searching. If} the gromacs build-in atom cut-off scheme supports
the tabulated potential, we expect better performance of ZMM. % {%
% How much? (just my interest)}
}

\section{The computation of benchmark quantities}

\label{appendix:benchmark}

In this section we provide the formula used for computing the benchmark
quantities. We assume that there are $N$ molecules in the water system. The
position and configuration of each molecule are fully described by $\{%
\mathbf{\mathit{r}}_{3i-2},\mathbf{\mathit{r}}_{3i-1},\mathbf{\mathit{r}}%
_{3i}\},\,i\in 1,\cdots ,N$, and the corresponding velocities are denoted by 
$\{\mathbf{\mathit{v}}_{3i-2},\mathbf{\mathit{v}}_{3i-1},\mathbf{\mathit{v}}%
_{3i}\},\,i\in 1,\cdots ,N$. The mass of each atom is denoted by $m_{i}$. 
% [ {%
% \color{blue} minor: I like bold font a little bit for $\mathbf{\mathit{r}}${%
% \ and }$\mathbf{\mathit{v}}${\ as in the first version, which is also seen
% in }$\boldsymbol{\mu }_{i}${. (In particular, }$\mathbf{\mathit{r}}${\ seems
% the distance, not vector. $\mathbf{\mathit{r}}$ appears below also. But if
% you like this font, please use it also for $\boldsymbol{\mu }_{i}$)}} {%
% \color{red} I prefer the original one, I did not change it. I guess your
% program did it automatically. I suggest you compile this file with standard
% latex that dose not do any automatic change to the source file!. Now I
% change the one of them back, please see if it works on your machine}][ {%
% \color{blue} Sorry, maybe this is just the problem of my program, now I can
% see }$\mathbf{\mathit{r}}_{3i-2}${\ to be bold in your original pdf.}]

\subsection{Radial distribution function}

The radial distribution function (RDF) $g(r)$ is a scalar function of
distance $r$, which indicates the probability of finding two atoms of
distance $r$ apart. It is a very important equilibrium structure property,
by which the X-ray scattering intensity and a mount of thermodynamic
properties can be computed. In this work, we investigate the center-of-mass
RDF in the water system. The RDFs presented in the main text were computed
from NVT simulations of system $L_1$.

% The results are plotted in Fig.~\ref{fig:rdf}. As the order of ZMM method goes higher, the precision of RDF improves. In addition, the precision of RF method is between $l=2$ and $l=3$.

\subsection{Pressure}

The pressure of the system calculated by the virial formula: 
% [ {\color{blue}
% Modified $3N\longrightarrow N$} {\color{red} should be $3N$!, each molecule
% contains 3 atoms. The pressure is computed by atom based virial formula. %
% \color{blue} OK, I was just confusing}]%
\begin{equation*}
P=\frac{1}{3V}\Bigg\langle\sum_{i=1}^{3N}\Big(m_{i}\mathbf{\mathit{v}}%
_{i}^{2}+\mathbf{\mathit{r}}_{i}\cdot \mathbf{\mathit{F}}_{i}\Big)%
\Bigg\rangle,
\end{equation*}%
where $V$ is the volume of the system. The presented pressures were computed
from NVT simulations of $L_{0}$. The dispersion correction~\cite{frenkel2001understanding} was
added to correct the value.

\subsection{Chemical potential}

The chemical potential is defined by 
\begin{align}
\mu = \Big(\frac{\partial A}{\partial N}\Big)_{V,T},
\end{align}
where $A$ is the Helmholtz free energy. In practice the excess chemical
potential $\mu^\text{ex}$, which is the chemical potential abstracted by the
kinetic contribution, is of special interest. We calculate the excess
chemical potential by the thermodynamic integration (TI). In this approach,
the interaction between an inserted testing molecule and the system is
denoted by $U_t(\lambda)$, where $\lambda$ is a coupling parameter in range $%
[0,1]$. When $\lambda = 1$ the testing molecule is fully coupled to the
system with the TIP3P interaction. When $\lambda=0$, the testing molecule is
decoupled with the system. Therefore the Hamiltonian of the system is
function of the coupling parameter. The excess chemical is calculated by the
free energy difference between the fully coupled and decoupled states: 
\begin{align}
\mu^\text{ex} = \int_0^1 \Big\langle \frac{\partial U_t(\lambda)}{\partial
\lambda} \Big\rangle_\lambda d\lambda
\end{align}
This integral is usually calculated by numerical integration, which equally
partitions the range of $\lambda$, simulates the system at each discretized $%
\lambda$, and computes chemical potential by trapezoidal formula. Here for
the sake of accuracy, we adopt a two step coupling approach: firstly couple
the van der Waals interaction of the testing particle with the rest of the
system (stage vdw), and then couple the electrostatic interaction (stage
ele). During stage vdw, 21 $\lambda$ values are equally distributed in $%
[0,1] $. During stage ele, 6 $\lambda$ values are equally distributed in $%
[0,0.05]$, while the other 20 $\lambda$ values are equally distributed in $%
(0.05, 1]$. Therefore, in total 47 simulations are performed for all $%
\lambda $ values. The free energy differences and the error estimates are
calculated by the Bennet's acceptance ratio method (BAR)~\cite%
{bennett1976efficient}. The NVT simulations were performed with system $L_0$.

\subsection{Constant volume/pressure molar heat capacity}

The constant volume heat capacity is defined by the infinitesimal increment
of the energy due to an infinitesimal increment of the temperature at
constant volume condition: 
\begin{align}
C_V = \Big(\frac{\partial E}{\partial T}\Big)_V
\end{align}
Since it is a extensive thermodynamics quantity, we normalize it by the
number molecules in the system $N$ to obtain the constant volume molar heat
capacity $C_{V,m} = C_V/N$. In the MD simulations, the constant volume molar
heat capacity is calculated by estimating the fluctuation of the Hamiltonian
in an NVT simulation: 
\begin{align}
C_{V,m} = \frac{1}{k_BT^2 N} \langle (\mathcal{H }- \langle\mathcal{H}%
\rangle)^2 \rangle,
\end{align}
where $\mathcal{H}$ is the Hamiltonian of the system. 
% It is calculated in system $\systemsb$.

The constant pressure molar heat capacity is defined in a similar way, but
at constant pressure condition. It is calculated by estimating the
fluctuation of the enthalpy in an NPT simulation~\cite{wang2011existence}: 
\begin{align}
C_{p,m} = \frac{1}{k_BT^2 N} \langle ( H - \langle H\rangle)^2 \rangle,
\end{align}
where $H$ is the enthalpy of the system, defined by $H = \mathcal{H }+ P%
\mathcal{V}$, and $\mathcal{V}$ is the instantaneous volume of the system.
It has been shown that the statistical error in estimating the fluctuations
does not decrease with respect to the system size~\cite%
{milchev1986fluctuations,ferrenberg1991statistical}. Therefore, both
constant volume and pressure molar heat capacities were calculated from the
simulation of the smallest system: $S_1$.

\subsection{Isothermal heat capacity}

The isothermal heat capacity is defined by the normalized increment of
volume due to an infinitesimal decrement of pressure under constant
temperature condition: 
\begin{align}
\kappa_T = - \frac 1V \Big(\frac{\partial V}{\partial P}\Big)_T
\end{align}
It can be calculated by estimating the fluctuation of the instantaneous
volume in an NPT simulation~\cite{wang2011existence}: 
\begin{align}
\kappa_T = \frac{1}{k_BT} \frac{\langle (\mathcal{V }- \langle \mathcal{V}%
\rangle)^2 \rangle}{\langle \mathcal{V}\rangle}.
\end{align}
Since it is also a fluctuation, the system $S_1$ was adopted for simulation.

\subsection{Thermal expansion coefficient}

The thermal expansion coefficient is defined by the normalized increment of
volume due to an infinitesimal increment of temperature under the constant
pressure condition: 
\begin{align}
\alpha_V = - \frac 1V \Big(\frac{\partial V}{\partial T}\Big)_P
\end{align}
It can be calculated by estimating the cross fluctuation of the enthalpy and
instantaneous volume in an NPT simulation: 
\begin{align}
\alpha_V = \frac{1}{k_BT^2\langle \mathcal{V}\rangle} \langle (H - \langle
H\rangle)\cdot(\mathcal{V }- \langle \mathcal{V}\rangle) \rangle
\end{align}
The system $S_1$ was adopted for simulation.

\subsection{The dielectric constant}

The dielectric constant can be estimated in a direct manner: 
\begin{align}
\varepsilon = 1 + \frac{1}{3L^3 k_BT} ( \langle \vert \mathbf{\mathit{M}}%
\vert^2\rangle - \vert\langle \mathbf{\mathit{M}}\rangle\vert^2 )
\end{align}
where 
\begin{align}
\mathbf{\mathit{M }} = \sum_i\boldsymbol{\mu}_i = \sum_{\alpha\in i} q_\alpha%
\mathbf{\mathit{r}}_\alpha
\end{align}
is the total dipole moment of the system. Since it is a fluctuation, and is
calculated by NVT simulations of $S_1$. 
% The results are listed in Tab.~\ref{tab:tmp2}. The dielectric constant calculated in system $\systemsb$ is reported in Table.~\ref{tab:thermo}.

\subsection{Kirkwood G-factor}

The Kirkwood G-factor is defined by~\cite{vanderSpoel2006origin} 
\begin{align}
G_k(r) = \Big\langle \frac 1N \sum_{i=1}^N \sum_{j, r_{ij} < r} \frac {%
\boldsymbol{\mu}_i \cdot \boldsymbol{\mu}_j}{\vert \boldsymbol{\mu}_i\vert
\cdot \vert\boldsymbol{\mu}_j\vert} \Big\rangle,
\end{align}
where $r_{ij}$ denotes the oxygen-oxygen distance between two water
molecules. The system for computing this quantity is $L_1$ in NVT ensemble.

\subsection{Diffusion constant}

The diffusion constant is calculated from the Einstein relation: 
\begin{align}
D = \lim_{t\rightarrow \infty}\frac {1}{6t} \langle \vert \mathbf{\mathit{r}}%
_i(t) - \mathbf{\mathit{r}}_i(0)\vert^2\rangle.
\end{align}
In practice, the mean-square-displacement $\langle\vert \mathbf{\mathit{r}}%
_i(t) - \mathbf{\mathit{r}}_i(0)\vert^2\rangle$ is calculated, then the
value is linearly fitted. It is calculated by NVT simulations of system $S_1$%
.

\subsection{Viscosity}

The viscosity is calculated from the Green-Kubo relation: 
\begin{align}
\eta = \frac{V}{k_BT}\int_0^\infty\langle P_{\alpha\beta}(0)
P_{\alpha\beta}(t)\rangle\,dt, \quad \alpha,\beta \in \{x, y, z\}
\end{align}
where $\alpha$ and $\beta$ denote the directions, and $P_{\alpha\beta}$
denotes the off-diagonal components of the pressure tensor. Since we used
the isotropic system setting, then it is obvious that $P_{xy}$, $P_{yz}$ and 
$P_{xz}$ are equivalent. Moreover, it has been pointed out that in addition $%
(P_{xx} - P_{yy})/2$ and $(P_{xx} - P_{yy})/2$ are two independent
components that are equivalent to the first three~\cite{alfe1998first}.
Therefore, the viscosity is calculated from the auto-correlation functions
of five independent components, and the statistical error is estimated from
the standard deviation of the viscosities calculated from the five
components.

We investigate the convergence of the integral of the auto-correlation
function w.r.t.~time $t$: 
\begin{equation*}
I_{\eta }(t)=\frac{V}{k_{B}T}\int_{0}^{t}\langle P_{\alpha \beta
}(0)P_{\alpha \beta }(t^{\prime })\rangle \,dt^{\prime }.
\end{equation*}%
The values listed in Table~\ref{tab:thermo} are computed with $I_{\eta }(10\,%
\text{ps})$. The quantity is calculated by NVT simulations of system $S_{1}$.

\newpage

%{\color{blue} Question1: I see that the statistical uncertainty is twice of
%the standard deviation of time series.\ Let $\{A(x(t),p(t))\}_{t\in \Lambda }
%${\ be time series of physical quantity }$A(x,p)${. Then I want to confirm
%how you defined the average and the standard deviation. (i) Are they for }$%
%\{A(x(t),p(t))\}_{t\in \Lambda }${\ itself? (ii) }Or are they {for }$\left\{ 
%\frac{1}{t}\int_{0}^{t}A(x(s),p(s))ds\right\} _{t\in \Lambda }${? }}
%
%{\color{blue}Maybe (ii) is correct. But "100~Bar" in the description "the
%fluctuation (measured by standard deviation) of the pressure is roughly
%100~Bar" should come from (i). If so, I proposed the presentation of the
%trajectory of pressures $\{P(x(t),p(t))\}_{t\in \Lambda }$ in SI.}
%
%{\color{blue}Question2 (this is just from my interest): When you did the ZMM
%NTP with P-R method, how was the cell shape? You obtained the good density,
%but how about the cell deformation? Was it keep almost the cubic structure?} 
%{\color{red} The box is kept cubic by resizing isotropically in x, y and z
%directions.}

% \bibliographystyle{plain}
\bibliographystyle{plain}
\bibliography{ref}
{}

\end{document}
