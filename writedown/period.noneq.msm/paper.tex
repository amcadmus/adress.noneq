\documentclass[aps, pre, preprint,unsortedaddress,a4paper,onecolumn]{revtex4}
% \documentclass[aps, pre, preprint,unsortedaddress,a4paper,twocolumn]{revtex4}
% \documentclass[acs, jctcce, a4paper,preprint,unsortedaddress,onecolumn]{revtex4-1}
% \documentclass[aps,pre,twocolumn,unsortedaddress]{revtex4-1}
% \documentclass[aps,jcp,groupedaddress,twocolumn,unsortedaddress]{revtex4}

\usepackage[fleqn]{amsmath}
\usepackage{amssymb}
\usepackage[dvips]{graphicx}
\usepackage{color}
\usepackage{tabularx}
\usepackage{algorithm}
\usepackage{algorithmic}

\makeatletter
\makeatother

\newcommand{\recheck}[1]{{\color{red} #1}}
\newcommand{\redc}[1]{{\color{red} #1}}
\newcommand{\bluec}[1]{{\color{red} #1}}
\newcommand{\greenc}[1]{{\color{green} #1}}
\newcommand{\vect}[1]{\textbf{\textit{#1}}}
\newcommand{\dd}[1]{\textsf{#1}}
\newcommand{\fwd}[0]{\textrm{fw}}
\newcommand{\bwd}[0]{\textrm{bw}}
\newcommand{\period}[0]{T_{\textrm{P}}}
\newcommand{\ml}[0]{\mathcal {L}}
\newcommand{\mo}[0]{\mathcal {O}}
\newcommand{\mbp}[0]{\mathbb {P}}
\newcommand{\mc}[0]{\mathcal {C}}
\newcommand{\dt}[0]{\Delta t}

\newcommand{\confaa}[0]{{\alpha_{\textrm{R}}}}
\newcommand{\confab}[0]{{\alpha_{\textrm{R}}'}}
\newcommand{\confba}[0]{{\textrm{C}7_{\textrm{eq}}}}
\newcommand{\confbb}[0]{{\textrm{C}5}}
\newcommand{\confc}[0]{{\alpha_{\textrm{L}}}}



\begin{document}

\title{Building Markov State Model for a Periodically Driven Non-Equilibrium System}
\author{Han Wang}
\email{han.wang@fu-berlin.de}
\affiliation{Zuse Institut Berlin, Germany}
\author{Christof Sch\"utte}
\email{Christof.Schuette@fu-berlin.de}
\affiliation{Institute for Mathematics, Freie Universit\"at Berlin, Germany}
\affiliation{Zuse Institut Berlin, Germany}
   
\begin{abstract}
\end{abstract}

\maketitle

\section{Introduction}
Non-equilibrium, especically periodically driven system. Interesting.

MSM tools for analyzing, provide profund understanding.

Current achievement of MSM in equilibrium cases.

Importance: first application of MSM in a non-equilibrium system.

\section{Discretization of the non-equilibrium molecular dynamics}

We consider  the following SDE form of the externally driven MD:
\begin{align}
  \label{eq:dis-1}
  d\vect x_t = \Big(-\nabla V(\vect x_t) + E(t) D(\vect x_t)\Big)dt + \sqrt{2\beta^{-1}} d\vect w_t, 
\end{align}
where the configurational space is denoted by $\Omega$, and the variable by
$\vect x$.   $V(\vect x)$ is the molecular interaction potential and
$E(t)D(x_t)$ the time-dependent external perturbation with the
$T$-periodic external field $E(t)$. $\beta$ is the inversed temperature,
i.e.~$\beta = 1/(k_B\mathcal T)$.
The propagation of probability
densities $\rho=\rho(\vect x,t)$ based on this kind of dynamics in the sense
of $\rho(\vect x,t)dx=\mathbb P[\vect x_t\in [\vect x,\vect x+d\vect x)]$ is governed by
Fokker-Planck equation:
\begin{align}
  \label{eq:dis-2}
  \frac{\partial \rho}{\partial t} = \ml^\dagger(t) \rho,
\end{align}
where $\ml^\dagger(t)$ is the adjoint of the generator
\begin{align}
  \label{eq:dis-3}
  \ml(t)=\beta^{-1}\Delta_{\vect x}+\Big(-\nabla V(\vect x) + E(t)D(\vect x)\Big)\cdot\nabla_{\vect x},
\end{align}
where $\Delta_{\vect x}$ denotes the Laplacian operator and $\nabla_{\vect x}$
the nabla-operator wrt to $\vect x$. 
The periodicity of the external driven indicates the periodicity of the generator,
i.e.~$\ml(t) = \ml(t+T)$.  Now
introduce a partition of the configurational space $\Omega$ into finite number of disjoint
sets $\{ \Omega_1, \cdots, \Omega_n\}$, which satisfy $\Omega = \cup_i \Omega_i$,
$\Omega_j\cap \Omega_j = \emptyset,\ \forall i\neq j$.
Now the original dynamics of the system can be discretized to a time-dependent
Markov jump process:
\begin{align}
  \label{eqn:tmp2}
  \frac{d\vect p(t)}{dt} = \vect M(t)\cdot \vect p(t)
\end{align}
where
\begin{align}
  \vect p(t) = (p_1(t), \cdots, p_n(t))^T, \quad \textrm{with}\ p_i(t) = \int_{\Omega_i} \rho(\vect x,t)d\vect x,
\end{align}
and the generator is approximated by finite difference scheme:
\begin{align}
  \label{eqn:tmp4}
  M_{ij}(t) \approx \frac{1}{\tau} \,[\, \mbp (\vect x_{t+\tau} \in \Omega_j \vert \vect x_{t} \in \Omega_i) - \delta_{ij} \,]
\end{align}
From Eqn.~\eqref{eq:dis-2} to Eqn.~\eqref{eqn:tmp2} involves several
approximations:
\begin{enumerate}
\item Eqn.~\eqref{eqn:tmp2} assumes Markovianity, which is not
  necessarily true for a reduce dynamics $\vect p(t)$, see e.g.~\cite{prinz2011markov}.
\item Numerical error from the finite difference introduced in
  Eqn.~\eqref{eqn:tmp4}. The lag-time $\tau$ should not be too large
  to control the finite difference error, and at the same time should
  not be too small, because otherwise the statistical error would be
  large, and the Markovianity would be disturbed.
\item Although the original dynamics defined by generator $\ml(t)$ is
  periodic, the reduced one defined by $\vect M(t)$ is not necessary
  periodic, i.e.~$\vect M(t) \neq \vect M(t+T)$ in general.  In the
  simulation of biomolecules, it is reasonable, in most cases to
  assume a time-scale separation in the system. For example, the
  conformational change of the molecules is much faster than the
  covalent bond vibrations. Therefore, if the discretion $\{ \Omega_1,
  \cdots, \Omega_n\}$ properly captures the slow dynamics of interest, in
  the sense that within the lag-time $\tau$, the system is fully
  relaxed to a local equilibrium in the set $\Omega_i$, then it is clear
  that the reduce system is periodic. It suggests that the lag-time
  should be right in the spectrum gap: it is shorter than the dominant
  implied timescale, and longer than the fast-relaxing time-scales.
\end{enumerate}


\subsection{Perodicity of the reduced dynamics: investigated by alanine dipeptide under oscillatory electric field}
\begin{figure}
  \centering
  \includegraphics[width=0.3\textwidth]{fig/confs/c-2.eps}
  \caption{A schematic plot of the alanine dipeptide molecule and the dihedral angles $\phi$ and $\psi$.}
  \label{fig:tmp1}
\end{figure}

At this stage we do have any analytical result regarding the quality
of approximating Eqn.~\eqref{eq:dis-2} by
Eqn.~\eqref{eqn:tmp2}.
It is therefore useful to investigate the discussed approximations
by a numerical example. The Markovianity of the reduce dynamics
was extensively discussed in e.g.~\cite{prinz2011markov}, so we 
will not touch this topic in the current paper. The choice of the lag-time
will be discussed later, and in this subsection we focus on the periodicity
of the reduced dynamics.
Throughout this paper, we take the alanine dipeptide system
under oscillatory EF, as an example, which was extensively studied by
Ref.~\cite{wang2014exploring}.
We choose the two dihedral angles $\phi$
and $\psi$ as slow variables (see Fig.~\ref{fig:tmp1}), and the discretization
is a uniform division of the $\phi$--$\psi$ plane. We denote the number of
division on each dihedral by $K$.

The alanine dipeptide was put into the local thermostating
environment, and was driven by a periodic electric field with period
$T = 10$~ps. The 20,000 branching trajectories started from 20,000
initial configurations that samples the equilibrium distribution.
\redc{Write a lot of details on non-equilibrium MD.}  The branching
trajectories were each 1000~ps long, and the system reaches
non-equilibrium steady state in roughly 300~ps. We compute the
generator from non-equilibrium trajectories in time interval $[t_1,
t_2]$. If the reduced dynamics were periodic, then generator $\vect
M(t)$ would not depends on the choice of the time interval. We compute
the reduced generator by two discretizations $K=2$ and $K=20$, and two
choices of time intervals $[0, 40]$ and $[320, 1000]$, and then
compare the results with the brute force result in
Fig.~\ref{fig:tmp2}. In the figure, we show the time-dependent
probability $\mathbb P\big(\phi\in[0,180), \psi\in [0,180)\big)$.
Using $K=2$ the dynamics depends on the time interval used for
calculating the generator: using time intervals $[320, 1000]$ the reduced
dynamics can only reproduces the brute force result in $[0, 40]$,
while using time intervals $[0, 40]$ the reduced dynamics can only
reproduces the brute force result after 300~ps.  This shows that the
reduced dynamics is lack of periodicity.  The reason is that the
discretization is too coarse so that the dynamics cannot be fully
equilibriated within the lag-time $\tau$ in each discretized set.  For
$K=20$, the reduced dynamics does not depend on the time interval of
calculating the generator, and is consistent with the brute force
non-equilibrium simulation within the error bar. Therefore, the $K=20$
is a reliable discretization for the alanine dipeptide system.


\begin{figure}
  \centering
  \includegraphics[width=0.4\textwidth]{fig/t010/discrete/fig-cg-prob.eps}  
  \caption{The time-dependent probability $\mathbb
    P\big(\phi\in[0,180), \psi\in [0,180)\big)$.  The brute force
    non-equilibrium MD simulation is compared with different
    discretization methods. The red shadow region indicates the
    statsitical uncertainty of the brute force simulation.}
  \label{fig:tmp2}
\end{figure}


\subsection{The choice of lag-time $\tau$}

As discussed before, the lag-time should be chosen a value that lies
in the spectrum gap of the original dynamics, so that it resolves the
slow dynamics, and at the same time relaxes the fast dynamics. In
practice, it is very difficult to estimate the lag-time in
priori. Therefore, we discretize the original non-equilibrium based on
different choices of $\tau$ (0.5, 1.0, 2.0 and 5.0~ps) and compare the
time dependent probability of $\mathbb P\big(\phi\in[0,180), \psi\in
[0,180)\big)$ estimated out of the reduced dynamics in
Fig.~\ref{fig:tmp3}.  All cases use idential discretization: $K=20$.
It is clear that when lag-time is close to the period (10~ps), the
reduced dynamics cannot resolve the probability change within a
period. However, the large lag-times are able to capture the
the overall log time behavior of the probability.
We observe no significant difference between $\tau=0.5$ and
$\tau=1.0$~ps, which means the reduced dynamics is not very sensitive
to the choice of $\tau$. 


\begin{figure}
  \centering
  \includegraphics[width=0.4\textwidth]{fig/t010/discrete/fig-cg-prob-tau.eps}  
  \caption{The time-dependent probability $\mathbb
    P\big(\phi\in[0,180), \psi\in [0,180)\big)$.  The brute force
    non-equilibrium MD simulation is compared with different
    discretization methods. The red shadow region indicates the
    statsitical uncertainty of the brute force simulation.}
  \label{fig:tmp3}
\end{figure}


% The definition of the conformations
% $\confaa$, $\confab$, $\confba$, $\confbb$ and $\confc$ are presented
% by Fig.~\ref{fig:tmp1}.  We are interested in the time-dependent
% probability of the conformations. In Fig.~\ref{fig:tmp2}, the brute
% force molecular dynamics simulation result
% (i.e.~Eqn.~\eqref{eq:dis-2}) is compared with different discretization
% methods, which give different generator in Eqn.~\eqref{eqn:tmp2}. For
% clarity, only the probability of conformation $\confc$ is plotted.  In
% the figure, ``5 sets'' means the $\phi$-$\psi$ space is discretized by
% 5 sets as shown in Fig.~\ref{fig:tmp1}, ``Uni. $M=$'' stands for the
% uniform discretization on $\phi$-$\psi$ space, and $M$ denotes the
% number of bins on both $\phi$ and $\psi$ directions.  Both $M=10$ and
% 20 methods present very good accuracy.  Please notice that for $M=10$,
% for example, the number of discretized sets is $n=M^2=100$.  The
% lag-time in all cases is chosen to be $\tau = 0.5$~ps. A test with
% $\tau = 1.0$~ps shows good consistency with $\tau = 0.5$~ps.
% The generator
% $\vect M(t)$ is assumed to be periodic, and is estimated from the
% trajectories from $t=300$ to 1000~ps.

% The phenomena presented by Fig.~\ref{fig:tmp2} is very similar to the
% equilibrium MSM: the quality of the MSM depends on how accurate the
% eigenfunctions are approximated by the discretization. The ``5 sets''
% method achieve better accuracy than the uniform $M=5$ method by using
% much less DOFs, because it captures the positive and negative signs of
% the eigenfunctions. With larger $M$ the discretization error of the
% eigenfunctions becomes smaller, so the accuracy of the MSM is improved.




\section{Floquet theory}

The reduced dynamics will not provide a lot further interesting
information until we introduce the MSM to analyze the reduced dynamics. The main
difficulty is that the reduced dynamics is time-inhomogeneous, however, this can
be overcome by the Floquet theory.
The Floquet theory states that for a periodic system, e.g.~Eq.~\eqref{eqn:tmp2}, the solution can be written as
\begin{align}
  \label{eqn:tmp5}
  \vect p(t)  = \vect Q(t) e^{t\vect B} \vect p(0)
\end{align}
where $\vect Q(t)$ is periodic and $\vect B$ is a constant matrix.  To
analyze the spectrum, it is not difficult to infer the implied
timescales, on which the system relaxes to the periodic steady states
defined by $\vect Q(t)$.
Assume that the fundamental solution to Eqn.~\ref{eqn:tmp2} is denoted by $\boldsymbol \Phi(t)$, then
the matrix $\vect B$ can be expressed by
\begin{align}
  e^{T\vect B} = \boldsymbol \Phi^{-1}(0)\cdot \boldsymbol \Phi(T)
\end{align}
It is fully justified by choosing $ \boldsymbol \Phi(0) = I$, then
$e^{T\vect B}$ is estimated by evolving the fundamental solutions by
one period.
Since it is obvious that $Q(kT) = Q(0) = I,\ k\in\mathbb N$, if we only consider the state of the system
at time $kT$, then from Eq.~\eqref{eqn:tmp5} we have
\begin{align}
  \label{eqn:tmp5a}
  \vect p(kT)  = \boldsymbol\Phi^k(T) \,\vect p(0),
\end{align}
which is a time-homogeneous Markov process with transition matrix $\boldsymbol\Phi(T)$.
Please notice that in general this Markov process is not reversible.
Calculating the eigenvalues of $\boldsymbol\Phi(T)$ reveals the
implied timescales on which the non-equilibrium dynamics converges to
the periodic steady state.
The Eq.~\eqref{eqn:tmp5a} allows one to use the available techniques for time-homogeneous
Markov jump process (especially for irreversible processes)
to analyze the time-inhomogeneous dynamics Eq.~\eqref{eqn:tmp2}.
However, it can only resolve the processes that are longer than the period $T$.
Therefore, rather than going directly from Eq.~\eqref{eq:dis-2} to Eq.~\eqref{eqn:tmp5a},
we would like to keep the intermediate step Eq.~\eqref{eqn:tmp2}.
In the following we compare the MSM with the brute force non-equilibrium simulation
by considering several properties that are of special interest.

% The largest non-trivial timescales is given in table
% Tab.~\ref{tab:tmp1}, which corresponds very well to the dynamics of the  $\confc$ probability.
% \begin{table}
%   \centering
%   \caption{The first four non-trivial timescales calculated from ${\vect B}$. The unit of time in the table is ps.
%     $\tau$ is the lag time used to calculate the generator~\eqref{eqn:tmp4}. $[T_0,T_1]$ is the time interval,
%     in which the generator is estimated from the MD trajectories. The 5th eigenvalue of uniform decomposition with $\tau=8.0$~ps is not real, so
%     the corresponding timescale is not calculated.
%   }
%   \begin{tabular*}{0.9\textwidth}{@{\extracolsep{\fill}}c ccc rrrr}\hline\hline
%     Method  &  $M$ & $\tau$ & $[T_0,T_1]$ &   $t_2$ &   $t_3$ &   $t_4$ &   $t_5$  \\\hline
%     5 sets  & --   & 0.5    & $[300,1000]$    &  72.4 & 17.8 &  2.1 & 2.8    \\
%     Uniform & $05$ & 0.5    & $[300,1000]$    &  53.6 & 17.4 &  9.4 & 2.8    \\
%     Uniform & $10$ & 0.5    & $[300,1000]$    & 107.2 & 27.0 & 16.4 & 3.6    \\
%     Uniform & $20$ & 0.5    & $[300,1000]$    & 118.6 & 30.4 & 20.3 & 4.3    \\
%     Uniform & $30$ & 0.5    & $[300,1000]$    & 120.6 & 31.1 & 20.9 & 4.4    \\\hline
%     Uniform & $20$ & 0.5    & $[0,40]$    & 156.5 & 34.4 & 20.7 & 4.4    \\
%     Uniform & $20$ & 0.5    & $[0,80]$    & 133.8 & 32.8 & 20.4 & 4.4    \\
%     Uniform & $20$ & 0.5    & $[0,120]$   & 127.5 & 32.0 & 20.3 & 4.3    \\
%     Uniform & $20$ & 0.5    & $[0,200]$   & 122.0 & 31.3 & 20.3 & 4.3    \\
%     Uniform & $20$ & 0.5    & $[0,320]$   & 120.4 & 31.0 & 20.3 & 4.3    \\\hline
%     Uniform & $20$ & 1.0    & $[300,1000]$    & 119.8 & 30.5 & 21.7 & 4.9   \\
%     Uniform & $20$ & 2.0    & $[300,1000]$    & 117.8 & 29.2 & 21.5 & 5.3   \\
%     Uniform & $20$ & 4.0    & $[300,1000]$    & 114.2 & 27.0 & 20.2 & 4.8   \\
%     Uniform & $20$ & 8.0    & $[300,1000]$    & 110.6 & 25.0 & 18.1 & --   \\
%     \hline\hline
%   \end{tabular*}
%   \label{tab:tmp1}
% \end{table}


% \section{Numerical investigation of the accuracy of the MSM}

% At this stage we do have any analytical result regarding the quality
% of approximating Eqn.~\eqref{eq:dis-2} by
% Eqn.~\eqref{eqn:tmp2}. However, it would be useful to do some
% numerical experiments. We have studied the alanine dipeptide system
% under oscillatory EF, which was studied by
% Ref.~\cite{wang2014exploring}. We use the two dihedral angles $\phi$
% and $\psi$ as slow variables. The discretization is the uniform division of
% the $\phi$--$\psi$ plan in to 2-dimentional bins. 
% The quality of the MSM~\eqref{eqn:tmp5a} can be checked by comparing the physical properties
% computed from the MSM and those computed by the brute force non-equilibrium molecular
% dynamics simulation from Ref.~\cite{wang2014exploring}.


\subsection{Steady state distribution and eigenfunctions}
We investigate the steady state properties distribution at time $kT$. To make it comparible to
the free energy in equilibrium case, we take the logrithm of the properties, i.e.
\begin{align}
  \label{eq:num-tmp1}
  F_{\textrm{st}}(\phi,\psi) = \lim_{k\rightarrow\infty} -k_B\mathcal T \log p(\phi,\psi,kT),
\end{align}
where $k_B$ is the Boltzmann constant and $\mathcal T$ is the temperature of the system.
The result is given in Fig.~\ref{fig:num-1}. A good consistency between the brute force
MD simulation and the MSM is observed.

\begin{figure}
  \centering  
  \includegraphics[width=0.4\textwidth]{fig/t010/cluster.marco.40/fig-dist.eps}
  % \includegraphics[width=0.23\textwidth]{fig/t010/cluster.marco.40/fig-dist-msm.eps}
  \includegraphics[width=0.4\textwidth]{fig/t010/cluster.marco.40/fig-floquet-vec-1.eps}
  \caption{The steady state distribution calculated by brute force non-equilibrium comparing with MSM.}
  \label{fig:num-1}
\end{figure}


% \begin{figure}
%   \centering  
%   \includegraphics[width=0.23\textwidth]{fig/t010/cluster.marco.40/fig-floquet-vec-2-simple.eps}
%   \includegraphics[width=0.23\textwidth]{fig/t010/cluster.marco.40/fig-floquet-vec-3-simple.eps}
%   \includegraphics[width=0.23\textwidth]{fig/t010/cluster.marco.40/fig-floquet-vec-4-simple.eps}
%   \caption{The first three non-trivial eigenfunctions cdalculated by the MSM.}
%   \label{fig:num-2}
% \end{figure}

\subsection{Coreset identification}

\begin{figure}
  \centering
  \includegraphics[width=0.4\textwidth]{fig/t040/cluster.marco/fig-cluster-2.eps}
  \caption{The coreset identification. Different colors indicate different coresets: $C_{R_\alpha}$ (yellow), $C_\beta$ (red) and $C_{L_\alpha}$ (green).
    The blue color means out of any coreset. (By Marco)}
  \label{fig:cluster}
\end{figure}

\redc{Write how to detact the core sets for irreversible Markovian process}.

The coresets (see Fig.~\ref{fig:cluster}) are denoted by $C_{R_\alpha}$ (yellow), $C_\beta$ (red) and $C_{L_\alpha}$ (green), which correspond to
right-handed alpha-helix, beta-sheet and left-handed alpha-helix, respectively.


\subsection{Committors}

The forward committor $q^\fwd_i(\phi,\psi)$ of a coreset $C_i,\ i\in\{R_\alpha, \beta, L_\alpha\}$ is defined as
the probability of visiting coreset $C_i$ next conditioned on being at configuration $(\phi,\psi)$.
The backward committor $q^\bwd_i(\phi,\psi)$ of a coreset $C_i,\ i\in\{R_\alpha, \beta, L_\alpha\}$ is defined as
the probability of last coming from $C_i$  conditioned on being at configuration $(\phi,\psi)$.
For  reversible Markov processes, the forward and backward committors are identical, however, it is in general not the case
for irreversible processes. Please see Fig.~\ref{fig:num-3}--\ref{fig:num-5} for the committors.

\begin{figure}
  \centering
  \includegraphics[width=0.23\textwidth]{fig/t010/cluster.marco.40/fig-commitor-fw-1.eps}
  \includegraphics[width=0.23\textwidth]{fig/t010/cluster.marco.40/fig-commitor-bw-1.eps}
  \includegraphics[width=0.23\textwidth]{fig/t010/cluster.marco.40/fig-commitor-diff-1.eps}\\
  \includegraphics[width=0.23\textwidth]{fig/t010/cluster.marco.40/fig-commitor-fw-msm-1.eps}
  \includegraphics[width=0.23\textwidth]{fig/t010/cluster.marco.40/fig-commitor-bw-msm-1.eps}
  \includegraphics[width=0.23\textwidth]{fig/t010/cluster.marco.40/fig-commitor-diff-msm-1.eps}
  \caption{The forward $q^\fwd_{L_\alpha}$ and backward committors $q^\bwd_{L_\alpha}$ computed by MSM (second row) is compared with those computed by brute force non-equilibrium simulations (first row).}
  \label{fig:num-3}
\end{figure}

\begin{figure}
  \centering
  \includegraphics[width=0.23\textwidth]{fig/t010/cluster.marco.40/fig-commitor-fw-2.eps}
  \includegraphics[width=0.23\textwidth]{fig/t010/cluster.marco.40/fig-commitor-bw-2.eps}
  \includegraphics[width=0.23\textwidth]{fig/t010/cluster.marco.40/fig-commitor-diff-2.eps}\\
  \includegraphics[width=0.23\textwidth]{fig/t010/cluster.marco.40/fig-commitor-fw-msm-2.eps}
  \includegraphics[width=0.23\textwidth]{fig/t010/cluster.marco.40/fig-commitor-bw-msm-2.eps}
  \includegraphics[width=0.23\textwidth]{fig/t010/cluster.marco.40/fig-commitor-diff-msm-2.eps}
  \caption{The forward $q^\fwd_{R_\alpha}$ and backward $q^\fwd_{R_\alpha}$ committors computed by MSM (second row) is compared with those computed by brute force non-equilibrium simulations (first row).}  
  \label{fig:num-4}
\end{figure}

\begin{figure}
  \centering
  \includegraphics[width=0.23\textwidth]{fig/t010/cluster.marco.40/fig-commitor-fw-3.eps}
  \includegraphics[width=0.23\textwidth]{fig/t010/cluster.marco.40/fig-commitor-bw-3.eps}
  \includegraphics[width=0.23\textwidth]{fig/t010/cluster.marco.40/fig-commitor-diff-3.eps}\\
  \includegraphics[width=0.23\textwidth]{fig/t010/cluster.marco.40/fig-commitor-fw-msm-3.eps}
  \includegraphics[width=0.23\textwidth]{fig/t010/cluster.marco.40/fig-commitor-bw-msm-3.eps}
  \includegraphics[width=0.23\textwidth]{fig/t010/cluster.marco.40/fig-commitor-diff-msm-3.eps}
  \caption{The forward $q^\fwd_{\beta}$ and backward $q^\fwd_{\beta}$ committors computed by MSM (second row) is compared with those computed by brute force non-equilibrium simulations (first row).}  
  \label{fig:num-5}
\end{figure}

\subsection{First mean hitting time}

The first mean hitting time is presented in Fig.~\ref{fig:num-6}.

\begin{figure}
  \centering
  \includegraphics[width=0.23\textwidth]{fig/t010/cluster.marco.40.fht/fig-fht-1.eps}
  \includegraphics[width=0.23\textwidth]{fig/t010/cluster.marco.40.fht/fig-fht-2.eps}
  \includegraphics[width=0.23\textwidth]{fig/t010/cluster.marco.40.fht/fig-fht-3.eps}\\
  \includegraphics[width=0.23\textwidth]{fig/t010/cluster.marco.40.fht/fig-fht-msm-1.eps}
  \includegraphics[width=0.23\textwidth]{fig/t010/cluster.marco.40.fht/fig-fht-msm-2.eps}
  \includegraphics[width=0.23\textwidth]{fig/t010/cluster.marco.40.fht/fig-fht-msm-3.eps}
  \caption{The first mean htting time comparsion between the brute force simulation (first row) and the MSM (second row).
  From left to right are first mean hitting time to coresets $C_{L_\alpha}$, $C_{R_\alpha}$ and $C_{\beta}$, respectively}  
  \label{fig:num-6}
\end{figure}



% \section{Discussions}
% \subsection{The periodicity of the reduced dynamics}


% As discussed before, if the reduced dynamics captures the slow
% dynamics of interest, then the reduced dynamics is also periodic,
% which implies that the generator $M(t)$ can be estimated from any time
% interval by Eq.~\eqref{eqn:tmp4}. In practice, the reduced dynamics
% may not be perfectly periodic. The generator calculated from
% different time intervals can be used to test the periodicity of the
% reduced dynamics. The results are shown in Tab.~\ref{tab:tmp1} and
% Fig.~\ref{fig:tmp2}.  The dependency with respect to the time
% interval, in which the generator is calculated, indicates the reduced
% dynamics is not perfectly periodic. The reduced dynamics from $[0,40]$
% reproduces the best probability in interval $[0,40]$, and worst in the
% steady state. Moreover, the leading non-trivial timescale is also off
% (see Tab.~\ref{tab:tmp1}), while the accuracy of the second and third
% non-trivial timescales are better, because they are comparable to the
% period, 40~ps, in which the dynamics is relaxed. The reduced dynamics
% from $[300,1000]$ reproduces the best steady state probability, while
% worst in reproducing the initial relaxation process. The reduced
% dynamics from $[0,320]$ is indistinguishable from the one from
% $[300,1000]$, which implies that in practice, one only have to
% simulation relatively short non-equilibrium trajectories (that are
% almost relaxed), and compute the generator from it to build the reduced dynamics. \redc{This saves a great
% amount of computational resources comparing with calculating the
% reduced dynamics only from steady state trajectories.}


% \begin{figure}
%   \centering
%   \includegraphics[width=0.5\textwidth]{fig/t040/fig-cg-prob-range.eps}  
%   \caption{The time-dependent probability of conformation
%     $\confc$. The brute force MD simulation is compared with reduced
%     dynamics, the transition matrix of which are calculated from
%     different ranges. The red shadow region is the statistical uncertainty of the brute force MD simulation.
%     The inserts show the zoom-ined probability at the relaxation stage $[0,120 ]~\textrm{ps}$ and at the steady state $[800,880 ]~\textrm{ps}$.}
%   \label{fig:tmp2}
% \end{figure}

% \subsection{The choice of lag time $\tau$}
% Table~\ref{tab:tmp1} shows a clear trend of worse accuracy with
% increased $\tau$, however, the timescales are not very sensitive to
% the choice of lag time $\tau$.



% \section{To do}

% \begin{itemize}
% \item Theoretical analysis of the accuracy of the discretization, a
% generalization of the equilibrium case. the numerical results indicate
% a very close link. The perodicity is very nice, if the generator is
% ``continuous'', then it should be uniformly continuous, then it should
% be possible to develop estimates that are similar to the equilibrium
% case.
% \item More tests on the effect of choosing different lag-time $\tau$ (0.5 and 1.0~ps are done).
% \item More theoretical consideration on the Floquet theory, what is the
% explict form of matrix $\vect B$, and does it has any nice properties,
% such as self-joint.
% \end{itemize}


\section{Approximation quality of the Markov state model}

The whole Floquet theory can be applied to the original dynamics, and we denote that the resulting transition matrix by $\psi(T)$, then
the time-dependent probability density can be written as
\begin{align*}
  \rho(\vect x, kT) = 
\end{align*}

In general, the original dynamics of the molecular system under perodical external driven
is not reversible, see the discussion in Sec.~\ref{sec:revs}. However, we can assume the, 



\section{Reversibility of the original dynamics}
\label{sec:revs}

Now we assume the overdamped dynamics
\begin{align}\label{eqn:tmp7}
  dx_t = -\nabla_x V(x_t,t) dt + \sigma dw_t
\end{align}
where $V$ is the time-dependent and $T$-periodic potential. For simplicity we denote the force by $F(x_t,t) = -\nabla_x V(x_t, t)$. $dw_t$ is
the standard Wiener process.
According to Girsanov, we have
\begin{align}
  \label{eq:tmp8}
  \frac{dp[x_t]}{dw[x_t]}  =
  \exp \bigg\{
  \frac 1{\sigma^2}\int_0^T F(x_t,t) dx_t -
  \frac1{2\sigma^2}\int_0^T F^2(x_t,t) dt
  \bigg\}
\end{align}
where $dp$ is the probability measure of trajectory $x_t$, while $dw$ is the
probability measure of the standard Wiener process $dx_t = \sigma dw_t$.
Assuming a discretization of the
stochastic process at time $0 < t_1 < t_2 < \cdots < t_N = T$, where
$t_i = iT / N$. We denote $x_i = x_{t_i}$, and $w_i = w_{t_i}$, then we have,
in the sense of Ito,
\begin{align}\label{eq:tmp9}
  \frac{dp[x_t]}{dw[x_t]}  \approx
  \exp\bigg\{\frac1{\sigma^2}\sum_{i=0}^{N-1} F(x_{i},t_{i})(x_{i+1} - x_i) -\frac1{2\sigma^2}\sum_{i=0}^{N-1}F^2(x_i,t_i)\dt\bigg\} 
\end{align}
Now, consider a conjugate trajectory $x^\dagger_t = x_{T-t}$ that starts at $x_T$, ends at $x_0$. The conjugate dynamics is driven by  $F^\dagger(x^\dagger_t,t) = F(x^\dagger_t, T-t)$.
Writting the Girsanov for the conjugate dynamics
\begin{align}\label{eq:dagger-0}
  \frac{dp^\dagger[x^\dagger_t]}{dw[x^\dagger_t]}  
  \approx\,&
  \exp\bigg\{
  \frac1{\sigma^2}\sum_{i=0}^{N-1} F^\dagger(x^\dagger_{i},t_{i})(x^\dagger_{i+1} - x^\dagger_i) -
  \frac1{2\sigma^2}\sum_{i=0}^{N-1}[F^\dagger(x^\dagger_i,t_i)]^2\dt\bigg\} \\ \nonumber
  =\,&
  \exp\bigg\{
  \frac1{\sigma^2}\sum_{i=0}^{N-1} F(x^\dagger_{i},T - t_{i})(x^\dagger_{i+1} - x^\dagger_i) -
  \frac1{2\sigma^2}\sum_{i=0}^{N-1}[F(x^\dagger_i, T-t_i)]^2\dt\bigg\} \\\nonumber
  =\,&
  \exp\bigg\{
  \frac1{\sigma^2}\sum_{i=0}^{N-1} F(x_{N-i},t_{N-i})(x_{N-i-1} - x_{N-i}) -
  \frac1{2\sigma^2}\sum_{i=0}^{N-1}[F(x_{N-i},t_{N-i})]^2\dt\bigg\} \\
  = \,&
  \exp\bigg\{
  \frac1{\sigma^2}\sum_{i=N}^{1} F(x_{i},t_{i})(x_{i-1} - x_i) -
  \frac1{2\sigma^2}\sum_{i=N}^{1}F^2(x_i,t_i)\dt\bigg\}
\end{align}
% Therefore,
% \begin{align}\nonumber
%   \frac{dp[x^\dagger_t]}{dw[x^\dagger_t]}  =
%   \,&
%   \frac{dp^\dagger[x^\dagger_t]}{dw[x^\dagger_t]} \\\nonumber
%   \approx\,&
%   \exp\bigg\{
%   \frac1{\sigma^2}\sum_{i=0}^{N-1} F^\dagger(x^\dagger_{i},t_{i})(x^\dagger_{i+1} - x^\dagger_i) -
%   \frac1{2\sigma^2}\sum_{i=0}^{N-1}[F^\dagger(x^\dagger_i,t_i)]^2\dt\bigg\} \\\nonumber
% \end{align}
Since it is obvious that $dw[x^\dagger_t] / dw[x_t] = 1$,
\begin{align}
  \label{eq:tmp10}
  \frac{dp^\dagger[x^\dagger_t]}{dw[x_t]}
  \approx \,&
  \exp\bigg\{
  \frac1{\sigma^2}\sum_{i=1}^{N} F(x_{i},t_{i})(x_{i-1} - x_i) -
  \frac1{2\sigma^2}\sum_{i=1}^{N}F^2(x_i,t_i)\dt\bigg\}
\end{align}
The difference between the single trajectory probabilities is
\begin{align}\label{eqn:tmp12}
  \frac{  dp^\dagger[x^\dagger_t] }{ dp[x_t]}
  \approx&\,
  \exp\bigg\{
  -\frac1{\sigma^2}\sum_{i=1}^{N-1}
  \bigg[
  F(x_i,t_i)(x_{i+1} - x_{i}) + F(x_i,t_i)(x_{i} - x_{i-1})
  \bigg]
  \bigg\}
\end{align}
Assuming the smoothness of the external perturbation, consider the differentiation:
\begin{align}\nonumber
  F(x_{i},t_{i}) - F(x_{i-1},t_{i-1}) =
  &\,
  F(x_{i},t_{i}) - F(x_{i-1},t_{i}) + F(x_{i-1},t_{i}) -  F(x_{i-1},t_{i-1})\\\nonumber
  =&\,
  \nabla_x F(x_{i-1},t_{i})(x_i - x_{i-1}) + \mo(\dt) \\\label{eqn:tmp13}
  =&\,
  \nabla_x F(x_{i-1},t_{i-1})(x_i - x_{i-1}) + \mo(\dt)
\end{align}
The second order expansion w.r.t.~$x_i - x_{i-1}$ is of order $\dt$, so it is absorbed into $ \mo(\dt)$.
Then the \eqref{eqn:tmp12} becomes
\begin{align}\label{eqn:tmp14}
  \frac{  dp^\dagger[x^\dagger_t] }{ dp[x_t]}
  \approx&\,
  \exp\bigg\{
 -\frac2{\sigma^2}\sum_{i=0}^{N-1} F(x_i,t_i)(x_{i+1} - x_{i}) 
 -\frac1{\sigma^2}\sum_{i=0}^{N-1}\nabla_xF(x_i,t_i)(x_{i+1} - x_{i})^2
 \bigg\}
\end{align}
Using the identity
$  dt = (dw_t)^2 = {\sigma^{-2}} dx_t^2$,
Eq.~\eqref{eqn:tmp14} is written in the integral form
\begin{align}\label{eqn:tmp14-0}
  \frac{  dp^\dagger[x^\dagger_t] }{ dp[x_t]}
  \approx&\,
  \exp\bigg\{
 -\frac2{\sigma^2}\int_0^T F(x_t,t) dx_t
 -\int_0^T\nabla_xF(x_t,t)dt
 \bigg\}  
\end{align}
One would not have the second integral on the exponent if the first integral of the exponent were defined in the sense of Stratonovich.

We notice that
\begin{align}\nonumber
  dV(x, t) = &\, \frac{\partial V}{\partial x} dx + \frac{\partial V}{\partial t} dt\\\nonumber
  =&\,
  \frac12 \sigma^2 \nabla^2_x V dt +  \nabla V dx_t + \frac{\partial V}{\partial t} dt \\
  =&\,
  -\frac12 \sigma^2 \nabla_x F dt -  F dx_t + \frac{\partial V}{\partial t} dt
\end{align}
Eq.~\eqref{eqn:tmp14-0} becomes
\begin{align}
  \frac{  dp^\dagger[x^\dagger_t] }{ dp[x_t]}
  =&\,
  \exp\bigg\{
  \frac2{\sigma^2}\bigg[
  V(x_T,T) - V(x_0,t_0) - \int_0^T\partial_tV(x_t,t)dt
  \bigg]
  \bigg\}
\end{align}
According to the  Einstein relation, the temperature $\mathcal T = \sigma^2/2$, we denote $\beta = 1/{\mathcal T} = 2/\sigma^2$.
Take the limit of infinite small time interval, and notice the equilibrium
invariant probability density with respect to potential $V(x,0)$ satisfies $\mu(x) \propto e^{-\beta V(x,0)}$,
\begin{align}
  \frac{  dp^\dagger[x^\dagger_t] }{ dp[x_t]}
  =&\,
  \frac{\mu(x_0)}{\mu(x_T)}\times
  \exp\bigg\{
  - \beta\int_0^T\partial_tV(x_t,t)dt
  \bigg\}
\end{align}

\subsection{Irreversibility of the perodical symmetrical dynamics}

In Eq.~\eqref{eq:dagger-0}, we assume the periodicity of the perturbation $F(x,t) = F(x,t+T)$, and the symmetry of the external perturbation, i.e.~$F(x, -t) = F(x, t)$, we have
\begin{align}
  \frac{dp^\dagger[x^\dagger_t]}{dw[x^\dagger_t]}  
  \approx\,&
  \exp\bigg\{
  \frac1{\sigma^2}\sum_{i=0}^{N-1} F(x^\dagger_{i},t_{i})(x^\dagger_{i+1} - x^\dagger_i) -
  \frac1{2\sigma^2}\sum_{i=0}^{N-1}[F(x^\dagger_i, t_i)]^2\dt\bigg\}  
\end{align}
By changing notation $x^\dagger$ back to $x$, and comparing with~\eqref{eq:tmp9}, the reversed dynamics is subject to the Eq.~\eqref{eqn:tmp7},
i.e.~$dp^\dagger = dp$.
Therefore
\begin{align}\nonumber
  p(x_0,T\vert x_T,0)
  =&\,\int_{\mc\{x_T,0;x_0,T\}}
  dp[x^\dagger_t] \\\nonumber  
  =&\,
  \int_{\mc\{x_0,0;x_T,T\}}
  \frac{  dp[x^\dagger_t] }{ dp[x_t]} \cdot dp[x_t] \\\nonumber
  =&\,
  % \lim_{N\rightarrow\infty} 
  % \frac{1}{(2\pi\sigma^2\dt)^{(N-1)/2}} \int dx_{N-1}\cdots\int dx_{1}
  \int_{\mc\{x_0,0;x_T,T\}}
  \frac{  dp^\dagger[x^\dagger_t] }{ dp[x_t]} \cdot dp[x_t] \\\label{eqn:tmp18}
  = &\,
  \frac{\mu(x_0)}{\mu(x_T)}
  \int_{\mc\{x_0,0;x_T,T\}}
  \exp\bigg\{
  -\beta\int_0^T \partial_t V(x_t,t)dt 
  \bigg\} \cdot dp[x_t]
\end{align}
where $\mc\{x_0,0;x_T,T\}$ denotes all continuous trajectories starting at $x_0$ and ending at $x_T$.
If $\partial_t V = 0$, i.e.~equilibrium, we have
\begin{align}
  p(x_0,T\vert x_T,0)e^{-\beta V(x_T,T)} =  p(x_T,T\vert x_0,0) e^{-\beta V(x_0,0)},
\end{align}
which proves the reversibility of the equilibrium dynamics.
The term
\begin{align}
  W[x_t] = \int_0^T \partial_t V(x_t,t)dt
\end{align}
is the non-equilibrium work associated to all possible
the dynamics $x_t$ starting at $x_0$ and ending at $x_T$ (see e.g.~Ref.~\cite{seifert2012stochastic}).
Therefore Eq.~\eqref{eqn:tmp18} is the detailed Jarzynski relation.
% Now the problem becomes if we can write a nice form (e.g.~the difference of a state function measured at $x_T$ and $x_0$) for the non-equilibrium work.
Noticing that
\begin{align}
  \label{eq:tmp21}
  p(x_T,T\vert x_0,0) = \int_{\mc\{x_0,0;x_T,T\}}dp[x_t],
\end{align}
From Eq.~\eqref{eqn:tmp18} we have
\begin{align}
  \label{eq:tmp22}
  \frac{p(x_0,T\vert x_T,0)}{  p(x_T,T\vert x_0,0)  }
  =
  \frac{\mu(x_0)}{\mu(x_T)}
  \mathbb E_{x_0\rightarrow x_T} [e^{-\beta W}]
\end{align}
% Maybe we want a more symmetric form.
% The expectation value of the reversed dynamics reads,
% \begin{align}\nonumber
%   \mathbb E_{x^\dagger_0\rightarrow x^\dagger_T} [e^{-\beta W^\dagger}]
%   =\,&
%   \int_{\mc\{x^\dagger_0,0;x^\dagger_T,T\}}
%   \exp\bigg\{
%   -\beta\int_0^T \partial_t V(x^\dagger_t,t)dt 
%   \bigg\} \cdot dp[x^\dagger_t]   \\\nonumber
%   =\,&
%   \int_{\mc\{x_0,0;x_T,T\}}
%   \exp\bigg\{
%   -\beta\int_0^T \partial_t V(x_{T-t},t)dt 
%   -\beta\int_0^T \partial_t V(x_{t},t)dt 
%   \bigg\}
%   \cdot dp[x_t]   \\  \nonumber
%   =\,&
%   \int_{\mc\{x_T,0;x_0,T\}}
%   \exp\bigg\{
%   \beta\int_0^T \partial_t V(x_t,t)dt 
%   \bigg\} \cdot dp[x_t]   \\
%   \label{eq:tmp23}
%   = \,&
%   \mathbb E_{x_0\rightarrow x_T} [e^{- 2\beta W}]  
% \end{align}
% Noticing that \eqref{eq:tmp22}  is true for the reversed dynamics, we have
% \begin{align}
%   \label{eq:24}
%   \frac{p(x_T,T\vert x_0,0)}{  p(x_0,T\vert x_T,0)  }
%   =
%   \frac{\mu(x_T)}{\mu(x_0)}
%   \mathbb E_{x_T\rightarrow x_0} [e^{\beta W}]  
% \end{align}


\bibliography{ref}{}
\bibliographystyle{unsrt}



\end{document}
