\documentclass[aps, pre, preprint,unsortedaddress,a4paper,onecolumn]{revtex4}
% \documentclass[aps, pre, preprint,unsortedaddress,a4paper,twocolumn]{revtex4}
% \documentclass[acs, jctcce, a4paper,preprint,unsortedaddress,onecolumn]{revtex4-1}
% \documentclass[aps,pre,twocolumn,unsortedaddress]{revtex4-1}
% \documentclass[aps,jcp,groupedaddress,twocolumn,unsortedaddress]{revtex4}

\usepackage[fleqn]{amsmath}
\usepackage{amssymb}
\usepackage[dvips]{graphicx}
\usepackage{color}
\usepackage{tabularx}
\usepackage{algorithm}
\usepackage{algorithmic}

\makeatletter
\makeatother

\newcommand{\recheck}[1]{{\color{red} #1}}
\newcommand{\redc}[1]{{\color{red} #1}}
\newcommand{\bluec}[1]{{\color{red} #1}}
\newcommand{\greenc}[1]{{\color{green} #1}}
% \newcommand{\vect}[1]{\textbf{\textit{#1}}}
\newcommand{\vect}[1]{#1}
\newcommand{\dd}[1]{\textsf{#1}}
\newcommand{\fwd}[0]{\textrm{fw}}
\newcommand{\bwd}[0]{\textrm{bw}}
\newcommand{\period}[0]{T_{\textrm{P}}}
\newcommand{\ml}[0]{\mathcal {L}}
\newcommand{\mo}[0]{\mathcal {O}}
\newcommand{\mbp}[0]{\mathbb {P}}
\newcommand{\mc}[0]{\mathcal {C}}
\newcommand{\dt}[0]{\Delta t}
\newcommand{\id}{\mathrm{Id}}
% \newcommand{\myphi}{\boldsymbol\Phi}
% \newcommand{\mymu}{\boldsymbol\mu}
\newcommand{\myphi}{\Phi}
\newcommand{\mymu}{\mu}
\newcommand{\prob}{\textrm{P}}

\newcommand{\confaa}[0]{{\alpha_{\textrm{R}}}}
\newcommand{\confab}[0]{{\alpha_{\textrm{R}}'}}
\newcommand{\confba}[0]{{\textrm{C}7_{\textrm{eq}}}}
\newcommand{\confbb}[0]{{\textrm{C}5}}
\newcommand{\confc}[0]{{\alpha_{\textrm{L}}}}



\begin{document}

\title{Building Markov State Model for a Periodically Driven Non-Equilibrium System}
\author{Han Wang}
\email{han.wang@fu-berlin.de}
\affiliation{Zuse Institut Berlin, Germany}
\author{Christof Sch\"utte}
\email{Christof.Schuette@fu-berlin.de}
\affiliation{Institute for Mathematics, Freie Universit\"at Berlin, Germany}
\affiliation{Zuse Institut Berlin, Germany}
   
\begin{abstract}
\end{abstract}

\maketitle

\section{Introduction}
Non-equilibrium, especially periodically driven system. Interesting.

MSM tools for analyzing, provide profound understanding.

Current achievement of MSM in equilibrium cases.

Importance: first application of MSM in a non-equilibrium system.

\section{Discretization of the non-equilibrium molecular dynamics}
\label{sec:disc}

We consider  the following SDE form of the externally driven MD:
\begin{align}
  \label{eq:disc-1}
  d\vect x_t = \Big(-\nabla V(\vect x_t) + E(t) D(\vect x_t)\Big)dt + \sqrt{2\beta^{-1}} d\vect w_t, 
\end{align}
where the configurational space is denoted by $\Omega$, and the variable by
$\vect x$.   $V(\vect x)$ is the molecular interaction potential and
$E(t)D(x_t)$ the time-dependent external perturbation with the
$T$-periodic external field $E(t)$. $\beta$ is the inverse temperature,
i.e.~$\beta = 1/(k_B\mathcal T)$.
The propagation of probability
densities $\rho=\rho(\vect x,t)$ based on this kind of dynamics in the sense
of $\rho(\vect x,t)dx=\prob[\vect x_t\in [\vect x,\vect x+d\vect x)]$ is governed by
Fokker-Planck equation:
\begin{align}
  \label{eq:disc-fp}
  \frac{\partial \rho}{\partial t} = \ml^\dagger(t) \rho,
\end{align}
where $\ml^\dagger(t)$ is the adjoint of the generator
\begin{align}
  \label{eq:disc-3}
  \ml(t)=\beta^{-1}\Delta_{\vect x}+\Big(-\nabla V(\vect x) + E(t)D(\vect x)\Big)\cdot\nabla_{\vect x},
\end{align}
where $\Delta_{\vect x}$ denotes the Laplacian operator and $\nabla_{\vect x}$
the nabla-operator wrt to $\vect x$. 
The periodicity of the external driven indicates the periodicity of the generator,
i.e.~$\ml(t) = \ml(t+T)$.  Now
introduce a partition of the configurational space $\Omega$ into finite number of disjoint
sets $\{ \Omega_1, \cdots, \Omega_n\}$, which satisfy $\Omega = \cup_i \Omega_i$,
$\Omega_j\cap \Omega_j = \emptyset,\ \forall i\neq j$.
Following Ref.~\cite{latorre2011structure}, the Fokker-Planck Eq.~\ref{eq:disc-fp}
is discretized, which results in a time-inhomogeneous Markov jump process in state
space $S = \{1, \cdots, n\}$ with time-dependent rate
matrix $\vect L(t) \in \mathbb R^{n\times n}$ satisfying
\begin{align}\label{eq:disc-4}
\sum\limits_{j=1}^n L_{ij}(t) & =  0\\ \label{eq:disc-5}
L_{ij}(t) & \ge  0, \quad i\not= j\\
L_{ij}(t) & =  L_{ij}(t+T)
\end{align}
for all real time $t\geq 0$.
{Moreover,
the rate matrix $L$ has the form $\vect L(t)=\vect L_0+E(t)\vect L_1$
where $E(t)$ is periodic with period $T>0$.}
In analogy to \eqref{eq:disc-fp}, the Markov jump process generated by
$\vect L(t)$ transports probability densities according to the associated Master equation
\begin{align}
  \label{eq:disc-master}
  \frac{d\vect p(t)}{dt} = \vect L^T(t)\cdot \vect p(t)
\end{align}
where $\vect L(t)^T$ denotes the matrix transpose of $\vect L(t)$ and
$p_i(t)$, for example, the probability to be in state $i$ at time $t$.
As usual the properties (\ref{eq:disc-4}) and (\ref{eq:disc-5}) of
$\vect L(t)$ guarantee that the total probability mass is conserved,
i.e., if $p_i(0)\ge 0$ componentwise, then $p_i(t)\ge 0$ and $\sum_i
p_i(t) = \sum_ip_i(0)$.
% The solution of the master equation need no be
% periodic.
It can be formally written
\begin{align}  \label{eq:disc-8}
\vect p(t)=\myphi(t)\vect p(0)
\end{align}
by using the
associated propagator matrix $\myphi(t)\in\mathbb R^{n\times n}$ that solves
\begin{align}
  \label{eq:disc-master-phi}
  \frac{d}{dt}\myphi(t) = \vect L^T(t)\myphi(t), \quad \myphi(0) = \id
\end{align}
Since the last equation can be considered column-wise, the propagator matrix inherits column-wise conservation properties:
$\Phi_{ij}(t) \ge  0$
and $\sum\limits_{i=1}^n \Phi_{ij}(t)  =  1$,
that is, $\myphi^T(t)$ is a
stochastic matrix satisfying $\myphi^T(t)\vect e=\vect e$
with $\vect e=(1,\ldots,1)^T\in \mathbb R^n$.
Regarding these considerations, we find
\begin{align}
\label{eq:disc-10}  
\myphi_{ij}(t)=\prob\left(\vect X_t=i\mid \vect X_0=j \right),
\end{align}
where $\vect X_t$ denotes the Markov process generated by $\vect L(t)$.
 The
discretization sets that we used to go from $\vect x_t$ and $\ml(t)$ to $\vect X_t$
and $\vect L(t)$, respectively, can be assumed to provide an arbitrarily fine
partition of the original state space; then the transport properties
of $\vect L(t)$ are almost perfect approximations of the transport properties
of $\ml(t)$, in particular $p_i(t)=\prob(x_t\in \Omega_i)$.


\section{Floquet theory}
\label{sec:floquet}

As an effect of the periodicity of $\vect L(t)$ the propagator $\myphi(t+T)$
satisfies
\begin{equation}\label{compo-1}
\myphi(t+T)=\myphi(t)\myphi(T),
\end{equation}
for all $t\ge 0$. This can be seen by considering $\vect Y(t)=\myphi(t+T)$. $\vect Y$ satisfies
\[
\frac{d\ }{dt}\vect Y(t)=\vect L(t+T)^T \vect Y(t)=\vect L(t)^T\vect Y(t),\quad \vect Y(0)=\myphi(T).
\]
When we consider this identity column-wise and use the propagator property of $\myphi(t)$ we get $\myphi(t+T)=\vect Y(t)=\myphi(t)\myphi(T)$. As a consequence of (\ref{compo-1}) we get for all integers $m=0,1,2,\ldots$ that 
\begin{equation}\label{compo-2}
\myphi(t+mT)=\myphi(t)\myphi^m(T).
\end{equation}
In combination with Eq.~\eqref{eq:disc-8}, we therefore
know the solution $\vect p(t)$ of the Master equation for all $t\ge 0$,
if we can compute $\myphi(t)$ for $t\in (0,T]$.  
In particular we get the long-term evolution of the propagator:
\begin{align}
\label{eq:floq-13}  
\myphi(mT)=\myphi^m(T),
\end{align}
and for the probability at integral period we have
\begin{align}
  \label{eq:floq-dynamics}
  p(mT) =  \myphi(mT)\, p(0) = \myphi^m(T)\, p(0).
\end{align}

Since $\myphi(T)$ is a stochastic matrix, the spectrum $\sigma(\myphi(T))$
of the matrix $\myphi(T)$ is contained in the circle in
the complex plane, i.e., each eigenvalue $\lambda\in \sigma(\myphi(T))$
satisfies $|\lambda|\le 1$. Furthermore $1\in\sigma(\myphi(T))$ is an
eigenvalue with left eigenvector $\vect e$ and a right eigenvector $\mymu$
satisfying
$\myphi(T)\mymu=\mymu$.
From now on, we assume $\myphi(T)$ to be irreducible and aperiodic such that the Perron Frobenius theorem the eigenvalue $\lambda=1$ is  non-negative componentwise, and unique (up to normalization $\sum_j\mu_j=1$). In this case $\mymu$ is the stationary measure in the sense that
\begin{align}
\label{eq:floq-14}  
\myphi(mT) \mymu = \mymu,\quad m=0,1,2,\ldots,
\end{align}
and (more precisely) the asymptotic evolution of an initial probability distribution $\vect p(t=0)$ by the process satisfies
\begin{align}
\label{eq:floq-15}  
\myphi(mT)\vect p(0)\to \mymu,\quad m\to\infty.
\end{align}

Using the Floquet theorem, the time-inhomogeneous Markov process $X_t$
is simplified into a \emph{time-homogeneous} (not necessarily
reversible) Markov jump process $\tilde X_{m} = X_{mT}, \ m\in\mathbb
N$.
Therefore, comparing with $X_t$, we prefer to consider the process $\tilde X_{m}$
that helps understand the original dynamics $x_t$, because the powerful theories and
computational tools for time-homogeneous processes can be directly applied.
There is no doubt that information within one period is lost by using this temporal
discretization, however, information regarding the long-term behaviors
of the system would be satisfactorily described as long as their time-scales
are much longer than the period.
At the same time,
the computational cost of generating $\tilde X_{m}$ is mush less demanding
than the brute force simulations of NEMD, which implies lower
statistically uncertainty in calculating the observables of interest.
% due to the discretization in both the spacial and temperal directions.
% In the numercal example in Sec.~\ref{sec:alanine}, we
% compute the 

\section{Building the Markov state model}
\label{sec:build-msm}

If the discretization cells $\Omega_i$, $i=1,\ldots,n$ form a fine
partition of the molecular state space, the Markov chain defined via
the transition matrix $\vect P=\myphi^T(T)$ still is a fine-scale description of the
dynamics.  Now we want to coarse our description much further by
constructing a Markov State Model (MSM) for $\vect P$ with $k\ll n$
macrostates: The resulting $k\times k$ MSM transition matrix $\hat{\vect P}$
then defines the coarse grained long term kinetics that should
approximate the original long term kinetics well. In Ref.~\cite{sarich2010approximation}
and \cite{schuette2011markov} it
has been shown how to do this if $\vect P$ satisfies the detailed balance
condition: (1) Identify the cores of the metastable sets of the
dynamics, (2) use them as milestones to construct an MSM in which the
macrostates are the metastable core sets and $\hat{\vect P}$ is the transition
matrix of the milestone process that models the jumping behavior of
the original dynamics between the metastable regions.


However, since we cannot assume $P$ to satisfy detailed balance, we
instead follow the approach recently proposed in Ref.~\cite{sarich2014utilizing}.
It allows to identify the metastable core sets for the
non-reversible transition matrix $P$. Assume that this approach leads
to the $k$ core sets $C_1,\ldots, C_k\subset S$, and we denote $C=S\setminus\cup_j C_j$. Following
Ref.~\cite{djurdjevac2010markov}, Thm. 3.1, the coarse grained transition
matrix $\hat{P}$ has to be computed as follows:
\begin{enumerate}
\item For the process associated with $P$ compute the forward and backward committors $q^\fwd_j$ and $q^\bwd_j$  for each core set $C_j$. This can be done by solving the linear equations
\begin{align}
(P-\id) q^\fwd_j(i) & =  0, \quad i\in C\\
q^\fwd_j(i) & =  1,\quad i\in C_j\\
q^\fwd_j(i) & =  0,\quad i\in C_l,l\not=j
\end{align}
and
\begin{align}
(P^b-\id) q^\bwd_j(i) & =  0, \quad i\in C\\
q^\bwd_j(i) & =  1,\quad i\in C_j\\
q^\bwd_j(i) & =  0,\quad i\in C_l,l\not=j
\end{align}
where $P^b$ denotes the transition matrix of the time-reversed process
given by $P^b_{ji}=\mu_i P_{ij}/\mu_j$.

Alternatively, the forward and
backward committors can be sampled via the definitions: The forward
committor $q^\fwd_j(i)$ is defined as the probability of visiting coreset
$C_j$ next conditioned on being at state $i$.  The
backward committor $q^\bwd_j(i)$  is defined as the probability
of last coming from $C_j$ conditioned on being at state $i$.
\item Compute $\hat{\mu}(j)=\sum_{i\in S} q^\bwd_j(i)\mu(i)$,
  which is the invariant measure of the MSM
\item Construct the MSM transition matrix $\hat{P}$ according to
  \begin{align}
    \label{eq:msm-tmatrix}
    \hat{\vect P}_{jk}
    = &
    \frac{1}{\hat{\mu}(j)}
    \langle (\vect P^b - \id) q^\bwd_j,q^\fwd_k \rangle_\mu,\qquad j\not= k, \\
    \hat{\vect P}_{jj}
    =&
    1-\sum_{k\not=j} \hat{\vect P}_{jk}
  \end{align}
where the inner product is defined by
$\langle u,v \rangle_\mu=\sum_{i\in S} u_i v_i \mu_i$.
\end{enumerate}





\section{Numerical example:
  Alanine dipeptide under oscillatory electric field}
\label{sec:alanine}

If the spacial discretization is fine enough, the Markov jump
process~\eqref{eq:disc-master} is a good approximation to the original
MD~\eqref{eq:disc-1}.  In practice, it is difficult to predict
how many discrete set is fine enough; Moreover, since the
total configurational dimension of the system is $3N$ ($N$ being the
number of atoms), it is prohibitive to do a very fine discretization
over all DOFs for most of systems of practical interest.
In the systems that possess a
time-scale separation, it is possible, in most cases, to pick up
a characteristic time (or lag-time) $\tau$ that distinguishes the slow
and fast time-scales.
For example, the conformational
transitions of a large biomolecule is much slower than the covalent bond
vibrations. 
% It should be noted that we do NOT assume that the fast time-scales are decoupled with the slow time-scales~\footnote{If the fast time-scales
%   are decoupled from the slow ones, then the story would become much simpler,
% however, this situation is lack of generality, we will not discuss.}.
In this context,
it is usually possible to construct a
relatively small number of discrete sets that
correctly describe the slow  dynamics, and 
in each set the fast dynamics relaxes within the lag-time $\tau$.
Then if the discretized dynamics~\eqref{eq:disc-master}
reproduces the slow time-scales and the corresponding transitions
of the original dynamics~\eqref{eq:disc-1},
the former is considered to be a good approximation of the later.
% Then the 
% dynamics~\eqref{eq:disc-master} built on top of the discretization
% is a good approximation to~\eqref{eq:disc-1}
% in the sense of reproducing the slow dynamics.
One possible way to define the discretized set is firstly to find a few collective
variables, and then to discretize the dynamics as fine as possible only on these
variables either by uniform or adaptive
discretization~\cite{chodera2007automatic, prinz2011markov}.
% This process also suggests that the choice of the lag-time should be 
% shorter than the dominant implied timescale so that they can be resolved,
% and be longer than the fast time-scales to relax the unresolved dynamics
% within the discretized sets.
However, it is difficult to give a general answer in prior regarding, e.g.
how large is the lag-time,
how to choose the collect variables, how fine the discretization should be.
For large or high dimensional systems, these questions usually become non-trivial. 

\begin{figure}
  \centering
  \includegraphics[width=0.3\textwidth]{fig/confs/c-2.eps}
  \caption{A schematic plot of the alanine dipeptide molecule and the dihedral angles $\phi$ and $\psi$.}
  \label{fig:tmp1}
\end{figure}

To illustrate how the discretization works in practice, it is therefore
useful to investigate the discussed approximations by a numerical
example.  We take the alanine dipeptide system under an oscillatory EF,
as an example, the NEMD simulation of which was
extensively studied by Ref.~\cite{wang2014exploring}.  
The alanine dipeptide was put into the local thermostating
environment, and was driven by a periodic electric field with period
$T = 10$~ps. The 20,000 branching trajectories were simulated from 20,000
initial configurations that sample the equilibrium distribution.
\redc{Write a lot of details on NEMD.}  The branching
trajectories were each 4000~ps long, and the system reaches
non-equilibrium stationary state in roughly 300~ps.

We choose the
two dihedral angles $\phi$ and $\psi$ as collective variables (see
Fig.~\ref{fig:tmp1}), and the discretization is a uniform division of
the $\phi$--$\psi$ plane. We denote the number of division on each
dihedral by $K$, then the discretized sets are
$\{\Omega_i\},\ i\in S = \{1,\cdots,K^2\}$.
One possible approximation to
the discretized generator $L(t)$ is via
the following forward finite difference scheme:
\begin{align}
  \label{eqn:tmp4}
  L_{ij}(t) \approx \frac{1}{\tau}
  \,[\, \prob (\vect X_{t+\tau} = i \vert \vect X_{t} = j) - \delta_{ij} \,],
  \quad i,j\in S
\end{align}
and $\tau$ is the lag-time.
In the following subsections~\ref{sec:alanine-disc} and \ref{sec:alanine-tau},
we investigate the discretization
quality with respect to  the choice of $K$ and  lag-time $\tau$.
Then
the discretized dynamics generated by the Floquet transition matrix
is compared with original NEMD simulation
via considering the stationary probability density, the first mean hitting time
and the forward and backward committors.


\subsection{Discretizing the dihedral angles}
\label{sec:alanine-disc}

\begin{figure}
  \centering
  \includegraphics[width=0.4\textwidth]{fig/t010/discrete/fig-cg-prob.eps}  
  \caption{The time-dependent probability $\mathbb
    P\big(\phi\in[0,180), \psi\in [0,180)\big)$.  The NEMD simulation is compared with different
    discretization methods. The red shadow region indicates the
    statistical uncertainty of the NEMD simulation.}
  \label{fig:tmp2}
\end{figure}

We estimate the discretized generator $L(t), \ t\in[0,T)$
from NEMD trajectories generated by $\ml(t)$ in time interval $[t_1,
t_2]$.
Due to the periodicity of $\ml(t)$, if the discretized dynamics approximate the original
dynamics good enough, generator  $L(t)$ should not depend on in which interval
it is estimated, provided that the initial state of the system
is not very far from the stationary state at long-time limit.
Therefore, this is an indicator for calibrating the discretization quality.
We compute $L(t)$ by two discretizations $K=2$ and $K=20$, and two
choices of time intervals $[0, 80]$~ps and $[320, 400]$~ps, and then
compare the time-dependent
probability $\prob\big(\phi_t\in[0,180), \psi_t\in [0,180)\big)$
with the NEMD result in
Fig.~\ref{fig:tmp2}.
The time derivative of the master equation~\eqref{eq:disc-master}
is discretized by the forward Euler scheme with a time step of 0.5~ps.
Using $K=2$ the dynamics depends on the time interval used for
calculating the generator: using time interval $[0, 80]$~ps the discretized
dynamics deviates from the NEMD result,
while using time interval $[320, 400]$~ps the discretized dynamics can only
reproduces the NEMD result after 300~ps.  This therefore indicates poor 
approximations to the original dynamics with $K=2$. The reason is that the
discretization is too coarse so that the dynamics cannot be fully
equilibriated within the lag-time $\tau$ in each discretized set,
therefore, the discretization presents state dependency.  For
$K=20$, the discretized dynamics does not depend on the time interval of
calculating the generator, and is consistent with the
NEMD simulation within the error bar. Therefore, throughout this paper we use $K=20$
to discretize  the dihedral angle space of the alanine dipeptide.
For a good statistical accuracy, if not stated otherwise,
we will use the full trajectories, i.e.~a time interval of
$[0,4000]$~ps for estimating the discretized generator $L(t)$.


\subsection{The choice of lag-time $\tau$}
\label{sec:alanine-tau}

\begin{figure}
  \centering
  \includegraphics[width=0.4\textwidth]{fig/t010/discrete/fig-cg-prob-tau.eps}  
  \caption{The time-dependent probability $\mathbb
    P\big(\phi\in[0,180), \psi\in [0,180)\big)$.  The NEMD simulation is compared with different
    discretization methods. The red shadow region indicates the
    statistical uncertainty of the NEMD simulation.}
  \label{fig:tmp3}
\end{figure}

As discussed before, the lag-time should be chosen a value that lies
in the spectrum gap of the original dynamics, so that it resolves the
slow dynamics, and at the same time relaxes the fast dynamics. In
practice, it is very difficult to estimate the lag-time in
prior. Therefore, we consider 
different choices of $\tau$ (0.5, 1.0, 2.0 and 5.0~ps)
discretizing the same original dynamics, and compare the
time-dependent probability $\prob\big(\phi\in[0,180), \psi\in
[0,180)\big)$ calculated from different time discretization with the NEMD simulation
of the original dynamics
(see Fig.~\ref{fig:tmp3}).  All cases use identical dihedral angle discretization: $K=20$.
It is clear that when lag-time is close to the period (10~ps), the
discretized dynamics cannot resolve the probability change within a
period. However, it is surprising  that the large lag-times are able to capture the
the overall long time behavior of the original dynamics.
We observe no significant difference between $\tau=0.5$ and
$\tau=1.0$~ps, which means the discretized dynamics is not very sensitive
to the choice of $\tau$.
Therefore, throughout this paper $\tau=0.5$~ps will be used.
We do not investigate the small $\tau$ limit, because saving trajectories
more frequently requires more hard disk space, and is usually not preferred in practice.

\subsection{Steady state distribution}


\begin{figure}
  \centering  
  \includegraphics[width=0.4\textwidth]{fig/t010/cluster.marco.3.steadyDist//fig-dist.eps}
  % \includegraphics[width=0.23\textwidth]{fig/t010/cluster.marco.3/fig-dist-msm.eps}
  \includegraphics[width=0.4\textwidth]{fig/t010/cluster.marco.3/fig-floquet-vec-1.eps}
  \caption{The color scale plot of the logarithm stationary probability
    of (a) the NEMD  and (b)
    the Floquet matrix $\Phi(T)$.
  }
  \label{fig:num-1}
\end{figure}

Comparing with the time-homogeneous Markov processes, there are not as
many tools for understanding the time-inhomogeneous Markov process.
Therefore, in the following subsections,
we consider the time-homogeneous process $\tilde X_{m}$ generated by
the Floquet transition matrix $P = \Phi^T(T)$, and investigate if it
reproduces the properties of the original  non-equilibrium process $x_t$.
In this context, only the configurations at the  integral periods $mT$ along the original process
are taken into consider.

An important check is the consistency between the
stationary probability density of $\Phi(T)$ (i.e.~the leading eigenvector $\mu$) and that 
estimated form the original NEMD simulation, along which only integral periods $mT$ are considered:
\begin{align}
  \label{eq:num-tmp1}
  \rho_{\textrm{st}}(\phi,\psi) = \lim_{m\rightarrow\infty} \rho (\phi,\psi,mT),
\end{align}
On each NEMD branching trajectory, the very beginning 320~ps is discarded, and
the rest of the trajectory in $[320,4000]$~ps is used to estimate
the stationary probability. $\Phi(T)$ is computed by solving~\eqref{eq:disc-master-phi}
in one period, where the discretized generator $L(t)$ was estimated
from the full NEMD trajectories of $4000$~ps long.
Moreover, to make it comparable to the free energy in equilibrium case, we take
the logarithm of the properties, i.e.~$F_{\textrm{st}}(\phi,\psi)=
-k_B\mathcal T\log \rho_{\textrm{st}}(\phi,\psi)$
for NEMD and $F_{\textrm{st}}(\phi,\psi)=
-k_B\mathcal T\log \mu(\phi,\psi)$ for $\Phi(T)$, where $k_B$ is the
Boltzmann constant and $\mathcal T$ is the temperature of the system.
% Under the aformentioned discretization, the stationary distribution is
% defined for the each bin of the dihedral angle space, i.e.~$F_{pq} =
% \int_{ph}^{(p+1)h}d\phi\int_{qh}^{(q+1)h}d\psi
% F_{\textrm{st}}(\phi,\psi), \ 0\leq p,q<K-1$, where $h = 360/K$ being
% the size of the bin, $p$ and $q$ here being the bin indexes.
The
results are  compared in Fig.~\ref{fig:num-1}. A good consistency between
the  NEMD simulation and 
$\Phi(T)$ is observed.  

\subsection{Coreset identification}

\begin{figure}
  \centering
  \includegraphics[width=0.4\textwidth]{fig/t010/cluster.marco.3/fig-cluster.eps}
  \caption{The coreset identification. Different colors indicate different coresets: $C_{\confaa}$ (green), $C_\beta$ (yellow) and $C_{\confc}$ (red).
    The blue color means out of any coreset. (By Marco)}
  \label{fig:cluster}
\end{figure}

\redc{Write how to detect the core sets for irreversible Markov process}.

The coresets (see Fig.~\ref{fig:cluster}) are denoted by $C_{\confaa}$ (yellow), $C_\beta$ (red) and $C_{\confc}$ (green), which correspond to
right-handed alpha-helix, beta-sheet and left-handed alpha-helix, respectively.

\subsection{First mean hitting time}
\label{sec:alanine-fmht}

\begin{figure}
  \centering
  \includegraphics[width=0.23\textwidth]{fig/t010/cluster.marco.3.fht/fig-fht-1.eps}
  \includegraphics[width=0.23\textwidth]{fig/t010/cluster.marco.3.fht/fig-fht-msm-1.eps}\\
  \vskip -.5cm
  \includegraphics[width=0.23\textwidth]{fig/t010/cluster.marco.3.fht/fig-fht-2.eps}
  \includegraphics[width=0.23\textwidth]{fig/t010/cluster.marco.3.fht/fig-fht-msm-2.eps}\\
  \vskip -.5cm
  \includegraphics[width=0.23\textwidth]{fig/t010/cluster.marco.3.fht/fig-fht-3.eps}
  \includegraphics[width=0.23\textwidth]{fig/t010/cluster.marco.3.fht/fig-fht-msm-3.eps}\\
  \caption{The first mean hitting time (FMHT) comparison between the NEMD
    simulation (first column) and the discretized dynamics (second column).  From up
    to down are first mean hitting time to coresets $C_{\confaa}$ and $C_{\beta}$
    and $C_{\confc}$, respectively}
  \label{fig:num-6}
\end{figure}

The first mean hitting time, as a function of the dihedral angles $(\phi,\psi)$,
is defined by the
averaged first time needed for hitting a certain coreset
$i\in\{\alpha_R,\beta,\alpha_L\}$, conditioned on starting from the
conformation $(\phi,\psi)$.  Since the largest first mean hitting
time (starting from coreset $\confaa$ to $\confc$) is longer than
600~ps, if we use the NEMD trajectories of length 4000~ps,
the results will be biased. Therefore, we use 100 NEMD
trajectories of $2\times 10^5$~ps instead. We also compute the first mean hitting
time of 
a  $10^9$~ps long (being $T=10$~ps, $m = 1,\cdots, 10^8$)
discretized trajectory $\tilde
X_{m}$ that is generated by the Floquet transition matrix $P=\Phi^T(T)$.
The first mean hitting time is presented in Fig.~\ref{fig:num-6}.  The
good consistency between the NEMD and the discretized Markov process
$\tilde X_{m}$ indicates a good approximation quality.
Since the length of the discretized process is substantially longer than
the total length of NEMD
simulation ($10^9$~ps v.s.~$2\times10^7$~ps), the statistically uncertainty
is much smaller. 
However, one should notice that systematic errors
are introduced during the discretization (estimating $L(t)$), and 
its potential influence on the observables,
e.g.~the first mean hitting time, is not straightforward to estimate.
This topic is out of the scope of the current paper,
and will not be discussed in detail.
In summary, very long discretized trajectories helps in calculating the observables
in a smoother  (less statistical error) but not necessarily more accurate manner.
On the other hand, the computational cost of
discretized process, if the cost for estimating $\Phi(T)$ is not included, is
essentially cheaper than NEMD:
it took roughly 10 minutes on one core of an Intel Xeon E31245 CPU for generating the discretized trajectory, while the
NEMD trajectories took $1.6\times 10^4$ core hours for Intel Xeon E5-4650 CPUs.


\subsection{Forward and backward committors}
\label{sec:alanine-committor}
\begin{figure}
  \centering
  \includegraphics[width=0.23\textwidth]{fig/t010/cluster.marco.3/fig-commitor-fw-1.eps}
  \includegraphics[width=0.23\textwidth]{fig/t010/cluster.marco.3/fig-commitor-bw-1.eps}
  \includegraphics[width=0.23\textwidth]{fig/t010/cluster.marco.3/fig-commitor-diff-1.eps}\\
  \includegraphics[width=0.23\textwidth]{fig/t010/cluster.marco.3/fig-commitor-fw-msm-1.eps}
  \includegraphics[width=0.23\textwidth]{fig/t010/cluster.marco.3/fig-commitor-bw-msm-1.eps}
  \includegraphics[width=0.23\textwidth]{fig/t010/cluster.marco.3/fig-commitor-diff-msm-1.eps}
  \caption{The forward $q^\fwd_{\confc}$ and backward committors
    $q^\bwd_{\confc}$ computed by discretized dynamics (second row) is compared with
    those computed by NEMD simulations (first
    row).}
  \label{fig:num-3}
\end{figure}

\begin{figure}
  \centering
  \includegraphics[width=0.23\textwidth]{fig/t010/cluster.marco.3/fig-commitor-fw-2.eps}
  \includegraphics[width=0.23\textwidth]{fig/t010/cluster.marco.3/fig-commitor-bw-2.eps}
  \includegraphics[width=0.23\textwidth]{fig/t010/cluster.marco.3/fig-commitor-diff-2.eps}\\
  \includegraphics[width=0.23\textwidth]{fig/t010/cluster.marco.3/fig-commitor-fw-msm-2.eps}
  \includegraphics[width=0.23\textwidth]{fig/t010/cluster.marco.3/fig-commitor-bw-msm-2.eps}
  \includegraphics[width=0.23\textwidth]{fig/t010/cluster.marco.3/fig-commitor-diff-msm-2.eps}
  \caption{The forward $q^\fwd_{\confaa}$ and backward
    $q^\bwd_{\confaa}$ committors computed by discretized dynamics (second row) is
    compared with those computed by NEMD
    simulations (first row).}
  \label{fig:num-4}
\end{figure}

\begin{figure}
  \centering
  \includegraphics[width=0.23\textwidth]{fig/t010/cluster.marco.3/fig-commitor-fw-3.eps}
  \includegraphics[width=0.23\textwidth]{fig/t010/cluster.marco.3/fig-commitor-bw-3.eps}
  \includegraphics[width=0.23\textwidth]{fig/t010/cluster.marco.3/fig-commitor-diff-3.eps}\\
  \includegraphics[width=0.23\textwidth]{fig/t010/cluster.marco.3/fig-commitor-fw-msm-3.eps}
  \includegraphics[width=0.23\textwidth]{fig/t010/cluster.marco.3/fig-commitor-bw-msm-3.eps}
  \includegraphics[width=0.23\textwidth]{fig/t010/cluster.marco.3/fig-commitor-diff-msm-3.eps}
  \caption{The forward $q^\fwd_{\beta}$ and backward $q^\bwd_{\beta}$
    committors computed by discretized dynamics (second row) is
    compared with those computed by NEMD
    simulations (first row).}
  \label{fig:num-5}
\end{figure}


Committors are very important statistical properties of Markov processes, and play
an important role in the later MSM building, therefore, it is worth
checking if the discretized process $\tilde X_m$ reproduces the NEMD committors.
The forward committor $q^\fwd_i(\phi,\psi)$ of a coreset $C_i,\
i\in\{\confaa, \beta, \confc\}$ is defined as the probability of
visiting coreset $C_i$ next conditioned on being at conformation
$(\phi,\psi)$.  The backward committor $q^\bwd_i(\phi,\psi)$ of a
coreset $C_i,\ i\in\{\confaa, \beta, \confc\}$ is defined as the
probability of last coming from $C_i$ conditioned on being at
configuration $(\phi,\psi)$.
For reversible Markov processes, the
forward and backward committors are identical, however, it is in
general not the case for irreversible processes.
The committors estimated from NEMD simulations ($20000$ trajectories $4000$~ps each) is compared with
those estimated from the discretized trajectory $\tilde X_m$, which is the same as the one used in Sec.~\ref{sec:alanine-fmht} for first mean hitting time.
Fig.~\ref{fig:num-3}--\ref{fig:num-5} presents both
committors as well as their difference corresponding to different coresets.
The committors of the discretized process is in good consistency with those of
the NEMD simulations. The non-zero values in the committor differences
indicate that the NEMD process, projected on the discretized
dihedral angle space, is irreversible, and
the discretized process is able to correctly describe this irreversiblity.
Also the accurate reproduction of the committors indicates it is reasonable to build the 
MSM out of the committors of the  discretized process.

\subsection{Building MSM}
Following the process described in Sec.~\ref{sec:build-msm}, we are able to build a 
three states MSM for the alanine dipeptide system,
where the stationary probability and the committors are
estimated by the discretized process $\tilde X_m$, see
Sec.~\ref{sec:alanine-committor}.
The leading eigenvalues of $P=\Phi^T(T)$ is compared with those of the MSM, i.e.~$\hat P$ in
Tab.~\ref{tab:tmp1}.
The MSM is able to accurately reproduce
the largest non-trivial eigenvalue, which means a precise reproduction
of the longest non-trivial implied time-scale. The accuracy of the second non-trivial
time-scale is not as good as the first, but is still acceptable. The reason
for the lower accuracy is that
the corresponding time scale is 26.5~ps (calculated by $-T/\log(\lambda_2)$),
which is NOT significantly longer than the time resolution $T=10$~ps.
It worth noting that although the discretized process $\tilde X_m$ is irreversible,
the MSM built out of it is almost reversible:
The magnitude of the anti-symmetric part of the
matrix $\textrm{diag}(\hat \mu)\cdot \hat P$ is only of order $10^{-4}$.

\begin{table}
  \centering
  \caption{
    The eigenvalue comparison
  }
  \begin{tabular*}{0.5\textwidth}{@{\extracolsep{\fill}}c rrr}\hline\hline
      &  $\lambda_2$ & $\lambda_3$ & $\lambda_4$ \\\hline
    $P$         &0.907  &0.686 & 0.553       \\
    $\hat P$    &0.911  &0.724 & --       \\
    \hline\hline
  \end{tabular*}
  \label{tab:tmp1}
\end{table}

% \begin{figure}
%   \centering
%   \includegraphics[width=0.23\textwidth]{fig/t010/cluster.marco.3/fig-eig-vec-2.eps}
%   \includegraphics[width=0.23\textwidth]{fig/t010/cluster.marco.3/fig-eig-vec-3.eps}
%   \includegraphics[width=0.23\textwidth]{fig/t010/cluster.marco.3/fig-eig-vec-4.eps}\\
%   \caption{The eigenfunctions of $P$.}
%   \label{fig:num-6}
% \end{figure}

\begin{figure}
  \centering
  \includegraphics[width=0.4\textwidth]{fig/t010/cluster.marco.3/fig-coreset-prob.eps}
  \caption{The time-dependent probability $\hat P_l$.
    Solid lines are from NEMD simulation, while the dashed lines are from MSM.}
  \label{fig:num-7}
\end{figure}

We study the time-dependent expectation values of
the properties, i.e.
\begin{align}
  \mathcal A(t) = \langle A(i)\rangle_t = \sum_{i\in S} A(i) p(i,t),
\end{align}
which are linear combinations of the backward
committor, i.e.
\begin{align}
  A(i) = \sum_{l=1}^k \alpha_l q^\bwd_l(i)
\end{align}
then
\begin{align}\nonumber
  \mathcal A(t) &=
  \sum_{i\in S} \sum_{l=1}^k \alpha_l q^\bwd_l(i)  p(i,t) \\\nonumber
  & =
  \sum_{i\in S} \sum_{l=1}^k \alpha_l \prob (\hat X_t = l \vert X_t = i) \prob (X_t = i) \\\nonumber
  & =
  \sum_{i\in S} \sum_{l=1}^k \alpha_l \prob (\hat X_t = l ,X_t = i) \\\nonumber
  & =
  \sum_{l=1}^k \alpha_l \prob (\hat X_t = l) \\\label{eq:num-28}
  & =
   \sum_{l=1}^k \alpha_l \,\hat p (l, t),
\end{align}
where the time-dependent projected probability is governed by the MSM:
\begin{align}\label{eq:num-29}
  \hat p(i, t+T) = \sum_{j\in S} \hat p(j,t)\hat P_{ji}
\end{align}
In Fig.~\ref{fig:num-7} we compare the numerical calculation of $\hat p (l, mT), \ m\in\mathbb N$ from NEMD and MSM calculations.
In the former case, the backward committor and probability density
on RHS of the projection $\hat p (l, mT) = \sum_{i\in S}  q^\bwd_l(i)  p(i,mT) $
are estimated from the
NEMD trajectories. For MSM, the projection of the initial probability is applied $\hat p (l, 0) = \sum_{i\in S}  q^\bwd_l(i)  p(i,0) $, then
the time-dependent probability at $mT$ is generated by Eq.~\eqref{eq:num-29}.
% The projected probability, e.g.~$\hat p (l', mT)$ is calculated
% by Eq.~\eqref{eq:num-28} letting $\alpha_l = \delta_{l'l}$.




\appendix



\section{Reversibility of the original dynamics}
\label{sec:revs}

We consider the governing dynamics Eq.~\eqref{eq:disc-1}.
For simplicity we denote the force by $F(x_t,t) = -\nabla_x V(x_t)+ E(t)D(x_t) $.  We denote $\sigma =  \sqrt{2\beta^{-1}} $.
According to Girsanov, we have
\begin{align}
  \label{eq:tmp8}
  \frac{dp[x_t]}{dw[x_t]}  =
  \exp \bigg\{
  \frac 1{\sigma^2}\int_0^T F(x_t,t) dx_t -
  \frac1{2\sigma^2}\int_0^T F^2(x_t,t) dt
  \bigg\}
\end{align}
where $dp$ is the probability measure of trajectory $x_t$, while $dw$ is the
probability measure of the standard Wiener process $dx_t = \sigma dw_t$.
Assuming a discretization of the
stochastic process at time $0 < t_1 < t_2 < \cdots < t_N = T$, where
$t_i = iT / N$. We denote $x_i = x_{t_i}$, and $w_i = w_{t_i}$, then we have,
in the sense of Ito,
\begin{align}\label{eq:tmp9}
  \frac{dp[x_t]}{dw[x_t]}  \approx
  \exp\bigg\{\frac1{\sigma^2}\sum_{i=0}^{N-1} F(x_{i},t_{i})(x_{i+1} - x_i) -\frac1{2\sigma^2}\sum_{i=0}^{N-1}F^2(x_i,t_i)\dt\bigg\} 
\end{align}
Now, consider a conjugate trajectory $x^\dagger_t = x_{T-t}$ that starts at $x_T$, ends at $x_0$. The conjugate dynamics is driven by  $F^\dagger(x^\dagger_t,t) = F(x^\dagger_t, T-t)$.
Writing the Girsanov for the conjugate dynamics
\begin{align}\label{eq:dagger-0}
  \frac{dp^\dagger[x^\dagger_t]}{dw[x^\dagger_t]}  
  \approx\,&
  \exp\bigg\{
  \frac1{\sigma^2}\sum_{i=0}^{N-1} F^\dagger(x^\dagger_{i},t_{i})(x^\dagger_{i+1} - x^\dagger_i) -
  \frac1{2\sigma^2}\sum_{i=0}^{N-1}[F^\dagger(x^\dagger_i,t_i)]^2\dt\bigg\} \\ \nonumber
  =\,&
  \exp\bigg\{
  \frac1{\sigma^2}\sum_{i=0}^{N-1} F(x^\dagger_{i},T - t_{i})(x^\dagger_{i+1} - x^\dagger_i) -
  \frac1{2\sigma^2}\sum_{i=0}^{N-1}[F(x^\dagger_i, T-t_i)]^2\dt\bigg\} \\\nonumber
  =\,&
  \exp\bigg\{
  \frac1{\sigma^2}\sum_{i=0}^{N-1} F(x_{N-i},t_{N-i})(x_{N-i-1} - x_{N-i}) -
  \frac1{2\sigma^2}\sum_{i=0}^{N-1}[F(x_{N-i},t_{N-i})]^2\dt\bigg\} \\
  = \,&
  \exp\bigg\{
  \frac1{\sigma^2}\sum_{i=N}^{1} F(x_{i},t_{i})(x_{i-1} - x_i) -
  \frac1{2\sigma^2}\sum_{i=N}^{1}F^2(x_i,t_i)\dt\bigg\}
\end{align}
% Therefore,
% \begin{align}\nonumber
%   \frac{dp[x^\dagger_t]}{dw[x^\dagger_t]}  =
%   \,&
%   \frac{dp^\dagger[x^\dagger_t]}{dw[x^\dagger_t]} \\\nonumber
%   \approx\,&
%   \exp\bigg\{
%   \frac1{\sigma^2}\sum_{i=0}^{N-1} F^\dagger(x^\dagger_{i},t_{i})(x^\dagger_{i+1} - x^\dagger_i) -
%   \frac1{2\sigma^2}\sum_{i=0}^{N-1}[F^\dagger(x^\dagger_i,t_i)]^2\dt\bigg\} \\\nonumber
% \end{align}
Since it is obvious that $dw[x^\dagger_t] / dw[x_t] = 1$,
\begin{align}
  \label{eq:tmp10}
  \frac{dp^\dagger[x^\dagger_t]}{dw[x_t]}
  \approx \,&
  \exp\bigg\{
  \frac1{\sigma^2}\sum_{i=1}^{N} F(x_{i},t_{i})(x_{i-1} - x_i) -
  \frac1{2\sigma^2}\sum_{i=1}^{N}F^2(x_i,t_i)\dt\bigg\}
\end{align}
The difference between the single trajectory probabilities is
\begin{align}\label{eqn:tmp12}
  \frac{  dp^\dagger[x^\dagger_t] }{ dp[x_t]}
  \approx&\,
  \exp\bigg\{
  -\frac1{\sigma^2}\sum_{i=1}^{N-1}
  \bigg[
  F(x_i,t_i)(x_{i+1} - x_{i}) + F(x_i,t_i)(x_{i} - x_{i-1})
  \bigg]
  \bigg\}
\end{align}
Assuming the smoothness of the external perturbation, consider the differentiation:
\begin{align}\nonumber
  F(x_{i},t_{i}) - F(x_{i-1},t_{i-1}) =
  &\,
  F(x_{i},t_{i}) - F(x_{i-1},t_{i}) + F(x_{i-1},t_{i}) -  F(x_{i-1},t_{i-1})\\\nonumber
  =&\,
  \nabla_x F(x_{i-1},t_{i})(x_i - x_{i-1}) + \mo(\dt) \\\label{eqn:tmp13}
  =&\,
  \nabla_x F(x_{i-1},t_{i-1})(x_i - x_{i-1}) + \mo(\dt)
\end{align}
The second order expansion w.r.t.~$x_i - x_{i-1}$ is of order $\dt$, so it is absorbed into $ \mo(\dt)$.
Then the \eqref{eqn:tmp12} becomes
\begin{align}\label{eqn:tmp14}
  \frac{  dp^\dagger[x^\dagger_t] }{ dp[x_t]}
  \approx&\,
  \exp\bigg\{
 -\frac2{\sigma^2}\sum_{i=0}^{N-1} F(x_i,t_i)(x_{i+1} - x_{i}) 
 -\frac1{\sigma^2}\sum_{i=0}^{N-1}\nabla_xF(x_i,t_i)(x_{i+1} - x_{i})^2
 \bigg\}
\end{align}
Using the identity
$  dt = (dw_t)^2 = {\sigma^{-2}} dx_t^2$,
Eq.~\eqref{eqn:tmp14} is written in the integral form
\begin{align}\label{eqn:tmp14-0}
  \frac{  dp^\dagger[x^\dagger_t] }{ dp[x_t]}
  \approx&\,
  \exp\bigg\{
 -\frac2{\sigma^2}\int_0^T F(x_t,t) dx_t
 -\int_0^T\nabla_xF(x_t,t)dt
 \bigg\}  
\end{align}
One would not have the second integral on the exponent if the first integral of the exponent were defined in the sense of Stratonovich.

We notice that
\begin{align}\nonumber
  dV(x, t) = &\, \frac{\partial V}{\partial x} dx + \frac{\partial V}{\partial t} dt\\\nonumber
  =&\,
  \frac12 \sigma^2 \nabla^2_x V dt +  \nabla V dx_t + \frac{\partial V}{\partial t} dt \\
  =&\,
  -\frac12 \sigma^2 \nabla_x F dt -  F dx_t + \frac{\partial V}{\partial t} dt
\end{align}
Eq.~\eqref{eqn:tmp14-0} becomes
\begin{align}
  \frac{  dp^\dagger[x^\dagger_t] }{ dp[x_t]}
  =&\,
  \exp\bigg\{
  \frac2{\sigma^2}\bigg[
  V(x_T,T) - V(x_0,t_0) - \int_0^T\partial_tV(x_t,t)dt
  \bigg]
  \bigg\}
\end{align}
% According to the  Einstein relation, the temperature $\mathcal T = \sigma^2/2$, we denote $\beta = 1/{\mathcal T} = 2/\sigma^2$.
Take the limit of infinite small time interval, notice the equilibrium
invariant probability density with respect to potential $V(x,0)$ satisfies $\mymu(x) \propto e^{-\beta V(x,0)}$, and replace $\sigma^2$ by $2\beta^{-1}$,
\begin{align}
  \frac{  dp^\dagger[x^\dagger_t] }{ dp[x_t]}
  =&\,
  \frac{\mymu(x_0)}{\mymu(x_T)}\times
  \exp\bigg\{
  - \beta\int_0^T\partial_tV(x_t,t)dt
  \bigg\}
\end{align}

\subsection{Irreversibility of the periodical symmetrical dynamics}

In Eq.~\eqref{eq:dagger-0}, we assume the periodicity of the perturbation $F(x,t) = F(x,t+T)$, and the symmetry of the external perturbation, i.e.~$F(x, -t) = F(x, t)$, we have
\begin{align}
  \frac{dp^\dagger[x^\dagger_t]}{dw[x^\dagger_t]}  
  \approx\,&
  \exp\bigg\{
  \frac1{\sigma^2}\sum_{i=0}^{N-1} F(x^\dagger_{i},t_{i})(x^\dagger_{i+1} - x^\dagger_i) -
  \frac1{2\sigma^2}\sum_{i=0}^{N-1}[F(x^\dagger_i, t_i)]^2\dt\bigg\}  
\end{align}
By changing notation $x^\dagger$ back to $x$, and comparing with~\eqref{eq:tmp9}, the reversed dynamics is subject to the Eq.~\eqref{eq:disc-1},
i.e.~$dp^\dagger = dp$.
Therefore
\begin{align}\nonumber
  p(x_0,T\vert x_T,0)
  =&\,\int_{\mc\{x_T,0;x_0,T\}}
  dp[x^\dagger_t] \\\nonumber  
  =&\,
  \int_{\mc\{x_0,0;x_T,T\}}
  \frac{  dp[x^\dagger_t] }{ dp[x_t]} \cdot dp[x_t] \\\nonumber
  =&\,
  % \lim_{N\rightarrow\infty} 
  % \frac{1}{(2\pi\sigma^2\dt)^{(N-1)/2}} \int dx_{N-1}\cdots\int dx_{1}
  \int_{\mc\{x_0,0;x_T,T\}}
  \frac{  dp^\dagger[x^\dagger_t] }{ dp[x_t]} \cdot dp[x_t] \\\label{eqn:tmp18}
  = &\,
  \frac{\mu(x_0)}{\mu(x_T)}
  \int_{\mc\{x_0,0;x_T,T\}}
  \exp\bigg\{
  -\beta\int_0^T \partial_t V(x_t,t)dt 
  \bigg\} \cdot dp[x_t]
\end{align}
where $\mc\{x_0,0;x_T,T\}$ denotes all continuous trajectories starting at $x_0$ and ending at $x_T$.
If $\partial_t V = 0$, i.e.~equilibrium, we have
\begin{align}
  p(x_0,T\vert x_T,0)e^{-\beta V(x_T,T)} =  p(x_T,T\vert x_0,0) e^{-\beta V(x_0,0)},
\end{align}
which proves the reversibility of the equilibrium dynamics.
The term
\begin{align}
  W[x_t] = \int_0^T \partial_t V(x_t,t)dt
\end{align}
is the non-equilibrium work associated to all possible
the dynamics $x_t$ starting at $x_0$ and ending at $x_T$ (see e.g.~Ref.~\cite{seifert2012stochastic}).
Therefore Eq.~\eqref{eqn:tmp18} is the detailed Jarzynski relation.
% Now the problem becomes if we can write a nice form (e.g.~the difference of a state function measured at $x_T$ and $x_0$) for the non-equilibrium work.
Noticing that
\begin{align}
  \label{eq:tmp21}
  p(x_T,T\vert x_0,0) = \int_{\mc\{x_0,0;x_T,T\}}dp[x_t],
\end{align}
From Eq.~\eqref{eqn:tmp18} we have
\begin{align}
  \label{eq:tmp22}
  \frac{p(x_0,T\vert x_T,0)}{  p(x_T,T\vert x_0,0)  }
  =
  \frac{\mu(x_0)}{\mu(x_T)}
  \mathbb E_{x_0\rightarrow x_T} [e^{-\beta W}]
\end{align}
% Maybe we want a more symmetric form.
% The expectation value of the reversed dynamics reads,
% \begin{align}\nonumber
%   \mathbb E_{x^\dagger_0\rightarrow x^\dagger_T} [e^{-\beta W^\dagger}]
%   =\,&
%   \int_{\mc\{x^\dagger_0,0;x^\dagger_T,T\}}
%   \exp\bigg\{
%   -\beta\int_0^T \partial_t V(x^\dagger_t,t)dt 
%   \bigg\} \cdot dp[x^\dagger_t]   \\\nonumber
%   =\,&
%   \int_{\mc\{x_0,0;x_T,T\}}
%   \exp\bigg\{
%   -\beta\int_0^T \partial_t V(x_{T-t},t)dt 
%   -\beta\int_0^T \partial_t V(x_{t},t)dt 
%   \bigg\}
%   \cdot dp[x_t]   \\  \nonumber
%   =\,&
%   \int_{\mc\{x_T,0;x_0,T\}}
%   \exp\bigg\{
%   \beta\int_0^T \partial_t V(x_t,t)dt 
%   \bigg\} \cdot dp[x_t]   \\
%   \label{eq:tmp23}
%   = \,&
%   \mathbb E_{x_0\rightarrow x_T} [e^{- 2\beta W}]  
% \end{align}
% Noticing that \eqref{eq:tmp22}  is true for the reversed dynamics, we have
% \begin{align}
%   \label{eq:24}
%   \frac{p(x_T,T\vert x_0,0)}{  p(x_0,T\vert x_T,0)  }
%   =
%   \frac{\mu(x_T)}{\mu(x_0)}
%   \mathbb E_{x_T\rightarrow x_0} [e^{\beta W}]  
% \end{align}


\bibliography{ref}{}
\bibliographystyle{unsrt}



\end{document}
