\documentclass{article}

\addtolength{\textwidth}{2.5cm}
\addtolength{\hoffset}{-.4in}

\usepackage{graphicx}
\usepackage[margin=10pt,font=small,labelfont=bf]{caption}
\usepackage{lipsum}

\renewcommand{\baselinestretch}{1.0}
% \setlength{\footskip}{0cm}
%\addtolength{\textheight}{10cm}
% \sloppy

\renewcommand{\arraystretch}{1.2}



\begin{document}


\hfill\begin{tabular}{@{}p{.4\linewidth}@{}}
\multicolumn{1}{@{}c@{}}{} \\
        Han Wang\\
        CAEP Software Center for \\
        High Performance Numerical\\
        Simulation\\
        Huayuan Road 6\\
        Beijing 100088, China\\
        \\
        \today
\end{tabular}
\vskip .1cm
\noindent
\begin{tabular}{@{}p{.45\linewidth}@{}}
\multicolumn{1}{@{}c@{}}{} \\
Prof. William L. Jorgensen \\
\\
Yale University \\
Department of Chemistry\\
P.O. Box 208107 \\
New Haven, CT  06520-8107 \\
  USA\\  
\end{tabular}\hfill

\vskip 1cm

\noindent
\textbf{\large Resubmission for manuscript ct-2014-00997y}

\vskip .5cm
\noindent
Dear Prof. Jorgensen,\\

On behalf of the co-author, I would like to resubmit our article:
\textit{``Building Markov State Models for
  Periodically Driven Non-Equilibrium Systems''}
by H.~Wang and C.~Sch\"utte to the
\textit{Journal of Chemical Theory and Computation}.
We gratefuly acknowledge the reviewers for the positive reports, and for
their insight and suggestions, which helped us improving the quality
of the manuscript.  In response to their comments, the revisions are
reported below in details.  To help the referees spot the changes in
the manuscript, we have submitted a separate file (highlight.pdf) with
all changes highlighted in red.

\vskip 1cm

\noindent Best regards,\\
Han Wang

\vskip 1cm
\newpage
\section*{Reply to Referee 1}


\textit{
A periodically driven electric field is but one of several
non-equilibrium situations that may be of interest to the simulation
community.  Can the authors comment on how extensible their framework
is to non-periodic forces?  For instance, what about steered molecular
dynamics?  What if the work is far from equilibrium?  I believe the
group of K. Schulten has made available SMD pulling/unfolding
trajectories for deca-alanine. These simulations were performed at
multiple velocities, including one simulation that was fully
reversible.  It would be most outstanding for the authors to look into
retrieving that data to reconstruct a non-equilibrium MSM for a
larger, slightly more realistic system and for a different
non-equilibrium regime.
}\\

The current theoretical framework firstly reduces to an
time-homogeneous Markov process by the Floquet theory.  The extension
to the non-periodic driving forces is not straightforward, because in
general the Floquet theory is not applicable to for arbitrary driving
force. Nevertheless, the question raised by the referee is very
important, and we intend to address in future research.
\\

\textit{ Of course, alanine dipeptide is a toy system.  And, as the
  authors show, researchers may not be able to arrive at a perfect
  answer using their methods even for this toy system. I.e., the
  second non-trivial eigenvalue for their example had some error,
  already; therefore it would be my expectation that this would
  essentially be unresolvable (realistically) for biological systems
  of real interest.  As someone more in the applications field, I find
  this worrisome. It would be helpful for the authors to discuss the
  potential application of their framework to a realistic biological
  system (e.g., a normal size protein).
}\\

% \begin{minipage}{.8\linewidth}
%   \centering
%   \includegraphics[width=0.6\textwidth]{hitting.eps}
%   \captionof{figure}{Schematic plot of comparison between a continuous and time-discretized trajectories in hitting a core set.}
% \end{minipage}

\begin{figure}
  \centering
  \includegraphics[width=0.6\textwidth]{hitting.eps}
  \caption{Schematic plot of comparison between a continuous and time-discretized trajectories in hitting a core set.}
  \label{fig:tmp1}
\end{figure}

The lower accuracy of the second non-trivial time scale stems from the
time-disretization problem in core set hitting, which is graphically
illustrated in Fig.~\ref{fig:tmp1}.  The milestone process is defined
to change state when the trajectory hits a core set that is not the
one it came from.  In Fig.~\ref{fig:tmp1}, if the trajectory is
continuum, then the milestone process changes to state ``2'' at the
first hitting at core set $C_2$. In comparison, the first hitting
event would not be recognized if the trajectory is discretized, and
the milestone process changes to state ``2'' only at the second
hitting at core set $C_2$.\\

This time-discretizations error would not be significant if the
interested time-scale is much longer than the discretized time-step.
However, it would be seen if the they are comparable. In our case, the minimum discretization
time-step is the period $T=10$~ps, which is not well separated from the second non-trivial time-scale
that is 26.5~ps.\\

\begin{table}
  \centering
  \begin{tabular*}{0.9\textwidth}{@{\extracolsep{\fill}}l cccccc}\hline\hline
    & NEMD & MSM & MSM &MSM &MSM &MSM \\
    Time Discr. [ps] & 0.5 &  $10$ &  $20$ & $40$ & $80$ & $120$\\
    $t_1$ [ps] & 100.2 & 104.8 & 108.8 & 115.5 & 127.6 & 142.4 \\\hline\hline
  \end{tabular*}
  \caption{The leading non-trivial time-scale computed by NEMD and MSM at different time-discretizations.
  The smallest time-discretization of MSM is constrained by the period of external driving $T$ that is 10~ps.}
  \label{tab:tmp1}
\end{table}
In order to further illustrate the artifact of time-discretizations,
we discretized the trajectory by increasing time-steps, and compare
the leading time-scale of the derived MSM with the NEMD, see Tab.~\ref{tab:tmp1}.
It can be easily concluded that as the time-discretizations increases and approaches
the leading non-trivial time-scale $t_1 = 100.2$~ps, the accuracy of MSM
in computing $t_1$ decreases. When the time-discretizations increases to $4T = 40$~ps, the relative
error is over 15\%.
\\

In the applications of the more complex and realistic problems, if the
system has time-scale separation between the external driving period and the interested implied
time-scales, then good accuracy is expected. Generally speaking, in
more complex molecular system, the interested time-scales are longer
than the alanine dipeptide toy system. Therefore, we would expect 
higher accuracy in more complex systems if the period of external field is 10~ps.
\\

\textit{ Also, the authors and others in the field have used the
  Chapman-Komolgorov tests to assess convergence or consistency of the
  MSMs with the underlying simulations.  The authors should include
  this metric, or the authors should state why they are not including
  it.  In particular, it would be helpful for the authors to discuss
  (if they are not using it) and clarify for the community, when it is
  appropriate to use the CK test to assess convergence and when it is
  not. Application of CK has been mysteriously lacking in all the
  “real world” examples of MSM and this is an opportunity to help
  clarify its application.
}\\

The discretizations time-step in the milestone MSM approach should be
distinguished from the lag-time in the space full partitioning MSM
approach. Larger time-discretizations in milestone MSM generally gives
worse prediction of implied time-scales, whereas larger lag-time in
space full partitioning MSM leads to better accuracy as long as the
lag-time is smaller than the interested time-scale. The
Chapman-Komolgorov test basically checks the consistency of full
partitioning MSM with different lag-times, and it does not apply to
the milestone MSM approach.

\begin{figure}
  \centering
  \includegraphics[width=0.48\textwidth]{fig-coreset-prob-chapman-ar.eps}
  \includegraphics[width=0.48\textwidth]{fig-coreset-prob-chapman-al.eps}
  \caption{The Chapman-Komolgorov test of state $\alpha_{\textrm{R}}$ (left) and $\alpha_{\textrm{L}}$ (right).}
\end{figure}

\end{document}
