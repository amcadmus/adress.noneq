\documentclass{article}

\addtolength{\textwidth}{2.5cm}
\addtolength{\hoffset}{-.4in}

\usepackage{graphicx}
\usepackage[margin=10pt,font=small,labelfont=bf]{caption}
\usepackage{lipsum}

\renewcommand{\baselinestretch}{1.0}
% \setlength{\footskip}{0cm}
%\addtolength{\textheight}{10cm}
% \sloppy

\renewcommand{\arraystretch}{1.2}



\begin{document}


\hfill\begin{tabular}{@{}p{.4\linewidth}@{}}
\multicolumn{1}{@{}c@{}}{} \\
        Han Wang\\
        CAEP Software Center for \\
        High Performance Numerical\\
        Simulation\\
        Huayuan Road 6\\
        Beijing 100088, China\\
        \\
        \today
\end{tabular}
\vskip .1cm
\noindent
\begin{tabular}{@{}p{.45\linewidth}@{}}
\multicolumn{1}{@{}c@{}}{} \\
Prof. William L. Jorgensen \\
\\
Yale University \\
Department of Chemistry\\
P.O. Box 208107 \\
New Haven, CT  06520-8107 \\
  USA\\  
\end{tabular}\hfill

\vskip 1cm

\noindent
\textbf{\large Resubmission for manuscript ct-2014-00997y}

\vskip .5cm
\noindent
Dear Prof. Jorgensen,\\

On behalf of the co-author, I would like to resubmit our article:
\textit{``Building Markov State Models for
  Periodically Driven Non-Equilibrium Systems''}
by H.~Wang and C.~Sch\"utte to the
\textit{Journal of Chemical Theory and Computation}.
We gratefuly acknowledge the reviewers for the positive reports, and for
their insight and suggestions, which helped us improving the quality
of the manuscript.  In response to their comments, the major questions raised by the referees are
reported below in detail.  

We revised the manuscript repsectively and reworked the English completely.
To help the referees spot the major changes in
the manuscript, we have submitted a separate file (highlight.pdf) with
all changes highlighted in red.

We hope that the revised manuscript is fit for publication now.

\vskip 1cm

\noindent Best regards,\\
Han Wang

\vskip 1cm
\newpage
\section*{Reply to Referee 1}


\textit{
A periodically driven electric field is but one of several
non-equilibrium situations that may be of interest to the simulation
community.  Can the authors comment on how extensible their framework
is to non-periodic forces?  For instance, what about steered molecular
dynamics?  What if the work is far from equilibrium?  I believe the
group of K. Schulten has made available SMD pulling/unfolding
trajectories for deca-alanine. These simulations were performed at
multiple velocities, including one simulation that was fully
reversible.  It would be most outstanding for the authors to look into
retrieving that data to reconstruct a non-equilibrium MSM for a
larger, slightly more realistic system and for a different
non-equilibrium regime.
}\\

The theoretical framework presented in the manuscript firstly reduces the NEMD process to a
time-homogeneous Markov process by utilizing Floquet theory.  An extension
to non-periodic driving forces is not straightforward, because in
general Floquet theory is not applicable to arbitrary driving
force. Nevertheless, apart from the aspects related to Floquet theory our general approach does well extend to more general driving forces so that
the question raised by the referee is very relevant: Applications of the approach presented in the manuscript to approach like the one of K. Schulten is one of the final goals. \\

However, one should be aware that our approach is \emph{the first} application of the MSM idea to NEMD. We think that the combination of the results presented here with the approach of the ideas presented in Sarich et al., Markov State Models for Rare Events in Molecular Dynamics. Entropy, 16 (1). pp. 258-286 (2013)  will allow for addressing the question of the referee in future research.\\


\textit{ Of course, alanine dipeptide is a toy system.  And, as the
  authors show, researchers may not be able to arrive at a perfect
  answer using their methods even for this toy system. I.e., the
  second non-trivial eigenvalue for their example had some error,
  already; therefore it would be my expectation that this would
  essentially be unresolvable (realistically) for biological systems
  of real interest.  As someone more in the applications field, I find
  this worrisome. It would be helpful for the authors to discuss the
  potential application of their framework to a realistic biological
  system (e.g., a normal size protein).
}\\

% \begin{minipage}{.8\linewidth}
%   \centering
%   \includegraphics[width=0.6\textwidth]{hitting.eps}
%   \captionof{figure}{Schematic plot of comparison between a continuous and time-discretized trajectories in hitting a core set.}
% \end{minipage}

\begin{figure}
  \centering
  \includegraphics[width=0.6\textwidth]{hitting.eps}
  \caption{Schematic plot of comparison between a continuous and time-discretized trajectories in hitting a core set.}
  \label{fig:tmp1}
\end{figure}

The lower accuracy of the second non-trivial timescale stems from the
time-disretization problem in core set hitting, which is graphically
illustrated in Fig.~\ref{fig:tmp1}.  The milestone process is defined
to change state when the trajectory hits a core set that is not the
one it came from.  In Fig.~\ref{fig:tmp1}, the milestone process based on the time-continuous trajectory changes to state ``2'' at the
first hitting at core set $C_2$. In comparison, the first hitting
event would not be recognized using the time-discretized trajectory; in this case 
the milestone process would change to state ``2'' only at the second
hitting at core set $C_2$. This respective error in detecting the correct hitting time would not be significant if the
first hitting time is much longer than the discretization time-step.
However, it will become relevant whenever they are comparable. In our case, the minimum discretization
time-step is the period $T=10$~ps, which is not well separated from the second non-trivial time-scale
that is 26.5~ps. Therefore, we observe an unavoidable deviation in the second timescale.\\

\begin{table}
  \centering
  \begin{tabular*}{0.9\textwidth}{@{\extracolsep{\fill}}l cccccc}\hline\hline
    & NEMD & MSM & MSM &MSM &MSM &MSM \\
    Time Discr. [ps] & 0.5 &  $10$ &  $20$ & $40$ & $80$ & $120$\\
    $t_1$ [ps] & 100.2 & 104.8 & 108.8 & 115.5 & 127.6 & 142.4 \\\hline\hline
  \end{tabular*}
  \caption{The leading non-trivial time-scale computed by NEMD and MSM at different time-discretizations.
  The smallest time-discretization of MSM is constrained by the period of external driving $T$ that is 10~ps.}
  \label{tab:tmp1}
\end{table}
In order to further illustrate the artifact of time-discretizations,
we discretized the trajectory by increasing time-steps, and compare
the \emph{leading} timescale $t_1$ of the derived MSM with the NEMD, see Tab.~\ref{tab:tmp1}.
It can be easily seen that as the discretization time-step increases and approaches
the leading non-trivial time-scale $t_1 = 100.2$~ps, the accuracy of MSM
in computing $t_1$ decreases. 
\\

In applications to more complex and realistic problems (larger systems), in particular for
systems exhibiting timescale separation between the external driving period and the interested implied
timescales, good accuracy is expected (and in fact guaranteed by general MSM theory). Generally speaking, in
more complex molecular system, the  timescales of interest are longer
than in the alanine dipeptide toy system. 
\\

\textit{ Also, the authors and others in the field have used the
  Chapman-Komolgorov tests to assess convergence or consistency of the
  MSMs with the underlying simulations.  The authors should include
  this metric, or the authors should state why they are not including
  it.  In particular, it would be helpful for the authors to discuss
  (if they are not using it) and clarify for the community, when it is
  appropriate to use the CK test to assess convergence and when it is
  not. Application of CK has been mysteriously lacking in all the
  “real world” examples of MSM and this is an opportunity to help
  clarify its application.
}\\

The discretization time-step in the milestoning MSM approach should be
distinguished from the lag-time in the full partition MSM
approach. Larger discretization time-steps in milestoning MSM generally result in
worse prediction of implied timescales (as discussed above), while larger lag-times in
full partition MSMs lead to better accuracy as long as the
lag-time is smaller than the time-scales of interest. The
Chapman-Komolgorov test basically checks the consistency of full
partition MSM with different lag-times, and thus does not apply to
milestone MSMs. See Schütte et al. (2011), Markov State Models Based on Milestoning, J. Chem. Phys., 134 (20). 204105 (2011) for more details.


\end{document}
