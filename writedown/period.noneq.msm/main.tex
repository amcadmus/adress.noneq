\documentclass[preprint,unsortedaddress,a4paper,onecolumn]{revtex4-1}
% \documentclass[acs, jctcce, a4paper,preprint,unsortedaddress,onecolumn]{revtex4-1}
% \documentclass[aps,pre,twocolumn,unsortedaddress]{revtex4-1}
% \documentclass[aps,jcp,groupedaddress,twocolumn,unsortedaddress]{revtex4}

\usepackage[fleqn]{amsmath}
\usepackage{amssymb}
\usepackage[dvips]{graphicx}
\usepackage{color}
\usepackage{tabularx}
\usepackage{algorithm}
\usepackage{algorithmic}

\makeatletter
\makeatother

\newcommand{\recheck}[1]{{\color{red} #1}}
\newcommand{\redc}[1]{{\color{red} #1}}
\newcommand{\bluec}[1]{{\color{red} #1}}
\newcommand{\greenc}[1]{{\color{green} #1}}
\newcommand{\vect}[1]{\textbf{\textit{#1}}}
\newcommand{\dd}[1]{\textsf{#1}}
\newcommand{\fwd}[0]{\textrm{fwd}}
\newcommand{\bwd}[0]{\textrm{bwd}}
\newcommand{\period}[0]{T_{\textrm{P}}}
\newcommand{\ml}[0]{\mathcal {L}}
\newcommand{\mo}[0]{\mathcal {O}}
\newcommand{\mbp}[0]{\mathbb {P}}
\newcommand{\mc}[0]{\mathcal {C}}
\newcommand{\dt}[0]{\Delta t}

\newcommand{\confaa}[0]{{\alpha_{\textrm{R}}}}
\newcommand{\confab}[0]{{\alpha_{\textrm{R}}'}}
\newcommand{\confba}[0]{{\textrm{C}7_{\textrm{eq}}}}
\newcommand{\confbb}[0]{{\textrm{C}5}}
\newcommand{\confc}[0]{{\alpha_{\textrm{L}}}}



\begin{document}

\title{Building Markov State Model for a Periodically Driven Non-Equilibrium System}
\author{Han Wang}
\email{han.wang@fu-berlin.de}
\affiliation{Zuse Institut Berlin, Germany}
\author{Christof Sch\"utte}
\email{Christof.Schuette@fu-berlin.de}
\affiliation{Institute for Mathematics, Freie Universit\"at Berlin, Germany}
\affiliation{Zuse Institut Berlin, Germany}
   
\begin{abstract}
\end{abstract}

\maketitle

\section{Discretization of the non-equilibrium molecular dynamics}

We denote the phase space by $\Omega$, and the variable by $\vect
x$. The time-dependent probability distribution on the phase space is
denoted by $\rho(\vect x, t)$. The dynamics of the system is governed
by
\begin{align}
  \label{eqn:tmp1}
  \frac{\partial \rho(\vect x, t)}{\partial t} = \ml(t) \rho(\vect x,t),
\end{align}
where the generator $\ml(t)$ is time-dependent and assumed to be
periodic, i.e.~$\ml(t) = \ml(t+T)$, with period denoted by $T$.  Now
we discretize the phase space $\Omega$ into finite number of disjoint
sets $\{ S_1, \cdots, S_n\}$, which satisfy $\Omega = \cup_i S_i$,
$S_j\cap S_j = \emptyset,\ \forall i\neq j$.
Now the original dynamics of the system can be discretized to a time-dependent
Markov jump process:
\begin{align}
  \label{eqn:tmp2}
  \frac{d\vect p(t)}{dt} = \vect M(t)\cdot \vect p(t)
\end{align}
where
\begin{align}
  \vect p(t) = (p_1(t), \cdots, p_n(t))^T, \quad \textrm{with}\ p_i(t) = \int_{S_i} \rho(\vect x,t)d\vect x,
\end{align}
and the generator is approximated by finite difference scheme:
\begin{align}
  \label{eqn:tmp4}
  M_{ij}(t) \approx \frac{1}{\tau} \,[\, \mbp (\vect x_{t+\tau} \in S_j \vert \vect x_{t} \in S_i) - \delta_{ij} \,]
\end{align}
From Eqn.~\eqref{eqn:tmp1} to Eqn.~\eqref{eqn:tmp2} involves several
approximations:
\begin{enumerate}
\item Eqn.~\eqref{eqn:tmp2} assumes Markovianity, which is not
  necessarily true for a reduce dynamics $\vect p(t)$.
\item Numerical error from the finite difference introduced in
  Eqn.~\eqref{eqn:tmp4}. The lag-time $\tau$ should not be too large
  to control the finite difference error, and at the same time should
  not be too small, because otherwise the statistical error would be
  large, and the Markovianity would be disturbed.
\item Although the original dynamics defined by generator $\ml(t)$ is
  periodic, the reduced one defined by $\vect M(t)$ is not necessary
  periodic, i.e.~$\vect M(t) \neq \vect M(t+T)$ in general.  In the
  simulation of biomolecules, it is reasonable, in most cases to
  assume a time-scale separation in the system. For example, the
  conformational change of the molecules is much faster than the
  covalent bond vibrations. Therefore, if the discretion $\{ S_1,
  \cdots, S_n\}$ properly captures the slow dynamics of interest, in
  the sense that within the lag-time $\tau$, the system is fully
  relaxed to a local equilibrium in the set $S_i$, then it is clear
  that the reduce system is periodic. It suggests that the lag-time
  should be right in the spectrum gap: it is shorter than the dominant
  implied timescale, and longer than the fast-relaxing time-scales.
\end{enumerate}

\begin{figure}
  \centering
  \includegraphics[width=0.6\textwidth]{fig/confs/fig-begin-1.eps}
  \caption{The Ramachandran plot presents the main conformations of the alanine dipeptide.}
  \label{fig:tmp1}
\end{figure}

At this stage we do have any analytical result regarding the quality
of approximating Eqn.~\eqref{eqn:tmp1} by
Eqn.~\eqref{eqn:tmp2}. However, it would be useful to do some
numerical experiments. We have studied the alanine dipeptide system
under oscillatory EF, which was studied by
Ref.~\cite{wang2014exploring}. We use the two dihedral angles $\phi$
and $\psi$ as slow variables. The definition of the conformations
$\confaa$, $\confab$, $\confba$, $\confbb$ and $\confc$ are presented
by Fig.~\ref{fig:tmp1}.  We are interested in the time-dependent
probability of the conformations. In Fig.~\ref{fig:tmp2}, the brute
force molecular dynamics simulation result
(i.e.~Eqn.~\eqref{eqn:tmp1}) is compared with different discretization
methods, which give different generator in Eqn.~\eqref{eqn:tmp2}. For
clarity, only the probability of conformation $\confc$ is plotted.  In
the figure, ``5 sets'' means the $\phi$-$\psi$ space is discretized by
5 sets as shown in Fig.~\ref{fig:tmp1}, ``Uni. $M=$'' stands for the
uniform discretization on $\phi$-$\psi$ space, and $M$ denotes the
number of bins on both $\phi$ and $\psi$ directions.  Both $M=10$ and
20 methods present very good accuracy.  Please notice that for $M=10$,
for example, the number of discretized sets is $n=M^2=100$.  The
lag-time in all cases is chosen to be $\tau = 0.5$~ps. A test with
$\tau = 1.0$~ps shows good consistency with $\tau = 0.5$~ps.
The generator
$\vect M(t)$ is assumed to be periodic, and is estimated from the
trajectories from $t=300$ to 1000~ps.

The phenomena presented by Fig.~\ref{fig:tmp2} is very similar to the
equilibrium MSM: the quality of the MSM depends on how accurate the
eigenfunctions are approximated by the discretization. The ``5 sets''
method achieve better accuracy than the uniform $M=5$ method by using
much less DOFs, because it captures the positive and negative signs of
the eigenfunctions. With larger $M$ the discretization error of the
eigenfunctions becomes smaller, so the accuracy of the MSM is improved.

\begin{figure}
  \centering
  \includegraphics[width=0.5\textwidth]{fig/t040/fig-cg-prob.eps}  
  \caption{The time-dependent probability of conformation
    $\confc$. The brute force MD simulation is compared with different
    discretization methods.}
  \label{fig:tmp2}
\end{figure}



\section{Floquet theory}

The Floquet theory states that for a periodic system
Eqn.~\ref{eqn:tmp2}, the solution can be written as
\begin{align}
  \label{eqn:tmp5}
  \vect p(t)  = \vect Q(t) e^{t\vect B} \vect p(0)
\end{align}
where $\vect Q(t)$ is periodic and $\vect B$ is a constant matrix.  To
analyze the spectrum, it is not difficult to infer the implied
timescales, on which the system relaxes to the periodic steady states
defined by $\vect Q(t)$.
Assume that the fundamental solution to Eqn.~\ref{eqn:tmp2} is denoted by $\boldsymbol \Phi(t)$, then
the matrix $\vect B$ can be expressed by
\begin{align}
  e^{T\vect B} = \boldsymbol \Phi^{-1}(0)\cdot \boldsymbol \Phi(T)
\end{align}
It is fully justified by choosing $ \boldsymbol \Phi(0) = I$, then
$e^{T\vect B}$ is estimated by evolving the fundamental solutions by
one period.  Calculating the eigenvalues of $e^{T\vect B}$ reveals the
implied timescales on which the non-equilibrium dynamics converges to
the periodic steady state. However, it can only resolve the processes that are longer than the period $T$.
The largest non-trivial timescales is given in table
Tab.~\ref{tab:tmp1}, which corresponds very well to the dynamics of the  $\confc$ probability.


\begin{table}
  \centering
  \caption{The first non-trivial timescales calculated from $e^{T\vect B}$.
  }
  \begin{tabular*}{0.3\textwidth}{@{\extracolsep{\fill}}c   c}\hline\hline
    Method      &        $t_2$ [ps]  \\\hline
    5 sets      & 72.4     \\
    Uni. $M=05$ & 53.6     \\
    Uni. $M=10$ &107.2     \\
    Uni. $M=20$ &113.8     \\
    Uni. $M=30$ &120.6     \\
    \hline\hline
  \end{tabular*}
  \label{tab:tmp1}
\end{table}


\section{To do}

\begin{itemize}
\item Theoretical analysis of the accuracy of the discretization, a
generalization of the equilibrium case. the numerical results indicate
a very close link. The perodicity is very nice, if the generator is
``continuous'', then it should be uniformly continuous, then it should
be possible to develop estimates that are similar to the equilibrium
case.
\item More tests on the effect of choosing different lag-time $\tau$ (0.5 and 1.0~ps are done).
\item More theoretical consideration on the Floquet theory, what is the
explict form of matrix $\vect B$, and does it has any nice properties,
such as self-joint.
\end{itemize}


\section{Reversibility of the original dynamics}

Now we assume the overdamped dynamics
\begin{align}\label{eqn:tmp7}
  dx_t = -\nabla_x V(x_t,t) dt + \sigma dw_t
\end{align}
where $V$ is the time-dependent and $t$-periodic potential. For simplicity we denote the force by $F(x_t,t) = -\nabla_x V(x_t, t)$. $dw_t$ is
the standard Wiener process.  Assuming a discretization of the
stochastic process at time $0 < t_1 < t_2 < \cdots < t_N = T$, where
$t_i = iT / N$. We denote $x_i = x_{t_i}$, and $w_i = w_{t_i}$, then we have the
conditional probability:
\begin{align}\nonumber
  p(x_T,T\vert x_0,0) =\lim_{N\rightarrow\infty} &\,
  \frac{1}{(2\pi\sigma^2\dt)^{(N-1)/2}} \int dx_1\cdots\int dx_{N-1}\\\nonumber
  &\,
  \exp\bigg\{\frac1\sigma\sum_{i=0}^{N-1} F(x_i,t_i)(w_{i+1} - w_{i}) - \frac1{2\sigma^2}\sum_{i=0}^{N-1}F^2(x_i,t_i)\dt\bigg\} \times\\\label{eqn:tmp8}
  &\,
  \exp\bigg\{- \frac1{2\sigma^2} \sum_{i=0}^{N-1} \frac{(x_{i+1} - x_i)^2}{\dt}\bigg\}.
\end{align}
Notice that
\begin{align}\label{eqn:tmp10}
  \sigma(w_{i+1} - w_{i}) =
  &\,
  x_{i+1} - x_i 
\end{align}
we have
% so with the formal identity $(dw_t)^2 = dt$,  we have
\begin{align}\nonumber
  p(x_T,T\vert x_0,0)
  =\lim_{N\rightarrow\infty} &\,
  \frac{1}{(2\pi\sigma^2\dt)^{(N-1)/2}} \int dx_1\cdots\int dx_{N-1}\\\nonumber
  &\,
  \exp\bigg\{\frac1{\sigma^2}\sum_{i=0}^{N-1} F(x_{i},t_{i})(x_{i+1} - x_i) -\frac1{2\sigma^2}\sum_{i=0}^{N-1}F^2(x_i,t_i)\dt\bigg\} \times\\\label{eqn:tmp10}
  &\,
  \exp\bigg\{- \frac1{2\sigma^2} \sum_{i=0}^{N-1} \frac{(x_{i+1} - x_i)^2}{\dt}\bigg\}.
\end{align}
Now, consider a conjugate trajectory $x^\dagger_t = x_{T-t}$ that starts at $x_T$, ends at $x_0$. The conjugate dynamics is driven by  $F^\dagger(x^\dagger_t,t) = F(x^\dagger_t, T-t)$.
Since we have the symmetry $F(x_t, -t) = -F(x_t, t)$, the conjugate dynamics also satisfies Eq.~\eqref{eqn:tmp7}.
Therefore, we have the probability:
\begin{align}\nonumber
  p(x_0,T\vert x_T,0) = &\,
  p^\dagger(x^\dagger_T,T\vert x^\dagger_0,0)  \\\nonumber
  = &\,
  \lim_{N\rightarrow\infty}
  \frac{1}{(2\pi\sigma^2\dt)^{(N-1)/2}} \int dx^\dagger_1\cdots\int dx^\dagger_{N-1}\\\nonumber
  &\,
  \exp\bigg\{\frac1{\sigma^2}\sum_{i=0}^{N-1} F^\dagger(x^\dagger_{i},t_{i})(x^\dagger_{i+1} - x^\dagger_i) -\frac1{2\sigma^2}\sum_{i=0}^{N-1}[F^\dagger(x^\dagger_i,t_i)]^2\dt\bigg\} \times\\\nonumber
  &\,
  \exp\bigg\{- \frac1{2\sigma^2} \sum_{i=0}^{N-1} \frac{(x^\dagger_{i+1} - x^\dagger_i)^2}{\dt}\bigg\} \\\nonumber
  =&\,
  \lim_{N\rightarrow\infty} 
  \frac{1}{(2\pi\sigma^2\dt)^{(N-1)/2}} \int dx_1\cdots\int dx_{N-1}\\\nonumber
  &\,
  \exp\bigg\{\frac1{\sigma^2}\sum_{i=N}^{1} F(x_{i},t_{i})(x_{i-1} - x_i) -\frac1{2\sigma^2}\sum_{i=N}^{1}F^2(x_i,t_i)\dt\bigg\} \times\\\label{eqn:tmp11}
  &\,
  \exp\bigg\{- \frac1{2\sigma^2} \sum_{i=N}^{1} \frac{(x_{i-1} - x_i)^2}{\dt}\bigg\}.
\end{align}
The difference between the single trajectory probabilities is
\begin{align}\label{eqn:tmp12}
  \log\bigg\{\frac{  p[x_t] }{ p[x^\dagger_t]} \bigg\}
  =&\,
  \frac1{\sigma^2}\sum_{i=1}^{N-1} F(x_i,t_i)(x_{i+1} - x_{i}) + F(x_i,t_i)(x_{i} - x_{i-1}) 
\end{align}
Now consider the differentiation:
\begin{align}\nonumber
  F(x_{i},t_{i}) - F(x_{i-1},t_{i-1}) =
  &\,
  F(x_{i},t_{i}) - F(x_{i-1},t_{i}) + F(x_{i-1},t_{i}) -  F(x_{i-1},t_{i-1})\\\nonumber
  =&\,
  \nabla_x F(x_{i-1},t_{i})(x_i - x_{i-1}) + \mo(\dt) \\\label{eqn:tmp13}
  =&\,
  \nabla_x F(x_{i-1},t_{i-1})(x_i - x_{i-1}) + \mo(\dt)
\end{align}
Then the \eqref{eqn:tmp12} becomes
\begin{align}\label{eqn:tmp14}
  \log\bigg\{\frac{  p[x_t] }{ p[x^\dagger_t]} \bigg\}
  =&\,
  \frac2{\sigma^2}\sum_{i=0}^{N-1} F(x_i,t_i)(x_{i+1} - x_{i}) +
  \frac1{\sigma^2}\sum_{i=0}^{N-1}\nabla_xF(x_i,t_i)(x_{i+1} - x_{i})^2
\end{align}
We notice that
\begin{align}\nonumber
  dV(x, t) = &\, \frac{\partial V}{\partial x} dx + \frac{\partial V}{\partial t} dt\\\nonumber
  =&\,
  \frac12 \sigma^2 \nabla^2_x V dt +  \nabla V dx_t + \frac{\partial V}{\partial t} dt \\
  =&\,
  -\frac12 \sigma^2 \nabla_x F dt -  F dx_t + \frac{\partial V}{\partial t} dt
\end{align}
and the identity
\begin{align}
  dt = (dw_t)^2 = \frac1{\sigma^2} dx_t^2
\end{align}
Eq.~\eqref{eqn:tmp14} becomes
\begin{align}
  \log\bigg\{\frac{  p[x_t] }{ p[x^\dagger_t]} \bigg\}
  =&\,
  -\frac2{\sigma^2}\bigg[
  V(x_N,t_N) - V(x_0,t_0) + \sum_{i=0}^{N-1}\partial_tV(x_i,t_i)\dt
  \bigg]
\end{align}
According to the  Einstein relation $T = \sigma^2/2$, by denoting $\beta = 1/T$
\begin{align}
  \log\bigg\{\frac{  p[x_t] }{ p[x^\dagger_t]} \bigg\}
  =&\,
  -\beta\bigg[
  V(x_N,t_N) - V(x_0,t_0) + \sum_{i=0}^{N-1}\partial_tV(x_i,t_i)\dt
  \bigg]  
\end{align}
Therefore
\begin{align}\nonumber
  p(x_0,T\vert x_T,0)
  =&\,
  % \lim_{N\rightarrow\infty} 
  % \frac{1}{(2\pi\sigma^2\dt)^{(N-1)/2}} \int dx_{N-1}\cdots\int dx_{1}
  \int_{\mc\{x_0,0;x_T,T\}}
  \frac{  p[x^\dagger_t] }{ p[x_t]} \cdot p[x_t] \\\label{eqn:tmp19}
  = &\,
  \frac{e^{-\beta V(x_0,t_0)}}{e^{-\beta V(x_N,t_N)}}
  \int_{\mc\{x_0,0;x_T,T\}}
  \exp\bigg\{
  \beta\int_0^T \partial_t V(x_t,t)dt 
  \bigg\} \cdot p[x_t]
\end{align}
where $\mc\{x_0,0;x_T,T\}$ denotes all continuous trajectories starting at $x_0$ and ending at $x_T$.
If $\partial_t V = 0$, i.e.~equilibrium, we have
\begin{align}
  p(x_0,T\vert x_T,0)e^{-\beta V(x_T,T)} =  p(x_T,T\vert x_0,0) e^{-\beta V(x_0,0)},
\end{align}
which proves the reversibility of the equilibrium dynamics.
The term
\begin{align}
  W[x_t] = - \int_0^T \partial_t V(x_t,t)dt
\end{align}
is the non-equilibrium work associated to all possible
the dynamics $x_t$ starting at $x_0$ and ending at $x_T$ (see e.g.~Ref.~\cite{seifert2012stochastic}).
Therefore Eq.~\eqref{eqn:tmp19} is the detailed Jarzynski relation.
Now the problem becomes if we can write a nice form (e.g.~the difference of a state function measured at $x_T$ and $x_0$) for the non-equilibrium work.


\bibliography{ref}{}
\bibliographystyle{unsrt}



\end{document}
