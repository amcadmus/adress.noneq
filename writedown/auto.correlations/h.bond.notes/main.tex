\documentclass[unsortedaddress,a4paper,onecolumn]{revtex4}
% \documentclass[aps, pre, preprint,unsortedaddress,a4paper,twocolumn]{revtex4}
% \documentclass[acs, jctcce, a4paper,preprint,unsortedaddress,onecolumn]{revtex4-1}
% \documentclass[aps,pre,twocolumn,unsortedaddress]{revtex4-1}
% \documentclass[aps,jcp,groupedaddress,twocolumn,unsortedaddress]{revtex4}

\usepackage[fleqn]{amsmath}
\usepackage{amssymb}
\usepackage[dvips]{graphicx}
\usepackage{color}
\usepackage{tabularx}
\usepackage{algorithm}
\usepackage{algorithmic}

\makeatletter
\makeatother

\newcommand{\recheck}[1]{{\color{red} #1}}
\newcommand{\redc}[1]{{\color{red} #1}}
\newcommand{\bluec}[1]{{\color{red} #1}}
\newcommand{\greenc}[1]{{\color{green} #1}}
\newcommand{\vect}[1]{\textbf{\textit{#1}}}
\newcommand{\dd}[1]{\textsf{#1}}
\newcommand{\fwd}[0]{\textrm{fw}}
\newcommand{\bwd}[0]{\textrm{bw}}
\newcommand{\period}[0]{T_{\textrm{P}}}
\newcommand{\ml}[0]{\mathcal {L}}
\newcommand{\mo}[0]{\mathcal {O}}
\newcommand{\mbp}[0]{\mathbb {P}}
\newcommand{\mc}[0]{\mathcal {C}}
\newcommand{\dt}[0]{\Delta t}

\newcommand{\confaa}[0]{{\alpha_{\textrm{R}}}}
\newcommand{\confab}[0]{{\alpha_{\textrm{R}}'}}
\newcommand{\confba}[0]{{\textrm{C}7_{\textrm{eq}}}}
\newcommand{\confbb}[0]{{\textrm{C}5}}
\newcommand{\confc}[0]{{\alpha_{\textrm{L}}}}

\newcommand{\oxy}[0]{{\textrm{O}}}
\newcommand{\hyd}[0]{{\textrm{H}}}


\begin{document}

\title{Note on the hydrogen-bond auto-correlation}
\author{Han Wang}
\email{han.wang@fu-berlin.de}
\affiliation{Zuse Institut Berlin, Germany}
   
\begin{abstract}
  In this note I describe the implementation details of the hydrogen-bond auto-correlation.
\end{abstract}

\maketitle

\section{Definitions}
Here I use the definition of hydrogen-bond population from
Ref.~\cite{luzar1996effect, luzar1996hydrogen}.  The population $h(t)$
is defined to be 1 if a certain pair of molecules forms hydrogen-bond,
otherwise is defined to be 0. Therefore, the averaged total number of
hydrogen-bond in an equilibrium system is $\frac12 N(N-1)\langle
h\rangle$, where $\langle\cdot\rangle$ denotes the ensemble average.
The auto-correlation
\begin{align}
  c(t) = \frac{\langle h(0) h(t)\rangle}{\langle h\rangle}
\end{align}
is studied. The physical meaning is that given a hydrogen-bond formed
at time $0$, the probability of observing a hydrogen-bond between the
same pair at time $t$.
The rate of relaxation to equilibrium is of particular interest:
\begin{align}\label{eqn:k-def}
  k(t) &= -\frac{dc}{dt}\\\label{eqn:k-exp}
  &= -\frac{\langle \dot h(0) [1 - h(t)] \rangle}{\langle h\rangle}
\end{align}
where $\dot h(0) = (dh/dt)\vert_{t=0}$. The Eq.~\eqref{eqn:k-exp} holds because
\begin{align} \label{eqn:tmp3}
  \langle h(0) \dot h(t) \rangle &= - \langle \dot h(0) h(t)\rangle \\\label{eqn:tmp4}
  \langle \dot h(0) \rangle &= 0
\end{align}
Eq.~\eqref{eqn:tmp4} is obvious because the system is in equilibrium. Eq.~\eqref{eqn:tmp3} holds because
\begin{align}
  0 = \frac{d [ h(\tau) h(\tau+t) ] }{d\tau} =  \langle h(\tau) \dot h(\tau+t)\rangle  + \langle \dot h(\tau) h(\tau+t)\rangle.
\end{align}
Letting $\tau \rightarrow 0$ on both sides yields Eq.~\eqref{eqn:tmp3}.

\section{Implementation details regarding the rate of relaxation}

The rate of relaxation to equilibrium $k(t)$ can be calculated by
definition~\eqref{eqn:k-def}, or calculated from Eq.~\eqref{eqn:k-exp}
with an explicit estimation of the function $\dot h$. I want to discuss the second way.

Two water molecules are hydrogen-bonded if (1) the O--O distance is less than a cut-off $r_c$, \emph{and} (2)
the O--H...O angle is less than a cut-off $\theta_c$. To mathematically define the function $h$, I introduce
the Heaviside function:
\begin{equation}\label{eqn:heavi}
  H(x) =
  \left\{
    \begin{alignedat}{3}
      &1 &\quad& x\geq0, \\
      &0 &     & x < 0.
    \end{alignedat}
  \right.
\end{equation}
And important property of the Heaviside function is that
\begin{align}
  \delta (x) = \frac{dH(x)}{dx}
\end{align}
If I denote the DOFs of two water molecules by $\vect x_1$ and $\vect x_2$, then the hydrogen-bond population function $h$ is expressed by
\begin{align}
  h(\vect x_1, \vect x_2) = H [r_c - R(\vect x_1, \vect x_2)] \cdot H [\theta_c - \theta(\vect x_1, \vect x_2)]
\end{align}
where function $R(\vect x_1, \vect x_2)$ computes the O--O distance
between the molecules, while $\theta(\vect x_1, \vect x_2)$ computes
the O--H...O angle. In practice, it is not easy to calculate the
angle, but the cosine angle is relatively easy. Therefore I use the
definition 
\begin{align}
  h(\vect x_1, \vect x_2) = H [r_c - R(\vect x_1, \vect x_2)] \cdot H[\cos(\theta(\vect x_1, \vect x_2)) - \cos \theta_c]
\end{align}
If noticing the fact that the DOFs of the molecules are functions of
time, i.e.~$\vect x_1 = \vect x_1(t)$, and $\vect x_2 = \vect x_2(t)$, then
taking the derivative w.r.t.~time on both sides of the equation, we have
\begin{align}
  \frac{d h}{dt}
  =
  \frac{dH [r_c - R(\vect x_1, \vect x_2)]}{dt} \cdot H[\cos(\theta(\vect x_1, \vect x_2)) - \cos \theta_c] +
  H [r_c - R(\vect x_1, \vect x_2)] \cdot \frac{dH[\cos(\theta(\vect x_1, \vect x_2)) - \cos \theta_c]}{dt}
\end{align}
The two derivatives in the equation is computed by:
\begin{align}
  \frac{dH [r_c - R(\vect x_1, \vect x_2)]}{dt}
  &=
  -\delta[r_c - R(\vect x_1, \vect x_2)]
  \cdot  \frac{dR(\vect x_1, \vect x_2)}{dt} \\
  \frac{dH[\cos(\theta(\vect x_1, \vect x_2)) - \cos \theta_c]}{dt}
  &=
  \delta[\cos(\theta(\vect x_1, \vect x_2)) - \cos \theta_c]
  \cdot \frac{d \cos(\theta(\vect x_1, \vect x_2))}{dt}
\end{align}
In practice, the delta function can be approximated by
\begin{align}
  \delta(x) =
  \left\{
    \begin{alignedat}{3}
      &\frac 1{\Delta L}, &\qquad& -\frac{\Delta L}{2} \leq x < \frac{\Delta L}{2}, \\
      & 0, &     & \textrm{otherwise},
    \end{alignedat}
  \right.
\end{align}
with a small enough $\Delta L$.
Now the problem is to compute $dR(\vect x_1, \vect x_2)/dt$ and $d
\cos(\theta(\vect x_1, \vect x_2))/dt$. To compute $dR(\vect x_1,
\vect x_2)/dt$, I firstly derive the time-derivative of the distance
between any two positions $r_{12} = \vert \vect r_{12}\vert  = \vert \vect r_2 - \vect r_1\vert$:
\begin{align} 
  \frac{d\vert \vect r_2(t) - \vect r_1(t)\vert}{dt}
  &=
  \frac{( \vect r_2(t) - \vect r_1(t) )}{  \vert \vect r_2(t) - \vect r_1(t)\vert} \cdot
  \frac{ d( \vect x_2(t) - \vect x_1(t) )}{dt}  
  =
  \frac{( \vect r_2(t) - \vect r_1(t) ) \cdot (\vect v_2(t) - \vect v_1(t))}{  \vert \vect r_2(t) - \vect r_1(t)\vert} 
  =
  \frac{\vect r_{12}\cdot \vect v_{12}}{r_{12}}.
\end{align}
Then
\begin{align}
  \frac{dR(\vect x_1, \vect x_2)}{dt} = \frac{d r^{\oxy\oxy}_{12}}{dt} = \frac{\vect r^{\oxy\oxy}_{12}\cdot \vect v^{\oxy\oxy}_{12}}{r^{\oxy\oxy}_{12}}.
\end{align}
where $r_{12}^{\oxy\oxy}$ denotes the oxygen--oxygen distance between molecules 1 and 2. For the derivative of the cosine angle:
\begin{align}\nonumber
  \frac{d \cos(\theta(\vect x_1, \vect x_2))}{dt}
  &=
  \frac{d}{dt}
  \Big (
  \frac{\vect r_{11}^{\oxy\hyd}\cdot \vect r_{12}^{\oxy\oxy}}{r_{11}^{\oxy\hyd} \, r_{12}^{\oxy\oxy}}
  \Big ) \\\nonumber
  &=
  \frac{(\vect r_{11}^{\oxy\hyd}\cdot \vect r_{12}^{\oxy\oxy})'r_{11}^{\oxy\hyd}\, r_{12}^{\oxy\oxy} - \vect r_{11}^{\oxy\hyd}\cdot \vect r_{12}^{\oxy\oxy} (r_{11}^{\oxy\hyd}\cdot r_{12}^{\oxy\oxy})'}{ ( r_{11}^{\oxy\hyd}\, r_{12}^{\oxy\oxy} )^2} \\
  & =
  \frac{\vect v_{11}^{\oxy\hyd}\cdot \vect r_{12}^{\oxy\oxy}  + \vect r_{11}^{\oxy\hyd}\cdot \vect v_{12}^{\oxy\oxy}}{ r_{11}^{\oxy\hyd}r_{12}^{\oxy\oxy}}
  -
  \frac{\vect r_{11}^{\oxy\hyd}\cdot \vect r_{12}^{\oxy\oxy} } { ( r_{11}^{\oxy\hyd}\, r_{12}^{\oxy\oxy} )^2 } \times
  \frac{ \vect r_{11}^{\oxy\hyd}\cdot \vect v_{11}^{\oxy\hyd} (r_{12}^{\oxy\oxy})^2 + \vect r_{12}^{\oxy\oxy}\cdot \vect v_{12}^{\oxy\oxy} (r_{11}^{\oxy\hyd})^2  }{ r_{11}^{\oxy\hyd}\, r_{12}^{\oxy\oxy} }
\end{align}
The meaning of the notations should be self-explanatory.

\bibliography{ref}{}
\bibliographystyle{unsrt}

\end{document}
