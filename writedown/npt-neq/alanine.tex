% \documentclass[aip,jcp,preprint,unsortedaddress,a4paper,onecolum]{revtex4-1}
\documentclass[a4paper,preprint,unsortedaddress,onecolumn]{revtex4-1}
% \documentclass[aps,pre,twocolumn,unsortedaddress]{revtex4-1}
% \documentclass[aps,jcp,groupedaddress,twocolumn,unsortedaddress]{revtex4}

\usepackage[fleqn]{amsmath}
\usepackage{amssymb}
\usepackage[dvips]{graphicx}
\usepackage{color}
\usepackage{tabularx}
\usepackage{algorithm}
\usepackage{algorithmic}

\makeatletter
\makeatother

\newcommand{\recheck}[1]{{\color{red} #1}}
\newcommand{\redc}[1]{{\color{red} #1}}
\newcommand{\bluec}[1]{{\color{blue} #1}}
\newcommand{\greenc}[1]{{\color{green} #1}}
\newcommand{\vect}[1]{\textbf{\textit{#1}}}
\newcommand{\dd}[1]{\textsf{#1}}
\newcommand{\fwd}[0]{\textrm{fwd}}
\newcommand{\bwd}[0]{\textrm{bwd}}



\begin{document}

\title{Exploring Conformational Dynamics and Related Time Scales via Nonequilibrium Molecular Dynamics Simulation}
% \title{The non-equilibrium molecular dynamics simulation of the alanine dipeptide under electric field}
\author{Han Wang}
\affiliation{Institute for Mathematics, Freie Universit\"at Berlin, Germany}
\author{Christof Sch\"utte}
\affiliation{Institute for Mathematics, Freie Universit\"at Berlin, Germany}
\affiliation{Zuse Institute Berlin, Germany}
\author{Giovanni Ciccotti}
\affiliation{Department of Physics, University of Roma "La Sapienza", Italy}
\author{Luigi Delle Site}
\affiliation{Institute for Mathematics, Freie Universit\"at Berlin, Germany}

\begin{abstract}
\end{abstract}

\maketitle


\section{Introduction}


The possible effect of the electric field (EF) on the conformation of
proteins gains increasing attention recently, because it is closely
related to the health problem of exposing human tissue in the
electromagnetic radiation. Despite the importance of this problem, a
complete understanding of how the EF influence the  protein
is still lacking, therefore, a considerable number of
experimental~\cite{bohr2000microwave, bohr2000microwave-1,
  dePomerai2000cell, inskip2001cellular, mancinelli2004non} and
theoretical~\cite{budi2005electric, budi2007effect,
  budi2008comparative, toschi2008effects, astrakas2011electric,
  astrakas2012structural, damm2012can, starzyk2013proteins,
  english2009nonequilibrium, solomentsev2012effects} studies have been devoted to this topic,
and the references listed above is far from complete.

The molecular dynamics (MD) simulation has been proved to be useful in
understanding the behavior of a protein (or a short peptide segment)
in both static and oscillatory EF~\cite{budi2005electric, budi2007effect, budi2008comparative,
  toschi2008effects, astrakas2011electric, astrakas2012structural,
  damm2012can, starzyk2013proteins, english2009nonequilibrium,
  solomentsev2012effects}. In these studies, the initial
configurations are usually taken from the Protein Data Bank (PDB).
And then the temperature controlled simulation are performed for 
a few to tens of nanoseconds, during which the interested
conformational changes can be observed. Along the trajectories, the
secondary structure, root mean square displacement (RMSD), the dipole
moment and some shape parameters are calculated and investigated by
either plotted as a function of time, or doing time average over the
whole simulation. This approach release useful information on how the
secondary structure is changed by external  EF.  However,
by plotting the quantities against time, one may observe strong
fluctuation in plots, which hintered the quantitative and conclusive
analysis on what is the trend of the conformational change, when
it happens and what is the strength.
Although the time average reduces the statistical uncertainty,
it is indeed not well defined for a nonequilibrium process
such as relaxing process to the new conformations under a static
EF, or the dynamics under an oscillating EF.

The dynamical nonequilibrium molecular dynamics simulation (D-NEMD)
was firstly developed by G. Ciccotti and his
colleagues~\cite{ciccotti1975direct, ciccotti1979thought}, and
recently successfully applied to the study of
hydrodynamics~\cite{orlandini2011hydrodynamics,
  orlandini2011hydrodynamics-01}. D-NEMD provides a way of evaluating
nonequilibrium observables by doing an ensemble of nonequilibrium MD
simulations.  It can be used for analyzing the
conformational changing under a generally time-dependent EF.
In the present paper, we study the minimal model peptide
segment -- alanine dipeptide. We take the approach of starting from a
single peptide piece rather than a more realistic chain, firstly because it
is easier to draw clear conclusions on the nonequilibrium properties
of a single peptide by excluding all possible interplay between
different peptide segments along a chain.
Secondly, the current computational resource available is not enough
to study peptide chains by D-NEMD, since one needs to simulate an
ensemble of MD trajectories (typically a few thousands), so it
requires much higher computational cost than producing a single
equilibrium MD trajectory.

In this paper, we firstly describe the idea of the D-NEMD, then
introduce the
observables for analyzing the nonequilibrium properties
(i.e., the dipole moment, time-dependent properties of being a certain
conformation and the net
probability flux). The numerical results of alanine
dipeptide under static and sine-shaped oscillating EF are
presented. By calculating the nonequilibrium properties from the D-NEMD
simulations, we are able to answer how the peptide
changes the conformation in a EF, when do these changes happens,
and how strong are they.
As far as we know, our simulation is the first D-NEMD simulation
of alanine dipeptide under EF, and provides new
information helping people better understand the nonequilibrium
effect of EF on the protein conformation.



\section{Methodology}
\subsection{Nonequilibrium molecular dynamics simulation}
Here we fastly review the approach of performing dynamical nonequilibrium
molecular dynamics (D-NEMD) that was initiated by Giovanni Ciccoti and his
colleagues~\cite{ciccotti1975direct, ciccotti1979thought,
  orlandini2011hydrodynamics, orlandini2011hydrodynamics-01}.
We denote the macroscopic nonequilibrium observable by $O(t)$. If at $t$
the configurational probability distribution is $\rho(\vect x, t)$, where
$\vect x$ is the phase space variable, then the observable can be
expressed by
\begin{align}\label{eqn:tmp1}
  O(t) = \int d\vect x\, \hat O(\vect x)\rho(\vect x, t)  = \langle \hat O(\vect x), \rho(\vect x, t)\rangle,
\end{align}
where $\hat O (\vect x)$ is the microscopic observable, 
which is measured at the phase space position $\vect x$,
comparing with the macroscopic observable that is not phase space independent.
In this work, we always assume that the initial probability distribution
$\rho(\vect x, 0)$ is known, and is equal to the equilibrium distribution
without any electric field.
The bracket on the right hand side of Eq.~\eqref{eqn:tmp1} denotes the inner produce in the
phase space.  We assume the dynamics of the system is governed by the
Hamiltonian equation, i.e. $\dot {\vect x} = J \cdot \nabla_{\vect x}
\mathcal H(\vect x)$, where $\mathcal H$ is the Hamiltonian, and $J$ is
the symplectic matrix, then the Liouville equation of the probability
distribution is
\begin{align}\label{eqn:tmp2}
  \frac{\partial \rho(\vect x, t)}{\partial t} = - iL(t) \rho(\vect x, t),
\end{align}
where $iL(t) = \{\cdot, \mathcal H\}$ is the Liouville operator.
The Eq.~\eqref{eqn:tmp2}
can be formally solved by $\rho(\vect x, t) = e^{-iL(t)} \rho(\vect x, 0)$.
On the other hand
\begin{align}
  \frac{d \hat O(\vect x(t))}{dt} = \nabla_{\vect x}\hat O\cdot \dot{\vect x}
  = \nabla_{\vect x}\hat O\cdot J\cdot \nabla_{\vect x}\mathcal H
  = iL(t) \hat O (\vect x(t))
\end{align}
This equation can be formally solved by $\hat O(\vect x(t)) = e^{iL(t)} O(\vect x, 0)$, therefore,
\begin{align}\nonumber
  O(t) & = \langle \hat O(\vect x), \rho(\vect x, t)\rangle
  = \langle \hat O(\vect x), e^{-iL(t)} \rho(\vect x, 0)\rangle
  = \langle e^{iL(t)}\hat O(\vect x), \rho(\vect x, 0)\rangle\\\label{eqn:tmp4}
  &= \langle \hat O(\vect x(t)), \rho(\vect x, 0)\rangle
\end{align}
Since we assume that the initial
probability distribution is the equilibrium distribution, an
explicit translation of Eq.~\eqref{eqn:tmp4} is that the
nonequilibrium observable is equal to the ensemble average of
microscopic observable measured along trajectory $\vect x(t)$, the
initial configuration of which is subject to the \emph{equilibrium}
distribution. In practice, we firstly run an equilibrium MD simulation
to generate a sampling of configurations that are subject to the
equilibrium distribution. And then by using these configurations as
initial configuration, the Hamiltonian dynamics are integrated until
time $t$. These trajectories are also called \emph{branching
  trajectories} in this paper. Finally, the macroscopic observable is
estimated by averaging the microscopic observable measured at the end
points of the trajectories.

Being more generalized, if the microscopic observable depends not only on the end
configuration at time $t$ but also depends on the historical
configurations along the trajectory, the expression~\eqref{eqn:tmp4} can be generalize to
\begin{align}
  O(t) = \int d\vect x\,\rho(\vect x, 0) \int_{\mathcal C\{\vect x, 0; t\}} \hat O[\vect x_s] \,d\mathcal P[\vect x_s] 
\end{align}
where the microscopic observable $ \hat O[\vect x_s] $ is now a
functional of the trajectory.  $\mathcal C\{\vect x, 0; t\}$ is the
set of all continuous trajectories starting at point $\vect x$ at time
0, and ending at time $t$. $\mathcal P[\vect x_s] $ is the measure of
the trajectory space $\mathcal C$.  In our case, it is the
$\delta$-function peaked around the trajectories governed by the
Hamiltonian dynamics. To be general, it can also be the measure of all
trajectories generated by, for example, a Langevin dynamics, then the
branching trajectories should be generated by the Langevin
dynamics rather than the Hamiltonian dynamics.


\subsection{Nonequilibrium temperature control}\label{sec:tmp2b}

The aforementioned D-NEMD simulation algorithm
does not describe how the boundary of the system should be set up.
The very naive idea of doing the simulation
is to simulate a infinitely large isolated system, or any large enough
isolated subsystem of it. In practice, it is both impossible and unnecessary to
do such a simulation, due to the unaffordable computational expense.
Therefore, proper boundary condition should be supplied
to truncated the simulation region. It is only possible when we
have a clear knowledge on
on how the nonequilibrium system is set up,
and what are the nonequilibrium observables to measure.

\begin{figure}
  \centering
  \includegraphics[width=0.3\textwidth]{fig/thermostat.eps}
  \caption{Schematic plot of the truncated nonequilibrium subsystem.
    The white part is the Hamiltonian dynamics subsystem
    and the gray part is thermostating region. The whole system
    is subject to the periodic boundary condition.}
  \label{fig:tmp2}
\end{figure}

Since we are mainly interested in the
nonthermal~\cite{delaHoz2005microwaves} effect of the electric field
to the configuration of a solvated alanine dipeptide, we assume the
alanine is embeded into an infinitely large solvent environment, and
the electric field is only applied to the neighborhood of the alanine.
Any extra heat generated by the electric field can be effectively
absorbed by the solvent environment.  It is argued that such an
accurate reaction temperature control is essential for being able to
perform reproducible experiments~\cite{damm2012can}.
Now the question comes to how to truncate the solvent environment
so that the system can be handled by the limited computer resource.
% In practice, it is impossible to simulate an infinitely large
% environment.
For an equilibrium simulation, the solution is to couple
the system to a thermostat, under which the canonical ensemble can be
sampled by only simulating a finite size system with periodic boundary
condition. For a nonequilibrium simulation, simply couple the system
to a thermostat may not be a good idea, because the \emph{dynamics}
generated by the thermostat is artificial, and deviation from the
Hamiltonian dynamics under investigation will introduce artificial
effects in the nonequilibrium observations. To solve this problem, we
observe that only preserving the Hamiltonian dynamics of the alanine 
itself and the water molecules in the nearby solvation shell is
crucial to the interested nonequilibrium observable
(in this work, it is the conformational change of alanine),
while the detailed dynamics of
the water molecules far away is of less importance. 
Therefore, instead of simulating an infinitely
large system, we only simulate a finite size system with periodic
boundary condition. The simulating region is divided into two
subregions: Near the alanine dipeptide, the dynamics is kept to be
Hamiltonian, and this subregion is called \emph{the Hamiltonian dynamics
region}. While far from the alanine, the dynamics of water is coupled
to a Langevin thermostat, and this region is called \emph{the thermostating
region} (see Fig.~\ref{fig:tmp2}).  For simplicity, the Hamiltonian
dynamics region is assumed to be spherical, centered at the
alpha-carbon of the alanine, and with radius of $R_{ex}$.  Further, we
freeze the movement of the alpha-carbon, so that the alanine is always
located at the center of the simulation region.
Therefore, the dynamics is preseved in the Hamiltonian dynamics region (where the properties of interest are observed)
and the artificial effect of thermostating is negligible.
At the same time, the thermostating region works as a infinitely large
environment that effectively absorbs the extra heat, so that
the nonthermal effect can be studied.
The validity of the above
statement will be later check latter by numerical examples showing that the
nonequilibrium observables do not depends on the size of the system
and the size of the Hamiltonian dynamics subregion, if they are
reasonably large and the finite size effect does not play a dominating role.


% Notice again we actually
% do not want to precisely calculate each trajectory starting from the
% initial configurations,
% but want to correctly calculate the nonequilibrium observable,
% we 

\section{Example I: Alanine dipeptide
  under a constant electric field}

\subsection{System settings and simulation protocol}

In this section,
we want to study the nonequilibrium properties of an alanine dipeptide
under a constant EF.  At time $t=0$~ps, the system
has been fully equilibriated without any electric field. From $t=0$ to
$t=t_{\textrm{init}}$, the eclectic field is switched on linearly, while
after $t=t_{\textrm{init}}$, the field is kept constant at
$E_{\infty}$.  In this work we consider $t_{\textrm{init}} = 10$~ps,
and $E_{\infty} = 1$~V/nm.
The direction of this field is arbitrarily chosen: along the
$x$ direction. Since the dipole of the equilibrium configurations of
alanine can be along any direction, applying the electric along $x$-axis is
not biased.
Being more general, in this paper we denote the
electric field as a function of time $\vect E(t)$.
Therefore, the constant electric field is
$\vect E(t) = (E_\infty\cdot t/t_{init},0,0)$ as $0\leq t < t_{init}$, and 
$\vect E(t) = (E_\infty,0,0)$
as $t \geq t_{init}$.
The system is set up in a $2.7\times 2.7\times
2.7\, \textsf{nm}^3$ periodic simulation region, with one alanine dipeptide
described by the CHARMM27 force field, and solved in 644 TIP3P
water molecules.
The electric field exserts force $\vect F = q \vect E(t)$ on
the partial charge $q$ in the system.
The size of the  Hamiltonian dynamics region is $R_{ex} = 1.0$~nm.
All simulations are performed by a home-modified GROMACS 4.5~\cite{pronk2013gromacs}.
Firstly, an equilibrium NVT simulation of
100~\textsf{ns} was performed with a Langevin thermostat (time-scale
$\tau_T = 0.5~\textsf{ps}$) coupled to the system for constant
temperature.  Along the trajectory, configurations were taken every
50~\textsf{ps}, so we use 2000 initial configurations for each nonequilibrium
MD simulation, if not stated otherwise.
% They server as starting point of the branching
% trajectories.
The branching trajectories were integrated by the
Leap-frog scheme with the aforementioned nonequilibrium
temperature control technique.  The
time step was $\Delta t = 0.002~\textsf{ps}$. The short-range
interaction was smoothed from $0.8$ to $1.0~\textsf{nm}$ by the
``switch'' method provided by Gromacs.  An energy conserving PME
method was applied to calculate the electrostatic interaction in this
periodic system. In the thermostating region, the original dynamics was
coupled to a Langevin thermostat with $\tau_T = 0.1~\textsf{ps}$.
The whole system is also coupled to a Parrinello-Rahman barostat with $\tau_P = 2.0~\textsf{ps}$ to keep
the pressure at the ambient condition (1~Bar). Since the
change of the system size is small and slow, the pressure control
does not have an obviously effect on the
dynamics of the system.


%  (see also Fig.~\ref{fig:tmp3} for
%  the vanished initial dipole moment)

\subsection{Metastable sets}

\begin{figure}
  \centering
  \raisebox{0.2\height}{\includegraphics[width=0.22\textwidth]{fig.ala/ext.mode1.010.Ex.01.00.t1000ps.recheck/c-2.eps}}
  \includegraphics[width=0.37\textwidth]{fig.ala/ext.mode1.010.Ex.01.00.t1000ps.recheck/fig-begin-1.eps}
  \includegraphics[width=0.37\textwidth]{fig.ala/ext.mode1.010.Ex.01.00.t1000ps.recheck/fig-end-1.eps}
  \caption{The population of metastable states of the alanine dipeptide.
    (a): the
    equilibrium state at $t=0$~\textsf{ps}. (b): The new equilibrium
    state at $t=1000$~\textsf{ps}, i.e., fully relaxed under the
    constant electric field 1~V/nm. The darker color
    indicates larger population on the Ramachandran histogram.}
  \label{fig:tmp4}
\end{figure}


On each branching trajectory, to easy the analysis of the conformation
of alanine dipeptide, the molecular configuration is projected on
the Ramachandran histogram, as
shown in Fig.~\ref{fig:tmp4}. The equilibrium distribution of the
conformation is given in the left plot, and the fully relaxed
conformation under $E_{\infty} = 1$~V/nm is given in the right plot.  It is
clear the conformations are populated in several clusters, and the
position of the clusters do not change when the external electric
field is applied.
This means that only a few of the conformations are likely to be observed,
while there is nearly no opportunity to see others. 
These conformations are called ``metastable states''.
The strict definition of the
metastability is not exactly the same as those conformations that are likely to be observed.
Since in this paper we do not focus on the discussion of the definition for the metastability of the system,
we assume they have the same meaning.
We manually divide the
Ramachandran histogram into 5 subregions, see Fig.~\ref{fig:tmp4},
then the observation on the Ramachandran histogram is projected to the
observation of these 5 states, i.e. $\{A_1, A_2, B_1, B_2, C\}$, which
correspond to the molecular conformations we are investigating in this work.
These subregions
correspond to different secondary structure of a peptide chain:
$A_1$ and $A_2$ are
corresponding to the alpha helix. $B_1$ and $B_2$ are
corresponding to the beta sheet. $C$ is corresponding to the
left-handed alpha helix.


From Fig.~\ref{fig:tmp4}, it is obvious that the population in each
metastable state changes due to the external electric field, for
example, the population of state $C$ vanishes in equilibrium,
but plays an important role when the electric field is applied. We
therefore want to study how the alanine changes from one conformation
to the other, how fast is the change, and the relations to the 
the electric field. 
Firstly, we calculate the probability of being in a certain state that
is defined by:
\begin{align}
  P_I(t) = \mathbb P (\vect x_t \in I), \quad  I \in \{A_1, A_2, B_1, B_2, C\},
\end{align}
please notice that the time dependency of the probability, so it is
a nonequilibrium observable.
% which is the time dependent probability of the alanine being in metastable
% state $I$.
% Secondly, we study the
% forward and backward transition probability of the metastable conformations, which is defined by
% \begin{align}
%   P^{\fwd}_{J,I}(t) & = \mathbb P( \vect x_{t+\Delta t} \in J | \vect x_t \in I) \\
%   P^{\bwd}_{J,I}(t) & = \mathbb P( \vect x_{t-\Delta t} \in J | \vect x_t \in I) 
%   % \qquad I, J \in \{A_1, A_2, B_1, B_2, C\},
% \end{align}
% where $\Delta t$ is the time-scale of the observation.
% For a revserible and homogeneous Markov process, it is easy to show
% that $P^{\fwd}_{J,I} = P^{\bwd}_{J,I}$, and both of them are time independent. For a general nonequilibrium
% process considered by the present work, this relation is genarlly not guarenteed.
Secondly,
we study the net probability flux from metastable state $I$ to $J$, defined by:
\begin{align}\nonumber
  F_{J,I}(t) & =\lim_{\Delta t\rightarrow 0} \frac1{\Delta t} [\mathbb P( \vect x_{t-\Delta t} \in J, \vect x_t \in I) - \mathbb P( \vect x_{t-\Delta t} \in I, \vect x_t \in J)], \\\label{eqn:tmp7}
  & J,I \in \{A_1, A_2, B_1, B_2, C\}.
\end{align}
which presents the rate of how much conformation $J$ is changed
into $I$. A positive value indicates the
net flux from $J$ to $I$,
while a negative value indicates a net flux from $I$ to $J$.
Now project the original nonequilibrium process $\vect x_t$
onto the discretized states $\{A_1, A_2, B_1, B_2, C\}$,
if the resulting reduced process 
is time-reversible, then it is obvious that
$ F_{J,I}(t) = 0$. In an other word, the quantity $ F_{J,I}(t) $ represents
the irreversibility of the reduced process on
$\{A_1, A_2, B_1, B_2, C\}$.
In practice, to measure $ F_{J,I}(t)$, one need firstly approximate the limit
in Eq.~\eqref{eqn:tmp7} by the finite difference,
then calculate the joint probabilities on
the right-hand-side from the molecular trajectories.
If one wants a precise finite difference approximation, a smaller $\Delta t$ is
preferable,
but then the statistical uncertainty of measuring the joint probabilities
is higher. In practice,
we use a $\Delta t$ that is small enough for a precise finite difference approximation,
and not too small so that the joint probabilities can be calculated properly.
In the oscillating EF case, the net probability flux is also highly oscillating,
so it is convenient to study the integrated probability flux:
\begin{align}
  Q_{J,I} (t) = \int_0^t F_{J,I}(\tau)d \tau,
\end{align}
which presents the time averaged effect of the EF to the molecular conformation.





\subsection{Results and discussions}


\begin{figure}
  \centering
  \includegraphics[]{fig.ala/fig-meta-npt.eps}
  \caption{The probability of being in a metastable states under a constant EF.
    The warm-up time $t_{init} = 10$~ps. The red line stands for state $A_1$,
    green for $A_2$, dark blue for $B_1$, purple for $B_2$ and light blue
    for $C$.
  }
  \label{fig:tmp5}
\end{figure}

\begin{figure}
  \centering
  % \includegraphics[width=0.19\textwidth]{fig.ala/fig-trans-010-fwd-1.eps}
  % \includegraphics[width=0.19\textwidth]{fig.ala/fig-trans-010-fwd-2.eps}
  % \includegraphics[width=0.19\textwidth]{fig.ala/fig-trans-010-fwd-3.eps}
  % \includegraphics[width=0.19\textwidth]{fig.ala/fig-trans-010-fwd-4.eps}
  % \includegraphics[width=0.19\textwidth]{fig.ala/fig-trans-010-fwd-5.eps}\\
  % \caption{The populations of the metastable sets as a function of time.
  %   The dashed vertical lines indicate the three different time scale of
  % metastable set population at $t_{\textrm{init}} = 10~\textsf{ps}$.}
  % \includegraphics[width=0.19\textwidth]{fig.ala/fig-trans-010-bwd-1.eps}
  % \includegraphics[width=0.19\textwidth]{fig.ala/fig-trans-010-bwd-2.eps}
  % \includegraphics[width=0.19\textwidth]{fig.ala/fig-trans-010-bwd-3.eps}
  % \includegraphics[width=0.19\textwidth]{fig.ala/fig-trans-010-bwd-4.eps}
  % \includegraphics[width=0.19\textwidth]{fig.ala/fig-trans-010-bwd-5.eps}\\
  \includegraphics[width=0.19\textwidth]{fig.ala/fig-trans-010-flux-1.eps}
  \includegraphics[width=0.19\textwidth]{fig.ala/fig-trans-010-flux-2.eps}
  \includegraphics[width=0.19\textwidth]{fig.ala/fig-trans-010-flux-3.eps}
  \includegraphics[width=0.19\textwidth]{fig.ala/fig-trans-010-flux-4.eps}
  \includegraphics[width=0.19\textwidth]{fig.ala/fig-trans-010-flux-5.eps}\\
  \includegraphics[width=0.19\textwidth]{fig.ala/fig-trans-010-iflux-1.eps}
  \includegraphics[width=0.19\textwidth]{fig.ala/fig-trans-010-iflux-2.eps}
  \includegraphics[width=0.19\textwidth]{fig.ala/fig-trans-010-iflux-3.eps}
  \includegraphics[width=0.19\textwidth]{fig.ala/fig-trans-010-iflux-4.eps}
  \includegraphics[width=0.19\textwidth]{fig.ala/fig-trans-010-iflux-5.eps}
  % \includegraphics[width=0.19\textwidth]{fig.ala/fig-trans-100-iflux-1.eps}
  % \includegraphics[width=0.19\textwidth]{fig.ala/fig-trans-100-iflux-2.eps}
  % \includegraphics[width=0.19\textwidth]{fig.ala/fig-trans-100-iflux-3.eps}
  % \includegraphics[width=0.19\textwidth]{fig.ala/fig-trans-100-iflux-4.eps}
  % \includegraphics[width=0.19\textwidth]{fig.ala/fig-trans-100-iflux-5.eps}\\
  % \includegraphics[width=0.19\textwidth]{fig.ala/fig-trans-500-iflux-1.eps}
  % \includegraphics[width=0.19\textwidth]{fig.ala/fig-trans-500-iflux-2.eps}
  % \includegraphics[width=0.19\textwidth]{fig.ala/fig-trans-500-iflux-3.eps}
  % \includegraphics[width=0.19\textwidth]{fig.ala/fig-trans-500-iflux-4.eps}
  % \includegraphics[width=0.19\textwidth]{fig.ala/fig-trans-500-iflux-5.eps}
  \caption{
    The net probability flux and its integration under the constant EF case.
    The warm-up time $t_{init} = 10$~ps.
    The first row shows the probability flux $F_{J,I}(t)$ with unit $\textrm{ps}^{-1}$, while the second
    shows the integrated  probability flux denoted by $Q_{J,I}(t)$.
    A constant $Q_{J,I}$ means vanishing net flux, while a increasing (deceasing)
    $Q_{J,I}$ indicates positive (negative) net flux.
    From left to right, the five
    columns present $I = A_1$, $A_2$, $B_1$, $B_2$ and
    $C$, respectively. In each plot, the red line stands for $J=A_1$,
    green for $J=A_2$, dark blue for $J=B_1$, purple for $J=B_2$ and light blue
    for $J=C$. 
    }
  \label{fig:tmp6}
\end{figure}

% \begin{figure}
%   \centering
%   \includegraphics[width=0.19\textwidth]{fig.ala/fig-trans-100-fwd-1.eps}
%   \includegraphics[width=0.19\textwidth]{fig.ala/fig-trans-100-fwd-2.eps}
%   \includegraphics[width=0.19\textwidth]{fig.ala/fig-trans-100-fwd-3.eps}
%   \includegraphics[width=0.19\textwidth]{fig.ala/fig-trans-100-fwd-4.eps}
%   \includegraphics[width=0.19\textwidth]{fig.ala/fig-trans-100-fwd-5.eps}\\
%   % \caption{The populations of the metastable sets as a function of time.
%   %   The dashed vertical lines indicate the three different time scale of
%   % metastable set population at $t_{\textrm{init}} = 10~\textsf{ps}$.}
%   \includegraphics[width=0.19\textwidth]{fig.ala/fig-trans-100-bwd-1.eps}
%   \includegraphics[width=0.19\textwidth]{fig.ala/fig-trans-100-bwd-2.eps}
%   \includegraphics[width=0.19\textwidth]{fig.ala/fig-trans-100-bwd-3.eps}
%   \includegraphics[width=0.19\textwidth]{fig.ala/fig-trans-100-bwd-4.eps}
%   \includegraphics[width=0.19\textwidth]{fig.ala/fig-trans-100-bwd-5.eps}\\
%   \includegraphics[width=0.19\textwidth]{fig.ala/fig-trans-100-flux-1.eps}
%   \includegraphics[width=0.19\textwidth]{fig.ala/fig-trans-100-flux-2.eps}
%   \includegraphics[width=0.19\textwidth]{fig.ala/fig-trans-100-flux-3.eps}
%   \includegraphics[width=0.19\textwidth]{fig.ala/fig-trans-100-flux-4.eps}
%   \includegraphics[width=0.19\textwidth]{fig.ala/fig-trans-100-flux-5.eps}
%   \caption{Forward and backward nonequilibrium transition probability of the case $t_{init} = 10$~ps.}
%   \label{fig:tmp6}
% \end{figure}


The time-dependent probability of the system in a certain metastable state is given in
Fig.~\ref{fig:tmp5}. The beta sheet conformation,
which presents at the beginning,
fades away as the system relaxes to the new equilibrium. The
probability of alpha helix $A_1$ conformation grows from 45\% to 70\%.
Alpha helix $A_2$  does not change a lot under the electric field.
The left-hand helix configuration $C$ noticeably grows from 0\% to 20\%.

We plot the net probability fluxes and their integration in Fig.~\ref{fig:tmp6}.
When $t$ is short, the net fluxes are generally non-zero.
When $t$ goes to infinity,
all fluxes converge to zero. This indicates
when the EF is switched on, the system firstly is driven from the old
equilibrium (in zero EF), and this is a nonequilibrium process.
After a long enough time, the system
is fully relaxed to the new equilibrium defined by the constant EF,
% it gradually relaxes to the new equilibrium,
so we do not see and change
in the population of the metastable states.

Even though the system is finally relaxed to the new equilibrium,
the speed of relaxation of each metastable state is different. 
From Fig.~\ref{fig:tmp6}, we observe mainly
three different timescales of the conformational relaxation:
(1) From
$t=0$ to approximately 10~\textsf{ps}, there is firstly
a flux from $A_1$ to $A_2$, then it goes to the opposite direction, i.e.
from $A_2$ to $A_1$.
A similar flux is also observed from $B_1$ to $B_2$.
% the population of $A_1$ firstly decreases and then increases to the original value,
% while the population of $A_2$ increases sharply. The plot of the net probability flux shows that 
% there is firstly a fast probability flux (conformational change) from
% $A_1$ to $A_2$, and then follows a probability flux with the same maganitude in the opposite direction, i.e.  from
% $A_2$ to $A_1$.
In the meanwhile, a net flux from all $\beta$-sheet conformations
($B_1$ and $B_2$) into $\alpha$-helix conformations ($A_1$, $A_2$ and
$C$) has been established.
% Also in this time-scale, strong and increasing probability flux from $B_2$ to $B_1$ is observed from Fig.~\ref{fig:tmp6}.
% The population of $B_1$ remains almost unchanged,
% because it  at the same time loses population to $\alpha$-helix conformations.
% Since  $B_2$ conformations not only changes into $B_1$, but also into $\alpha$-helix conformations, we observe a charp
% decrease of $B_2$ population between 10 and 20~ps.
(2), From $t=10$ to
100~\textsf{ps},
The net flux from  conformation $A_2$ to $A_1$ and from $B_2$ to $B_1$
gradually decays to 0. Conformational change from $B_2$ to $A_1$, $A_2$ and $C$
vanishes in about 50~ps.
The net flux from $B_1$  to $A_1$ and $C$ decay slower and 
will be discussed later.
At this stage, the probability of $B_2$ decreases to almost 0,
while the probability of $A_2$
converge to a non-vanishing value of approximately 0.06.
% the population of $B_2$ decreases by 0.15, and this
% configuration almost disappears. Interestingly, the population of $A_1$
% increases by the same amount, and the population of $A_2$ goes back to
% the initial value. The populations of $B_1$ and $C$ start increasing
% and decreasing, correspondingly. The speed of these changes are much
% slower than that of $B_2$ and $A_1$.
(3) At the time scale of $
500~\textsf{ps}$, the population of $A_1$, $B_1$ and $C$ converge to
the new equilibrium (system with $E(\infty) = 1~\textsf{V/m}$).
We observed slowing decaying probability flux from $B_1$ 
to $A_1$ and $C$.
The magnitude of the flux from $B_1$ to  $C$ is larger than that from  $B_1$ to  $A_1$.
% It is interesting to see that at nonequilibrium steady state, $A_1$ exchange
% population with $A_2$, and $B_1$ exchange
% population with $B_2$. However, when the $C$ conformation hardly changes
% to any other conformation, as indicated by the almost 1 value of the forward
% transition probability of set $C$. This implies that at the left-handed $\alpha$-helix is
% acutally more stable than  other conformations.
This time-scale is actually 50 times slower than speed, at which
the electric field is switched on ($t_{\textrm{init}} = 10~\textsf{ps}$).

% After aligning to the external EF, the alanine slowly adjust its confromation,
% and the new equilibrium is reached at slower time scales.

% To further test the dynamics of metastable population, two simulations
% of $t_{\textrm{init}} = 100~\textsf{ps}$ and $t_{\textrm{init}} =
% 500~\textsf{ps}$ are performed. The first time scale of $t =
% 20~\textsf{ps}$ disappears in both of the testing simulations. In the
% simulation of $t_{\textrm{init}} = 100~\textsf{ps}$, the population of
% $A$, $B_1$ and $C$ converges to the non-equilibrium steady state at
% around $500~\textsf{ps}$. This further verifies internal time scale of
% $A_1$, $B_1$ and $C$ presented in the testing case of
% $t_{\textrm{init}} = 10~\textsf{ps}$.  The dynamics of $B_2$ is
% faster: it decays to zero as soon as the external electric field is fully
% switched on. In the case of $t_{\textrm{init}} = 500~\textsf{ps}$, the
% growth of the external electric field is comparable to the longest
% internal time scale of system, which means that the changes of the
% perturbation to the system is so slow that the system can catch up, so
% the process can be treated as  quasi-equilibrium.
% The numerical results show the same story: the relaxation
% of all populations are around $500 \sim 600~\textsf{ps}$.


\begin{figure}
  \centering
  % \includegraphics[width=0.3\textwidth]{fig.ala/ele.field/field.eps}
  \includegraphics[width=0.45\textwidth]{fig.ala/field.dipole/fig-dipol-order-conf.eps}
  % \\
  % \includegraphics[width=0.18\textwidth]{fig.ala/confs/ext.mode1.010.Ex.01.00.t1000ps.recheck/a1-1.eps}
  % \includegraphics[width=0.18\textwidth]{fig.ala/confs/ext.mode1.010.Ex.01.00.t1000ps.recheck/a2-1.eps}
  % \includegraphics[width=0.18\textwidth]{fig.ala/confs/ext.mode1.010.Ex.01.00.t1000ps.recheck/b1-1.eps}
  % \includegraphics[width=0.18\textwidth]{fig.ala/confs/ext.mode1.010.Ex.01.00.t1000ps.recheck/b2-1.eps}
  % \includegraphics[width=0.18\textwidth]{fig.ala/confs/ext.mode1.010.Ex.01.00.t1000ps.recheck/c-1.eps}
  \caption{The
    $x$-component of the dipole moment, and the orientational order parameter
    of the alanine molecule as a
    function of time. The dashed line presents the strengths of the
    external electric field, while the solid line presents the dipole
    moments. The left vertical axis is for the dipole moment, while
    the right is for the orientational order parameter.
  }
  \label{fig:tmp3}
\end{figure}

The Fig.~\ref{fig:tmp3}
shows that the $x$-component of the dipole moment of the alanine dipeptide
reaches 85\% of the maximum value in only 20~ps, which is comparable to
the warm-up time $t_{init}$, and is 25 times
smaller than the slowest time scale of the conformational relaxation.
Then in the following 400~ps, the dipole slowly relaxes to the 
maximum value, that reads
6.8~Debye, and is  corresponding to a dipole energy of ca.
$-13.6$~kJ/mol. The main free energy barriers among the
metastable states are reported to be of order $20\sim 40$~kJ/mol~\cite{bohner2012algorithm},
which is comparable to the dipole energy. This explains the 
conformational changes under the electric field.
We calculate the averaged dipole moment of different metastable state
under the constant EF, and they are 6.8, 6.0, 5.1, 3.1 and 7.1~Debye
for $A_1$, $A_2$, $B_1$, $B_2$ and $C$, respectively.
Under a constant EF, the system will be driven towards those
conformations that with higher dipole moment, because the
energy of the system will be lowered by aligning the the dipole
to the external EF.
It is clear that
the $\alpha$-helix and left-handed $\alpha$-helix are of higher
dipole moments than $\beta$-sheet conformations, so the probability
of being in the $\alpha$-helix conformations were
observed being increased under the EF.  $A_1$ and $C$ are the most
probable conformations because their dipole moments are the highest
among all metastable conformation. Moreover, the $B_2$  vanishes, because
its dipole is significantly lower than other conformations.
% This also explains the two-time scale behavior of the molecular dipole
% moment.
From 0 to 20~ps, the probability of
lowest dipole conformation $B_2$ quickly
vanishes, so there is a sharp increment of the average molecular
dipole moment. Then until 500~ps, since the conformation slowly migrates from
$B_1$ to $A_1$ and $C$, and also since the dipole moment $A_1$ and $C$
is higher than $B_1$, the average molecular dipole moment slowly increases
to the maximum value.

We also consider the orientation of the alanine dipeptide as a function
of time.
The geometric direction of the alanine is defined by the red vector in
Fig.~\ref{fig:tmp3}, which is the angle bisector of the two black vectors.
The black vectors are define to connect the $\alpha$-carbon and the carbons
on the methyl groups. We define the orientational order parameter
\begin{align}
  S_\theta = \langle 3\sin^2\theta - 2\rangle,
\end{align}
where $\theta$ is the angle between the orientation of the alanine dipeptide
and the direction of the electric field. The order parameter
indicate if the molecule is perpendicular to the electric field.
If the molecule is perfectly perpendicular to the electric field, then $S_\theta = 1$;
If the molecule has no directional
preference at all, then $S_\theta = 0$;
If the molecule is perfectly parallel to the electric field, then $S_\theta = -2$.
From $t=0$ to roughly 10~ps, the orientational order parameter fastly
decreases from 0 to $-0.15$, which means a weak alignment of the molecule to the external field.
Then from $t=0$ to 500~ps, the molecule slowly changes to the orientation
that is perpendicular to the electric field.

% This indicate that when the EF is switched on, the alanine almost immediately
% changes its conformation so that the molecular dipole is aligned to the
% external EF.
% align itself to the
% direction of the external EF without much conformational change.
% And then, the conformation is slowly adjusted according to the
% external field (now it is the same as the molecular dipole),
% finally  the new equilibrium is reached at a much slower time scale
% (approximately 500~ps).


\subsection{Testing the finite-size effect}

\begin{figure}
  \centering
  \includegraphics[]{fig.ala/fig-meta-conv.eps}
  \caption{The populations of the metastable states as a function of time.
    The dashed vertical lines indicate the three different time scale of
  metastable state population at $t_{\textrm{init}} = 10~\textsf{ps}$.}
  \label{fig:tmp7}
\end{figure}

Just described in Sec.~\ref{sec:tmp2b}, we perform nonequilibrium MD
simulations in a finite size periodic system, and further divided it
into a Hamiltonian dynamics region and a thermostating region.
Since the periodic boundary condition and the division of the system
are artificial, 
we want to understand the finite-size effect of
these settings and the their influence to the simulation result.
Therefore, we
perform two additional simulations: one has box size $L=4.0$~nm and a
Hamiltonian dynamics region of radius $R_{ex} = 1.0$~nm, while the
other has box size $L=4.0$~nm and a Hamiltonian dynamics region of
radius $R_{ex} = 1.5$~nm, and compare them to the system
we used before, i.e. $L=2.7$~nm and $R_{ex} = 1.0$~nm.
Fig.~\ref{fig:tmp7} presents very good consistency of
the simulation results of the three systems, so the finite-size effect is
actually negligible in the system we used in the previous sections.



\section{Example II:
  periodically oscillating electric field}

We have tested the periodically oscillating electric field which has
a $\sin$-wave shape:
\begin{align}
  \vect E(t) = (E_0\sin(2\pi \omega t), 0, 0)
\end{align}
where $E_0$ is the strength of the field, which is chosen to be
1.0~V/nm.  $\omega$ is the frequency of the field, which is related to
the period $T$ by $\omega = 1/T$.  Here we tested three different
periods: 10, 40 and 200, which corresponding
to frequency 100, 25 and 5~GHz.
% The periods of the later two are
% commensurate to the time scales discovered by the nonequilibrium MD
% simulation of constant electric field. The first is to test the if the
% period is further reduced to smaller than 20~ps, there is any other
% timescale of interest would play a role. Although the periods are
% chosen according to the discovered time scales in the constant
% electric field simulation, the nonequilibrium phenonema of oscillating
% electric field cannot be derived from the constant field system.
% As far as the electric field changes, the internal time scale of the
% system also changes, therefore, the nonequilibrium probabilities is a
% complicated interplay of different internal time scales under
% different EF.
% We do not try to theoretically predict
% what will happen under the oscillating EF by a series constant EF simulation
% revealing the internal dynamics under different strength of EF, but just directly
% do the nonequilibrium MD simulation for the oscillating EF, and discuss
% the observed phenonema.


\begin{figure}
  \centering
  \includegraphics[width=0.48\textwidth]{fig.ala/fig-mode2-dipol.eps}
  \includegraphics[width=0.48\textwidth]{fig.ala/fig-mode2-order.eps}
  \caption{The nonequilibrium dipole moment (a) and
    orientation order parameter (b) of alanine dipeptide as a
    function of time. In plot (a),
    only the $x$-component of the dipole moment is
    shown. The black line: the static electric field with turning on
    time $t_{init} = 10$~ps. The red line: oscillating field with period
    10~ps. The green line: period 40~ps. The blue line: period 200~ps.}
  \label{fig:tmp8}
\end{figure}


Fig.~\ref{fig:tmp8} (a)
presents the $x$-component of the molecular dipole moment as
a function of time. The red, green and blue lines stands for the
results of $T=10$~ps, 40~ps and 200~ps, respectively. The black lines
shows the dipole moment under constant electric field
for reference.
It demonstrates that the periods of the molecular dipole moment
is the same as the periods of the oscillating EF. Therefore,
the molecular dipole moment
is able to response to the external EF almost immediately.
At $T=200$~ps, the maximum magnitude of the molecular
dipole moment is almost the same as the case of constant electric field, which
means that the changing of the electric field is so slow that the
alanine have enough time to 
relax its internal dipole (although not completely). However,
for $T=10$ and 40~ps, as the electric field changes faster,
although the alanine dipeptide can response to the electric field immediately,
it does not have enough time to fully relax the internal dipole, so we
see the maximum alanine's dipole moment reached is smaller.
% For even smaller period $T=10$~ps, the alanine does not have enough
% time  to properly align the molecular dipole according to the external
% field, so we see the maximum dipole moment is only about one third of the
% constant EF case.
Plot (b) of Fig.~\ref{fig:tmp9} presents the orientational order
parameter. The notations are the same as plot (a).
% From the constant
% EF simulation, we see that the response of order parameter has
% two time scales: In the fast time scale, the 
For all periodically oscillating cases the order parameter is much
smaller than the constant EF case. One possible reason is that the
relaxation of the order parameter is very slow, and the molecule does
not expose to a strong enough EF for a long enough time in a oscillating
EF. For  $T=10$ and 40~ps, the orientation of the alanine is only weakly
parallel to the EF. Please notice that for the constant EF case,
we also observe a quit alignment of molecular orientation to the EF
at time scale 10~ps. For $T=200$~ps, The molecule periodically
becomes perpendicular to the EF, but this directional preference is much
weaker than the constant EF case.

\begin{figure}
  \centering
  \includegraphics[width=0.32\textwidth]{fig.ala/fig-meta-mode2-0010.eps}
  \includegraphics[width=0.32\textwidth]{fig.ala/fig-meta-mode2-0040.eps}
  \includegraphics[width=0.32\textwidth]{fig.ala/fig-meta-mode2-0200.eps}
  % \includegraphics[width=0.49\textwidth]{fig.ala/fig-meta-mode2-2000.eps}
  \caption{ The probability of being in a certain metastable state for
    periodically oscillating electric field. Different periods,
    i.e. $T=10$~ps, 40~ps and 200~ps, are presented here.  The
    probability of being in a certain metastable state $I$, which is
    denoted by $P_I$ in this paper, is plotted by colored lines against
    time. The red line: $I = A_1$, the green line: $I = A_2$, the dark
    blue line: $I = B_1$, the pink line: $I = B_2$ and the light blue
    line: $I = C$. For $T=10$~ps and 40~ps, the nonequilibrium
    simulations of time 1000~ps are presented, while for $T=200$~ps,
    nonequilibrium simulations of time 3200~ps is presented.  }
  \label{fig:tmp9}
\end{figure}


\begin{figure}
  \centering
  % \includegraphics[width=0.19\textwidth]{fig.ala/fig-trans-mode2-0010-flux-1.eps}
  % \includegraphics[width=0.19\textwidth]{fig.ala/fig-trans-mode2-0010-flux-2.eps}
  % \includegraphics[width=0.19\textwidth]{fig.ala/fig-trans-mode2-0010-flux-3.eps}
  % \includegraphics[width=0.19\textwidth]{fig.ala/fig-trans-mode2-0010-flux-4.eps}
  % \includegraphics[width=0.19\textwidth]{fig.ala/fig-trans-mode2-0010-flux-5.eps}\\
  \includegraphics[width=0.19\textwidth]{fig.ala/fig-trans-mode2-0010-iflux-1.eps}
  \includegraphics[width=0.19\textwidth]{fig.ala/fig-trans-mode2-0010-iflux-2.eps}
  \includegraphics[width=0.19\textwidth]{fig.ala/fig-trans-mode2-0010-iflux-3.eps}
  \includegraphics[width=0.19\textwidth]{fig.ala/fig-trans-mode2-0010-iflux-4.eps}
  \includegraphics[width=0.19\textwidth]{fig.ala/fig-trans-mode2-0010-iflux-5.eps}\\
  % \includegraphics[width=0.19\textwidth]{fig.ala/fig-trans-mode2-0040-flux-1.eps}
  % \includegraphics[width=0.19\textwidth]{fig.ala/fig-trans-mode2-0040-flux-2.eps}
  % \includegraphics[width=0.19\textwidth]{fig.ala/fig-trans-mode2-0040-flux-3.eps}
  % \includegraphics[width=0.19\textwidth]{fig.ala/fig-trans-mode2-0040-flux-4.eps}
  % \includegraphics[width=0.19\textwidth]{fig.ala/fig-trans-mode2-0040-flux-5.eps}\\
  \includegraphics[width=0.19\textwidth]{fig.ala/fig-trans-mode2-0040-iflux-1.eps}
  \includegraphics[width=0.19\textwidth]{fig.ala/fig-trans-mode2-0040-iflux-2.eps}
  \includegraphics[width=0.19\textwidth]{fig.ala/fig-trans-mode2-0040-iflux-3.eps}
  \includegraphics[width=0.19\textwidth]{fig.ala/fig-trans-mode2-0040-iflux-4.eps}
  \includegraphics[width=0.19\textwidth]{fig.ala/fig-trans-mode2-0040-iflux-5.eps}\\
  % \includegraphics[width=0.19\textwidth]{fig.ala/fig-trans-mode2-0200-flux-1.eps}
  % \includegraphics[width=0.19\textwidth]{fig.ala/fig-trans-mode2-0200-flux-2.eps}
  % \includegraphics[width=0.19\textwidth]{fig.ala/fig-trans-mode2-0200-flux-3.eps}
  % \includegraphics[width=0.19\textwidth]{fig.ala/fig-trans-mode2-0200-flux-4.eps}
  % \includegraphics[width=0.19\textwidth]{fig.ala/fig-trans-mode2-0200-flux-5.eps}\\
  \includegraphics[width=0.19\textwidth]{fig.ala/fig-trans-mode2-0200-iflux-1.eps}
  \includegraphics[width=0.19\textwidth]{fig.ala/fig-trans-mode2-0200-iflux-2.eps}
  \includegraphics[width=0.19\textwidth]{fig.ala/fig-trans-mode2-0200-iflux-3.eps}
  \includegraphics[width=0.19\textwidth]{fig.ala/fig-trans-mode2-0200-iflux-4.eps}
  \includegraphics[width=0.19\textwidth]{fig.ala/fig-trans-mode2-0200-iflux-5.eps}\\
  % \includegraphics[width=0.19\textwidth]{fig.ala/fig-trans-mode2-2000-flux-1.eps}
  % \includegraphics[width=0.19\textwidth]{fig.ala/fig-trans-mode2-2000-flux-2.eps}
  % \includegraphics[width=0.19\textwidth]{fig.ala/fig-trans-mode2-2000-flux-3.eps}
  % \includegraphics[width=0.19\textwidth]{fig.ala/fig-trans-mode2-2000-flux-4.eps}
  % \includegraphics[width=0.19\textwidth]{fig.ala/fig-trans-mode2-2000-flux-5.eps}\\
  % \includegraphics[width=0.19\textwidth]{fig.ala/fig-trans-mode2-2000-iflux-1.eps}
  % \includegraphics[width=0.19\textwidth]{fig.ala/fig-trans-mode2-2000-iflux-2.eps}
  % \includegraphics[width=0.19\textwidth]{fig.ala/fig-trans-mode2-2000-iflux-3.eps}
  % \includegraphics[width=0.19\textwidth]{fig.ala/fig-trans-mode2-2000-iflux-4.eps}
  % \includegraphics[width=0.19\textwidth]{fig.ala/fig-trans-mode2-2000-iflux-5.eps}
  \caption{
    Integrated probability flux $Q_{J,I}(t)$ of the periodically
    oscillating EF is plotted against time. The unit of the
    time (horizontal axis) is picosecond.
    The integrated
    probability flux is defined by $Q_{J,I}(t) = \int_0^t F_{J,I}(\tau) d\tau$, where
    $F(t)$ is the net probability flux from metastable state $I$ to $J$.
    From up to down the rows present period 10, 40 and 200~ps, respectively.
    From left to right, the five
    columns show integrated flux $Q_{J,A_1}$, $Q_{J,A_2}$,
    $Q_{J,B_1}$, $Q_{J,B_2}$ and $Q_{J,C}$, respectively. In each plot,
    the red line stands for $J=A_1$, green for $J=A_2$, dark blue for $J=B_1$,
    purple for $J=B_2$ and light blue for $J=C$. In a certain time interval,
    if the integrated flux $Q_{J,I}$ increases, then there exist a (time-averaged)
    non-zero net flux from metastable state $I$ to $J$.
  }
  \label{fig:tmp10}
\end{figure}


The Fig.~\ref{fig:tmp9} shows the probability of being in a certain
metastable state for periodically oscillating electric field.  The
Fig.~\ref{fig:tmp10} presents the integrated probability flux among
the metastable states. We do not show the probability flux itself,
because the profiles are highly oscillating and no more
information can be obtained than the integrated probability flux.
For all periods investigated,
the observed time-dependent probabilities of being in a certain metastable
state are basically the same.
The probability in $A_1$, $A_2$, $B_1$ and $B_2$ are
highly oscillating and the average value does not change
too much with respect to time,
however, the probability in metastable state $C$ astonishingly
increase to approximately 0.17 for $T=10$~ps, 0.27 for  $T=40$~ps, and
0.25 for $T=200$~ps.
Notice that in the constant EF case, although the dipole moment is
larger than the  $T=40$~ps case (see Fig.~\ref{fig:tmp8} for comparison), the
probability of being in $C$ is indeed lower.
For the constant EF case, the probability of state $A_1$
increases from 0.45 to almost 0.7, and that of $B_1$ almost vanishes.
These phenomena are not observed in the oscillating EF case.
In the case of $T=10$ and $40$~ps, the probability of being in a state
reaches the steady value in around 300~ps, while it costs 1200~ps
for the $T=200$~ps case to reach the steady probability. That means
the intrinsic time scales of the nonequilibrium processes are commensurate
for $T=10$ and $40$~ps, while they are much longer for $T=200$~ps.

\begin{figure}
  \centering
  \includegraphics[width=0.5\textwidth]{fig.ala/graphs/graph-t0010-1.eps}
  \caption{The schematic plot of the main probability flux among the metastable
    states on the Ramachandran histogram
    for periodically oscillating EF, $T=10$~ps. The thickness 
    of the arrows approximately presents the strength of the flux.
    The numbers near the arrows indicates the strength of the averged
    probability flux, the unit of which is $10^{-3}\textrm{ps}^{-1}$.
  }
  \label{fig:tmp11}
\end{figure}

Fig.~\ref{fig:tmp10} shows some ever increasing
integrated probability net fluxes, which imply some ever lasting and directional
flux among the metastable states, see
Fig.~\ref{fig:tmp11} that presents a schematic plot of the main probability
fluxes for $T=10$~ps.
The thickness of the arrow and the number nearby indicate
the averaged strength of the net fluxes.
This demonstrate an interesting phenomenon: although
the oscillating EF is not directional and vanishes in time-average,
it generates directional probability fluxes among the metastable states.
Moreover, some loops of the flux are detected: $A_1 \rightarrow B_1
\rightarrow B_2 \rightarrow A_2 \rightarrow A_1$, $A_1 \rightarrow B_1
\rightarrow B_2 \rightarrow A_1$,  $A_1 \rightarrow B_1
\rightarrow B_2 \rightarrow C \rightarrow A_1$ and $B_1
\rightarrow B_2 \rightarrow C \rightarrow B_1$.
The probability flux of $T=40$~ps is quantitatively
comparable to case $T=10$~ps, except that the probability fluxes
going into state $C$ is stronger, which actually results in
a more dominant probability in $C$.
The probability flux of case $T=200$~ps is qualitatively
similar to $T=10$~ps and $T=40$~ps, however,
the strength is much lower than the latter two cases.
Also, we see that the integrated flux reaches the linearly increasing
stage after about 2400~ps, which is much longer than the time scale,
at which probability of $C$ reaches its steady value. This indicates
a even longer intrinsic time scale in the $T=200$~ps case.

We do not test any period longer than $T=200$~ps, because
from the trend that longer period indicates even longer intrinsic time scales.
The computer power available to us is not enough for a longer
simulation that can discover those intrinsic time scale.
However, the long-period-limit case is rather easy: When
the EF changes so slow that at each time point the system
can be viewed as in equilibrium, then the process is a quasi-equilibrium
process, and no nonequilibrium phenomenon will be observed. For example,
the probability being in a  metastable state will change
smoothly from equilibrium value to the new equilibrium of $E=1$~V/nm, as
the EF increases.
And then smoothly goes back to equilibrium, as the EF vanishes.
the time-averaged net flux will be infinitely small,
and the integrated flux will be periodical, and being constant in time-average.


% similar to $T=10$~ps. For $T=40$~ps, the probability
% flux is quantiatively comparable to the case $T=10$~ps, while
% the strength of fluxex for $T=200$~ps is much weaker.

% The time needed to reach a steady state is
% roughly 400~ps that is the same as the
% constant EF case.



% When the period is $T=200$~ps, the phenonema are substentially different
% from the high frequency cases ($T=10$ and 40~ps). Firstly, from Fig.~\ref{fig:tmp9}, the system reaches staedy state after around 2000~ps, while
% Fig.~\ref{fig:tmp10} suggests an even long time scale, but we cannot draw
% a very clear conclusion on it
% due to the lack of computational power to perform even longer simulations. 
% The steady probability of being in state $C$ is roughly the same as the
% case of $T=40$~ps, although the maximum dipole moment is actually higher
% than the latter. And the steady probability  of being in state $C$ is
% also higher than the constant EF case, even the maximum dipole are roughly
% the same. The time spend to reach the steady state is much longer than
% the higher frequency cases, so this indicates a \emph{dynamically} longer time
% scale in the system, comparing with the static EF case, in which all time scales
% are actually decided by the \emph{static} external field. We also observe
% a suden change in the probability flux from state $A_2$ to $A_1$ at
% around 1600~ps, which may imply a \emph{dynamically} rare event.
% Since we do not have the computer power to perform even longer simulations,
% this hypothesis is not further checked.


\section{Concluding remarks}

The D-NEMD method. The design of the nonequilibrium thermostating.

Time scales discovered in the constant EF case.

The discover of the directional probability flux introduced by the
oscillating EF.

Comment on the force field. cite A.Mey's paper.

Comment on the real life-strength and frequency of the EF, if it is
possible in experiment. such a strong field may distroy the molecule.

Comment on the limitation of the non-polarizable model. 

% Clearly two time-scales of relaxation are observed.  For
% configurations $A_2$ and $B_2$, the relaxations are very fast and
% comparable to $t_{\textrm{init}}$. For configurations $A_1$, $B_1$ and
% $C$, the time-scales of relaxation are roughly 600~\textsf{ps}.

% \section{Testing the effect of the thermostat and barostat}

% % \section{System set up}

% % $R_{\textrm{ex}}$ was chosen to be
% % $1.0$~\textsf{nm}, which means that a Langvin thermostat ($\tau =
% % 0.1~\textsf{ps}$) was applied to water molecules that were more than
% % $1.0$~\textsf{nm} away from the center of the simulation box. In
% % addition, a Parrinello-Rahman barostat was couple to the system with a
% % time-scale of $\tau_{\textrm{P}} = 2.0~\textsf{ps}$. 


% Let's consider an infinitely large NVE system (here we call it
% "universe"). Some interesting process is happening in a small
% subsystem. For equilibrium phenomena, this subsystem is the usually
% considered as NVT (or NPT, $\mu$VT, depends on how the question is
% raised) system.  In simulations, we only simulate the subsystem and
% use the thermostat to mimic the embedment into the universe.  For
% non-equilibrium simulation, we also want to simulate only the
% subsystem of interest, but the physical meaning of directly applying a
% thermostat designed for generating equilibrium ensemble is not clear
% to a non-equilibrium process. Moreover, there is no unique way of
% defining how the subsystem is coupled into the universe: It is subject
% to how the perturbation comes into the subsystem, and how the energy
% of the system dissipates into the universe. In the case of switching on
% electric field system, two limiting scenarios are considered by us: (1)
% The subsystem is isolated from the universe: The electric field is
% applied to a system that conserves the energy, volume and number of
% particle (i.e. a NVE subsystem in short).  The isolation here means
% that the subsystem does not exchange energy, volume and number of
% particle with the universe, but the perturbation (electric field) does
% come in to the subsystem.  The periodic boundary condition is applied
% here only to avoid the difficulty of defining a experimental
% meaningful boundary in simulation.  In this scenario, the size the
% subsystem is somewhat arbitrary (it depends on experimental setting),
% therefore, it is important to consider the convergence of the
% simulation result with respect to the size of the subregion.  (2) The
% system exchange the energy, volume and particle with the universe, and
% the universe is so big that the perturbation in the subsystem will not
% kick the universe from equilibrium.  In practice, it is impossible to
% simulate the infinite large universe, Instead, we want to simulate the
% considerably smaller subsystem instead.  The idea is to divide the
% simulation region into two parts (see Fig.~\ref{fig:tmp2}): the first
% is called \emph{Newtonian dynamics region} that is a region around the
% molecule of interest, where the dynamics is Newtonian. Note that the
% molecular trajectories in the Newtonian dynamics region are the same
% as if it were couple to the universe, only when a proper boundary
% condition is provided. Strictly speaking, the boundary should be
% provided by the out side infinitely large universe, which immediately
% leads back to the numerical difficulty of simulating a infinite large
% system. However, we notice that the detailed information
% (i.e. position and velocity) of the every molecule outside the
% Newtonian dynamics region is not relevant to the non-equilibrium
% behavior of the molecule of interest, because we are only interest in
% the non-equilibrium averages, which does not depends on the individual
% trajectory, but on the trajectorial ensemble sampled by the
% non-equilibrium simulation.  Therefore, the universe, as a boundary to
% the Newtonian dynamics region, can be replace by
% an efficiently thermo- and barostated subsystem, as indicated by the gray
% part of Fig.~\ref{fig:tmp2}. 
% % That is to say the subsystem is coupled to a
% % thermostat, a barostat and a particle bath with infinitely high
% % efficiency. Practically, it is impossible to simulation with a
% % ``infinitely fast'' thermostat, barostat and particle bath.
% We indicate the radius of the Newtonian dynamics region by
% $R_{\textrm{ex}} = 1~\textsf{nm}$ (see Fig.~\ref{fig:tmp2} for
% explanation), and couple it to very sufficient thermo- and
% barostat. We use a Langevin thermostat coupled to the thermostating
% region with a time-scale of $\tau_{T} = 0.1~\textsf{ps}$, and a
% Parrinello-Rahman barostat with a time-scale of $\tau_{P} =
% 2.0~\textsf{ps}$.

% The non-equilibrium steady temperature and pressure of the
% non-thermostated simulation are $330$ K and $-500$ Bar,
% correspondingly. Those of the thermostated system are $300$ K and 1
% Bar, correspondingly. Therefore, we can see from Fig.~\ref{fig:tmp6}
% that the relaxation of the thermostated system is faster than the
% non-thermostated system. The three time-scale of metastable population
% dynamics in the thermostated system is consistent with that of the
% non-thermostated system. This means that thermostating, in this case,
% does not qualitatively change the population dynamics of metastable
% states.

% % \begin{figure}
% %   \centering
% %   \includegraphics[]{fig.ala/fig-meta-stat-10.eps}
% %   \caption{ Comparison between simulations with and without
% %     temperature and pressure control at $t_{\textrm{init}} =
% %     10~\textsf{ps}$. The The populations of the metastable sets as a
% %     function of time are plotted.  The dashed vertical lines indicate
% %     the three different time scale of metastable set population at
% %     $t_{\textrm{init}} = 10~\textsf{ps}$.}
% %   \label{fig:tmp6}
% % \end{figure}



% % To make the non-equilibrium feasible, we assume the
% % non-equilibrium process is \emph{local}, i.e. the non-equilibrium perturbation
% % of the system happens only locally, and the rest of the system
% % stays the unperturbed (usually assumed under ambient conditions).
% % This provides us the possibility of simulating only the small subsystem
% % of the universe: We can simulate only subsystem by providing proper
% % boundary condition for it, so that it behaves as if it were embeded into
% % the universe.

% % \subsection{The water structure around methane under impluse }
% % In this study, we firstly consider the system of a methane molecule solvated
% % in the liquid water environment. A local density perturbation is implemented
% % by applying an impluse  in a shell surronding methane, see Fig.~\ref{fig:tmp1}.
% % % As far as we concerned, the situation is: we truncate the universe to the
% % % subsystem, and provide proper boundary condition for this
% % % subsystem.
% % Here we denote the size of this subsystem by $R_{\ex}$.
% % Notice this subsystem is simulated by the Newtonian
% % dynamics. If the size of the
% % subsystem is large enough so that the perturbation is
% % small enough when it travels to the boundary of the subsystem, then a
% % trivial boundary condition would be: coupling the system to a boundary
% % layer equilibriated by the thermostat (and barostat or particle reservior),
% % which mimics the infinitely large equilibrium universe.

% % \begin{figure}
% %   \centering
% %   \includegraphics[width=0.3\textwidth]{fig/kick.eps}
% %   \caption{Schematic plot of the density perturbation.}
% %   \label{fig:tmp1}
% % \end{figure}

\bibliography{ref}{}
\bibliographystyle{unsrt}





\end{document}
