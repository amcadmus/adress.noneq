% \documentclass[aip,jcp,preprint,unsortedaddress,a4paper,onecolum]{revtex4-1}
\documentclass[a4paper,preprint,onecolumn]{revtex4-1}
% \documentclass[aps,pre,twocolumn]{revtex4-1}
% \documentclass[aps,jcp,groupedaddress,twocolumn,unsortedaddress]{revtex4}

\usepackage[fleqn]{amsmath}
\usepackage{amssymb}
\usepackage[dvips]{graphicx}
\usepackage{color}
\usepackage{tabularx}
\usepackage{algorithm}
\usepackage{algorithmic}

\makeatletter
\makeatother

\newcommand{\recheck}[1]{{\color{red} #1}}
\newcommand{\redc}[1]{{\color{red} #1}}
\newcommand{\bluec}[1]{{\color{blue} #1}}
\newcommand{\greenc}[1]{{\color{green} #1}}
\newcommand{\vect}[1]{\textbf{\textit{#1}}}
\newcommand{\dd}[1]{\textsf{#1}}

\newcommand{\AT}{{\textrm{{AT}}}}
\newcommand{\EX}{{\textrm{EX}}}
\newcommand{\ex}{{\textrm{ex}}}
\newcommand{\CG}{{\textrm{CG}}}
\newcommand{\HY}{{\Delta}}
\newcommand{\rdf}{{\textrm{rdf}}}



\begin{document}

\title{Report: the non-equilibrium molecular dynamics simulation}
\author{Han Wang}
\affiliation{Institute for Mathematics, Freie Universit\"at Berlin, Germany}
\author{Christof Sch\"utte}
\affiliation{Institute for Mathematics, Freie Universit\"at Berlin, Germany}
\author{Luigi Delle Site}
\affiliation{Institute for Mathematics, Freie Universit\"at Berlin, Germany}

\begin{abstract}
\end{abstract}

\maketitle

\newpage


\section{Alanine dipeptide
  under turning on electric field}
\begin{figure}
  \centering
  \includegraphics[width=0.3\textwidth]{fig.ala/ele.field/field.eps}
  \caption{The magnitude of the external electric field as
    a function of time.}
  \label{fig:tmp3}
\end{figure}

We study the response of the configuration of the alanine dipeptide
with respect to a turning on electric field. The direction of this
field is along the $x$ direction and the magnitude with respect to time
is plotted in Fig.~\ref{fig:tmp3}. From $t=0$ to
$t=t_{\textrm{init}}$, the strength of the field grows linearly, while
after $t=t_{\textrm{init}}$, the strength is kept constant at
$E_{\textrm{max}}$.  The size of the system is $2.67\times 2.67\times
2.67\, \textsf{nm}^3$, with one alanine dipeptide solved in 644 TIP3P
water molecules. Firstly, an equilibrium NVT simulation of
100~\textsf{nm} was performed with a Langevin thermostat (time-scale
$\tau_T = 0.5~\textsf{ps}$) coupled to the system for constant
temperature.  Along the trajectory, configurations were taken every
5~\textsf{ps}.  They server as starting point of the branching
trajectories. The branching trajectories were integrated by the
Leap-frog scheme without any temperature and pressure control.  The
time step was $\Delta t = 0.002~\textsf{ps}$. The short-range
interaction was smoothed from $0.8$ to $1.0~\textsf{nm}$ by the
``switch'' method provided by Gromacs.  An energy conserving PME
method was applied to calculate the electrostatic interaction in this
periodic system.  On each branching trajectory, to easy the analysis,
the conformation of the alanine dipeptide is projected on the
$\psi$--$\phi$ plane, as shown in Fig.~\ref{fig:tmp4}. The equilibrium
distribution of the conformation on the $\psi$--$\phi$ plane is given
in the Left plot of Fig.~\ref{fig:tmp4}.  It is clear the
conformations are populated in several clusters that are called
``metastable states'' in this research. For a clearer definition, we
manually divide the $\psi$--$\phi$ plane into 5 subregions. $A_1$ and
$A_2$ are corresponding to the alpha helix conformation. $B_1$ and
$B_2$ are corresponding
to beta sheet conformation. $C$ is corresponding to ill-folding.\\

\begin{figure}
  \centering
  \includegraphics[width=0.45\textwidth]{fig.ala/ext.mode1.100.Ex.01.00.t1200ps/fig-begin-1.eps}
  \includegraphics[width=0.45\textwidth]{fig.ala/ext.mode1.100.Ex.01.00.t1200ps/fig-end-1.eps}
  \caption{The population of different metastable state. Left: the equilibrium state at $t=0$~\textsf{ps}. Right: The non-equilibrium study state at $t=1200$~\textsf{ps}, i.e., fully relaxed when the electric field is switched on. The darker color indicates larger population on the $\psi$--$\phi$ phase space.}
  \label{fig:tmp4}
\end{figure}


For all testing cases, the maximum strength of the electric field
$E_{\textrm{max}}$ is chosen to be 1~\textsf{V/nm}.  Three values of
$t_{\textrm{init}}$ (10~\textsf{ps}, 100~\textsf{ps} and
500~\textsf{ps}) are chosen to study the response of the system to the
external electric field perturbation. The result is given in
Fig.~\ref{fig:tmp5}.  No matter how fast is the electric field turned
on, the beta sheet configuration, which presents at the beginning,
fades away as the system relaxes to the new equilibrium. The
population of alpha helix $A_1$ configuration grows from 45\% to 70\%.
alpha helix $A_2$  does not change a lot under the electric field.
The ill-folding configuration $C$ noticeably grows from 0\% to 20\%.
\\

In the case of fastly switching on electric field
($t_{\textrm{init}}=10~\textsf{ps}$), the dynamics of the population
of the metastable sets shows three different time-scales. (1) From
$t=0$ to 10~\textsf{ps}, the population of $A_2$ fastly increases,
while the population of $A_1$ decreases with the same amount, so it is
reasonable to argue that there is a fast configurational change from
$A_1$ to $A_2$. Please see, in Fig.~\ref{fig:tmp5}, the green and read
peak around $t = 10~\textsf{ps}$. At the same time, the population of
other metastable sets remains unchanged.  (2), From $t=10$ to
40~\textsf{ps}, the population of $B_2$ decreases by 0.15, and this
configuration almost disappears. Interestingly, the population of $A_1$
increases by the same amount, and the population of $A_2$ goes back to
the initial value. The populations of $B_1$ and $C$ start increasing
and decreasing, correspondingly. The speed of these changes are much
slower than that of $B_2$ and $A_1$.  (3) At the time scale of $t =
300~\textsf{ps}$, the population of $A_1$, $B_1$ and $C$ converge to
the non-equilibrium steady state (system with $E(\infty) = 1~\textsf{V/m}$).
This time is actually thirty times large than when the electric field
is fully switched on ($t_{\textrm{init}} = 10~\textsf{ps}$). This 
indicates that the internal time scale of $A_1$, $B_1$ and $C$
relaxation is around $300~\textsf{ps}$.\\

To further test the dynamics of metastable population, two simulations
of $t_{\textrm{init}} = 100~\textsf{ps}$ and $t_{\textrm{init}} =
500~\textsf{ps}$ are performed. The first time scale of $t =
10~\textsf{ps}$ disappears in both of the testing simulations. In the
simulation of $t_{\textrm{init}} = 100~\textsf{ps}$, the population of
$A$, $B_1$ and $C$ converges to the non-equilibrium steady state at
around $300~\textsf{ps}$. This further verifies internal time scale of
$A_1$, $B_1$ and $C$ presented in the testing case of
$t_{\textrm{init}} = 10~\textsf{ps}$.  The dynamics of $B_2$ is
faster: it decays to zero as fast as the external electric field is
switched on. In the case of $t_{\textrm{init}} = 500~\textsf{ps}$, the
growth of the external electric field is slower than the longest
internal time scale of system, which means that the changes of the
perturbation to the system is so slow that the system can catch up
with that. The numerical results show the same story: the relaxation
of all populations are around $500 \sim 600~\textsf{ps}$.
\\

% Clearly two time-scales of relaxation are observed.  For
% configurations $A_2$ and $B_2$, the relaxations are very fast and
% comparable to $t_{\textrm{init}}$. For configurations $A_1$, $B_1$ and
% $C$, the time-scales of relaxation are roughly 600~\textsf{ps}.
\begin{figure}
  \centering
  \includegraphics[]{fig.ala/fig-meta.eps}
  \caption{The populations of the metastable sets as a function of time.
    The dashed vertical lines indicate the three different time scale of
  metastable set population at $t_{\textrm{init}} = 10~\textsf{ps}$.}
  \label{fig:tmp5}
\end{figure}

\section{Testing the effect of the thermostat and barostat}
\begin{figure}
  \centering
  \includegraphics[width=0.3\textwidth]{fig/thermostat.eps}
  \caption{Schematic plot of the truncated non-equilibrium subsystem.
    The white part is the subsystem and the gray part is the boundary
    condition, implemented by the thermostat and barostat mimicing
    the infinitely large universe.}
  \label{fig:tmp2}
\end{figure}




% \section{System set up}

% $R_{\textrm{ex}}$ was chosen to be
% $1.0$~\textsf{nm}, which means that a Langvin thermostat ($\tau =
% 0.1~\textsf{ps}$) was applied to water molecules that were more than
% $1.0$~\textsf{nm} away from the center of the simulation box. In
% addition, a Parrinello-Rahman barostat was couple to the system with a
% time-scale of $\tau_{\textrm{P}} = 2.0~\textsf{ps}$. 


Let's consider an infinitely large NVE system (here we call it
"universe"). Some interesting process is happening in a small
subsystem. For equilibrium phenomena, this subsystem is the usually
considered as NVT (or NPT, $\mu$VT, depends on how the question is
raised) system.  In simulations, we only simulate the subsystem and
use the thermostat to mimic the embedment into the universe.  For
non-equilibrium simulation, we also want to simulate only the
subsystem of interest, but the physical meaning of directly applying a
thermostat designed for generating equilibrium ensemble is not clear
to a non-equilibrium process. Moreover, there is no unique way of
defining how the subsystem is coupled into the universe: It is subject
to how the perturbation comes into the subsystem, and how the energy
of the system dissipates into the universe. In the case of switching on
electric field system, two limiting scenarios are considered by us: (1)
The subsystem is isolated from the universe: The electric field is
applied to a system that conserves the energy, volume and number of
particle (i.e. a NVE subsystem in short).  The isolation here means
that the subsystem does not exchange energy, volume and number of
particle with the universe, but the perturbation (electric field) does
come in to the subsystem.  The periodic boundary condition is applied
here only to avoid the difficulty of defining a experimental
meaningful boundary in simulation.  In this scenario, the size the
subsystem is somewhat arbitrary (it depends on experimental setting),
therefore, it is important to consider the convergence of the
simulation result with respect to the size of the subregion.  (2) The
system exchange the energy, volume and particle with the universe, and
the universe is so big that the perturbation in the subsystem will not
kick the universe from equilibrium.  In practice, it is impossible to
simulate the infinite large universe, Instead, we want to simulate the
considerably smaller subsystem instead.  The idea is to divide the
simulation region into two parts (see Fig.~\ref{fig:tmp2}): the first
is called \emph{Newtonian dynamics region} that is a region around the
molecule of interest, where the dynamics is Newtonian. Note that the
molecular trajectories in the Newtonian dynamics region are the same
as if it were couple to the universe, only when a proper boundary
condition is provided. Strictly speaking, the boundary should be
provided by the out side infinitely large universe, which immediately
leads back to the numerical difficulty of simulating a infinite large
system. However, we notice that the detailed information
(i.e. position and velocity) of the every molecule outside the
Newtonian dynamics region is not relevant to the non-equilibrium
behavior of the molecule of interest, because we are only interest in
the non-equilibrium averages, which does not depends on the individual
trajectory, but on the trajectorial ensemble sampled by the
non-equilibrium simulation.  Therefore, the universe, as a boundary to
the Newtonian dynamics region, can be replace by
an efficiently thermo- and barostated subsystem, as indicated by the gray
part of Fig.~\ref{fig:tmp2}. 
% That is to say the subsystem is coupled to a
% thermostat, a barostat and a particle bath with infinitely high
% efficiency. Practically, it is impossible to simulation with a
% ``infinitely fast'' thermostat, barostat and particle bath.
We indicate the radius of the Newtonian dynamics region by
$R_{\textrm{ex}} = 1~\textsf{nm}$ (see Fig.~\ref{fig:tmp2} for
explanation), and couple it to very sufficient thermo- and
barostat. We use a Langevin thermostat coupled to the thermostating
region with a time-scale of $\tau_{T} = 0.1~\textsf{ps}$, and a
Parrinello-Rahman barostat with a time-scale of $\tau_{P} =
2.0~\textsf{ps}$.

The non-equilibrium steady temperature and pressure of the
non-thermostated simulation are $330$ K and $-500$ Bar,
correspondingly. Those of the thermostated system are $300$ K and 1
Bar, correspondingly. Therefore, we can see from Fig.~\ref{fig:tmp6}
that the relaxation of the thermostated system is faster than the
non-thermostated system. The three time-scale of metastable population
dynamics in the thermostated system is consistent with that of the
non-thermostated system. This means that thermostating, in this case,
does not qualitatively change the population dynamics of metastable
states.

\begin{figure}
  \centering
  \includegraphics[]{fig.ala/fig-meta-stat-10.eps}
  \caption{ Comparison between simulations with and without
    temperature and pressure control at $t_{\textrm{init}} =
    10~\textsf{ps}$. The The populations of the metastable sets as a
    function of time are plotted.  The dashed vertical lines indicate
    the three different time scale of metastable set population at
    $t_{\textrm{init}} = 10~\textsf{ps}$.}
  \label{fig:tmp6}
\end{figure}



% To make the non-equilibrium feasible, we assume the
% non-equilibrium process is \emph{local}, i.e. the non-equilibrium perturbation
% of the system happens only locally, and the rest of the system
% stays the unperturbed (usually assumed under ambient conditions).
% This provides us the possibility of simulating only the small subsystem
% of the universe: We can simulate only subsystem by providing proper
% boundary condition for it, so that it behaves as if it were embeded into
% the universe.

% \subsection{The water structure around methane under impluse }
% In this study, we firstly consider the system of a methane molecule solvated
% in the liquid water environment. A local density perturbation is implemented
% by applying an impluse  in a shell surronding methane, see Fig.~\ref{fig:tmp1}.
% % As far as we concerned, the situation is: we truncate the universe to the
% % subsystem, and provide proper boundary condition for this
% % subsystem.
% Here we denote the size of this subsystem by $R_{\ex}$.
% Notice this subsystem is simulated by the Newtonian
% dynamics. If the size of the
% subsystem is large enough so that the perturbation is
% small enough when it travels to the boundary of the subsystem, then a
% trivial boundary condition would be: coupling the system to a boundary
% layer equilibriated by the thermostat (and barostat or particle reservior),
% which mimics the infinitely large equilibrium universe.

% \begin{figure}
%   \centering
%   \includegraphics[width=0.3\textwidth]{fig/kick.eps}
%   \caption{Schematic plot of the density perturbation.}
%   \label{fig:tmp1}
% \end{figure}





\end{document}