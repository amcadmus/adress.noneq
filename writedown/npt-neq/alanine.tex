% \documentclass[aip,jcp,preprint,unsortedaddress,a4paper,onecolum]{revtex4-1}
\documentclass[aip,jcp,a4paper,preprint,onecolumn]{revtex4-1}
% \documentclass[aps,pre,twocolumn]{revtex4-1}
% \documentclass[aps,jcp,groupedaddress,twocolumn,unsortedaddress]{revtex4}

\usepackage[fleqn]{amsmath}
\usepackage{amssymb}
\usepackage[dvips]{graphicx}
\usepackage{color}
\usepackage{tabularx}
\usepackage{algorithm}
\usepackage{algorithmic}

\makeatletter
\makeatother

\newcommand{\recheck}[1]{{\color{red} #1}}
\newcommand{\redc}[1]{{\color{red} #1}}
\newcommand{\bluec}[1]{{\color{blue} #1}}
\newcommand{\greenc}[1]{{\color{green} #1}}
\newcommand{\vect}[1]{\textbf{\textit{#1}}}
\newcommand{\dd}[1]{\textsf{#1}}
\newcommand{\fwd}[0]{\textrm{fwd}}
\newcommand{\bwd}[0]{\textrm{bwd}}



\begin{document}

\title{The non-equilibrium molecular dynamics simulation of the alanine dipeptide under electric field}
\author{Han Wang}
\affiliation{Institute for Mathematics, Freie Universit\"at Berlin, Germany}
\author{Christof Sch\"utte}
\affiliation{Institute for Mathematics, Freie Universit\"at Berlin, Germany}
% \affiliation{Zuse Institute Berlin, Germany}
\author{Luigi Delle Site}
\affiliation{Institute for Mathematics, Freie Universit\"at Berlin, Germany}

\begin{abstract}
\end{abstract}

\maketitle

\newpage


\section{Methodology}
\subsection{Nonequilibrium molecular dynamics simulation}
Here we fastly review the approach of performing nonequilibrium
molecular dynamics that was initiated by Giovanni Ciccoti and his
colleagues~\cite{ciccotti1975direct, ciccotti1979thought,
  orlandini2011hydrodynamics, orlandini2011hydrodynamics-01}.
We denote the macroscopic nonequilibrium observable by $O(t)$. If at $t$
the configurational probability distribution is $\rho(\vect x, t)$, where
$\vect x$ is the phase space variable, then the observable can be
expressed by
\begin{align}\label{eqn:tmp1}
  O(t) = \int d\vect x\, \hat O(\vect x)\rho(\vect x, t)  = \langle \hat O(\vect x), \rho(\vect x, t)\rangle,
\end{align}
where $\hat O (\vect x)$ is the microscopic observable. Compared with
the macroscopic observable, it is measured at the phase space position $\vect x$.
The bracket on the right hand side denotes the inner produce in the
phase space.  We assume the dynamics of the system is governed by the
Hamiltonian equation, i.e. $\dot {\vect x} = J \cdot \nabla_{\vect x}
\mathcal H(\vect x)$, where $\mathcal H$ is the Hamiltonian and $J$ is
the symplectic matrix, then the Liouville equation of the probability
distribution is
\begin{align}\label{eqn:tmp2}
  \frac{\partial \rho(\vect x, t)}{\partial t} = - iL(t) \rho(\vect x, t),
\end{align}
where $iL(t) = \{\cdot, \mathcal H\}$ is the Liouville operator and
$\mathcal H$ is the Hamiltonian of the system. The Eq.~\eqref{eqn:tmp2}
can be formally solved by $\rho(\vect x, t) = e^{-iL(t)} \rho(\vect x, 0)$.
On the other hand
\begin{align}
  \frac{d \hat O(\vect x(t))}{dt} = \nabla_{\vect x}\hat O\cdot \dot{\vect x}
  = \nabla_{\vect x}\hat O\cdot J\cdot \nabla_{\vect x}\mathcal H
  = iL(t) \hat O (\vect x(t))
\end{align}
This equation can be formally solved by $\hat O(\vect x(t)) = e^{iL(t)} O(\vect x, 0)$, therefore,
\begin{align}\nonumber
  O(t) & = \langle \hat O(\vect x), \rho(\vect x, t)\rangle
  = \langle \hat O(\vect x), e^{-iL(t)} \rho(\vect x, 0)\rangle
  = \langle e^{iL(t)}\hat O(\vect x), \rho(\vect x, 0)\rangle\\\label{eqn:tmp4}
  &= \langle \hat O(\vect x(t)), \rho(\vect x, 0)\rangle
\end{align}
If the inital state of the system is in equilibrium, i.e. the initial
probability distribution is the equilibrium distribution. Then an
explicit translation of Eq.~\eqref{eqn:tmp4} is that the
nonequilibrium observable is equal to the ensenble average of
microscopic observable measured along trajectory $\vect x(t)$, the
inital configuration of which is subject to the equilibrium
distribution. In practice, we firstly run an equilibrium MD simulation
to generate a collection of configurations that are subject to the
equilibrium distribution. And then by using these configurations as
inital configuration, the Hamiltonian dynamicses are integrated until
time $t$. These trajectories are also called \emph{branching
  trajectories} in this paper. Finially, the macroscopic observable is
estimated by averaging the microscopic observable measured at the end
points of the trajectories.

Being more gneralize, if the microscopic depends not only on the end
configuation of at time $t$ but also depends on the historical
configurations along the trajectory, the experssion~\eqref{eqn:tmp4} can be generalize to
\begin{align}
  O(t) = \int d\vect x\,\rho(\vect x, 0) \int_{\mathcal C\{\vect x, 0; t\}} \hat O[\vect x_s] \,d\mathcal P[\vect x_s] 
\end{align}
where the microscopic observable $ \hat O[\vect x_s] $ is now a
functional of the trajectory.  $\mathcal C\{\vect x, 0; t\}$ is the
set of all continuous trajectories starting at point $\vect x$ at time
0, and ending at time $t$. $\mathcal P[\vect x_s] $ is the meansure of
the trajectory space $\mathcal C$.  In our case, it is the
$\delta$-function peaked around the trajectory generated by the
Hamiltonian dynamics. To be general, it can also be the measure of all
trajectories generated by, for example, a Langevin dynamics, then the
branching trajectories should be generated by the Langevin
dynamics rather than the Hamiltonian dynamics.


\subsection{Nonequilibrium temperature control}

The aformentioned nonequilibrium molecular dynamicses simulation algorithm
does not describe how the boundary of the system should be set up. The 100\% save
method is to simulate a infinitely large isolated system, or any large enough
isolated subsystem of it. In practice, it is both impossible and unnecessary to
do such a simulation. In practice, proper boundary condition should be developped
based on how the nonequilibrium system is set up, and what are the nonequilibrium
observables to measure.
Since we are mainly interested in the non-thermal effect of
the electric field to the configuration of a solvated alanine dipeptide,
we assume the alanine is embeded into an infinitely large solvent envrionment,
and the electric field is only applied to the neighborhood of the alanine. Therefore, 
extra heat generated by turing on the electric field can be effectively absorbed
by the envrionment. 

\begin{figure}
  \centering
  \includegraphics[width=0.3\textwidth]{fig/thermostat.eps}
  \caption{Schematic plot of the truncated nonequilibrium subsystem.
    The white part is the subsystem and the gray part is the boundary
    condition, implemented by the thermostat and barostat mimicing
    the infinitely large universe.}
  \label{fig:tmp2}
\end{figure}

In practice, it is impossible to simulate an infinitely large
environment.  For an equilibrium simulation, the solution is to couple
the system to a thermostat, under which the canonical ensember can be
sampled by only simulate a finite size system with periodic boundary
condition. For a nonequilibrium simulation, simply couple the system
to a thermostat may not be a good idea, because the \emph{dynamics}
generated by the thermostat is artificial, and deviation from the
Hamiltonian dynamics under investigation will introduce artificial
effects in the nonequilibrium observations. To solve this problem, we
observe that only the Hamiltonian dynamics of the alanine molecule
itself and the water molecules in the nearby solvation shell is
crutial to the configuration change, while the detailed dynamics of
the water molecules far away has less effect. Notice again we actually
do not want to precisely calculate each trajectory starting from the
initial configurations, but want to sample correctly the trajectory
ensemble correctly, so that the nonequilibrium observable can be
calculated correctly.  Therefore, instead of simulating an infinitely
large system, we only simulate a finite size system with periodic
boundary condition. The simulating region is divided into two
subregions: Near the alanine dipeptide, the dynamics is kept to be
Hamiltonian, and this subregion is called the Hamiltonian dynamics
region. While far from the alanine, the dynamics of water is coupled
to a Langevin thermostat, and this region is called the thermostating
region (see Fig.~\ref{fig:tmp2}).  For simplicity, the Hamiltonian
dynamics region is assumed to be spherical, centered at the
alpha-carbon of the alanine, and with radius of $R_{ex}$.  Further, we
freeze the movment of the alpha-carbon, so that the alanine is always
located at the center of the simulation region.  The validity of this
setting will be later check by numerical examples showing that the
nonequilibrium observables do not depends on the size of the system
and the size of the Hamiltonian dynamics subregion, if they are
reasonably large and the finite size effect does not play a role.


\section{Example I: Alanine dipeptide
  under a linearly switching on electric field}

\subsection{System setting up }
\begin{figure}
  \centering
  % \includegraphics[width=0.3\textwidth]{fig.ala/ele.field/field.eps}
  \includegraphics[]{fig.ala/fig-field-dipol.eps}
  \caption{The magnitude of the external electric field, and the
    x-component of the dipole moment of the alanine molecule as a
    function of time. The dashed lines are the strengths of the
    external electric field, while the solid lines are the dipole
    moments. Three test cases are indicated by different colors.}
  \label{fig:tmp3}
\end{figure}

We want to study the nonequilibrium properties of an alanine dipeptide
under a changing electric field. As a model system, we study the
linearly switching on electric field: From $t=0$ to
$t=t_{\textrm{init}}$, the strength of the field grows linearly, while
after $t=t_{\textrm{init}}$, the strength is kept constant at
$E_{\textrm{max}}$. Fig.~\ref{fig:tmp3} shows the three different
cases considered in this work: $t_{\textrm{init}} = 10$, 100, and
500~ps, and 
in all cases the maximum strength of the electric field is
$E_{\textrm{max}} = 1$~V/nm.
The direction of this field is arbitarily chosen: along the
$x$ direction. Since the dipole of the equilibrium configurations of
alanine can be along any direction (see also Fig.~\ref{fig:tmp3} for
the vanished initial dipole moment), applying the electric along x-axis is
not biased.
The size of the system is $2.67\times 2.67\times
2.67\, \textsf{nm}^3$, with one alanine dipeptide described by the CHARMM27 force field, and solved in 644 TIP3P
water molecules.
All simulations are performed by a home-modified GROMACS 4.5~\cite{pronk2013gromacs}.
Firstly, an equilibrium NVT simulation of
100~\textsf{nm} was performed with a Langevin thermostat (time-scale
$\tau_T = 0.5~\textsf{ps}$) coupled to the system for constant
temperature.  Along the trajectory, configurations were taken every
5~\textsf{ps}.  They server as starting point of the branching
trajectories. The branching trajectories were integrated by the
Leap-frog scheme with the aformentioned nonequilibrium temperature control technique.  The
time step was $\Delta t = 0.002~\textsf{ps}$. The short-range
interaction was smoothed from $0.8$ to $1.0~\textsf{nm}$ by the
``switch'' method provided by Gromacs.  An energy conserving PME
method was applied to calculate the electrostatic interaction in this
periodic system. In the thermostating region, the original dyanmics was
coupled to a Langevin thermostat with $\tau_T = 0.1~\textsf{ps}$.
The whole system is also coupled to a Parrinello-Rahman barostat with $\tau_P = 2.0~\textsf{ps}$ to keep
the pressure at the ambient condition (1~Bar). Since the
change of the system size is small and slow, it does not have an obviously effect on the
dyanmics of the system.

\subsection{Metastable sets}

On each branching trajectory, to easy the analysis, the conformation
of the alanine dipeptide is plotted by the Ramachandran histogram, as
shown in Fig.~\ref{fig:tmp4}. The equilibrium distribution of the
conformation is given in the left plot, and the fully relaxed
conformation at $E_{\max} = 1$~V/nm is given in the right plot.  It is
clear the conformations are populated in several clusters, and the
position of the clusters do not change when the external electric
field is applied. This means that only a few of the configurations are likely to be observed, while there is nearly
no oppotunity to see others. 
These configurations are called ``metastable states''. The strict definition of the
metastability is not exactly the same as those conformations that are likely to be observed.
Since in this paper we do not focus on the discussion of the definition for the metastability of the system,
we assume they have the same meaning.
We manually divide the
$\psi$--$\phi$ plane into 5 subregions, see Fig.~\ref{fig:tmp4},
then the observation on the Ramachandran histogram is projected to the
observation of these 5 subregions.  These subregions
correspond to different secondary structure of a peptide chain:
$A_1$ and $A_2$ are
corresponding to the alpha helix. $B_1$ and $B_2$ are
corresponding to beta sheet. $C$ is corresponding to the
left-handed alpha helix.

\begin{figure}
  \centering
  \includegraphics[width=0.45\textwidth]{fig.ala/ext.mode1.100.Ex.01.00.t1200ps/fig-begin-1.eps}
  \includegraphics[width=0.45\textwidth]{fig.ala/ext.mode1.100.Ex.01.00.t1200ps/fig-end-1.eps}
  \caption{The population of different metastable state. Left: the equilibrium state at $t=0$~\textsf{ps}. Right: The non-equilibrium study state at $t=1200$~\textsf{ps}, i.e., fully relaxed when the electric field is switched on. The darker color indicates larger population on the $\psi$--$\phi$ phase space.}
  \label{fig:tmp4}
\end{figure}

From Fig.~\ref{fig:tmp4}, it is obvious that the population in each
metastable sets changes due to the external electric field, for
example, the population of conformations C vanishes in equilibrium,
but plays an important role when the electric field is applied. We
therefore want to study how the alanine changes from one conformation
to the other, how fast is the change, and the relation to the speed of
the increasing electric field (i.e. the value of $t_{init}$). 
Firly, we study the population of the metastable conformations
Being more strict, we calculate the nonequilibrium observable
\begin{align}
  p_I(t) = \mathbb P (\vect x_t \in I), \quad  I \in \{A_1, A_2, B_1, B_2, C\}
\end{align}
Secondly, we study the
forward and backward transition probability of the metastable conformations, which is defined by
\begin{align}
  p^{\fwd}_{J,I}(t) & = \mathbb P( \vect x_{t+\Delta t} \in J | \vect x_t \in I) \\
  p^{\bwd}_{J,I}(t) & = \mathbb P( \vect x_{t-\Delta t} \in J | \vect x_t \in I)
  \qquad I, J \in \{A_1, A_2, B_1, B_2, C\},
\end{align}
where $\Delta t$ is the time-scale of the observation.
For a revserible and homogeneous Markov process, it is easy to show
that $p^{\fwd}_{J,I} = p^{\bwd}_{J,I}$, and both of them are time independent. For a general nonequilibrium
process considered by the present work, this relation is genarlly not guarenteed.


\subsection{Results and discusstions}
\begin{figure}
  \centering
  \includegraphics[]{fig.ala/fig-meta-npt.eps}
  \caption{The populations of the metastable sets as a function of time.
    The dashed vertical lines indicate the three different time scale of
  metastable set population at $t_{\textrm{init}} = 10~\textsf{ps}$.}
  \label{fig:tmp5}
\end{figure}

\begin{figure}
  \centering
  \includegraphics[width=0.19\textwidth]{fig.ala/fig-trans-010-fwd-1.eps}
  \includegraphics[width=0.19\textwidth]{fig.ala/fig-trans-010-fwd-2.eps}
  \includegraphics[width=0.19\textwidth]{fig.ala/fig-trans-010-fwd-3.eps}
  \includegraphics[width=0.19\textwidth]{fig.ala/fig-trans-010-fwd-4.eps}
  \includegraphics[width=0.19\textwidth]{fig.ala/fig-trans-010-fwd-5.eps}\\
  % \caption{The populations of the metastable sets as a function of time.
  %   The dashed vertical lines indicate the three different time scale of
  % metastable set population at $t_{\textrm{init}} = 10~\textsf{ps}$.}
  \includegraphics[width=0.19\textwidth]{fig.ala/fig-trans-010-bwd-1.eps}
  \includegraphics[width=0.19\textwidth]{fig.ala/fig-trans-010-bwd-2.eps}
  \includegraphics[width=0.19\textwidth]{fig.ala/fig-trans-010-bwd-3.eps}
  \includegraphics[width=0.19\textwidth]{fig.ala/fig-trans-010-bwd-4.eps}
  \includegraphics[width=0.19\textwidth]{fig.ala/fig-trans-010-bwd-5.eps}\\
  \caption{Forward and backward nonequilibrium transition probability of the case $t_{init} = 10$~ps.}
  \label{fig:tmp6}
\end{figure}


For all testing cases, the maximum strength of the electric field
$E_{\textrm{max}}$ is chosen to be 1~\textsf{V/nm}.  Three values of
$t_{\textrm{init}}$ (10~\textsf{ps}, 100~\textsf{ps} and
500~\textsf{ps}) are chosen to study the response of the system to the
external electric field perturbation. The result is given in
Fig.~\ref{fig:tmp5}.  No matter how fast is the electric field turned
on, the beta sheet configuration, which presents at the beginning,
fades away as the system relaxes to the new equilibrium. The
population of alpha helix $A_1$ configuration grows from 45\% to 70\%.
alpha helix $A_2$  does not change a lot under the electric field.
The ill-folding configuration $C$ noticeably grows from 0\% to 20\%.
\\

In the case of fastly switching on electric field
($t_{\textrm{init}}=10~\textsf{ps}$), the dynamics of the population
of the metastable sets shows three different time-scales. (1) From
$t=0$ to 10~\textsf{ps}, the population of $A_2$ fastly increases,
while the population of $A_1$ decreases with the same amount, so it is
reasonable to argue that there is a fast configurational change from
$A_1$ to $A_2$. Please see, in Fig.~\ref{fig:tmp5}, the green and read
peak around $t = 10~\textsf{ps}$. At the same time, the population of
other metastable sets remains unchanged.  (2), From $t=10$ to
40~\textsf{ps}, the population of $B_2$ decreases by 0.15, and this
configuration almost disappears. Interestingly, the population of $A_1$
increases by the same amount, and the population of $A_2$ goes back to
the initial value. The populations of $B_1$ and $C$ start increasing
and decreasing, correspondingly. The speed of these changes are much
slower than that of $B_2$ and $A_1$.  (3) At the time scale of $t =
300~\textsf{ps}$, the population of $A_1$, $B_1$ and $C$ converge to
the non-equilibrium steady state (system with $E(\infty) = 1~\textsf{V/m}$).
This time is actually thirty times large than when the electric field
is fully switched on ($t_{\textrm{init}} = 10~\textsf{ps}$). This 
indicates that the internal time scale of $A_1$, $B_1$ and $C$
relaxation is around $300~\textsf{ps}$.\\

To further test the dynamics of metastable population, two simulations
of $t_{\textrm{init}} = 100~\textsf{ps}$ and $t_{\textrm{init}} =
500~\textsf{ps}$ are performed. The first time scale of $t =
10~\textsf{ps}$ disappears in both of the testing simulations. In the
simulation of $t_{\textrm{init}} = 100~\textsf{ps}$, the population of
$A$, $B_1$ and $C$ converges to the non-equilibrium steady state at
around $300~\textsf{ps}$. This further verifies internal time scale of
$A_1$, $B_1$ and $C$ presented in the testing case of
$t_{\textrm{init}} = 10~\textsf{ps}$.  The dynamics of $B_2$ is
faster: it decays to zero as fast as the external electric field is
switched on. In the case of $t_{\textrm{init}} = 500~\textsf{ps}$, the
growth of the external electric field is slower than the longest
internal time scale of system, which means that the changes of the
perturbation to the system is so slow that the system can catch up
with that. The numerical results show the same story: the relaxation
of all populations are around $500 \sim 600~\textsf{ps}$.
\\

% Clearly two time-scales of relaxation are observed.  For
% configurations $A_2$ and $B_2$, the relaxations are very fast and
% comparable to $t_{\textrm{init}}$. For configurations $A_1$, $B_1$ and
% $C$, the time-scales of relaxation are roughly 600~\textsf{ps}.

\section{Testing the effect of the thermostat and barostat}




% \section{System set up}

% $R_{\textrm{ex}}$ was chosen to be
% $1.0$~\textsf{nm}, which means that a Langvin thermostat ($\tau =
% 0.1~\textsf{ps}$) was applied to water molecules that were more than
% $1.0$~\textsf{nm} away from the center of the simulation box. In
% addition, a Parrinello-Rahman barostat was couple to the system with a
% time-scale of $\tau_{\textrm{P}} = 2.0~\textsf{ps}$. 


Let's consider an infinitely large NVE system (here we call it
"universe"). Some interesting process is happening in a small
subsystem. For equilibrium phenomena, this subsystem is the usually
considered as NVT (or NPT, $\mu$VT, depends on how the question is
raised) system.  In simulations, we only simulate the subsystem and
use the thermostat to mimic the embedment into the universe.  For
non-equilibrium simulation, we also want to simulate only the
subsystem of interest, but the physical meaning of directly applying a
thermostat designed for generating equilibrium ensemble is not clear
to a non-equilibrium process. Moreover, there is no unique way of
defining how the subsystem is coupled into the universe: It is subject
to how the perturbation comes into the subsystem, and how the energy
of the system dissipates into the universe. In the case of switching on
electric field system, two limiting scenarios are considered by us: (1)
The subsystem is isolated from the universe: The electric field is
applied to a system that conserves the energy, volume and number of
particle (i.e. a NVE subsystem in short).  The isolation here means
that the subsystem does not exchange energy, volume and number of
particle with the universe, but the perturbation (electric field) does
come in to the subsystem.  The periodic boundary condition is applied
here only to avoid the difficulty of defining a experimental
meaningful boundary in simulation.  In this scenario, the size the
subsystem is somewhat arbitrary (it depends on experimental setting),
therefore, it is important to consider the convergence of the
simulation result with respect to the size of the subregion.  (2) The
system exchange the energy, volume and particle with the universe, and
the universe is so big that the perturbation in the subsystem will not
kick the universe from equilibrium.  In practice, it is impossible to
simulate the infinite large universe, Instead, we want to simulate the
considerably smaller subsystem instead.  The idea is to divide the
simulation region into two parts (see Fig.~\ref{fig:tmp2}): the first
is called \emph{Newtonian dynamics region} that is a region around the
molecule of interest, where the dynamics is Newtonian. Note that the
molecular trajectories in the Newtonian dynamics region are the same
as if it were couple to the universe, only when a proper boundary
condition is provided. Strictly speaking, the boundary should be
provided by the out side infinitely large universe, which immediately
leads back to the numerical difficulty of simulating a infinite large
system. However, we notice that the detailed information
(i.e. position and velocity) of the every molecule outside the
Newtonian dynamics region is not relevant to the non-equilibrium
behavior of the molecule of interest, because we are only interest in
the non-equilibrium averages, which does not depends on the individual
trajectory, but on the trajectorial ensemble sampled by the
non-equilibrium simulation.  Therefore, the universe, as a boundary to
the Newtonian dynamics region, can be replace by
an efficiently thermo- and barostated subsystem, as indicated by the gray
part of Fig.~\ref{fig:tmp2}. 
% That is to say the subsystem is coupled to a
% thermostat, a barostat and a particle bath with infinitely high
% efficiency. Practically, it is impossible to simulation with a
% ``infinitely fast'' thermostat, barostat and particle bath.
We indicate the radius of the Newtonian dynamics region by
$R_{\textrm{ex}} = 1~\textsf{nm}$ (see Fig.~\ref{fig:tmp2} for
explanation), and couple it to very sufficient thermo- and
barostat. We use a Langevin thermostat coupled to the thermostating
region with a time-scale of $\tau_{T} = 0.1~\textsf{ps}$, and a
Parrinello-Rahman barostat with a time-scale of $\tau_{P} =
2.0~\textsf{ps}$.

The non-equilibrium steady temperature and pressure of the
non-thermostated simulation are $330$ K and $-500$ Bar,
correspondingly. Those of the thermostated system are $300$ K and 1
Bar, correspondingly. Therefore, we can see from Fig.~\ref{fig:tmp6}
that the relaxation of the thermostated system is faster than the
non-thermostated system. The three time-scale of metastable population
dynamics in the thermostated system is consistent with that of the
non-thermostated system. This means that thermostating, in this case,
does not qualitatively change the population dynamics of metastable
states.

\begin{figure}
  \centering
  \includegraphics[]{fig.ala/fig-meta-stat-10.eps}
  \caption{ Comparison between simulations with and without
    temperature and pressure control at $t_{\textrm{init}} =
    10~\textsf{ps}$. The The populations of the metastable sets as a
    function of time are plotted.  The dashed vertical lines indicate
    the three different time scale of metastable set population at
    $t_{\textrm{init}} = 10~\textsf{ps}$.}
  \label{fig:tmp6}
\end{figure}



% To make the non-equilibrium feasible, we assume the
% non-equilibrium process is \emph{local}, i.e. the non-equilibrium perturbation
% of the system happens only locally, and the rest of the system
% stays the unperturbed (usually assumed under ambient conditions).
% This provides us the possibility of simulating only the small subsystem
% of the universe: We can simulate only subsystem by providing proper
% boundary condition for it, so that it behaves as if it were embeded into
% the universe.

% \subsection{The water structure around methane under impluse }
% In this study, we firstly consider the system of a methane molecule solvated
% in the liquid water environment. A local density perturbation is implemented
% by applying an impluse  in a shell surronding methane, see Fig.~\ref{fig:tmp1}.
% % As far as we concerned, the situation is: we truncate the universe to the
% % subsystem, and provide proper boundary condition for this
% % subsystem.
% Here we denote the size of this subsystem by $R_{\ex}$.
% Notice this subsystem is simulated by the Newtonian
% dynamics. If the size of the
% subsystem is large enough so that the perturbation is
% small enough when it travels to the boundary of the subsystem, then a
% trivial boundary condition would be: coupling the system to a boundary
% layer equilibriated by the thermostat (and barostat or particle reservior),
% which mimics the infinitely large equilibrium universe.

% \begin{figure}
%   \centering
%   \includegraphics[width=0.3\textwidth]{fig/kick.eps}
%   \caption{Schematic plot of the density perturbation.}
%   \label{fig:tmp1}
% \end{figure}

\bibliography{ref}{}
\bibliographystyle{unsrt}





\end{document}
