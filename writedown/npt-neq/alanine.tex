% \documentclass[aip,jcp,preprint,unsortedaddress,a4paper,onecolum]{revtex4-1}
\documentclass[aip,jcp,a4paper,reprint,onecolumn]{revtex4-1}
% \documentclass[aps,pre,twocolumn]{revtex4-1}
% \documentclass[aps,jcp,groupedaddress,twocolumn,unsortedaddress]{revtex4}

\usepackage[fleqn]{amsmath}
\usepackage{amssymb}
\usepackage[dvips]{graphicx}
\usepackage{color}
\usepackage{tabularx}
\usepackage{algorithm}
\usepackage{algorithmic}

\makeatletter
\makeatother

\newcommand{\recheck}[1]{{\color{red} #1}}
\newcommand{\redc}[1]{{\color{red} #1}}
\newcommand{\bluec}[1]{{\color{blue} #1}}
\newcommand{\greenc}[1]{{\color{green} #1}}
\newcommand{\vect}[1]{\textbf{\textit{#1}}}
\newcommand{\dd}[1]{\textsf{#1}}

\newcommand{\AT}{{\textrm{{AT}}}}
\newcommand{\EX}{{\textrm{EX}}}
\newcommand{\ex}{{\textrm{ex}}}
\newcommand{\CG}{{\textrm{CG}}}
\newcommand{\HY}{{\Delta}}
\newcommand{\rdf}{{\textrm{rdf}}}



\begin{document}

\title{Report: the non-equilibrium molecular dynamics simulation}
\author{Han Wang}
\affiliation{Institute for Mathematics, Freie Universit\"at Berlin, Germany}
\author{Christof Sch\"utte}
\affiliation{Institute for Mathematics, Freie Universit\"at Berlin, Germany}
\author{Luigi Delle Site}
\affiliation{Institute for Mathematics, Freie Universit\"at Berlin, Germany}

\begin{abstract}
\end{abstract}

\maketitle

\newpage

\section{System set up}


Let's consider an infinitely large NVE system (here we call it
"universe"). Some interesting process is happening in a small subsystem. For
equilibrium phenomena, this subsystem is the usually considered as NVT
(or NPT, $\mu$VT, depends on how the question is raised) system.
In simulations, we only simulate the subsystem and
use the thermostat to mimic the embedding to the universe.
For non-equilibrium simulation, the physical meaning of directly applying
a thermostat designed for generating equilibrium ensemble is still
not clear. To make the non-equilibrium feasible, we assume the
non-equilibrium process is \emph{local}, i.e. the non-equilibrium perturbation
of the system happens only locally, and the rest of the system
stays the unperturbed (usually assumed under ambient conditions).
This provides us the possibility of simulating only the small subsystem
of the universe: We can simulate only subsystem by providing proper
boundary condition for it, so that it behaves as if it were embeded into
the universe.

\subsection{The water structure around methane under impluse }
In this study, we firstly consider the system of a methane molecule solvated
in the liquid water environment. A local density perturbation is implemented
by applying an impluse  in a shell surronding methane, see Fig.~\ref{fig:tmp1}.
% As far as we concerned, the situation is: we truncate the universe to the
% subsystem, and provide proper boundary condition for this
% subsystem.
Here we denote the size of this subsystem by $R_{\ex}$.
Notice this subsystem is simulated by the Newtonian
dynamics. If the size of the
subsystem is large enough so that the perturbation is
small enough when it travels to the boundary of the subsystem, then a
trivial boundary condition would be: coupling the system to a boundary
layer equilibriated by the thermostat (and barostat or particle reservior),
which mimics the infinitely large equilibrium universe.

\begin{figure}
  \centering
  \includegraphics[width=0.3\textwidth]{fig/kick.eps}
  \caption{Schematic plot of the density perturbation.}
  \label{fig:tmp1}
\end{figure}

\begin{figure}
  \centering
  \includegraphics[width=0.3\textwidth]{fig/thermostat.eps}
  \caption{Schematic plot of the truncated non-equilibrium subsystem.
    The white part is the subsystem and the gray part is the boundary
    condition, implemented by the thermostat and barostat mimicing
    the infinitely large universe.}
  \label{fig:tmp2}
\end{figure}


\subsection{Alanine dipeptide under electric field}

We study the response of the configuration of the alanine dipeptide
with respect to a turning on electric field. The direction of this field
is along the $x$ dirction and the magnitude with respect to time is plotted
in Fig.~\ref{fig:tmp3}. From $t=0$ to $t=t_{\textrm{init}}$, the strength
of the field grows linearly, while after $t=t_{\textrm{init}}$, the strength
is kept constant at $E_{\textrm{max}}$. 
\begin{figure}
  \centering
  \includegraphics[width=0.3\textwidth]{fig.ala/ele.field/field.eps}
  \caption{The maganitude of the external electric field as
    a function of time.}
  \label{fig:tmp3}
\end{figure}
The size of the system is $2.67\times 2.67\times 2.67\,
\textsf{nm}^3$, with one alanine dipeptide solved in 644 TIP3P water
molecules. Firstly, an equilibrium simulation of 100~\textsf{nm} was
performed, along which configurations were taken every
5~\textsf{ps}. They server as starting point of the branching
trajectories. The time step was $\Delta t = 0.002~\textsf{ps}$. The
short-range interaction was switched from $0.8$ to $1.0~\textsf{nm}$.
An energy conserving PME method was applied. Langvin thermostat was
coupled to the system, with a time-scale of $\tau = 0.5~\textsf{ps}$.
On each branching trajectory, $R_{\textrm{ex}}$ was chosen to be
$1.0$~\textsf{nm}, which means that a Langvin thermostat ($\tau =
0.1~\textsf{ps}$) was applied to water molecules that were more than
$1.0$~\textsf{nm} away from the center of the simulation box. In
addition, a Parrinello-Rahman barostat was couple to the system with a
time-scale of $\tau_{\textrm{P}} = 2.0~\textsf{ps}$. To easy the
analysis, the conformation of the alanine dipeptide is projected on
the $\psi$--$\phi$ plane, as shown in Fig.~\ref{fig:tmp4}. The
equilibrium distribution of the conformation on the $\psi$--$\phi$
plane is given in the Left plot of Fig.~\ref{fig:tmp4}.  It is clear
the conformations are populated in several clusters that are called
``metastable states'' in this research. For a clearer definition, we
manually devide the $\psi$--$\phi$ plane into 5 subregions. $A_1$ and $A_2$ are
corresponding to the alpha helix conformation. $B_1$ and $B_2$ are corresponding
to beta sheet conformation. $C$ is corresponding to ill-folding.\\
\begin{figure}
  \centering
  \includegraphics[width=0.45\textwidth]{fig.ala/ext.mode1.100.Ex.01.00.t1200ps/fig-begin-1.eps}
  \includegraphics[width=0.45\textwidth]{fig.ala/ext.mode1.100.Ex.01.00.t1200ps/fig-end-1.eps}
  \caption{The population of different metastable state. Left: the equilibrium state at $t=0$~\textsf{ps}. Right: The non-equilibrium stady state at $t=1200$~\textsf{ps}. The darker color indicates larger population on the $\psi$--$\phi$ phase space.}
  \label{fig:tmp4}
\end{figure}


For all testing cases, the maximum strength of the electric field
$E_{\textrm{max}}$ is chosen to be 1~\textsf{V/nm}.  Three values of
$t_{\textrm{init}}$ (10~\textsf{ps}, 50~\textsf{ps} and
100~\textsf{ps}) are chosen to study the response of the system to the
external electric field perturbation. The result is given in
Fig.~\ref{fig:tmp5}.  No matter how fast is the electric field turned
on, the beta sheet configuration, which presents at the beginning,
fades away as the system relaxes to the new equilibrium. The
population of alpha helix $A_1$ configuration grows from 45\% to 70\%.
alpha helix $A_2$  does not change a lot under the electric field.
The ill-folding configuration $C$ noticably grows from 0\% to 20\%.
Clearly two time-scales of relaxation are observed.  For
configurations $A_2$ and $B_2$, the relaxations are very fast and
comparable to $t_{\textrm{init}}$. For configurations $A_1$, $B_1$ and
$C$, the time-scales of relaxation are roughly 600~\textsf{ps}.
\begin{figure}
  \centering
  \includegraphics[]{fig.ala/fig-meta.eps}
  \caption{The populations of the metastable sets as a function of time.}
  \label{fig:tmp5}
\end{figure}


\end{document}