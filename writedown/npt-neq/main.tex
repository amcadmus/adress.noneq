% \documentclass[aip,jcp,preprint,unsortedaddress,a4paper,onecolum]{revtex4-1}
\documentclass[aip,jcp,a4paper,reprint,onecolumn]{revtex4-1}
% \documentclass[aps,pre,twocolumn]{revtex4-1}
% \documentclass[aps,jcp,groupedaddress,twocolumn,unsortedaddress]{revtex4}

\usepackage[fleqn]{amsmath}
\usepackage{amssymb}
\usepackage[dvips]{graphicx}
\usepackage{color}
\usepackage{tabularx}
\usepackage{algorithm}
\usepackage{algorithmic}

\makeatletter
\makeatother

\newcommand{\recheck}[1]{{\color{red} #1}}
\newcommand{\redc}[1]{{\color{red} #1}}
\newcommand{\bluec}[1]{{\color{blue} #1}}
\newcommand{\greenc}[1]{{\color{green} #1}}
\newcommand{\vect}[1]{\textbf{\textit{#1}}}
\newcommand{\dd}[1]{\textsf{#1}}

\newcommand{\AT}{{\textrm{{AT}}}}
\newcommand{\EX}{{\textrm{EX}}}
\newcommand{\ex}{{\textrm{ex}}}
\newcommand{\CG}{{\textrm{CG}}}
\newcommand{\HY}{{\Delta}}
\newcommand{\rdf}{{\textrm{rdf}}}



\begin{document}

\title{Report: the non-equilibrium molecular dynamics simulation}
\author{Han Wang}
\affiliation{Institute for Mathematics, Freie Universit\"at Berlin, Germany}
\author{Christof Sch\"utte}
\affiliation{Institute for Mathematics, Freie Universit\"at Berlin, Germany}
\author{Luigi Delle Site}
\affiliation{Institute for Mathematics, Freie Universit\"at Berlin, Germany}

\begin{abstract}
\end{abstract}

\maketitle

\newpage

\section{System set up}

\begin{figure}
  \centering
  \includegraphics[width=0.3\textwidth]{fig/kick.eps}
  \caption{Schematic plot of the density perturbation.}
  \label{fig:tmp1}
\end{figure}

Let's consider an infinitely large NVE system (here we call it
"universe"). Some interesting process is happening in a small subsystem. For
equilibrium phenomena, this subsystem is the usually considered as NVT
(or NPT, $\mu$VT) system.
In simulations, we ONLY simulate the subsystem and
use the thermostat to mimic the embedding to the universe.

\begin{figure}
  \centering
  \includegraphics[width=0.3\textwidth]{fig/thermostat.eps}
  \caption{Schematic plot of the truncated non-equilibrium subsystem.
    The white part is the subsystem and the gray part is the boundary
    condition, implemented by the thermostat and barostat mimicing
    the infinitely large universe.}
  \label{fig:tmp2}
\end{figure}


% If some non-equilibrium perturbation is going on in the subsystem,
In this study, we consider the system of a methane molecule solvated
in the liquid water environment. A local density perturbation is implemented
by applying an impluse  in a shell surronding methane, see Fig.~\ref{fig:tmp1}.
As far as we concerned, the situation is: we truncate the universe to the
subsystem, and provide proper boundary condition for this
subsystem.
Here we denote the size of this subsystem by $R_{\ex}$.
Notice this subsystem is simulated by the Newtonian
dynamics. If the size of the
subsystem is large enough so that the perturbation is
small enough when it travels to the boundary of the subsystem, then a
trivial boundary condition would be: coupling the system to a boundary
layer equilibriated by the thermostat (and barostat or particle reservior),
which mimics the infinitely large equilibrium universe.

The simulation result of the universe is important
for us, because it servers as the reference of all testing simulations.
Practically speaking, it is impossible to simulate the infinitely large
universe. We alway trancate it to a finite size system servered
by the periodic bounday condition. Therefore, it is important to
test the convergency of properties of interest with respect to the
size of the truncated universe.





\end{document}